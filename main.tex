\documentclass[gtsf,useotf,dvipdfmx,%
% Galois,%   ガロアの設定のとき
% VDN%,      現在未設定
% VUP%,      現在未設定
%uplatex,%% LaTeX文書がutf8の場合でUNICODE文字を対象した場合
]{mathplus}

\usepackage{graphicx,xcolor}
\usepackage{amsmath,amssymb}
\usepackage{uline--,macrosMP}
%\usepackage[utf8]{inputenc}%% LaTeX文書がutf8の場合でUNICODE文字を対象した場合
\usepackage{makeidx,multicol}
%

%% 装幀者
\装幀{装幀者名}

\usepackage[general]{fontmap}
%% fontmap.styのoptionの[general]を指定でipaexgを使用.これはどの環境でも処理可能.
%% [general]を取ると章に小塚ゴシックBと節にMが使われるようになる.

\includeonly{
MathPlus-sample,
%%	Chap01,
%%	Chap02,
%%	Chap03,
%%	Chap04,
%%	Chap05,
	}

\begin{document}
\frontmatter
\tableofcontents
\mainmatter
\documentclass[4x6t/12Q,gtsf,%
%              uplatex,%% LaTeX文書がutf8の場合
              useotf,dvipdfmx]{mathpulus}
\usepackage{graphicx,xcolor}
\usepackage{amsmath,amssymb}
%\usepackage[utf8]{inputenc}%% LaTeX文書がutf8の場合
\usepackage{textcomp}
\usepackage[LY1,TS1,OT1,T1]{fontenc}
\usepackage{lmodern}
\usepackage[deluxe,scale=1]{otf}[2013/11/17]
\usepackage{uline--,makeidx,multicol}

\usepackage{comment}
\usepackage{okumacro}
\usepackage{fancybox}
\usepackage{ascmac}
\usepackage{macrosMP}



%% ipaexと小塚フォント設定
\def\KozuGo{KozGoPro-Bold.otf}%% 章見出しに割り当てる小塚ゴシックのファイル名
\usepackage{fontmap}

\includeonly{
	Chap01,
%	Chap02,
%	Chap03,
%	Chap04,
%	Chap05,
	}

\begin{document}
\frontmatter
\tableofcontents
\mainmatter
\chapter{�x���k�[�C���}

\section{�x���k�[�C�E�t�@�~���[�Ɓe�o�[�[���g�f}

�X�C�X�ό��Ƃ����ƃ}�b�^�[�z�����A�����O�t���E�A�����u�����A�s�s�Ɖ]���Ύ�s�x�����A
�o�ς̒��S�`���[���b�q�A���������c�F�������v�������ԁB
�������A�o�ϊw�ҁA���w�҂Ȃ�W���l�[�u�A
�����Đ��w�҂Ȃ�A�c�c�������o�[�[���ł���B
�����̔��ϕ��w�̑̌n�̔��z�����������΁e�o�[�[���Y�f�ł���B
�o�[�[���ɂ͂����̊w��I���т��L�O���āu�x���k�[�C�ʂ�v(Bernoullistrasse)��
�u�I�C���[�ʂ�v(Eulerstrasse)������B
���炭�����Ȃ����ł����������������A�I�C���[�ɂ����
\[
\dfrac{1}{1^2}+\dfrac{1}{2^2}+\dfrac{1}{3^2}+\cdots=\dfrac{\pi^2}{6}
\]
�Ƃ��ĉ�����\footnote{
�I�C���[�̍ŏ��́e�㐔�I��@�f���܂߂ĕ��@�͍����܂�3,\, 4�ʂ肠��B}
���A���͏ے��I�Ɂu�o�[�[�����v�ƌĂ΂�Ă���B
�����ϕ��w�̓��C�v�j�b�c�ƃj���[�g���ɂ���Ĉ�܂ꂽ���A
���ꂩ��q�ׂ�x���k�[�C�Ƃ���уI�C���[�́e�o�[�[���g�f�ɂ���đ̌n�I�A�Z�p�I�ɕҐ�����A
���i�������ނˊ��������B
����ǂ��납�A���Z�̐��w�ł����x�����グ��
\begin{align*}
&1^4+2^4+3^4+4^4+\cdots +n^4
\intertext{�‚��ł�}
&1^{10}+2^{10}+3^{10}+4^{10}+\cdots +n^{10}
\end{align*}
�͂ǂ����낤���B
�������e�o�[�[���Y�f�̌����ʼn�����B

�o�[�[���̓X�C�X�恜�̓s��ł���B
�`���[���b�q�ɂ͓��{���璼�s�ւ͂��邪�A�o�[�[���ւ̓h�C�c�̃t�����N�t���g�Ȃǂ��o�R����B
��ԗ��s���y���݂�����΃p���E�������w����ł��悢�B
���̃o�[�[���̒��S�X�̈�pFreie Strasse 20�Ԓn�ɁA
�u�x���k�[�C��ǁv(����Goldene Apotheke M.Bernoulli)���X���\���Ă���B
�u�x���k�[�C�Ɓv�i�����j�͊w�҂̓V�ˉƌn�Ƃ��ÂƂɒm���Ă��邪�A
�ƋƂ͑�X��폤�ł��̖����ł���B
�V�ˉƌn�̍ŏ��̃o�b�^�[�ł��郄�R�u�E�x���k�[�C�����܂ꂽ�̂́A
���傤�ǂ��̗L���ȃp�X�J�����t�F���}�[�̊m���_�̉������Ȃ�1654�N�Ƃ�������A
�{�N��361�N�ڂɂȂ邪�A
��폤�͈ȑO��肾������ۂɂ͂���ɒ����B

�u��폤�v(�papothecary, ��Apotheke)�Ƃ͐���܎t�̂���u��ǁv�ł���B
�����Ƃ�'pharmacy', \, 'chemist'�Ƃ̐��x��̈Ⴂ�͂悭�킩��Ȃ����A
�Ƃɂ������w�E�����w�̐��ȊO�Ɉ�w�A�����w�A�A���w�������ł������̂����̂������ł��낤�B
���E�G���[�g�ł��邱�Ƃɉ����A�����炭�͖�폤�̕t�����l�͑傫���A
���‚��ꂾ���̘V�܂ł���Ώ��H��c���̗L�͎҂ł����������Ƃ��z���ɓ�Ȃ��B
���j�㑽���̐��w�ҁE�����w�҂������ɋ��X�Ƃ����邢�͑�w�|�X�g�ŋ�J�������ƂƂ���ׂ�ƁA
�܂��ɗ]�T�̐����œ��փP���J�⒇�Ⴂ�̗]�n�����������킯�ł���B
�t�Ɍ����΁A�x�M�Ɉ����������̐��E�j���̊w�҂�y�o���A
�ƋƂ������̒����ɏ‚��葱�����͈̂̋ƂƂ������Ȃ�\footnote{
���Ƃ����{�̗{�q���x�͂Ȃ��������Ƃɒ��ӂ���B}�B

���āA���w�╨���w���w��ł��āA���́e�x���k�[�C�f�̖��ɏo���Ȃ����Ƃ͂Ȃ����낤�B
���Ƃ���
\begin{itemize}
\item �x���k�[�C���s�A�x���k�[�C�̓񍀕��z
\item �x���k�[�C��
\item �x���k�[�C�̔���������
\item �x���k�[�C�̒藝
\item �u�Z���g�E�y�e���u���N�̋t���v�̃x���k�[�C�ɂ�����
\end{itemize}
�Ȃǂ́A�݂ȃx���k�[�C�Ƃ̊w�҂����̖��O�������Ă���B
�܂��A���R�u�E�x���k�[�C�i�����j�A���̒탈�n���E�x���k�[�C�i�����j�A
���̎q�_�j�G���E�x���k�[�C��3�l�������Ȃ��Ă͂Ȃ�Ȃ��B
�����ŁA�_�j�G���̏]�Z��j�R���E�X(II)�E�x���k�[�C�i�����j�������Ă������B
���̌�w�҂̓`���͈����p�����w�A�N�w�A�_�w���邢�͌��z�w�Ȃǂ̊e����Ɋ��􂷂�l�ނ��y�o���A���݂Ɏ����Ă���B
�������h�C�c�̕��w�҃w���}���E�w�b�Z�v�l�Ȃǂ�y�o���Ă���B
���̉ؗ�Ȋw�҉ƌn�́A���y�ƃo�b�n�Ƃƕ���ň�`�w�̃e�L�X�g�ɂ��o������B
%�܂����v�w�̑n�n�҂̈�lF�E�S���g���i�����j�́w�����x�Ŏ��グ���Ă��邭�炢�ł���B

�ł́A�V�ˉƌn�̈�`�𓝌v�I�ɕ��͂�������F.�S���g���w��`�I�V�ˁx
(Heredetary Genius,����)�ɂ���x���k�[�C�ƕ]�����Ă݂悤�B
�������A�ƕ�\footnote{
���S�l�����猩�āAB�͌Z��AN�͉��Ȃǂ�\���B}
�̘A���ɒ��ڂ��A����������]���S�Ŋw��I�ɂ͕s�\���A�s�ύt�ȓ_�����邪�A
���i�`�ʂɓ��ݍ���ł���_�́i�q�ϐ��͂Ƃɂ������j���ڂ������B
\[
\text{��������܂����H}
\]

{\gt ���R�u�E�x���k�[�C}�@��̑�ɕ��O�ꂽ���̍����Ȑ��w�҂ƉȊw�҂����𐶂񂾃X�C�X�̉ƌn�̂����A�����̍ŏ�������B
��Ƃ͑S�ʓI�ɂ��񂩑������Œ݂��ɒ������������B
�����Ȃ�ґ����A80�΂𒴂���3�l�𐔂���B
���R�u�͋���E�҂̗\��Ȃ���A�������琔�w�ɐ�S���邪���̓��@�͂��Ƃ͋��R�ł������B
���̋C���͒_�`���i�q�|�N���e�X��4�̉t���ސ��ŁA����I�œ{����ۂ��U���I�j�ŗJ�T�^�ł���B
���������s���͂ނ���x���B
��������邪���̑���ȑԓx�͋ɓx�ɒ��������B
���񂩂Ƒ΍R�S�͕K�R�B
���w�҂Ƃ��ẴI���W�i���e�B�[�ƍ˔\�͍ō����x���B
�t�����X�̃A�J�f�~�[����B

B {\gt ���n���E�x���k�[�C}�@�����͏��Ƃ̓��̗\��ł��������A
�i�H��ύX���m���R�n�Ȋw�Ɖ��w�ɐi�ށB
�t�����X�E�A�J�f�~�[����i�_�����x�[���ɂ��̐l�^������j�B
�ȉ�5�l�̑c�ƂȂ�B
�m���F�������w�̋Ɛт͊֐S�̑Ώۂ���O��Ă���B�n

N.{\gt �j�R���X�E�x���k�[�C}�i���N31�΁j�@��͂葽��Ȑ��w�I�˔\�Ɍb�܂�邪�A�Z���g�E�y�e���u���O�ɂĚ�܁B
���n�̐V���A�J�f�~�[�Ɍ��ʂ�^������l�B

N.{\gt �_�j�G���E�x���k�[�C}�@���Ȉ�A�A���w�ҁA��U�w�҂ŁA���̗͊w�̗L������������B
���ɑ��n�B
5��̎�܂����邪�A1��͕����n���������A�q�̐����Ɏ��i���������B
�t�����X�E�A�J�f�~�[����i�R���h���Z�ɂ��̐l�^������j�B�@(�ȉ����j�B

\noindent
\Fig[�q�n�}�r�q�ʐ^�r]{\textwidth}{10\baselineskip}

\section{�ƋƂ͖�폤�A�����ė���鐸�_�͉��v�h}

�x���k�[�C�Ƃ̂��������̃��[�c�́A�k������ł́A
17���I�̃x���M�[�i�����Ƃ��Ă̓X�y�C���̃l�[�f�������h�j�̃A���g���[�v�݂̖�폤�ł�����
�i������ȑO�ɁA�I�����_�̃A���X�e���_���o�g�Ƃ̏��������邪�A�m���łȂ��j�B
�������d���ꂽ�����̓��[���b�p�ł͎Y�o���ꂸ�A���m�f�՗R���̖��[���i�͂���߂č����t�����l���傫�������B
��Ɨǂ��䗦�ŗL���ɓ�����������Ă������炢�ł���B

�ƋƂ���폤�ł������w��ɂ́A���E�j�̋��ȏ��ɂ��o�Ă��郈�[���b�p�Ɠ��m�Ƃ̍����f�Ղ̗��j���������B
���E�j�Ŋw�Ԃ悤�ɁA1453�N�r�U���`���鍑�i�����[�}�鍑�j�̎�s�R���X�^���`�m�[�v�����C�X�����̎x�z�Ɋח����A
�I�X�}���E�g���R�鍑�����m�Ƃ̌��՘H�̊Ԃɓ��������ʁA
���m�f�Ղ͒n���C���S����V�q�H���J���Ɛ肵���X�y�C���A�|���g�K���̎�Ɉڂ����B
���n�A���g���[�v�ɂ̓I�[�X�g���A����уX�y�C���̗L���ȃn�v�X�u���N�Ƃ��N�Ղ��Ă�������A
�A���g���[�v�͂��̖f�ՏW�ϒn�Ƃ��Ĕɉh��搉̂����B

�u�n�v�X�u���O�Ɓv�Ƃ����΁A����͉��邪���̃}���[�E�A���g���l�b�g�𐶂݁A
�܂����[�c�@���g�A�x�[�g�[�x���A�V���[�x���g�̌����ȁA
���n���E�V���g���E�X�̃����c��A�z���邠�̉؂₩�ȃE�B�[�����v���‚����A
�����n�v�X�u���O�鍑�̎x�z�͍L�͂��‹���ŁA�����ȗ��������[���b�p�̑啕���̎�u���S�[�j����Ƃ̍�����ʂ��āA
���̃I�����_�E�x���M�[�̒n��܂ł����̗L���Ă����̂ł���B
�����ցA���łɑO���I�Ɏn�܂����@�����v�́A�@���A�����A�Љ�A�����ɑ傫�ȉe���������炵���B
���ƂɁA�I�����_�A�x���M�[����уh�C�c�ł́A
�����ȗ��̃��[�}�E�J�g���b�N���i�����j�ɑ΍R���ĉ��v�h�v���e�X�^���g�i�V���j�̐Z���͑傫���A
�l�X�̐��_�̊o����ʂ��āA�����琭���A�Љ�𓮂����G�l���M�[�ƂȂ��Ă����B
����ɑ΂��ăn�v�X�u���N�Ƃ̑�X�N�傽���̓��[�}�E�J�g���b�N�̌싳�҂�C���āA���܂��܂Ȉ������������̂ł���B

�x���k�[�C�Ƃ͐V���k�Ƃ��ĉ��v�I�Ȑi��̋C�ۂ������Ă�������A
���n�̎x�z�҂ɂ��V���i�J�����@���h�j�����͋����I�A�ꐧ�I�Ɗ�����ꂽ�B
���łɈ��@�������u�O�\�N�푈�v�i1618--1648�j�͏I�����@���n�}�͈ꉞ�̈�������Ă����̂ŁA
�x���k�[�C�Ƃ�����17���I�ȍ~�X�C�X�̃o�[�[���ɐV�������Z�̒n�����o�����B
�X�C�X�͒����ȗ��̒����Ɨ��̐킢�̖��A
�O�\�N�푈�̍u�a�i�E�F�X�g�t�@���A���j��Ɨ��������Ƃ����ւ荂�����‚�����Ƃ����V�����ł������B
�悭�m����悤�ɃX�C�X�̉i�������͂���ȍ~�����܂ň�x������ꂽ���Ƃ͂Ȃ��B

�o�[�[���s���ɂ����Ă��x���k�[�C�Ƃ͏d�����Ȃ��A
���S�̎s�X���ɂ��u�x���k�[�C�ʂ�v(Bernoullistrasse)�����邱�Ƃ͂��łɏq�ׂ��B
�X�C�X��{���Ƃ����@�����v�҃J�����@�����ꎞ�o�[�[��������̒n�̈�‚Ƃ������A
���̃o�[�[���̒m�I�A���_�I�`���͐V���k�x���k�[�C�Ƃɂ͂ӂ��킵�����̂ł������B
�J�����@���̋֗~�I�ȉ��v��`�͑����ɉ՗�ŁA
�u���e�v�̐��_���Ȃđ΍R�����G���X���X�͐��Ɂu�J�����@���΃G���X���X�v�̑Ό��ł��m���邪�A�������o�[�[���Ő��U���߂����B
���̂悤�ɏd�w���Ȃ��_�w�I���͋C�̒��ŎႫ���ɏC����ς񂾂̂��x���k�[�C�Ƃ̐l�X�ł���A
����͏I���ς�邱�ƂȂ����̐l�X�̐��_�̔w���ƂȂ����B

\noindent
\Fig[���@���ʐl���W�@��]{\textwidth}{10\baselineskip}

\section{�Ȃ�$dx=0$�ł͂Ȃ��̂�}

�{���̖ړI�ł͂Ȃ����A�o�[�[���̂�����l�̃r�b�O�E�l�[���̓I�C���[�ł���B
�I�C���[�͏��N���ォ��x���k�[�C�Ƃɏo���肵�A�_�j�G���E�x���k�[�C�Ƃ͂��ꂱ���e�|�n�̗F�f�ł������B
���n�����ƒ닳�t�Ƃ��A�_�j�G���Ɗ�����ׂ��B
���ƂɘA�Ȃ�҂Ƃ��ăI�C���[�͏I���v���[�����p�Ɖ]����قǐM�‚ɓĂ��h����Ȑl�ł������B
�w�L���X�g���l�����T�x�ɂ��u�I�C���[�v�̍�������B

�l�̏@����̐M�O�͊w��ɂ��e����^����B
������$dx$�͖{�����C�v�j�b�c�䂸��ł������B
���w�j�Ɖ����Б��Y�̓��C�v�j�b�c�̐��w���u�_�I���w�v�ƕ]�������A�������A
��̃C�M���X�̊ϔO�I�o����`�N�w�҃o�[�N���[�ɉ]�킹��Ȃ�A
�����ɏ�����$dx$�Ƃ͂ǂ��l���Ă�����0�ɑ��Ȃ�Ȃ��̂ł͂Ȃ����B
���Ƃɓ�K���֐�$d(dx)$�‚܂�$(dx)^2$�Ƃ͉����Ƌ򂢉�����o�[�N���[��
�I�C���[�͏������������i�_���ɂ��Ȃ炸�j�_�w�I�ɑނ��Ă���B
�l�̗L���v�l�Łu�����v�܂Ŋ��S�ɉ��؂邱�Ƃ͂ł��Ȃ��B
�������̊T�O�ɂ͐l�Ԃ̊ϔO�v�l����͋ߕt����_�I�[��������߂��Ă���B
�������ɐl�Ԃ̊ϔO�ɂ��L���v�l�ɏ]���Ȃ�o�[�N���[�͐������A$dx=0$�ƂȂ��āA
���Z�̐��w�̐搶������������ʂ݂̂Ȃ炸�A���݂̔����ϕ��w�͂̂�������j�]���Ă��܂��B
�I�C���[�����҂͂��̕s�Žv�c�ɂ܂��܂��������邾�낤�B
�S�����ϕ��w��_�w�I�x�_�݂̂ŋ~�ς����I�C���[�̗͗ʂɂ͋��Q����B
�l�̗L���v�l�ɍ����I�Ȕ��f��~�𖽂���@���͈�ʂɍl������قǕs�����Ȃ��̂ł͂Ȃ��B

\section{���[�y���e���C}

�x���k�[�C�ƂƐe�D������I�C���[�ɂ��e����^�����̂́A���������l�I�ɂȂ邪�A
�������o�[�[���̕����w�ҁA���w�҂Łu�ŏ���p�̌����v�̃��[�y���e���C�ł���B
���[�y���e���C�́A�K�����C�A�z�C�w���X�A�j���[�g���A���C�v�j�b�c�A�x���k�[�C�ƁA
�I�C���[�ɋ��ʂɌ����鎞��v�z��T�^�I�ɑ̌������l�ŁA�_�́u�ۗ��v�Ƃ��Đ��E��
�őP�A�ŗǂ̏�Ԃɑn�������ƋK�肵���i���E�G�w�����w�j�x���Q�Ɓj�B
���ہA���ꂪ�ϕ��@�̎v�z�I���^�ƂȂ����B
���̔��z�͂��܂�ɂ��_��I�ŁA���j�I�ɂ͖Y�p���ꂪ���ł���B
���ہA�͊w�̃e�L�X�g�ɂ��ꕔ���������͂�o�����Ȃ����A
�傫�Ȕ��z�̎n�܂�͂����܂��ł͂����肵�Ȃ����Ƃ����΂��΂ł���B
���n���E�x���k�[�C�͍ŏ��̕ϕ��@���̈�‚Ƃ��āu�ő��~�����v�̉ۑ��^�����̉��̓��o�ɍv���������A
���܂��������Ƃ͂����A���[�y���e���C�A�����ăx���k�[�C�ƁA
�I�C���[�̃o�[�[���g�ɂ���Ďn�܂����ϕ��@�̓��O�����W���Ɏ󂯌p����A
��͗͊w��͊w�n�̌����Ƃ��ĕ����w�̑S����ł̗L�͂ȕ��@�Ƃ��āA
����ɂ͐����o�ϊw�ɂ����Ď��ԍœK���̕W���I���_�Ƃ��Č��݂Ɏ����Ă���B

�������A�u���O�����W���֐��v�A�u�n�~���g���֐��v���������ɂȂ�����͗͊w�ł́A
�{�ƍŏ���p�̌����͈�藝�܂ŗ��Ƃ����߈��𖡂���Ă���B
����ɂ́A���O�����W�������[�y���e���C������������Ƃ��`�����邪�A����΂���ł͂Ȃ��낤�B
���[�y���e���C�̔ӔN�̏Z�����܂��͍������o�[�[���̒��S�s�X�̈�p�Ɍ�������B

\noindent
\Fig[�q�o�[�[���ʐ^�r�q�n�}�r]{\textwidth}{10\baselineskip}

\section{�x���k�[�C�Ƃ𐶂ݏo��������}

���������Ă��̂��Ƃ͐�����B
�M�҂͐��w�̎g�p�҂ł��舤�D�҂ł��邪�A���w���߂��邢�����͋C��y����������Ă�������������̂́A
�G�z�ł��邪�A�R���ė���Ƃ��낪�Y����Ă��邩��ł͂Ȃ����Ɗ�����B
��㐔�w�̎w���I�n�ʂɂ�������i���g���w���㐔�w�̊�b�T�O�x�i1944�j�ɏq�ׂĂ���Ƃ�����Љ�悤�B

\begin{quote}
---�T�O�̌`���I������^����̂ł͂Ȃ��A���̗R���ė���Ƃ���A���̐��w�S�̂ɂ�����݂肩�������A
�����Ắu�w��v�ɂ����邻�̒n�ʂ������炩�ɂ���̂ɖ𗧂‚̂��ڕW�ł���B
����͏��Ȃ��Ƃ����w���̂��̗̂��j���Ȃ݂Ȃ���Ȃ���Ȃ���΂Ȃ�Ȃ��B
���j�̔��Ȃ͔���ȊՎ��Ƃł͌����ĂȂ��B
�w��̖{���̔c���̂��߂ɂނ���K�R�I�Ȏ�i�ł���B
���‚͌ÓT�ɂ����đn���҂̌��ɐڂ���̂͐S�y�����Ƃł���B---
\end{quote}

���w�҂łȂ��Ƃ������̊֐S�����������n���҂̌��ɐڂ���̂͊y�����B
�n���҂����̎���ɂ͂��̎�����L�̕��͋C�����肻�̒��ɑn���҂������������Ƃɂ́A���䂩�����C����������B

�ł́A�x���k�[�C�Ƃ̐l�X��I�C���[����ł������͋C�͂ǂ̂悤�Ȃ��̂ł������낤���B

---�����ʂɊ��‚��~�߂��Ƃ��Ă������҂̏@���I�����͉����̐��ł��ނ�̕����w��̊����ɉe����^�����B
������񌤋��Ҍl�̌����͌����̋���̋��`�Ƃ͂������ɕʌ‚̂��̂ł���A
�ނ���N�w�I�ɉe�����ꂽ�Ƃ݂�ׂ����̂ł��낤�B
�P�v���[�A�f�J���g�A���C�v�j�b�c�A�j���[�g���݂͂Ȃ��̎����𕠑��Ȃ��q�ׂĂ���B
�����18���I�Ɂi���[�y���e���C�́j�ŏ���p�̌����̂���������������B
���̎����ɃJ���g�̓N�w���Ȋw�I�F���Ə@���I�M�‚̊��S�ȓƗ�����錾����
�i�w���������ᔻ�x�͉Ȋw�҂̂��߂ɏ����ꂽ�ʂ�ے�ł��Ȃ��j�B
����ł��̎������ɂȂ�ƕ����w�̒���̒��ɏ@���I�Ȃ��̂͑S�R��������Ȃ��Ȃ�B

������Ƃ����Č����Č㐢�̎��R�Ȋw�҂̌����������S�̉��[���ނ�̏@���S��
���т‚��Ă��Ȃ������Ƃ������_�ɂ͂Ȃ�Ȃ��B
�w���̐^���̑̌������牻�̈Ӗ��ɂ�����theoria���Ȃ킿�u�_�̐ۗ��v�ł���Ƃ�������́A
�܂��ɂ����̂Ȃ���{\gt �ō��̂��̂ɑ΂���S����̌Ăт���}�ɈႢ�Ȃ��B
�m���Ɍ������Ă��ꂪ���p�����Ƃ����l�����邱�ƂȂ��ɁA�Ђ��ނ��ɓw�͂��邱�Ƃ�
�u����N��ʂ��Đl�Ԃ̖{���I�X���ł���A�l�Ԃ̍��M�Ȗ{�������̏ے��v�iK�D���X�p�[�X�j�ł���
�iv.���E�G\footnote{
1953�NWatson, Crick�����̊j�_�̓�d�点��̍\���𔭌������Ƃ�X�������wXray-
Crystallography�̑傫�ȏ��������������A����X�������w��1914�N���E�G�̌������ȂĎn�܂�B
���E�G�������w�j���u������x����̌����̂��ƂɁv���������̂��w�����w�j�x(1947)�ł��邪�A
�����ɏq�ׂ�ꂽ�Ȋw�҂̐��_�̓x���k�[�C�Ƃɂ����o�����B}
�w�����w�j�x�j�B

\section{����͎n�܂�Ȃ��n�܂�i�o�����{�C���j}

�����͉]���Ă��A�̑�Ȃ��Ƃ���̎n�܂�͂��΂��΃n�b�L�����Ȃ��B
���̂��Ƃ̌`���͌`������悤�ł��Ă���Ƃ͂킩��Ȃ��B
�n�܂�̕s���S�����p���Ă��ꂩ��N���邱�Ƃ̈̑傳�𘺂߂����Ă���B

�x�[�g�[�x����\ruby{������}{�V���t�H�j�[}�����ׂđ�5�ԁw�^���x�̂悤��
�e�h�A�h�A�h�A�h�[���f�Ƃ����������‚킯�ł͂Ȃ��B
��9�ԁw�����t���xmit Schlusschor�̖`���ł́A�ǂ�����Ƃ��Ȃ��A�����ł������ȁA�ɂ��������炸
�������܂��΂͂�����Ɛ����ɗL�@�I�Ɍ������錷�̂����߂��ŁA���������n�܂��Ă��邫�������_��I�ɂق̂߂������B
It begins without beginning. \, 
����w���҂͂������B

��ȉƂł���]�_�Ƃł�����������O�Y�̕]�F

\begin{quote}
---�����̒掦�ɐ旧���Č���鏘�́A�����ɑ΂�����߂ėL�@�I�֘A�������A
���‚܂����I�Ȑ��i���I�݂ɕ\�����Ă���_�ɂ����Ă����ꂽ���̂ł���B
����͂܂�A����E���Ƃ̍�銮�S�ܓx�ɂ���Ďn�߂��邪
�a���Ƃ��đ�O���������Ă��邽�߂ɂ͂Ȃ͂��s����ȁi���������ď��I�ȁj�C�������o���Ă���B
���̌ܓx�������s�A�j�V���łȂ��Ă��邤���ɁA�����`���̍ł��d�v�ȉ��`����������Ă���B---
\end{quote}

\noindent
\Fig[�y��]{\textwidth}{10\baselineskip}

�ق�Ƃ��������΁A���ϕ��̗��j�̓x���k�[�C�Ƃ��ȂĎn�܂�A�قǂȂ��o�g�̃I�C���[�����������������B
�j���[�g���ƃ��C�v�j�b�c
---�ǂ���ł��邩�͂Ƃɂ�����---
�ŁA\kenten{�Ƃɂ�����}���ϕ����n�܂������Ƃ͌����̗��j�N�\��̎����ł͂��邪�A
����𕶎��ʂ�́u�n�܂�v�ƌ���ɂ͂��܂�ɂ��ّ��ł���A
���҂Ƃ͂�������Ă���B
�j���[�g���́w�v�����L�s�A�E�}�e�}�e�B�J�x�͔��ϕ��w�̋��񐹏��u�n���L�v�Ƃ������悤���A
�������Ȃ���f�J���g�̊v���I�ȑ㐔�L�@�‚܂�u�����v�͂܂��قڊF���A����Җ{�l�̋Ό������A
�ĉ�ȃL�����N�^�[���������̂悤�Ɂw�􉽊w���{�x�̘_�ؗ��V�����̂܂ܓ��P���Ă���B
�����̓��e����A���ʐl�ɂƂ��Ă��ǂ߂���̂ł͂Ȃ��A���ꂾ���ɔ��ϕ��w�e�L�X�g�Ƃ������̂ł͂Ȃ��B
��i�͂ނ��둑�����ɖ����Ă���A�n���[�i���̖��̕t�����d���̔����ҁj�ɂ��u�̐l�^���v�Ɍ���悤�ɁA
�_�̑n���̒����A���̐ۗ��ƌo�ς�̂�����̂Ƃ��āA���j�コ��R�ƋP���Ă���B


�����A�j���[�g���ɑ΂��D�搫�𑈂����C�v�j�b�c���w�����x�ɂ���Ĕ��ϕ��w�̓�����\���������B
�����A��i�͑S�̂Ƃ��Ė��m���������A
���_�̌`���Ȃ��ɂ͌p���ҁi�����j�x���k�[�C�A�I�C���[�̍˔\�������˂΂Ȃ�Ȃ������B
�����A�x���k�[�C�ɂƂ��Ắu�����Ȃ��̂悤�Ȃ��́v�Ɍ������Ƃ����B
�ɂ�������炸�A����̔��ϕ��ɂ‚Ȃ�����Ƃ��ẮA�ϕ��i���͘a�j�L��$\int $\footnote{
s�̕ό`�B���e����\�L��s�͏㉺�Ɉُ�ɒ����L�сAf�Ƃ̌�����������ł������B}
�̓����i���C�v�j�b�c�j�A�p��u�ϕ��v�̓����i�x���k�[�C�j�A�����������̋��ρi���j�A
�̌n�w��������́x�̊����i�I�C���[�j�ɂ���āA
���`�Ƃ��Ă͖����Ɂe���C�v�j�b�c�̐��f�̏�Ɍ��オ����Ƃ����Ă悢�B

�w�҂̃p�[�\�i���e�B�[�������̔g���v�f�����A�����͈̑�Ȏn�܂�̒��ł̓G�s�\�[�h�ǂ܂�ł���B
�j���[�g���������u���i�������l�v�Łi�z�[�L���O�j�A�u���ɐS����j�����v�i�j���[�g���B���C�v�j�b�c�f�̐܁j�Ȃǂ�
���₨��Ƃ����v�킹��B
���̃j���[�g�����A�u�C�M���X�l�̏����͕K�v�Ƃ��Ȃ��v�Ƃ����x���k�[�C�Ƃ͂ǂ����B
�������̎��i�ƓG�ΓI�ȑ΍R�S�Ɠ��փP���J�����ʓI�ɂ͌�葐�ɂ����Ȃ��B
���オ����΁A�x���k�[�C�ƃj���[�g���̊Ԃɂ���M��K�₪�������悤�ŁA��͂�w�҂̐����͂ӂ‚��l�̊֌W����͐����ʂꂸ�A
����ł��ĕ\�ʏ���͕����u�Ă��Ȃ��_�́A�w�҂Ƃ����l�ނ̂��肪�������T�ł���B
���ہA�w�҂���J�͑������A���̍K�����Ȃ���Έ̑�Ȏ��т͎c���Ȃ��B

\chapter{���R�u�E�x���k�[�C}

\section{���R�u�E�x���k�[�C}

���R�u�E�x���k�[�C�i�����j���ȂāA�x���k�[�C�Ƃ̓V�˗��͎n�܂�B
���j�R���X�E�x���k�[�C�i�����j�̓��R�u�ɐ_�w�҂ɂȂ邱�Ƃ�]��ł����B
�u�_�w�v�itheology�j�͓��{�l�ɂ͑z���ł��Ȃ����A�u�_�v�ɂ‚��Ă̊w�A
���ƂɃL���X�g���̋��{�A���j�A���`�A�M�����̘_���ɂ‚��g�D�I�Ɍ�������w��ł���B
�L���X�g���͌Ñ�A�����ȗ��A�������[���b�p�̐��_�A�����̊�Ղł���������A
�w��̑̌n�ɂ����Ă����ׂĂ̊w��̕M���i�̏d���ƌ��Ђ������Ă����B
���E�҂����O�̐M�‚̗ǂ�������A����Ƃ��đ��h���W�߁A�Љ�I�АM�ɂ��������̂�����A�E�ƂƂ��Ă����肵���I���ł������B
�ߑ㏉���̑����̐��w�҂͐”N����ɐ_�w�̏C�����󂯁A���̉F���_�͔��z�̊�b���{�ƂȂ�����͑����B
���Ƃɐe���_�w���E�҂̃L�����A��]��Ŋw����^���A�������Ȃ���{�l�̊֐S���ӂ���ݐ_�w�ɖ����ł��Ȃ��Ƃ����i�H�������܂�ł������B
�����Ƃ����w�҂ƂȂ��Ă�����_�w�_���̒�����c�����̂̓I�C���[�A�����Ăقړ�����̊m���_�̃x�C�Y�i�����j�ł������B
���Ƀx�C�Y�͎��g���V���i�v���e�X�^���g�j�̖q�t�ł���A�I�C���[���q�t�̎q�ł����ďI���M�[���l���𑗂����B
���R�u�͐��E�҂������w�̓���I�ԁB

���āA���R�u�E�x���k�[�C�Ƃ����΁u�A���X�E�R���C�F�N�^���f�B�vArs Conjectandi�i�󂵂āw�����p�x�j��
�����m��l���m��m���_�̈��ÓT�����A���̑�삨��т��̑�II���Ɋ܂܂��_��I��
---�����͌����Ȃ���---
���Ȃ����w�̂����������Ɏp��������u�x���k�[�C���v�͌�ł܂Ƃ߂Ă������q�ׂ邱�Ƃɂ��A
�����ł́A���R�u�E�x���k�[�C�̍L�ĂȋƐяЉ�̈�[�Ƃ��āA
�ނ̔��z���ȂĎn�܂�탈�n���A����ɃI�C���[�A���O�����W���Ɏ󂯌p����č����́u�ϕ��@�v�ƂȂ����A
�ŏ��̉ۑ�u\ruby{�ő��i�Z�j�~����}{�u���L�X�g�N���[�l}�v�����������茩�Ă������B

�ϕ��@�͌��݂́u��͗͊w�v�u�͊w�n�v���x���鐔�w�I���@�ƂȂ��Ă��邪�A
�ŏ��͊􉽊w�̋Ȑ��_�Ƃ��ďo�������B���̒a���ɂ͎��̂悤�Ȋ􉽌��w�̑O�j������B

\begin{quote}
---���͕���A���̓_$\mathrm{P}(0, \, 1)$����o������A�̒����E���ɐi�݁A
����B�Ƃ̋��E$x$����̓_$(x, \, 0)$�ŕ���B�֓��˂��A
����B���ł͋��܂��ē_$\mathrm{Q}(1, \, -1)$�֒B������̂Ƃ���B
���܁A���E�֓��˂���p�i���ˊp�j��$i$�A
���E������܂��ďo�čs���p�i���܊p�j��$r$�A
����A���AB���ł̑��������ꂼ��$v_\mathrm{A},\, v_\mathrm{B}$�Ƃ���Ƃ��A
���̐i�HP��X��Q�����肹��---
\end{quote}

����X�i���Ȃ킿$x$�j���߂�@���͊􉽌��w�Łu�X�l���̖@���v�Ƃ����A
\[
\frac{\sin i}{\sin r}=\frac{v_\mathrm{A}}{v_\mathrm{B}} %\label{1}
\]
�ł���B
���Ƃ���$v_\mathrm{A} > v_\mathrm{B}$�̂Ƃ��͐}�̔@������PQ���㑤�ɋ��Ȃ���
�i$v_\mathrm{A} < v_\mathrm{B}$�̂Ƃ��́A�t�Ɍo�H�͉����֋��Ȃ���j

\noindent
\Fig[����2��]{\textwidth}{10\baselineskip}

�v�񂷂�ƁA���͑����̑傫�������֋��Ȃ��A���A�o�H�̕����i�p$i, \, r$�j��$\sin$�͑����ɔ�Ⴕ�Ă���B

�z�C�w���X�ƃt�F���}�[�̓X�l���̖@���͌o�H$\mathrm{P} \to \mathrm{Q}$�̓��B����
\[
 \frac{\mathrm{PX}}{v_\mathrm{A}}
+\frac{\mathrm{XQ}}{v_\mathrm{B}} %\label{3}
\]
���ŒZ�ƂȂ�悤��P��X��Q�����܂�Ƃ���΁A������邱�Ƃ��������B
���ہA�㎮��$x$��
\[
 \frac{\sqrt{1+ x^2}}{v_\mathrm{A}}
+\frac{\sqrt{1+(1-x)^2}}{v_\mathrm{B}} %\label{4}
\]
�ƕ\�킵�A�����$x$�Ŕ��������
\[
 \frac{  x}{v_\mathrm{A} \sqrt{1+ x^2}}
-\frac{1-x}{v_\mathrm{B} \sqrt{1+ (1-x)^2}}=0 %\label{5}
\]
����A��������
\[
\frac{\sin i}{v_\mathrm{A}}-\frac{\sin r}{v_\mathrm{B}}=0 %\label{6}
\]
���o��B

���Ƃ���A����C�AB�����Ȃ�$v_\mathrm{A} : v_\mathrm{B}=1.333:1$�ł����āA
\begin{align*}
x     &= \\
\sin i&=\qquad (i= \qquad ) \\
\sin r&=\qquad (r= \qquad )
\end{align*}

���̂悤�ɁA�����ɂ���Čo�H�i���邢�͊֐��̃O���t�j���߂���@���u�ϕ��@�v�ł���B
���̗�ł́A��͌o�H�ɂ����鎞�Ԃł���B
J�E�x���k�[�C�͂��̌��̐i�H����Ɋւ���z�C�w���X�ƃt�F���}�[�̌��ʂɃq���g�𓾂āA
���̂�����Ȑ��ɉ����ė�������Ƃ��́u�ŒZ���Ԃ̖��v���������B

�ŒZ���Ԃ̖��͏]�O���~�ʂ̂悤�Ȍ�������i�K�����I�j���o����Ă������ł���B
�������Ɍ��̋��Ȃɂ����āA�o�H�������̑傫�����֋��Ȃ��邱�ƂŁA�o�ߎ��Ԃ�Z�����邱�Ƃ��ł���悤��
---���傤�ǁA���H�ʼn^�]����Ƃ��A���������I�ɉ����ł��������H
---�n�C�E�F�C�Ƃ��^�[���p�C�N�ȂǂƉ]����---
�ɏ�邱�ƂŁA���Ԃ�Z�k�ł���悤��---
���̗����̊􉽊w�ɂ����Ă��A�d�͉����x�ɂ���ĉ����ɂ����鑬�����傫���̂ŁA�����o�H�Ƃ��Ď΂߂ɐ��`�ɗ�����������A
����ɉ����ɋ��Ȃ��邱�ƂŁA�������Ԃ��ŒZ�ɂł���B
���炽�߂āA���̍ŒZ���Ԃ̋Ȑ��ł���u�ŒZ�~�����v�𐳂������߂邱�Ƃ��ۑ�ƂȂ�B

���́A�탈�n���E�x���k�[�C����Â����e�R���y�����f�Ō������ꂽ��
�i�Z�ɑ΂��钧��̓��S�̈Ӑ}���������j�A���R�u�����n�������������ɒB���Ă����B
���̍ŒZ�~�����̗��j�I�������㕗�ɉ����Ă݂悤�B
�t�F���}�[�̌����̃A�i���W�[����A�����ɔ��������‚̋��ܑw������ƍl����ƁA

\noindent
\Fig[����7�C�}���@Vanner p.165]{\textwidth}{5\baselineskip}

\[
\frac{v}{\sin \alpha}=K
\]
�ƂȂ��Ă���B
$\alpha$�͋Ȑ��̕����i�����ƂȂ��p�j�A$v$�͂��̓_�ł̑����ł���B
���̗̂͊w����A$g$���d�͉����x�Ƃ���
\[
v=\sqrt{2gy} \qquad (y\text{�͐�����������})
\]

�܂��A$y'$������W���Ƃ���ƁA
\[
\sin \alpha =1 \Big/ \sqrt{1+{y'}^2} %\label{8}
\]
����������
\[
\sqrt{1+{y'}^2} \cdot \sqrt{2gy}=K %\label{9}
\]
���ꂩ��A$y'=dy/dx$�������o���ƁA����������
\[
dx=\sqrt{\dfrac{y}{c-y}} dy \qquad
\left( c=\dfrac{K^2}{2g} \right) %\label{10}
\]
�𓾂邪�A$\sqrt{}$ ���̕�����������邽��
\[
y=c \cdot \sin^2 s = (c/2)(1-\cos^2 s)
\]
�ƕϊ�����ƁA�e�Ղɐϕ��i�����������̉��j�͋��߂���
\[
x-x_0=cs-(c/2)\sin^2 s
\]
�ƂȂ�B
$s$�����ϐ��Ƃ���_$(c(s), \, y(s))$�̋O�Ղ́u�T�C�N���C�h�v�icycloid�j�Ƃ�΂�A
���ꂪ�ŒZ�~�����̉��ł���B
���́u�T�C�N���C�h�v�́e�[�~�f�i"-oid"�͋[�`�j���Ӗ�����B
����$x_0\equiv 0$�Ȃ�A
\[
(n-cs)^2+(y-c/2)^2=(c/2)^2
\]
����A$(x(s),\, y(s))$�͒��S��$c/2$�̍��������ԂƂƂ��ɑ���$c$�Ő����ɕ��s�ړ�����~����̓_�ł�����
���]�Ԃ̗ւɔ��˔‚�t���A����𓮉�Ƃ��ĎB�e����Ɠ�����B
$X$�Ɉړ���$cs$���Ȃ���Ή~�ɂȂ�B
������~�ƌ��ԈႦ���K�����C�̌��_���S���I�O��ł͂Ȃ������B

���́A�Z���R�u�����������������Ă���A���n���̒���ɂ���čēx�̎��݂ł��̐����ɒB�������̂ł���B
���ۂ����ɏq�ׂ��V�˓I��@�͒탈�n���ɂ����̂ŁA
�탈�n���̒��z�ɂ������Z���܂����Ă����x���k�[�C�Z��Ԃ̎��i�Ƌ����S�������ɂ�������Ă���B
�ϕ��@�ɂ‚��Ă͂��̒i�K�ł͂����܂łƂ��A��ɔ��W���q�ׂ邱�Ƃɂ��悤�B

\section{�w�����p�x�̐��E}

���R�u�E�x���k�[�C(Jacob Bernoulli, 1654--1705)�Ƃ����΁A
���������́eArs Conjectandi�f(1713)�̑��������Ȃ��Ă͂Ȃ�Ȃ��B
Ars�͉p���(art)���Ȃ킿�u�p�v�Aconjectandi\footnote{
conjectandi �ɂ‚��Ē��A����H}
��of conjecturing���Ȃ킿�u�\���́v�ł���B
�����Ƃ��A�������w�ł́econjecture�f�́u�\�z�v�Ƃ����Ă���B
������ɂ���u�\���p�v�Ƃ��󂵂��邪�A�‚܂�́u�m���_�v�̚���ł���B

���̑��͒��҂̎���i1705�v�j�̏o�łł���A�̐l�̌��e�������Ă������̃j�R���X�E�x���k�[�C�����͂��犩�߂��A
�Z��̔����Ȋ֌W�̒��ł��̘J���Ƃ����Ƃ܂������ɏq�ׂ��Ă���B
�����Ƃ��A���ł��邱�Ƃ͊ԈႢ�Ȃ����̂́A����̍ŏ��̑��ł��邱�Ƃɂ͈٘_������A
�o�Ŏ������߂������������[��(����)�́w���R�̃Q�[�����͗��_�x�̃��x���̍������w�E�����������B
�܂�����قǎ����o�����ďo�ł��ꂽ�h�E���A�u���́eThe Doctrine of Chances�f(�w���R�_�c�c�x)��
�\�����e�̏ڍׁA�������ɂ����āA����i��ł��邪�A���ꂾ����Ars Conjectandi�̐�쐫�͂ނ��낢���������������ł��낤�B
����4�����琬�钘�삪�A�旧�ƒz�C�w���X�́w���R�̃Q�[���ɂ�����v�Z�ɂ‚��āx
De Ratiociniis In Ludo Aleae�ɐG������Ă͂��邪�A
���̌�̊m���_�̔��W�̑b�΂ƂȂ�d�v�T�O�������Ă��邱�Ƃ́A�����������邱�Ƃ͂Ȃ����낤�B
\begin{enumerate}
\item[1.]�u���Ғl�v�̊T�O�𓱂���
\item[2.] ����������ނɁA�u�g�ݍ��킹���w�vcombinatorics�ɂ�鐔���̊�b�������
\item[3.]�u�x���k�[�C���v���`����
\item[4.] �����ł����u�吔�́i��j�@���v���ŏ��ɓ�����
\item[5.] �Љ�I�A�o�ϓI�Ȏ��ۂ̕���
\end{enumerate}

\section{�w�����p�x��I��}


�܂��́A4����̕M���ł���A����͕����ʂ莟�̒ʂ�F

\begin{quote}
---�z�C�w���X�̘_���w���R�̃Q�[���ɂ�����v�Z�ɂ‚��āx���܂ޑ�I���A
���R�u�E�x���k�[�C�ɂ�钍�߂‚�\, 
complectens�@Tractatum Hugenii De Ratiociniis In Ludo Aleae, Cum Annotationibus, Jacobi Bernoullj---
\end{quote}

ludo�́u�Q�[���v�Aalea�͕����ʂ�u��������v(aleae�͂��̑��i�A���i�Łu��������́v)�ł��邪�A
�����ł́u��������v�́u���R�v�Ǝ����I�ɂ�---�J�[�h������---
���`�ŁA���ۃt�����X��aleatoire�́u���R�I�v�u�m���I�v�Ƃ����M�󂪒蒅���Ă���B
�m���_�̎v�z�I���[�c�Ƃ��Ēm���Ă����ׂ��ł��낤�B
�‚��łȂ���A�u�Q�[���v�͍�����von Neumann���̂�����헪�I�u�Q�[�����_�v�̓��e�͊܂܂��A
�قڂ��̊m�������̖ʂ��w���B
de�͑O�u���i�D�i�x�z�j�ő��`�ł��邪�w��薼�ł́u�`�ɂ‚��āA�ւ��āv���Ӗ����A���΂��΁u�`�_�v\footnote{
����͐��w�A���R�Ȋw�Ɍ������󋵂ł͂Ȃ��A�l���A�Љ�Ȋw�̕���ł��c�c�B}
�Ɩ�o�����B
ratiocinium�i�����ł͒D�i�j��ratio�Ɠ������u�v�Z�v�����A�ނ���L���u�l�@�v�u�_���v�̖ʂ��܂ށB

\noindent
\Fig[�\���ʐ^]{\textwidth}{10\baselineskip}

���̑�I���͐�y�i�̓V�˃z�C�w���X���������Q�[���̌v�Z���Љ�icompectens�u�܂ށv�j�A
���̂����Łe���Ȃ炱������f�Ɖ������X�}�[�g�ŋZ�I�I��@��掦�����͂ŁA�Q�[���������@�W�ł���B
���҃p�X�J���ƃt�F���}�[�̖�����i�ƃ��x���͍����A��@���i�������Œn���I�ɂȂ��Ă���B
�e���ɂ͂��ꂼ��m��I�Ɋm������̌`���œ��Ƃ��ė^������_���炵�āA
�������Ɂu�\���v�̖��ɒp���Ȃ����e�ł���B
�ȉ��A���ڂ����f����B

\begin{enumerate}
\item[����I]   ���Ғl�̑㐔���i$a,\, b$�����ɉ”\�ȂƂ��j
\item[����II]  ���Ғl�̑㐔���i$a,\, b, \, c$�����ɉ”\�ȂƂ��j
\item[����III] ���Ғl�̑㐔���i$a,\, b$�e$p,\, q$�̓����ɉ”\���̂Ƃ��j
\end{enumerate}
�ȏ�͏��l�̗��V�ɂ�銨��ŁA���������玩�R�ɓ�����A�����̃e�L�X�g�ɂ���悤�ȓV����̒�`�ł͂Ȃ����������[���B
�܂��u�”\���v��$p:q$�̂悤�ɔ�ŗ^�����A������1�ɋK�i�����ꂽ�m���̊T�O�͈�ʓI�ł͂Ȃ������B
\begin{enumerate}
\item[����IV]  3�Q�[�����ŏ��ƒ��[���ŁA�e2���A1���ŃQ�[�������f����Ƃ��̌����ȓq�������z�̕��@
\item[����V]   ��1�̃v���[���[��1�Q�[���A��2��3�Q�[����v����ꍇ
\item[����VI]  ��1�̃v���[���[��2�Q�[���A��2�̃v���[���[��3�Q�[����v����ꍇ
\item[����VII] ��1�̃v���[���[��2�Q�[���A��2�̃v���[���[��4�Q�[����v����ꍇ \\
\end{enumerate}
����͂��܂�ɂ��L���ȃp�X�J�����t�F���}�[�̉������ȂŘ_����ꂽ�u���z���v�i���邢�́u�q���̒��f���v�j�ł���B
�����ł���͂蕪�z�̌��������L�[���[�h�ƂȂ��Ă���B
���l�Ԃ̎���ł͍ő�̗��Q�֐S�ł������̂��낤�B

���̋@��Ɉ�‚̃G�s�\�[�h���q�ׂĂ������B
�p�X�J�����t�F���}�[�̏��Ȃ�1654�N�̂��Ƃł��������A���傤�ǂ��̔N�Ƀ��R�u�E�x���k�[�C�͐���Ă���B

�ȍ~�͂�������\footnote{
�Ȃ��A�u��������v���O���̂悤�Ɂu�T�C�R���v�Ə����̂͏��߂��Ȃ��B
����͖{�����{��ł���u\ruby{��}{����}�v�ɗR������B
���{�̌Ñ�̖������͏�c����ΐ�̐��A�o�Z���΁A�R�@�t��V���̎O��s�@�ӂƂ����}�b�͂悭�m���Ă���B}
�̃Q�[���ł���B
\begin{enumerate}
\item[����VIII] ��1�A��2�̃v���[���[��1�Q�[���A��3�̃v���[���[��2�Q�[����v����ꍇ
\item[����IX]   ��1�A��2�̃v���[���[�̓x�X�g��1���邢��2�Q�[����v���A��3�̃v���[���[��5�܂ł̃Q�[����v����ꍇ
\end{enumerate}
����͂����܂ł̈�ʉ��ł���ȊO�A�Ƃ藧�ĂăR�����g���ׂ���ނ͂Ȃ��B

\noindent
\Fig[�\1.(2��)]{\textwidth}{10\baselineskip}

\vspace{5\baselineskip}

\begin{enumerate}
\item[����X]    �q����$a$�́A��������𓊂��������߂�6���o���Ƃ��Ɏ擾�ł���B
��������̊e��܂łɎ擾�ł���z�����߂�i���ۂɂ́A$1,\, 2,\, 3,\, 4$��ڂ܂Łj
\item[����XI]   �������A2�‚̂�������𓊂��������߂Ęa12�i6--6�̃y�A�j���o���Ƃ��A�Ƃ���
\item[����XII]  ���������2���6�𓾂邽�߂Ɂm����ɓq���邱�Ƃ��L���A���Ȃ킿�m����1/2�ȏ�Łn�K�v�ȉ�
\item[����XIII] 2�‚̂�������𓊂��A2�l�̃v���[���[���a�����ꂼ��$7, \, 10$�̂Ƃ��q�����𓾁A����ȊO�͕����z������
\item[����XIV]  ���肪���ŁA����͐��6���o�����Ƃ��A������͐��7���o�����Ƃ��A�q�����𓾂�
\end{enumerate}

\begin{enumerate}
\item[���I]   $\mathrm{A},\, \mathrm{B}$��2�‚̂�������Ńv���[���A�a��6�Ȃ�A�̏����A7�Ȃ�B�̏����Ƃ���B
�܂�A�������A����B��������2�񓊂���B
�‚��ŁAA��2�񓊂��A�ȉ����l�Ƃ��A�O�҂��邢�͌�҂����҂ƂȂ�܂ő�����B
A�̌����ݑ�B�̌����݂̔�͂ǂ��Ȃ邩�B
���F10,577��12,276�B
\item[���II]  3�l�̃v���[���[���A12���̃g�[�N���������A����4���͔��A8���͍��ł���A���̏����Ńv���[����F���̂����N�ł��ډB���������܂ܔ���I�񂾎҂������Ƃ��āA�܂��A�ŏ���A�A2�Ԗڂ�B�A3�Ԗڂ�C�������B
3�l�̌����݂͂ǂ��Ȃ邩�B
\item[���III]  A��B�Ƌ����Ă��āA���ꂼ��10���̃J�[�h��4�ʂ�̑g47������A���ꂼ��1����4���I�Ԃ��Ƃ�錾���Ă���B
A�̌����݂�B�̌����݂̔��1000��8139�ƌv�Z�����B
\item[���IV]   �ȑO�Ɠ��l�A��4���A��8���̃g�[�N��12���������AA��B�ɑ΂��A�ډB��������7���̃g�[�N���̂���3�������ł���悤�ɑI�Ԃ��Ƃ�q���Ă���B
A�̌����݂�B�̌����݂ɑ΂����قǂ̔�ƂȂ邩�B
\item[���V]    A��B�͂��ꂼ��12���̃R�C���������A3�‚̂�������Řa��11�Ȃ��A��B�ɃR�C��1����n���A14�Ȃ�B��A�ɃR�C��1����n�������Ńv���[����B
���ׂẴR�C���𓾂��������B
A�̌����݂�B�̌����݂̔��
\[ 244, \, 140, \, 625 \, \text{��}\,  282, \, 429, \, 536, \, 481 \]
�ƌv�Z�����B

�����́A�z�C�w���X����͂Ƃ��������x���k�[�C������̕��@�ʼn���^�������̂ł���B
\end{enumerate}

%%%%%%%%%%%%%%%%%%%%%
\section{�w�����p�x��II��}

\begin{comment}
��1���͐�y�i�œV�˔��̃z�C�w���X���������Q�[���̌v�Z�ɑ΂��A
�e���Ȃ炱������f�Ɖ������X�}�[�g�ŋZ�I�I��@��掦�����͂ŁA�Q�[���������@�W�ł���B
���҃p�X�J���ƃt�F���}�[�̖�����i�ƃ��x���͍����A��@���i�������Œn���I�ɂȂ��Ă���B
�e���ɂ͂��ꂼ��m��I�Ɋm��---�����I�p�@�łȂ��A��̌`����---�����Ƃ��ė^������_���炵�āA
�������Ɂu�\���v�̖��ɒp���Ȃ����e�ł���B
\end{comment}

�Z�J���h�E�p�[�g�́A�����薼��
\begin{quote}
---����Ƒg�ݍ��킹�̏����_���܂ޑ�II��\, Doctrinam De Permutationibus \& Combinationibus---
\end{quote}
�ł���B
���̑�II���́A��I���̔��W�Ƃ������ނ���t�ɁA�\���̊�b�Ƃ��Ắe����Ƒg�ݍ��킹�f�̊�b�����_�̉���ł���B
���ہA���ʂ���u�����_�vdoctrinam�idoctrina�̕����Ίi�j�Ɩ��ł��Ă��āA�܂��͐��������m�ɐ����邱�Ƃ��甭�W���āA
�����鏇��iP�j�A�g�ݍ��킹�iC�j�̕��“I�����𒴂��A�}�`���A�p�X�J���̎Z�p�O�p�`�A���R���̗ݏ�a�A
�x���k�[�C���Ȃǐ��w�҂ɂ͋����[����i�ƌ@�艺������ނ����ԁB
�Z�I�I�Ƃ�����肻�̊�b�I���e�́A�h�E���A�u���i���S�Ɍ��藝�j�A���v���X�i�m���̒�`�A�����t�m���A��֐��j��
�c���Ɋ��􂷂���ݒ肷����̂Ƃ��č����]�����^������B
�����Ƃ��A����͐����I�ʂ̍����I�]���ł����āA���Ƃ��ƍ�҂̍ŏI�ړI�͎Љ��������ɓ��ꂽ���@�v���ł������B
�c�O�Ȃ���A����͍�҂̎��ɂ���ē����΂ɏI������i���O�o�łł͂Ȃ��ꗝ�R�Ɛ��������j�B

���āA�܂��ꌩ�e�m���_�f�I�ɂ͌����Ȃ����A���������̊�b�Ƃ��āA���R���̗񂩂�n�߂�B
\begin{align}
&1,1,1,1,1,1,\ldots 
\intertext{����A�����A�����a}
&1,2,3,4,5,6,\ldots 
\intertext{����ɂ��̕����a}
&1,3,6,10,15,21,\ldots 
\intertext{�����A����𓯗l��}
&1,4,10,20,35,56,\ldots \\
&1,5,15,35,70,126,\ldots \\
&\ldots \ldots \ldots \notag
\end{align}
�̂悤�ɂ���Ԃ��B
��6�i�ȉ��͗�����2�A������
\[
\text{��}n+1\text{�i�̑�}k\text{��}=\text{��}n\text{�i�̑�}k\text{���܂ł̘a}
\]
�̂悤�ɕ��ԁB
�x���k�[�C�́A�������c�ɁA���������炵��

\noindent
\Fig[�}�}��]{\textwidth}{10\baselineskip}

\[ \ldots \ldots \ldots \]
�̂悤�ɎO�p�`�ɕ��ׂ�B
����́A�܂��Ɂu�p�X�J���̎Z�p�O�p�`�v�ɑ��Ȃ炸�A�ʖ��u�񍀌W���v���邢�͑g�ݍ��킹�̐�${}_n\mathrm{C}_r$�̕\�ł���B
�����āA��̕����a�̊֌W�́A����ǂ�
\[
\text{��}n+1\text{��̑�}k+1\text{��}=\text{��}n\text{��̑�}k\text{���܂ł̕����a}
\]
�ƂȂ��Ă���B
���邢��${}_n\mathrm{C}_r$�ł́A����͎��͓���
\[ {}_{n+1}\mathrm{C}_{k+1}={}_k\mathrm{C}_k+{}_{k+1}\mathrm{C}_k+\cdots+{}_n\mathrm{C}_k \]
�ł���A$n=5, \, k=3$�Ȃ�
\[
20=1+3+6+10
\]
�̂��Ƃ��ł���B

���̓����̏ؖ��͓���Ȃ��B
\begin{align*}
a&=\text{dummy} \\
b&=\text{dummy} \\
c&=\text{dummy} \\
d&=\text{dummy} %\label{11}
\end{align*}

���̂悤�Ɋ֐���p���ČW���𔭐���������@�͈�ʂɁu��֐��vgenerating function�Ƃ����A
����e�򓹋�Ƃ��āf�悭�p������i�Ⴆ�΁A���v���X�j�B

�e��I,\, II,\, III,\, IV, $\ldots$�͐}�`�I�ɂ����܂��֌W�ƂȂ��Ă���B
II�̐������X�ƕt�������Ă����ƁA�O�p�`�̗�

\noindent
\Fig[����12���@�iIII�j]{\textwidth}{3\baselineskip}

\noindent
�����������̂ŁAIII�́u�O�p���v�Ƃ�΂��B

�����O�p���iIII�j�����X��1�i�A2�i�A3�i�A4�i�́e�s���~�b�h�f��ɗ��̓I�ɐς�ōs���ƁA���R�A����3������

\noindent
{\gt ������I��} ���̌����͍��Z���w�I�ɂ���Ȃ�������B
�悭�m��ꂽ�֌W��
\[
{}_n\mathrm{C}_k+{}_n\mathrm{C}_{k+1}={}_{n+1}\mathrm{C}_{k+1}
\]
��ό`�����A�����i�Q�����j�̊֌W
\[
{}_{n+1}\mathrm{C}_{k+1}-{}_n\mathrm{C}_{k+1}={}_n\mathrm{C}_k
\]
�𗘗p����B
�����ŁA$n \to n-1, \, n+1 \to n$�̓Y������������$n=k$�ƂȂ�܂ŌJ��Ԃ�
�i������${}_k\mathrm{C}_{k+1}=\circ $�Ƃ���j�A���������ׂĕӁX�������
\[
{}_{n+1}\mathrm{C}_{k+1}={}_n\mathrm{C}_{k}+{}_{n-1}\mathrm{C}_k+\cdots+{}_k\mathrm{C}_k
\]
�𓾂�B
\begin{equation}
1, \, 4,\, 10,\, 20,\, 35,\ldots \text{�i�j} %\label{(IV)}
\end{equation}
�ςݏオ�������l�ʑ̂��������������B
�����IV�́u�l�ʑ̐��v�u�s���~�b�h���v�ƌĂ΂��B
���Ɏl�ʑ̐�IV�̐ςݏグ�ŁAV����������邪�A�����͏������I�Ȃ��̂ŁA���ϓI�ɔF���͂ł��Ȃ��B

�����S�̂Ƃ��āAI,\, II,\, III,\, IV,$\ldots$���e�����́u�}�`���v(figurate numbers)�Ƃ����B
�}�`���͌×�����m���邪�A�x���k�[�C������炪�}�`���ł��邱�Ƃ��w�E���Ă���B
���Ȃ킿�A�u�p�X�J���̎O�p�`�v�̂�������ɂ͐}�`�����_��I�ɉB����Ă���B
������ɂ���A�}�`��V,\,VI,$\ldots$�͋�ԓI�ɐ����ł��Ȃ��̂ŁA���̂���肵�Ȃ��B

\begin{table}[htb]
\begin{tabular}{c|l}
    & �ʏ�̖���               \\ \hline 
I   & �i�萔�j                 \\
II  & ���R���i��������j       \\
III & �O�p��                   \\
IV  & �l�ʑ̐��i�s���~�b�h���j \\
V   & �s���~�b�h�I�l�ʑ̐�     \\
\end{tabular}
\end{table}

�ȏ�̍\�����琔��A���̘a�Ƃ��Ă̐���������B
II�͓�������ł��邩��A���̘a�͂悭�m����悤�ɁA�e�U�ς݂̌����f
\[
\dfrac{n(n+1)}{2}, \quad n=1,2,3,\ldots
\]
�ł����āA���̌����͕����ʂ�O�p���i�U�ςݎ��j�ɑ��Ȃ�Ȃ��B
����${}_{n+1}\mathrm{C}_2$�ł��邩���̕����a�֌W����
\[
{}_{n+1}\mathrm{C}_3+{}_2\mathrm{C}_2+{}_3\mathrm{C}_2+\cdots+{}_n\mathrm{C}_2\quad (n \geqq 2)
\]
�‚܂�A�P����
\[
\frac{(n+1)n(n-1)}{6}
=\sum_{1}^n \frac{k(k-1)}{2} %\label{13}
\]
���o�āA���ӂ̑��삩��$\sum k^2$��
\[
\sum k^2 
=\frac{n(n+1)(2n+1)}{3} %\label{14}
\]
���e�ꔭ�Łf���o�����B
�x���k�[�C�͂��̗v�̂ŏ���$p=3,\, 4,\, 5,\ldots$�ɑ΂��āA�֌W��
\[
{}_{n+1}\mathrm{C}_{p+1}
={}_p\mathrm{C}_p+{}_{p+1}\mathrm{C}_p + \cdots + {}_n\mathrm{C}_p \quad (n \geqq p)
\]
�̍���
\[
\sum_{k=1}^n \frac{k(k-1)(k-2)\cdots (k-p+1)}{1 \cdot 2 \cdots p} %\label{15}
\]
�𕪉����A���R���̗ݏ�a$s_p$�̈�ʎ�
\[
\sum_{k=1}^n k^p=(n\text{��}p+1\text{���̑�����})
\]
�ɒB�����B
�ȉ��͂��̌��ʂł���B

\noindent
\begin{table}[tbh]
\caption{������$\psi$�̌W��}
{\tiny
  \begin{tabular}{rrrrrrrrrrrr}
���� &                  &                &          &           &           &           &       &       &       &        &        \\
$p$  & 1                & 2              & 3        & 4         & 5         & 6         & 7     & 8     & 9     & 10     & 11     \\ \hline
1    & $(1/2)^{\, *}$   & (1/2)          &          &           &           &           &       &       &       &        &        \\
2    & $(1/6)^{\, *}$   & (1/2)          & (1/3)    &           &           &           &       &       &       &        &        \\
3    &                  & (1/4)          & (1/2)    & (1/4)     &           &           &       &       &       &        &        \\
4    & $(-1/30)^{\, *}$ &                & (1/3)    & (1/2)     & (1/5)     &           &       &       &       &        &        \\
5    &                  & $(-1/12)$      &          & (5/12)    & (1/2)     & (1/6)     &       &       &       &        &        \\
6    & $(1/42)^{\, *}$  &                & $(-1/6)$ &           & (1/2)     & (1/2)     & (1/7) &       &       &        &        \\
7    &                  & (1/12)         &          & $(-7/24)$ &           & (7/12)    & (1/2) & (1/8) &       &        &        \\
8    & $(-1/30)^{\, *}$ &                & (2/9)    &           & $(-7/15)$ &           & (2/3) & (1/2) & (1/9) &        &        \\
9    &                  & $(-3/20)^{**}$ &          & (1/2)     &           & $(-7/10)$ &       & (3/4) & (1/2) & (1/10) &        \\
10   & $(5/66)^{\, *}$  &                & $(-1/2)$ &           & 1         &           & $-1$  &       & (5/6) & (1/2)  & (1/11) \\
\end{tabular}}
\noindent
*�F�x���k�[�C���A\qquad **�F������
\end{table}

\begin{align*}
\sum_{k=1}^n 1   &=n, \\
\sum_{k=1}^n k   &=\dfrac{n(n+1)}{2}, \\
\sum_{k=1}^n k^2 &=\dfrac{(2n+1)n(n+1)}{6}, \\
\sum_{k=1}^n k^3 &=\dfrac{n^2(n+1)^2}{4}, \\
\sum_{k=1}^n k^4 &=\dfrac{(2n+1)n(n+1)(3n^2+3n-1)}{30}, \\
\sum_{k=1}^n k^5 &=\dfrac{n^2(n+1)^2(2n^2+2n-1)}{12}, \\
\sum_{k=1}^n k^6 &=\dfrac{(2n+1)n(n+1)(3n^4+6n^3-3n+1)}{42}, \\
\sum_{k=1}^n k^7 &=\dfrac{n^2(n+1)^2(3n^4+6n^3-n^2-4n+2)}{24}, \\
\sum_{k=1}^n k^8 &=\dfrac{n(n+1)(2n+1)(5n^6+15n^5+5n^4-15n^3-n^2+9n-3)}{90}
\end{align*}

���́A�x���k�[�C�̌v�Z�ɂ͌�肪����B
�����‚��̃`�F�b�N�E�|�C���g��
\begin{enumerate}
\item $n$�܂ł̗ݏ�a�ł��邩��A$n=1$�ɑ΂��Ă�1
\item ���o�̏�ŁA$n(n+1)$�͂��‚��ێ�����邩��A����$n(n+1)$�������Ƃ��Ċ܂ށB
����������$n=-1$��0
\item ���L�̗אڔ�̃p�^�[���ȏォ��A$p=9$�ň�J���̌������o�����B
\end{enumerate}

\section{���ܖ͗l�̃t�H�����̕s�v�c}

���̌W���̕\�ɂ͒����������Əd�v�Ȑ��w�I�^���i�x���k�[�C���A���[�}����$\zeta$�֐��Ȃǁj���B����Ă���B
�܂��ꌩ���Ă킩�邱�Ƃ́G

$n$�܂ł�$p$��a$s_p$��
\begin{enumerate}
\item[(1)] $n$��$p+1$���������ł���
\item[(2)] �ō���$p+1$���̌W����$1/(p�{1)$�A$p$���̌W���͂‚˂�1/2
\item[(3)] $p-1$���̍��������Ȃ�
\end{enumerate}
�����$p+1, \, p, \, p-1$����3�����܂Ƃ܂�����p�^�[���ƂȂ��Ă���B
����Ɏ����᎟�̍���
\begin{enumerate}
\item[(4)] $p-2, \, p-4, \, p-6, \ldots$���̍�������
\item[(5)] $p-3, \, p-5, \, p-7, \ldots$���̍��͎c��я�̃X�b�L�������e���ܖ͗l�f�̃t�H�����ƂȂ��Ă���B
\end{enumerate}
%%%%%%%%%%%%%%%%%%
���āA�e�W���ɂ‚��Č�������B
�Ίp���ɂ�$1/2$��$p$���̍��Ƃ��ĕ��сA���̈�i��͍ō���$(p�{1)$���̍���$1/2, \, 1/3, \, 1/4, \ldots$�Ƃ��ĕ��Ԃ��A
���̗אڂ��鍀�̔��
\[
2/3, \, 3/4, \, 4/5, \, \ldots
\]
�ƂȂ��Ă���B
�O�̂��ߑΊp����$1/2$�̍��̗אڍ��̔�́A�������
\[
1, \, 1, \, 1, \ldots
\]
�ł���B
���ɁA�Ίp���̈�i����$p-1$���̍��̕��тł́A�אڍ��̔�́A����ǂ�
\[
3/2, \, 4/3, \, 5/4, \, \ldots
\]
�ƂȂ��Ă��邱�ƂɋC�Â����낤�B
���đz�肳���
$4/2\, (=2), \, 5/3, \, 6/4, \ldots$�ɑ�������$p-2$���̍��͌����A
���i��$p-3$���֍s���ƁA�אڍ��̔�
\[
5/2, \, 6/3, \, 7/4, \ldots
\]
���\����B
����$6/2\, (=3), \, 7/3, \, 8/4, \ldots$�̔�͌����邪�A��i���ł�$p-5$����
\[
7/2, \, 8/3, \, 9/4, \ldots
\]
�������ė���B

\section{�x���k�[�C���̗R��}

���̂悤�ɍ��ォ��E���֌W���̕��т��t�H���[���čs���ƁA�������������P���ȗאڔ�̃p�^�[���������Ă���B
���������āA�e���тɂ����Ă����אڔ�����X�ƘA�悵�A����\kenten{����}��^����΂����̌W���͊��S�Ɍ��肳���B

\noindent
{\gt ���၄} $p-3$���̌W���̕��сi�Ίp����3�i���j�����悤�B
�אڔ�$5/2, \, 6/3,$ $7/4, \ldots$��4�Ԗڂ܂ł̘A���
\[
\frac{5}{2} \cdot 
\frac{6}{3} \cdot 
\frac{7}{4} \cdot 
\frac{8}{5} %\label{17}
\]
����ɏ���$-1/20$���悶�Ď���悭$-7/15 \, (p=8)$�𓾂�B
���l��$p-1$���ɂ‚��Ă��m���߂�Ƃ悢�B

��ʂ�$p-3$���̌W���́A������$B=-1/30$�Ƃ���
\[
B\cdot \frac{p(p-1)(p-2)}{1\cdot 2\cdot 3\cdot 4}
\]
�ł���A���l��$p-5$���̍��̌W���́A������$B=1/42$�Ƃ���
\[
C \cdot \frac{p(p-1)(p-2)(p-3)(p-4)}{1�E2�E3�E4�E5�E6}
\]
�ȂǂŁA�ȉ��A���l�ł���B
�Ȃ��A�����̃��[���́A�Ίp���̒����i��$p-1$���̌W���ɂ��ʂ��A�ȒP�ɁA$A=1/6$�������Ƃ���
\[ A \cdot \frac{p}{2} \]
�ŗ^������B
���̂悤�ɁA�W�������ォ��E���ɕ��ׂ��Ƃ��̗אڔ�Ƀ��[��������ȏ�A
�����W�������肷��̂́A�W���\�̑�1��ɕ���
\[
A=1/6, \, B=-1/20, \, C=1/42, \, D=-1/30, \, E=5/66, \ldots
\]
�i����$F=691/2730, \, G=7/6, \ldots$�j�ł��邱�Ƃ��킩��B
������---�����āA�x���k�[�C�ɂ��΁A�����̐������𕄍����܂߂�---�u�x���k�[�C���v�Ƃ����B
�ŏ��̗�O�������āA������$p$�ɑΉ����Ă̂ݒ�`�����B
\begin{align*}
1^p+2^p+\cdots +n^p
=& \frac{n^{p+1}}{p+1}+\frac{n^p}{2}+\frac{p}{2}                   An^{p-1}
  +\frac{p(p-1)(p-2)}{2 \cdot 3 \cdot 4}                           Bn^{p-3} \\
 &+\frac{p(p-1)(p-2)(p-3)(p-4)}{2 \cdot 3 \cdot 4 \cdot 5 \cdot 6} Cn^{p-5}+\cdots %\label{18}
\end{align*}

�I�C���[���v�Z�����������A���̃x���k�[�C���������Ă����ƁA
\begin{align*}
B_0   &=1, \quad 
B_1    =-\frac{1}{2}�C\text{�i�ȏ�̓I�C���[�ɂ��j} \\
B_2   &= \frac{1}{6}, \quad 
B_4    =-\frac{1}{30},\quad 
B_6    = \frac{1}{42},\quad 
B_8    =-\frac{1}{30}, \\
B_{10}&= \frac{5}{66},\quad 
B_{12} =-\frac{691}{2730},\quad 
B_{14} = \frac{6}{7},     \quad 
B_{16} =-\frac{3617}{510},\quad 
B_{18} = \frac{43867}{798}, \\
B_{20}&=-\frac{174611}{330},     \quad 
B_{22} = \frac{854513}{138},     \quad 
B_{24} =-\frac{236364091}{2730}, \\
B_{26}&= \frac{8553103}{6},      \quad 
B_{28} =-\frac{23749461029}{870},\quad 
B_{30} = \frac{8615841276005}{14322} %\label{19}
\end{align*}

\noindent
{\gt ������II��}

�x���k�[�C���́A�������i���ہA���z���j�ł���$\pi, \, e$�ȂǂƈقȂ�L�����ł��邱�Ƃ������ł��邩��A
�܂����̕���A���q�ɂ͓��ʂ̊֐S����������B���Ƃɕ��q�ɂ‚���
\[
\text{�i��������H�j}
\]
���m������B

�x���k�[�C���̕s�v�c�ȈЗ́i���́j�́A��͊w�i�����ϕ��j�A�����_�ɏo�����邱�ƂŁA
���̂ق�̈�[�́A��ɂ��Љ��
\[
\sum_{k=1}^{\infty} \frac{1}{k^2}=\frac{\pi^2}{6}\qquad \text{�i�o�[�[�����j} %\label{20}
\]
�ŁA����$1/6$���ŏ��̃x���k�[�C��$A$�ł���B
����ȊO�ł�
%%%%%%%%%%%%%%%%%%
\begin{enumerate}
\item $\sum_{k=1}^{\infty} \dfrac{1}{k^{2q}}$�i��ʂ̐������j
\item $\sum_{k=1}^{\infty} \dfrac{1}{k^s}$�i$s=\text{���f��}$�j�m���[�}��$\zeta$�֐��n%\label{21}
\item �t�O�p�֐�$\arctan$�Ȃǂ̋����W�J
\item �I�C���[���}�N���[�����̘a����
\end{enumerate}
�Ȃǂő劈�􂷂邪�A�X�y�[�X�̓s���Ŗ{�͂ł͈���Ȃ��B

�x���k�[�C���͓����x���k�[�C�̒�`��肸���Əd�v�Ȑ��ł����Đ��w�̕������������Ƃ������Ă���B
�u�A���X�E�R���C�F�N�^���f�B�v�͂ӂ‚��m���_�̏��Ƃ���Ă��邪�A���w�̗��j�I��{���ł�����A
��������R�u�E�x���k�[�C�̑傫�Ȍ��тł���B
�o�[�[���ɂ����āA���N���ォ��x���k�[�C�Ƃɏo�����Ă����I�C���[�̈��ɂ�������ΉB�ꂪ���ł��邪�A
�I�C���[�̌����ォ�牟�����̂̓x���k�[�C�E�t�@�~���[�̐l�X�ł���B
���m�̂��Ƃ킴�̂��Ƃ��u������Ė�����v�ł���B

���āA�x���k�[�C���𐶐����郋�[���͂ǂ̂悤�Ȃ��̂ł��낤���B
�I�C���[�̌v�Z�ł́A���łɏq�ׂ��悤��
\[
\mathit{1}, \, \mathit{-1/2}, \, 1/6, \, -1/30, \, 0, \, 1/42, \, 0, \, -1/30, \, \ldots
\]
�ł����āA�x���k�[�C�̒�`�͑�3���ȉ��A$1, \, -1/2$�̓I�C���[�ɂ����̂ł���B
������ɂ���A���ʂɁu�x���k�[�C���v�Ƃ͂������̗̂L�����ł��邱�ƁA��������シ�邱�ƂȂǂ𒘂������F�Ƃ���B
�����Ƃ��A�����̃e�L�X�g�ɂ́A�������Ȃ�����A�����0���J�b�g���ĕt�Ԃ���\����������A�����̍����������Ă���B
�����ňꉞ�A�ȉ��ł͍��ؒ厡�w��͊T�_�x�ɂ��������B

�������ɂ�炸�d�v�Ȃ̂͑�2����$-1/2$�ł���B
����$-1/2$�����Ƃ��đ�W�J���N��̂ŁA�x���k�[�C�̃I���W�i���̒�`�͑�3���ȉ�������ǂ��ł��������̂́A
�����$1/2$�Ƃ���̂ł̓h���}�͌����Ȃ��B

\noindent
\Fig[�f�g�̂�4�s���f]{\textwidth}{4\baselineskip}

\section{�x���k�[�C���̐���}

�x���k�[�C���͗ݏ�a$s_p=\sum k^p$�̕\������R�����邪�A����炪�������钼�ړI�֌W���͂ǂ̂悤�Ȃ��̂��B
���ꂪ�킩��΁A�߂�ǂ��ȗݏ�a�̎��ɋ�J���邱�Ƃ͂Ȃ����A���́A�x���k�[�C���𐶐������֐��͂���B
�����A����𓱂��o�����Ƃ͂Ȃ��Ȃ�---�ł�������---���܂��ł��Ȃ��B
�w��͊T�_�x����ۂ悭������^���Ă���i��64�́j�B
�������A���؂͕��������������̂�$B$�A���ꂽ���̂�$b$�ƕ\�L���Ă���B
�ȉ��ł́A$b$��$B$�ƕ\�L���邱�ƂƂ���B

�����ŁA�����V����I�����A���̂悤�ȍ��ӂ̊֐��������W�J����
$B_0, \, B_1,$ $B_2, \ldots$������ꂽ�Ƃ��悤�B
\[
(\sharp) \qquad 
\frac{z}{e^z-1}
=B_0+B_1 z+\frac{B_2}{2!}z^2+\frac{B_3}{3!}z^3+\cdots %\label{22}
\]
$e^z-1$��W�J���A����
\begin{align*}
&\left( z+\frac{1}{2!}z^2+\frac{1}{3!}z^3+\cdots \right) \\
& \times \left( 
        B_0+B_1 z+\frac{B_2}{2!}z^2+\frac{B_3}{3!}z^3+\cdots \right)=z %\label{23}
\end{align*}
�Ɏ������݁A$z$�A�‚���$z^2, \, z^3, \ldots$�ƌW�������킹�čs���΁A���ۂ�
\[
B_1=1, \, B_2=-1/2, \, B_3=1/3, \, B_4=-1/30, \, B_5=1, \, B_6=-1/42, \ldots
\]
������������B
�����$B_0, \, B_1, \, B_2, \, B_3, \ldots$��
\[
\sum k^p=1^p+2^p+\cdots+n^p
\]
�𐶐����邱�Ƃ��ؖ����悤�B
�i�Ȃ��A��֐��̓W�J�́A�����ς�㐔�I�����̖ړI�̂��߂ł����āA���̎����͖��Ȃ��̂��ʏ�ł���B
���Ȃ��Ƃ��A���v���X�܂ł͂����ł������B
�����������ɗv�������̂̓R�[�V�[�A���C�G���V���g���X����ł����āA���̌�͕��f�֐��_�̒��֎��e����邱�ƂƂȂ����B
�܂������ł����Ă����A�u$\zeta$�֐��v�ɂ����ăx���k�[�C���ɏd�v�Ȗ������^������̂ł���B�j

���������āA����$(\sharp)$���u�x���k�[�C���̕�֐��v�Ƃ����B�i���i�j

������ؖ����邽�߂ɁA���̕�֐����g�������i$x$���܂ށj���̐V���ȕ�֐����l���悤�B
$B_0, \, B_1, \, B_2, \ldots$�́A���R�A�萔�łȂ�$x$�̊֐��ƂȂ�B
���Ȃ킿
\begin{align*}
\frac{ze^{zx}}{e^z-1}
=&B_0(x)+B_1(x)z+\frac{B_2(x)}{2!}z^2 \\
 &+\frac{B_3(x)}{3!}z^3+\cdots %\label{24}
\end{align*}
�����$B_0(x), \, B_2(x), \, B_3(x), \ldots$���u�x���k�[�C�������v�Ƃ����B
���炩�ɁA$x=0$�Ƃ������ƂŁA�x���k�[�C����
$B_0(0)=B_0, \, B_1(0)=B_1, \, B_2(0)=B_2, \, B_3(0)=B_3, \ldots$�Ƃ��ċ��߂���B

���̍��ӂ�
\begin{align*}
&\left( B_0+B_1 z+\frac{B_2}{2!}z^2+\frac{B_3}{3!}z^3+\cdots \right) \\
&\times 
 \left( 1  +  x z+\frac{x^2}{2!}z^2+\frac{x^3}{3!}z^3+\cdots \right) %\label{25}
\end{align*}
�ŁA������E�ӂɓ������Ƃ����ƁA��ʂ�
\[
B_n(x)= B_0x^n 
       +B_1\binom{n}{1}x^{n-1}
       +B_2\binom{n}{2}x^{n-2}+\cdots +B_n %\label{26}
\]
���Ƃ���
\[
B_0(x)=1, \, B_1(x)=x-1/2, \ldots
\]
�Ȃǂ��o��B�i���i�j

�Ƃ����
\begin{align*}
\frac{ze^{z(x+1)}}{e^z-1}
&=\frac{ze^{zx}\cdot e^z}{e^z-1} \\
&=\frac{ze^{zx}}{e^z-1} +z \cdot e^{zx} %\label{27}
\end{align*}
���A$z^p$�̍����r����
\[
B_p(x�{1)-B_p(x)=px^{p-1}
\]
������$p$��$p+1$�Ƃ��āA�e���ꂢ�Ȋ֌W�f
\[
B_{p�{1}(x�{1)-B_{p�{1}(x)=(p�{1)x^p
\]
������ꂽ�B
�����܂ŗ���΁A���Ƃ�$x=0, \, 1, \, 2, \ldots , n$�Ƃ��ĉ�����΂悢�B
���Ȃ킿�A
\begin{align*}
1^p+2^p+\cdots +n^p
=&\frac{1}{p+1} \{ B_{p+1}(n)-B_{p+1}(0) \}+n^p \\
=&\frac{n^{p+1}}{p+1}
 +\frac{n^p}{2}
 +\binom{p}{1}\frac{B_1}{2} n^{p-1}
 -\binom{p}{3}\frac{B_3}{4} n^{p-3}+\cdots %\label{28}
\end{align*}
�ƂȂ�A����悭�x���k�[�C�̗^�������ɓ��B�����B�i�ؖ��I�j

�ȏ�A�u�x���k�[�C���v���g�������u�x���k�[�C�̑������v���o�R����`�ƂȂ������A�I���ȑ������^�p���܂܂�Ă���_�A
���̑������Ȃ��Ō��ʂɒB���邱�Ƃ͔ے�I�Ǝv����B
�܂��A�x���k�[�C���������Ȃ��֐��ŗ^�������Ƃɂ������̈����ڂ���������B
�Ƃ͂����A�ȏ�̏ؖ����ŋ߂̃e�L�X�g���Ɍ��o���̂͋H�Ȃ̂ŁA������₳����������邱�Ƃɂ͈�[�̉��l�͂���ł��낤�B

\section{���̂���}

�����܂ł́u�x���k�[�C���v��V���莮�ɕ�֐��ɂ���Ē�`���A���ꂩ��ݏ�a�𓱂��o�����B
���ہA�قƂ�ǂ̃e�L�X�g���x���k�[�C�����֐��Œ�`���Ă���B
�������A�{���́u�x���k�[�C���v�͂ނ���ݏ�a�ɂ���Ē�`���ꂽ����A����ł͋t�ő����Ɉ�a�����c��B
�����ŁA�i�{���́j�x���k�[�C�����A���̕�֐������R�ɓ������Ƃ��������B

$(k�{1)^{p+1}$�̓񍀓W�J���
\[
(k+1)^{p+1}-k^{p+1}
=\sum_{j=0}^p \binom{p+1}{j} k^j %\label{29}
\]
������$k=n, \, n-1, \, \ldots, \, 1$�������ĉ������
\[
(\sharp) \qquad 
(n+1)^{p+1} -1
=\sum_{j=0}^p \binom{p+1}{j} s_j %\label{30}
\]
�������A$s_j$��$j$��̘a��
\[
s_j=\sum_{k=1}^n k^j \qquad(j=0, \, 1,\, \ldots,\, p)
\]
�𓾂�B
$s_j$��$n$�̑������ŁA$s_0=n$�A�����
$s_1=n(n�{1)/2, \, 
 s_2=n(n�{1) \times (2n�{1)/6,\ldots $�͊��Ɍ����B
�����Ł���$n$�̈ꎟ�̍��ɒ��ڂ���ƁA���̌W��$B_j$�ɂ‚�
\[
p+1=\sum_{j=0}^p \binom{p+1}{j} B_j %\label{31}
\]

������$B_0=1,\, B_1=1/2$�ł���A�܂�$B_2, \, B_3, \, \ldots $��
�i�x���k�[�C�̒�`�ɂ��j�x���k�[�C���ł���B
���ӂ��E�ӂ�$j=1$�̍��ɌJ�����$B_1-1=-1/2$�����炽�ɂ�����
\[
\sum_{j=0}^p \binom{p+1}{j} B_j=0 
\qquad (B_j=-1/2) %\label{32}
\]

{\gt ����A�����ւ��y�[�W����}

\newpage

\section{�x���k�[�C�̓ƒd��}

�u���[�}���\�z�v�ł��܂˂��m����u$\zeta$�֐��v
\[
\zeta(s)=\sum_{k=1}^{\infty} \frac{1}{k^s}, \quad \mathrm{Re}(s)>1
\]
��$\Gamma$�֐���
\[
\zeta(s)=\frac{1}{\Gamma(s)} \int_0^\infty \frac{u^{s-1}}{e^u-1} du %\label{33}
\]
�Ƃ��ĕ\�킳��邱�Ƃ͂悭�m���Ă���B
���͂��ꂪ�x���k�[�C���̊���̏���J���̂ł��邪�A�܂�����������Ă������B

����͈ӊO�ɏ����ŁA�܂�
\begin{align*}
\Gamma(s)
&=\int_0^\infty e^{-t}t^{s-1} dt \\
&=k^s \int_0^\infty e^{-ku} u^{s-1} du 
\qquad (t=ku) %\label{34}
\end{align*}
�ł��邩��A������g����
\[
\sum_{k=1}^N \frac{1}{k^s}
=\frac{1}{\Gamma(s)} \int_0^\infty \sum_{k=1}^N e^{-ku} \cdot u^{s-1} du %\label{35}
\]
�𓾂�B
���̐ϕ����̓��䐔��̘a�́A�������
\[
\sum_{k=1}^N e^{-ku}
=\frac{1}{e^u-1}-\frac{e^{-Nu}}{e^u-1} %\label{36}
\]
�ƂȂ�B
�̂ɁA�ϕ���2�‚ɕ�����
\[
 \int_0^\infty \frac{u^{s-1}}{e^u-1} du 
-\int_0^\infty \frac{e^{-Nu}}{e^u-1} u^{s-1} du %\label{37}
\]
�ŁA���ʂ͂��������Ă��邪�A��2���́A
$e^{-Nu} \to 0\, (N \to \infty)$�ɒ��ڂ���ƁA$N \to \infty$�̂Ƃ�$\to 0$�ƂȂ�B��

�������A���̋c�_�͖{����$\mathrm{Re}(s)>1$�̌��Ō����ȋc�_��K�v�Ƃ���̂����A
��܂��Ȍ��ʂł������肢����\footnote{������i�w���f��́x������w�o�ʼn�ȂǎQ�ƁB}�B

\newpage
{\gt �Q�y�[�W����������ւ�}

\newpage
{\gt �Q�y�[�W����������ւ�}

\newpage
�u�R�^���v�̐��E�̓x���k�[�C���Ɛ[���ʂ��Ă���B
���łɌ������ł���B
���́u�R�^���̐��E�v�̓[�[�^�֐�$\zeta(s)$�Ƃ��ʂ��Ă��āA���ꂪ�����Ŏ��������ŏ��̂��Ƃ���ł���A
���ꂩ��u���[�}���\�z�v�ɏo������L���ȁu���[�}���̊֐������v���������B

�����ł͂������Ɂu���[�}���\�z�v�̖{�_�܂œ��邱�Ƃ͂��Ȃ����A�R�^���̐��E��ʂ��āA
���͍��܂łɂ��܂��Ė{�i�I�ȈӖ��ɂ�����$\zeta$�֐����x���k�[�C���ɍ��������Ă��邱�ƁA
�u$\zeta$�֐��̗�_�v�Ƃ������[�}�������̌��֌��܂ōs���Ă݂悤�B

\noindent
\Fig[����38��]{\textwidth}{5\baselineskip}

�܂��A������͂���
\begin{align*}
&\frac{1}{2i} \int_{\Gamma_n} \frac{\cot \pi z}{z^s} dz
 =\sum_{k=1}^n \frac{1}{k^s} \\
&\text{�������A}\Gamma_n \text{�͗̈�} %\label{39}
\end{align*}
\[
G_n : \, |z| \leqq n+\frac{1}{2}, \, \mathrm{Re}(z)>a \, (0<a<1)\text{�̋��E} %\label{40}
\]
���������Ƃ��ł���B
���ہA��ϕ��֐���$z=1, \, 2, \, \ldots,\, n$�Ɉ�ʂ̋ɂ����—L���^�֐��ŁA
$z=k$�ɂ����闯����$1/k^s$����B

�����ŁA�u$n$�͉~�ʕ���$k_n$�Ƌ��$[a-iy_n, \, a+iy_n]$���琬�邪�A
$\int k_n \to 0 \, (n \to \infty)$���������Ƃ��ł���B

\noindent
\Fig[����41���}]{\textwidth}{5\baselineskip}

�䂦��
\begin{align*}
\frac{1}{2i} \int_{a+iy_n}^{a-iy_n} \frac{\cot \pi z}{z^s} dz
=-\frac{1}{2i} \int_{a}^{a+iy_n} \frac{\cot \pi z}{z^s} dz
 +\frac{1}{2i} \int_{a}^{a-iy_n} \frac{\cot \pi z}{z^s} dz %\label{42}
\end{align*}
���������B
�E�ӂ̐ϕ��ɂ����Ă͂��ꂼ��
\begin{align*}
\frac{\cot \pi z}{z^s}
=-\frac{1}{2}-\frac{1}{e^{-2\pi iz }-1}, \quad 
  \frac{1}{2}+\frac{1}{e^{ 2\pi iz }-1} %\label{43}
\end{align*}
�ł��邩��A�ŏ��̐ϕ���
\begin{align*}
-\frac{1}{2i} \int_{a}^{a+iy_n} \frac{\cot \pi z}{z^s} dz
&=\int_{a}^{a+iy_n} \left( \frac{z^{-s}}{2} + \frac{z^{-s}}{e^{-2\pi iz }-1} \right) dz \\
&= \frac{1}{2} \frac{a^{1-s}}{s-1}
  +\frac{1}{2} \frac{(a+iy_n)^{1-s}}{1-s}
  +\int_{a}^{a+iy_n} \frac{z^{-s}}{e^{-2\pi iz }-1} dz %\label{44}
\end{align*}
�ƂȂ�B
������$1-\mathrm{Re}(s)<0$������$\text{��}2\text{��}\to 0 \, (n \to \infty)$�B

2�Ԗڂ����l�ł���A2�‚𕹂��āA$n \to \infty$�̂Ƃ��A
\begin{align*}
\zeta(s)
&=\sum_{k=1}^\infty \frac{1}{k^s} \\
&=\frac{a^{1-s}}{s-1} 
  +\int_{a}^{a+iy_n} \frac{z^{-s}}{e^{-2\pi iz }-1} dz
  +\int_{a}^{a-iy_n} \frac{z^{-s}}{e^{ 2\pi iz }-1} dz
  +o \left( \dfrac{1}{n} \right) %\label{45}
\end{align*}
�𓾂��B
�ȏ�A���f�̈�ł̐ϕ������A�Ӗ����鏊�͂����ނˏ����ɗ����ł��悤�B

$0<a<1$�ł������$a$�͖��֌W������$a \to 0$�Ƃ���
\begin{align*}
\lim_{a \to 0}
  \int_{a}^{a-i\infty} \frac{z^{-s}}{e^{ 2\pi iz }-1} dz
=i\int_{0}^{ - \infty} \frac{|y|^{-s} e^{i\pi s/2}}{e^{-2\pi y }-1} dy \\
\lim_{a \to 0}
  \int_{a}^{a+i\infty} \frac{z^{-s}}{e^{-2\pi iz }-1} dz
=i\int_{0}^{   \infty} \frac{  y^{-s} e^{-i\pi s/2}}{e^{2\pi y }-1} dy %\label{46}
\end{align*}
����āA
\begin{align*}
\zeta(s)
&=ie^{-i\pi s/2}
   \int_{0}^{ \infty} \frac{y^{-s}}{e^{2\pi y}-1} dy
  +ie^{ i\pi s/2}
   \int_{0}^{-\infty} \frac{|y|^{-s}}{e^{-2\pi y}-1} dy \\
&=\frac{1}{i} (e^{ i\pi s/2}-e^{-i\pi s/2})
   \int_{0}^{ \infty} \frac{y^{-s}}{e^{2\pi y}-1} dy \\
&=2 \sin \frac{\pi s}{2}
   \int_{0}^{ \infty} \frac{y^{-s}}{e^{2\pi y}-1} dy \\
&=2 (2\pi)^{s-1} \sin \frac{\pi s}{2}
   \int_{0}^{ \infty} \frac{x^{-s}}{e^x -1} dx %\label{47}
\end{align*}
�Ƃ����܂ł͂悢���A�Ō�̐ϕ�����ł���B
�����ōH�v������B
���Ȃ킿�A���łɒm���Ă���

\noindent
\Fig[����47�f���}]{\textwidth}{5\baselineskip}

\[
\zeta(1-s)
=\frac{1}{\Gamma(1-s)}
   \int_{0}^{ \infty} \frac{x^{-s}}{e^x -1} dx %\label{48}
\]
�𗘗p����΁A���̐ϕ����������邱�Ƃ��ł�
\[
\zeta(1-s)
=\frac{1}{\Gamma(1-s)}
 \frac{\zeta(s)}{2 (2\pi)^{s-1} \sin \dfrac{\pi s}{2}} %\label{49}
\]
���Ȃ킿�A�����̋��߂�$\zeta$�֐��̊֌W��
\[
\zeta(s)
=2 (2\pi)^{s-1} \sin \frac{\pi s}{2} \Gamma(1-s) \zeta(1-s) %\label{50}
\]
�𓾂��B
���ꂪ�悭�m����u���[�}���̊֐������v�idie Riemanische Funktional gleichnung�j�ł���B

\section{�o�[�[����� I}

�Ƃ���œ����́u�o�[�[�����v�Ƃ́A��������
\[
\zeta(2)
=\dfrac{1}{1^2}+\dfrac{1}{2^2}+\dfrac{1}{3^2}+\cdots +\dfrac{1}{n^2}+\cdots %\label{50'}
\]
�����߂���ł������B

���̔w�i�ɂ͂�����e�ՂȖ�������
\[
\dfrac{1}{1}+\dfrac{1}{2}+\dfrac{1}{3}+\cdots +\dfrac{1}{n}+\cdots %\label{50''}
\]
�͗L���Șa�������Ȃ����Ƃ����炩�ɂ���Ă������Ƃ�����B
���ہA���̘a��$\log n$�Ɋ֌W��
\[
\text{����51��������} %\label{51}
\]
�ł����āA$\gamma$�͂�����u�I�C���[�萔�v�i$\gamma=\cdots$�j�ł���B
$\gamma$��$e,$���ƕ���ł悭�m����d�v�萔�ł��邪�A���̐����ɂ‚��Ă͂قƂ�ǂ킩���Ă��Ȃ��B
���Ȃ݂ɁA�u�I�C���[���}�N���[�����̘a�����v�ɂ���
\[
\text{����52��������} %\label{52}
\]
�ŁA$n \to \infty$�ł̔��U�͂���߂Ēx�����A����ł��a�͔��U�ł���i���܂���Ă͂����Ȃ��j�B

���āA�{���$\zeta(2)$�͗L����$<2$�ł��邱�Ƃ͂킩���Ă����B
���Ȃ킿�A���̎���ɂ����Ă��łɖ��������̘a�̗L���A�����̋�ʂ͏\���Ɉӎ�����Ă����B
���̉ۑ�ɑ΂��A�I�C���[�̉��́e�h�̂𔲂��f�����ŁA����Ȃ��Ƃ��������̂��A
�Ƃ����㐢����̃l�K�e�B�u�ȕ]���i18���I�̐��w�́u�����v�łȂ��j�������������̂ł���B
$\sin x$��$x$�̑������i�����ł���ƌ��Ȃ��A������$\sin x=0$�̎��́j����
\[
x=0, \, \pm \pi, \, \pm 2\pi, \, \pm 3\pi, \, \ldots
\]
�ł��邩��A$x$�ȊO�Ɉ�������
\begin{align*}
&\left( 1-\dfrac{x}{ \pi} \right) \left( 1+\dfrac{x}{ \pi} \right)
 \left( 1-\dfrac{x}{2\pi} \right) \left( 1+\dfrac{x}{2\pi} \right) \cdots \\
&=\left( 1-\dfrac{x^2}{ \pi^2} \right) \left( 1-\dfrac{x^2}{4\pi^2} \right) 
  \left( 1-\dfrac{x^2}{9\pi^2} \right) \cdots \qquad (\sharp) %\label{53}
\end{align*}
�����‚͂��ł���B
����
\[
\sin x/x=1-\frac{1}{6} x^2+\cdots
\]
�ł��邩��A���Ɗ֐��̊֌W����
\[
\dfrac{1}{\pi^2} \left( 
\dfrac{1}{1^2}+\dfrac{1}{2^2}+\dfrac{1}{3^2}+\cdots \right)
=\dfrac{1}{6} %\label{54}
\]
���ꂩ��A���]��
\[
\dfrac{1}{1^2}+\dfrac{1}{2^2}+\dfrac{1}{3^2}+\cdots 
=\dfrac{\pi^2}{6} %\label{55}
\]
�𓾂邱�ƂɂȂ�̂ł���B
���̌��ʂ݂̂Ȃ炸�A$(\sharp)$��$\sin x/x$�́u������ϓW�J�v�Ƃ����A���ې������B

����͐����������Ƃ��������łȂ��A18���I�̐��w�͌����łȂ������A�Ƃ����ᔻ�́A
�K�������������Ȃ��Ƃ������_�͐��w�j�Ƃ̖₢������B
����ǂ��납�A���Z���w�Ƒ�w���w�̊Ԃ̓��e�̒������M���b�v�́A�{���A
���w�̒���18���I�̐��w�ɑ΂���K�؂ȗ������������Ă��邩��ł��낤�B
���w��3K���̉������헪���l�����ł��̔F���͏d�v�ł���B

�����Ƃ��A�I�C���[�͂������ɋC�ɂȂ����炵���A�������������ƂȂ�
\[
\text{����55'��������} %\label{55'}
\]
���l����ʉ���^���Ă��āA������V�˂̗]�T�̖ʖږ��@���鏊�ł���B

�ł͗]���������āA����$p \geqq 3$�ɑ΂�
\[
\zeta(p)
=\dfrac{1}{1^p}+\dfrac{1}{2^p}+\dfrac{1}{3^p}+\cdots +\dfrac{1}{n^p}+\cdots %\label{56}
\]
�͂ǂ��Ȃ邩�Ƃ����ۑ肪����B
�����̓x���k�[�C��$B_1$��p����$\zeta(2)=B_1 \pi^2$�𓾂����A
��ʂɋ�����$p=2m$�ɑ΂��Ă��A�Ăуx���k�[�C����
\[
\zeta(2m)=\text{dummy} %\label{58}
\]
�Ƃ���������^���Ă����B
���Ȃ킿�A�o�[�[�����̈�ʉ�
\[
\text{����59��������} %\label{59}
\]
�ɍs���������B

���̓x���k�[�C���̊���͂���ǂ���ł͂Ȃ��B
����ɂ‚��Ă͖{���ɏq�ׂ邱�ƂƂ��悤�B

\section{�o�[�[����� II}

�I�C���[�́A�e�����̕��䂩���э~��Č�����f�悤�ȕ��@��
\[
\dfrac{1}{1^2}+\dfrac{1}{2^2}+\dfrac{1}{3^2}+\cdots 
=\dfrac{\pi^2}{6} %\label{60}
\]
�������ւ�ɁA��w�̔��ϕ��̃N���X�E���[���ɂӂ��킵���e�ӂ‚��f�̂���������B
�����̃e�L�X�g�ɏ�������Ă���t�[���G�����̕��@�ŁA���̋Z�@�ɏ]���A�����֐�
\[
f(x)=\text{dummy} %\label{61}
\]
���O�p�����ɓW�J���悤�B
�t�[���G�W����
\begin{align*}
a_n&=\text{dummy} \\
b_n&=\text{dummy} %\label{62}
\end{align*}
�ł��邩��A�t�[���G������
\[
\text{����63��������} %\label{63}
\]
�ƂȂ�B
���ꂪ�͂����Ď���������$f(x)$�ɓ������Ȃ邩�͂܂��ʖ��ł���B
�����̈ʑ��i���ώ����A��l�����j�̊m�F�A���ʐϕ��̕ۏ؂����߂��A�u���[�}�������x�[�O�̒藝�v�������ł͖������ʂ����A
�����ł͂ӂ�Ȃ����ƂƂ��A�t�[���G������$f(x)$�Ɏ������邱�Ƃ������邩��A������$x=\cdots$�Ƃ����ƁA
\[
\text{����64��������} %\label{64}
\]
�𓾂ďI��B
�����̐l�X�ɂƂ��āA���̂�����$\pi^2/6$�𓾂���@���܂��͕��Ղł��낤�B

\noindent
\Fig[����65���@�}]{\textwidth}{8\baselineskip}

\section{�Ȃ��e���[�}���́f$\zeta$�֐���}

$\zeta$�֐���
\[
\zeta(s)
=\dfrac{1}{1^s}+\dfrac{1}{2^s}+\dfrac{1}{3^s}+\cdots %\label{66}
\]
�ƒ�`����Ă��邪�A���̒�`�̓��[�}���ɂ����̂ŁA�I�C���[�̌��藝�͂���ƈقȂ��Ă����B
��̒�`�ŗႦ��
\[
(\sharp) \qquad 120^{-s}=(2^{-s})^3 \cdots 3^{-s} \cdot 5^{-s}
\]
�́A�W�J��
\begin{align*}
\dfrac{1}{1-2^{-s}}=1+2^{-s}+(2^{-s})^2+(2^{-s})^3+\cdots \\
\dfrac{1}{1-3^{-s}}=1+3^{-s}+(3^{-s})^2+(3^{-s})^3+\cdots \\
\dfrac{1}{1-5^{-s}}=1+5^{-s}+(5^{-s})^2+(5^{-s})^3+\cdots %\label{67}
\end{align*}
�̎�ɂ�����x�����o������B
���̂��Ƃ���A�I�C���[�ɂ��Ζ������
\begin{align*}
\zeta(s)
=&\dfrac{1}{1-2^{-s}} \cdot \dfrac{1}{1-3^{-s}} \cdot 
  \dfrac{1}{1-5^{-s}} \cdot \dfrac{1}{1-7^{-s}} \cdots \\
 &\text{�i�ȉ�}(\text{�f��})^{-s}\text{�ɂ���ρj} %\label{68}
\end{align*}
��$\zeta$�֐��i���͕̂ʂƂ��āj�̒�`�ł���B
$\zeta$�֐����f���ɐ[��������肪���邱�Ƃ́A���コ��ɂ͂����肷�邾�낤�B

\section{��͓I�����Ȃ����͉�͐ڑ��̃t�V�M}

%\footnote{�w��͊T�_�x�͂������p����B}

�u�J�C�Z�L�E�Z�c�]�N�v�͊֐��_---�������͉�͊֐��_---�̍ł��ؖ��̉s���L�p�Ȓ藝�ł���B
���ϐ��̔��ϕ��ł͏o���Ȃ��v�������Ȃ��A���́e�ǂ����Ă���Ȃ��Ƃ�������́H�f�Ƃ����Ƃ܂ǂ���^���A
�������Ă��̐������e���킩��Ȃ��Ȃ邭�炢�ł���B

���Ƃ��΁A$s>0$�i���͕��f���ł��悭$\mathrm{Re}(s)>0$�j�ɑ΂�
\[
\Gamma(s)
=\int_0^\infty e^{-x}x^{s-1} dx \qquad \text{�i�K���}�֐��j} %\label{69}
\]
���A�֐�����
\[
\Gamma(s+1)=s\Gamma(s), \quad s>0
\]
�𖞂����Ƃ͂悭�m���邪�A���̊֐��������e�t�p�f�����Q���֌W
\[
(\sharp) \qquad \Gamma(s)=\Gamma(s+1)/s
\]
�͑S�R�ʂ̓���������B
���Ȃ킿$s+1>0$�Ő��藧�‚���A$0>s>-1$�ɑ΂��Ă����ӂ͒�`�����B

����{��$(\sharp)$����`���ƂȂ�A���͂�ϕ��ɂ���`�͗p�����Ȃ��B
���ہA������$s$�ɑ΂��Ă͉���$\Gamma(s)<0$�ł���B
$\Gamma( \, \cdot \, )$�̒�`�悪$0>s>-1$�܂ʼn������ꂽ�̂ŁA
$(\sharp)$��$0>s+1>-1$�ł悭�A����Ē�`���$-1>s>-2$�܂ł���ɉ��������B
���̂悤��$\Gamma$�֐��͕��̐������������ׂĂ�$s$�ɑ΂��Ē�`����邱�ƂɂȂ�B

���ꂾ���ł͂Ȃ��B
����$s>0$�ɑ΂��Đϕ��Œ�`���ꂽ���֐��͂��̒�`�𒴂��Ă��ׂĂ�$s$�ɑ΂��Ē�`���ꂽ���A
���͂��̉����͂��ׂĂ̕��f��$s$�i���f���ʁj�܂Ŏ��邱�Ƃ��ؖ��ł���B

���̐܁A�����̒�`�͌`�𗯂߂Ȃ��̂ŁA��u�Ƃ܂ǂ��Ȃ����͍�����������B
�����͊֐������ɂ���čs���邪�A���ꂾ���ł͂Ȃ��B
�u��͓I�����v�Ƃ����悤�ɕ��f��$s$�̊֐��Ƃ���
\[
\Gamma(s)
=\int_0^\infty e^{-x}x^{s-1} dx, \qquad 
\mathrm{Re}(s)>0 %\label{70}
\]
����͊֐��ł��邱�Ƃ����������Ă���B
���Ȃ݂ɂ��̒l�������‚��^���Ă������B

\noindent
\Fig[����71���i�����j�i�}�j]{\textwidth}{10\baselineskip}

\section{$\zeta$�֐��̉�͓I����}

$\zeta$�֐�$\zeta(s)$������
\[
\zeta(s)
=\dfrac{1}{1^s}+\dfrac{1}{2^s}+\dfrac{1}{3^s}+\cdots, \qquad 
\mathrm{Re}(s)>1 %\label{72}
\]
�Œ�`���ꂽ�B
$\zeta(2)=\pi^2/6, \, 
 \zeta(3)=\cdots, \, 
 \zeta(4)=\cdots $�Ȃǂ��m���Ă��āA$\zeta(2m)$�̓x���k�[�C���ł���킳�ꂽ�B
�������A$\zeta(-1), \, \zeta(-2), \, \ldots$�͂ǂ����낤���B
�ꌩ���Ē�`�ɍ���Ȃ��ǂ��납�i���Z���X�Ƃ������ƂɂȂ�B
�������A����͓����̒�`�����Ƃ��āi���̌���ŁA�Y��j�A���[�}���̊֐�����
\[
\text{����73��������} %\label{73}
\]
��ʂ̐V������`���Ƃ���
---�������A�����̒�`�Ƃ�$\mathrm{Re}(s)>1$�ł͈�v��---��͓I�������s���B
���̂Ƃ��͓�����$\zeta(s)$�̒�`�͓K�p����Ȃ����Ƃɒ��ӂ��悤�B

�ł͌v�Z���Ă݂悤�B
���R�����ł��x���k�[�C����������
\begin{align*}
\zeta(1-2m)
&=2(2\pi)^{-2m} \text{���Ǖs�\} 
  \dfrac{\pi (1-2m)}{2} \Gamma(2m) \zeta(2m) \\
&=2(2\pi)^{-2m} 
  (-1)^m (2m-1)! \dfrac{(2\pi)^{2m}B_m}{2 \cdot (2m)!} \\
&=\dfrac{(-1)^m B_m}{2m}, \qquad m \geqq 1, %\label{74}
\end{align*}
�ȂǂŁA�ڗ��‚͕̂��̋���$-2m$�ɑ΂��ẮA�֐���������
\[
\zeta(-2m)=0, \quad k=1, \, 2, \ldots
\]
�ł��邱�Ƃł���B
������$\zeta$�֐��́u�����ȗ�_�v�Ƃ����B
���f���ʂ̍������ʂ̗�_�͂����Ɍ���B

�ʗp���Ȃ��Ƃ͂����A$\zeta(s)$�̒�`����0�ɂȂ�Ƃ͑z�����ł��Ȃ��������A���������ϐ��̔��ϕ��Ƃ͕ʐ��E
---�������A�������牄�����ׂ荇�����i�ڑ������j���E---����䂦��ł���B
�Ƃ���ŁA�����ȗ�_�ȊO��
\begin{equation*}
\zeta(s)=0 %\label{75}
\end{equation*}
�ƂȂ�$s$�͂Ȃ����낤���B
����͂����āA���f���ʂ̒���
\[
s=\dfrac{1}{2}+ti \quad (t\text{�͎���})
\]

�‚܂�R���s���[�^�E�T�[�`��$\mathrm{Re}(s)=1/2$�̏�ɋɂ߂đ������o����邱�Ƃ��킩���Ă���B
������ƌ����āA�e$\mathrm{Re}(s)=1/2$��$s$�Ɍ�����f�Ƃ͌��_�ł����A�����܂Łu�\�z�v�ɂƂǂ܂�B
������u���[�}���\�z�v�iRiemann's hypothesis�j�Ƃ������A���󂷂�΁u���[�}�������v�ł���B
���[�}�������͂��܂��ؖ�����Ă��Ȃ����㐔�w�̈����ł���B

���[�}���������ؖ������΁A�f���̕��z�֐�
\[
\pi(x)=x \text{�܂ł̑f���̌�}
\]
�ɑ傫�Ȑi����^����B
$\zeta$�֐����f���̕��z�Ɋ֌W���邱�Ƃ́A���[�}�������ނ���I�C���[�́i�f����p�����j��`������[���ł��鏊�ł��낤�B

�Ƃ͂����A$\zeta$�֐��̒m��ꂽ�����Ȓl�̓x���k�[�C���ƃ΂ŕ\�킳��Ă��邱�Ƃ͒m���Ă����������悢�B
�e�[�[�^�̐��E�̓x���k�[�C���ŕ~���‚߂��Ă���f�Ƃ����������Ƃ��ł��悤�B

\section{�x���k�[�C���̐��E�͑���}

���R�u�E�x���k�[�C�͎����̑n�o������������قNj���ȑ��݂ɂȂ�Ƃ͎v���Ă��݂Ȃ������ł��낤�B
�������A���z��$\pi$��$e$�Ƃ͈���ăx���k�[�C���͗L�����ł���A���ꂪ�������ăt�V�M�ł���B
���̌v�Z�������A���҂��ǂ񂾁w��͊T�_�x�i19�����N�Łj�ɂ́A�����Ԗڂ܂Ōv�Z���ꕪ�q�́c���A����́c���A�ƋL���Ă���B
���ł͂���ɐ�܂ő����Ă���ł��낤�B
�����ł͉��������Ȃ��B

���̗��_�I�˒����܂��悪���邪�A���Ƃ��΁A
�u�t�F���}�[�̍ŏI�藝�v�̏ؖ��̒��ɏo������i���ǂ͏ؖ��͂ł��Ȃ��������j�u�N���}�[�̗��z���v�́A
�x���k�[�C�������Ȃ�n�[�h�ɗ��p����B
�܂��A�ʑ��􉽊w�ɂ����Ă��A�x���k�[�C���̊��p������悤�����A���͂�����g�s�b�N�ł���A���҂��֐S���񂹂Ă���B

���������ƁA�x���k�[�C���͂��Ȃ艏�������Y�J�V�C�ِ��l�̃I�n�i�V�Ǝv�������m��Ȃ��B
�������A�u�I�C���[���}�N���[�����̘a�����v�̂悤�ɁA���ϕ��̃N���X�E���[���Ő������Ă������L�p�Ȍ����ɂ�
�x���k�[�C����x���k�[�C���������o������B
���̌����͂悭�m����}�N���[�����W�J�i�e�C���[�W�J�j�̕ό`�ł��邪�A�ނ��뉞�p�Ƃ��Ă͕ʕ��ƍl���������悢�B
�܂��A
\[
\text{����76��������} %\label{76}
\]
�ł��邪�A$f'$�̂Ƃ���ɂ��̎����̂����q�̂悤�Ɏg���A���ŒW���̐ςݏd�Ȃ�����肷��ƁA�܂�
\[
\text{����77��������} %\label{77}
\]
��������B
���̌`�ł����Ȃ�̌��p�����邪�A���͏o���Ƃ��Ă͐��ς��ŁA�������Ȍ��ʂ�h���؂�Ȃ��B
���Ƃ���
\[
\text{����78��������} %\label{78}
\]
�ł���B
�I�C���[���g���u$\cdots \cdots $�v�i���e�����$\cdots \cdots $�j�ƒ��ӂ��āA����ɏ�]������������ƕt��
\[
\text{����79��������} %\label{79}
\]
�̂悤�ɍ��グ�Ă���B
���ꂪ�����̌����ł���B

���p�Ƃ���
\[
\text{����80��������} %\label{80}
\]
�Ȃǂ��l������B

\section{��������ɐ��ރx���k�[�C�A���E�X�s���b�g}

���j�I�ɂ͂��ׂĂ͂�������Ɏn�܂�Ɖ]���Ă������߂��ł͂Ȃ��B
�x���k�[�C����������̖ڂ̘a�̌���

$N$�‚̂�������ɂ‚���
\[
n\text{�‚̂�������̖ڂ̘a}=k
\]
�̊m�����z�i���ۂɂ͏ꍇ�̐��j�����܂��v�Z�������Ƃ���A
���̋Ɛт��܂߂āA�����Ԃ�Ђ����ڂ����邪�A18���I�̑����̐��w��̔��W���n�܂�A
���ꂪ���ʂ̔g��̂悤�ɔg�y�����B
���ہA�悭�����Ă���B
�������Ȃ���A�m���_�ɂƂ��Ċ̗v�ł���ɂ�������炸�ӊO�Ɖ�͓I�ɋl�߂��Ă��Ȃ��B
�x���k�[�C��$n=10$�܂Ŏ������̂̍Ō�̕��͕����I�ł���B
�h�E���A�u��(de Moivre)�A�����̃t�F���[(W. Feller)�͎w�j�͎������A���ʂ܂ŒB���Ă��Ȃ��B
��莩�̂͊ȒP�����A��֐��A�d���g�ݍ��킹�A�}�`���A�x���k�[�C���A�������p���鎮�������K�v�ł���B
���̌���
%\begin{enumerate}
%\item 

(1) �����Ȃ�$n$�ɑ΂��Ă��ŏ���6��$k=n, \, n+1, \, \ldots, \, n+6$�ɂ‚��Ă͉��������̂͂Ȃ��A
$1,\, 1,\, 1,\, 1,\, \ldots $����a���d���I�ɐς݂������i$n$�d�a���j������u�}�`���v
\[
\text{�i��������H�j}
\]
�ƂȂ�B
�������u�}�`���vfigurate numbers�ɂ‚��Ă͎�X�̌ď̂�����B


(2) ��ʂ�$k$�ɂ‚��ẮA$n$�‚̂�������̖ڂ�$\text{�a}=k$�ƂȂ�ꍇ�̐��́A��֐�
\[
(t^1+ t^2+ t^3+ t^4+ t^5+ t^6)^n
=t^n\cdot (1-t^6)^n\times (1-t)^{-n}
\]
��$t^k\, (k=n,\, n+1,\,\ldots, \, 6n)$�̌W���ł���B
���̕�֐���

\begin{enumerate}
\item $(1-t)^{-n}\, \diamond$
\item $t^n \cdot(1-t^6)^n \, \diamond$
\end{enumerate}
��2���q���琬��A1.��$n$�d�a���̕�֐��ɂ���Đ}�`�����Q���I�ɗݐς��鉉�Z�ŁA
�g�ݍ��킹�_�I�Ɂu�d���g�ݍ��킹�̐��v${}_n\mathrm{H}_r$�i�����̓񍀌W���j�ŕ\������A
2.�́A�㔼��1.�ɑ΂���C���ŁA
\[
\text{�i��������H�j}
\]
����A$\Delta k=6$���ɁA1.�ɂ��}�`������������������ŎZ�����邢�͍����������Z���i��֐��Łj�\���Ă���B
�Ȃ�$t^n$�͂�������̖ڂ̘a�̕��z�̍ŏ�$\text{�i�ŏ��j}=n$���������ƂŁA���z�̍����w�b�_�[�̈ʒu�����߂�B
�ȏ�̌��ʂ���A�񍀌W���ȏォ��A�ŏI�I�Ȍv�Z�̉�͓I�\����
\[
\text{�i��������H�j}
\]
�ƂȂ�B
%\end{enumerate}

\noindent
{\gt �q��r}\, $n=16$�̃P�[�X

\vspace{10\baselineskip}

\noindent
{\gt �q��r}\, ��ʂ̏ꍇ
\vspace{10\baselineskip}

\section{�w�����p�x��III��}

��III���́A���H�E���p�ʂł���ɐi�݁A�q���ɂ�����m���ɂ������ȕ��z�@�A���Ҋz�̌v�Z�ւƓW�J����B
�����薼��

\begin{quote}
---���܂��܂ȓq���̕��z����ы��R�̃Q�[���Ɋւ���O�q�̗��_�̉��p����������III��\, 
Usum Praecedentis Doctrinae In Variis Sortitionibus \& Ludis Aleae---
\end{quote}
�ł���B
sortitio��sors�u�q���v�̕��z���Ӗ�����B
���̂��߁A��II���̑g�ݍ��킹���w����Ăъm���̉ۑ�֖߂��āA��III���͎���24�₩�琬��B
����߂ďڍׂł��邪�A����]���΁A�����̃f�J���g���̑㐔�I���L�@�̔��B���\���łȂ����߁A�璷�̈�ۂ��ے�ł��Ȃ��B
�܂��A�����Q���҂������v���[����ǖʂ������̃Q�[�����_�ł����u�W�J�`�Q�[���v�̌`���ł���A
������u�c���[�v�̕\����p����΂�薾�m�ɂȂ�B
�����Ƃ��A����͖{���I�ł͂Ȃ����낤�B

�����ł͍ŏ���10��݂̂��Љ��B

\begin{enumerate}
\item[��I] 
��I������11�̓��ʃP�[�X�m���A���̃g�[�N��\footnote{
���̑�p�d�݁B�����ł́A�w���̌��ʋ@�ւ̗��p�����̎x�����ȂǂɎg����B}
�e1���������Ă���‚ڂ���A3�l�������񕜌��Œ��o���A�����������҂������Ƃ���B����́u�����v�ɂ�����n

\noindent
\Fig[�}]{0.8\textwidth}{5\baselineskip}

\item[��II] 
���A3�l�Ƃ����������Ȃ��ꍇ�͊l���z�m�{���n�E�X�ɑ�����n��3�l�ɕ��z�����.

\item[��III] 
�����@�m���������n���K�p�����B
A,\, B,\, C,\, D,\, E,\, F�̂����A�܂�A,\, B�̏��҂�C�Ƒΐ�A���̏��҂�D�Ƒΐ�A�c�c�Ƃ���B

\item[��IV] 
���A���҂ɂ͂��̓s�x2�{�̏��P�[�X�����蓖�Ă���
�m$1:1, \, 2:1,\, 4:1,\, \ldots $�̂悤�ɏ��������i����n\\
���F

\noindent
\Fig[�}]{0.8\textwidth}{5\baselineskip}

\item[��V] 
�z�C�w���X�̕t�͑�3��ő�I���ɂ��邪�A����i�H�j�ʼn����B

\item[��VI] ����4��B
\item[��VII] 
I�Ɠ��A�������G�D1�����܂ރJ�[�h�E�f�b�N��������B
�m��ӂł̓J�[�h�̖����ɂ͒�߂��Ȃ��A�K�v�ɉ����āu�J�[�h�E�Q�[���v�̃J�[�h�����Z�b�g���p�ӂ���B�n
\item[��VIII] 
���A�G�D�͕������Ƃ���B
\item[��IX] 
���A�G�D�̍ő喇���𓾂��҂������Ƃ��A�^�C�̏ꍇ�͕�����������B���ʖ����ɂ͕��z���Ȃ��B
�m���z�͏ꍇ�̐��𕪔z������̂Ƃ���B�n
\item[��X] 
A,B,C,D4�l�ɁA�G�D16������ȊO�̃J�[�h36�����܂ރJ�[�h�����̏��Ԃɔz��B
23���z��I��������_�ŁAA,\,B,\,C,\,D�͂��̂���4,\,3,\,2,\,1���̊G�D�𓾂��B
D�͒f�O���āA���̌�����A,\,B,\,C�̂����ꂩ�ɔ�����̂Ƃ���B
���̉��i�͂�����ɂȂ邩�B
�����̊m���͂ǂ��Ȃ邩�B
\end{enumerate}

\section{�w�����p�x��IV��}

��҂��^�ɂ߂������ɖړI���W�Ԃ���Ă���ŏI���ł���A�薼�������ʂ�
\begin{quote}
---�Љ�I�A���_�I�A�o�ϓI(���Ƃ���)�Ɋւ���O�q�̗��_�̉��p���������IV�� \, 
Usum \& Applicationem Praecedentis Doctorina�@In Civilibus, Moralibus \& Oeconomicis---
\end{quote}
�Ƃ����܂łƂ��Ȃ苿�����قȂ�B�������A�����́u������v�œ��e���Љ�I�ȗL�p�������‚��Ƃ��������Ă���B

���w���_�̓L�`���Ƃ�邪�A���ꂾ�����ړI�����ł͂Ȃ��B
���ہAcivilus�i�����ł͒D�i�j�́u�����v�u���Ɓv�u�Љ�v�u�s���v���Ӗ�����B
�����́A���ꂳ�ꂽ���Ƃ̌`���悤�₭�`������‚‚������ߑ�̓�����̎���ŁA
���̋C�^���u�Љ�I�v�u�s���I�v�i���̌���Łu�S�̓I�v�j�Ƃ������Ƃ΂ɍ��߂��Ă���B
��������́u���n�v�Ȃ̂ł���B
moralia�����݂́u�ϗ��I�v�Ƃ͑����قȂ�
---�͂�����A�ʕ��ƍl���Ă悢---
�u�����I�v�łȂ��Ƃ����Ӗ��Łu���_�I�v�u�S���I�v�A��荡�����ɂ́u�F�m�I�v�Ƃ����悤���B
economicus�����Ƃ́i�ƌv�́j�؂萷��A��肭��ł��邱�Ƃ͑����̒m��Ƃ���ł��낤�B
���ꂪ�Љ��ɂȂ��āu�o�ρveconomics�ɂȂ����B
�����ł́u�o�ς̌v�Z�v�Ƃ����Ă������B

���̂悤�ɑ�IV�����߂������͎Љ�́u�O�����h�E�Z�I���[�v�ł���A�������匾�s��ł͂Ȃ��A
���̕��@�̊�b�����ɂ̌��ł����ւ�B
���̈�Ԏ肪�u�吔�̖@���v�ŁA���ꂾ���ł��̑�Ȗ@���ł��邪�A��҂͂��̐���s���O�ɖv�����B
�����̓��v�w�́w�����p�x�̊�{���_�̕����ł���B

���̑�IV����5�͂ɕ�����Ď����ɏ������ꂽ�Y��ȍ�i�ŁA�܂��Љ�ɂ�����u�m���v�̃R�g�o�g������n�߂��Ă���B

\noindent
{\gt ��I��}\, ���Ƃ���̊m�����A�m���炵���A�K�R������ъW�R���ɂ‚��Ă̂����‚��̊�{�v�f

Praeliminaria Quaedam De Certitudine, Probabilitate, Necessitate \&�@Contingentia Rerum

\newpage

\begin{table}[ht]
\caption{���{��A�p��A�������e����ɂ��u�m���v�֘A��b}
  \begin{tabular}{lll} \hline
�m���炵���m�m���炵���n & probable                  & probabile \\
���m���炵������       & more probable             &           \\
�”\                     & possible                  & possibile \\
�F����m��               & morally certain           & moraliter certum \\
�F����s�”\             & morally impossible        & moraliter impossibile \\
�K�R�m�I�n               & necessary�@�@�@�@�@�@     & necessitate \\
������K�R               & physically necessary      & necessitate vel physica \\
������K�R               & hypothetically necessary  & necessitate vel hypothetica \\
�_���A���x��K�R       & contractually,            & necessitate pacti seu instituti \\
                         & institutionally necessary &           \\
�W�R�m�I�n               & contingent                & contingens \\
                         &                           & liberum,\,fortuitum, casuale \\
�m�C�Ӂm�I�n���邢��     & free                      & \\
�F��m�I�n�A���R�I�n     &                           &
 \end{tabular}
\end{table}

\noindent
{\gt ��II��}\, �m���Ɛ����ɂ‚��āA�����p�ɂ‚��āA�����p�̋c�_�ɂ‚��āA�֘A���邢���‚��̈�ʓI����

De Scieintia \& Conjectura, De Arte Conjectandi. De Argumentis Conjecturarum. 
Axiomata Quaedam Generalia Huc Pertinentia

����Љ�ł��\���ɒʗp����ӎv����̏��������ʂȂ��K�؂ɏq�ׂ��Ă���A�����ɂ̓X�g�A��`�I�ȗϗ�������������B

1.���_�ɓ��B�ł���ꍇ�͐����̗]�n�͂Ȃ�

2.���ꂱ��̋c�_�̏d�v�x�m�E�F�C�g\footnote{
�����I�Ӗ��ɂ�����”\���̊m���i���Ƃ���0.6�Ȃǁj���l����΂悢�B}�n��
�i���ꂼ��j�]������݂̂ł͕s�\���ŁA�����𑍍����Ēm���𓾂��Ƃ���̏؋��Ƃ��ׂ��ł���B

3.�������̋c�_�݂̂Ȃ炸���Ε����̋c�_���l�����ׂ��ł����āA
�����̏d�v�x�𐳂����ʂ�΂ǂ���̋c�_���D�邩�m���ɂȂ�ł��낤�B

4.���ՓI�Ȃ��Ƃ���Ɋւ��锻�f�ɂ͊ԐړI�ŕ��ՓI�ȋc�_�ŏ\���ł��邪�A
�•ʂɊւ��鐄���ɂ͒��ړI�Ō•ʓI�ȋc�_���A�”\�Ȕ͈͓��ŁA�K�v�ł���B

5.�s�m���ŋ^�₪���邱�Ƃ���ɂ‚��Ă͏�񂪓�����܂Ŕ��f��ۗ����ׂ��ł��邪�A
�P�\��������Ȃ��ꍇ�ɂ����ẮA��҂̂����ǂ�����ϋɓI�ɗǂ��Ȃ��ꍇ�ł����Ă��A
��͂���K�؂ŁA���S�ŁA�v���[���A�m���炵���s�ׂ�I�����ׂ��ł���B

6.�v���Ȃ��Ƃ��Q�̂Ȃ����肪�]�܂����B

7.���ʂ݂̂Ől�Ԃ̍s�ׂ𔻒f���Ă͂Ȃ�Ȃ��B

8.�t�^���ׂ��d�v�x�m�E�F�C�g�n�̕]��������Ă͂Ȃ�Ȃ��B
���ΓI�Ɋm�����ΓI�ƌ���Ă͂Ȃ�Ȃ����A����𑼐l�Ɏ咣���Ă͂Ȃ�Ȃ��B

9. ���ׂĂ̈Ӗ��ƕK�v���ƗL�p���ɂ����Ċ��S�Ȋm�����𓾂邱�Ƃ͋H�ł���ȏ�A
�F����m�S�̒��ł���́n�m�����ΓI�m���ƒ�߂邪�悢�B

\noindent
{\gt ��III��}\, ���܂��܂ȋc�_�@����т��Ƃ���̊m���炵�����Z�o���邽�߂̏d�v�x���]���������@�ɂ‚���

De Variis Argumentorum Generibus \& Quomodo Eorum Pondera
Aestimentur Ad Supputandas Rerum Probabilitates

�������琔���I�i�K�ɓ�����H�I�������Ȃ����B
quomodo�́u�����ɂ��āv�̈ӂŌ���I�ɂ�how to�ƍl����΂悢�B
�ȉ��͊e���v��ł���B

\begin{enumerate}
\item ���R�I�ɑ��݂��邪����ɕK�R�����������Ƃ���̌v�Z�m�ȉ��v�Z���n
\item �K�R�I�ɑ��݂��邪����ɋ��R�����������Ƃ���̌v�Z
\item 1,\, 2�𑍍������v�Z
\item ����̂��Ƃ���ɂ‚��A�����̂��Ƃ��炪���݂���ꍇ�ւ̈�ʉ�
\item ���A���ꂼ��̋c�_�����ׂĈقȂ邱�Ƃ���̏ꍇ
\item ���A���ꂼ��̋c�_��4,5����������ꍇ
\item �؋��A���Ώ؋�����������Ƃ��A���ꂼ��̊m���炵���̌v�Z
\end{enumerate}

\noindent
{\gt ��IV��}\, �ꍇ�̐������o����‚̕��@�ɂ‚��āB
����ɂ‚��ĉ���m��ׂ����B�ώ@�Ɋ�Â�������߂��邩�B
���̕��@�Ɋւ�����ʂ̉ۑ�B

De Duplici Modo Investigandi Numeros Casuum.Quid Sentiendum De Illo, 
Qui Instituitur Per Experimenta. Problema Singulare Eam In Rem Propositum, \& c.

�\�z���n�߂ċ�̓I�ɖ��炩�ɂ����B
20�N�����߂Ă����\�z�ł���Ƃ����B

�l�̎����l���Ă݂悤�B
�����̌����ɂ���Đl�͎��ʂ��炻�̌�������l�̎��́u�m���v��m�邱�Ƃ͌����Ăł��Ȃ��B
�������A�����ɍ���‚̕��@������B
�ώ@�ɂ����
��������@�ŁA����͕ʒi�V�������̂ł͂Ȃ�\footnote{
�|�[���E�����C�����_���w�̎w����A.�A���m�[�́w�v�l�p�xArs Cognitandi�B
���ہA�x���k�[�C�́w���_�p�xArs Conjectandi�͂��̏����ɕ�����Ɖ]���Ă���B}�B
���܂Ő_��I�������u�m���v�͊ώ@����v�l�ɂ���đ��邱�Ƃ��ł���B

����𗝉����邽�߂Ɏ��̎������l���Ă݂悤�B

�|�‚ڂ������āA�����ɔ��̃g�[�N���m�F�̕t�����d�݂͍l�����Ȃ��̂Łn3000���ƍ��̃g�[�N��2000���������Ă���B
��������ƈ���Ă����ł͒P���ɔ���������2�ʂ肵���Ȃ��B
��������1�����‚Ƃ�o���B
�����������ώ@�������ƋL�^���A�g�[�N���͂��Ƃ֖߂��B
����āA�‚ڂ̂Ȃ��̃g�[�N���̖����͕s�ςɕۂ����B
����𑽐��񂭂�Ԃ��ƁA���A���̐��̊ώ@�L�^��������B---

�����Œ��ӂ������B
���̔��A���̗����m���ɂȂ�킯�ł͂Ȃ��B
���ł�\kenten{�m��}�͂��łɒ�܂��Ă��āA����ꂽ���A���̊ώ@���ꂽ�䗦�͂���ɋ߂��A
������킸���ɑ傫�����Ƃ�����킸���ɏ��������Ƃ̊Ԃɑ��݂���A���Ƃ��m�F�������̂ł���B
�������A�Ƃ�o��������������ƁA���̊Ԃ̊Ԋu�͋����Ȃ�A���̊O���ɑ��݂��邱�Ƃ͂܂��܂��N����ɂ����Ȃ�B

���Ƃ��΁A���̖�����
\begin{align*}
 500\text{���̎��_�ŁA}\quad &  299 \text{����}  301 \text{���̊�} \\
1000\text{���̎��_�ŁA}\quad & 2999 \text{����} 3001 \text{���̊�}
\end{align*}
�ƂȂ���\footnote{
�R���s���[�^�E�V�~�����[�V�����ł͎��ۂɂ͂���قǐ��m�ɂ͂Ȃ�Ȃ����A�_�Ƃ��Ă͐�������B}
����������Ƃ̔�$3 : 2$�ɋ߂��A���̔���킸���ɑ傫����Ə�������̊Ԃɓ���B
���Ȃ킿�A3:2�̔䂪������邱�Ƃ��m���炵���Ȃ�A�������ώ@�������傫���Ȃ�ɂ��������A
�܂��܂����̊m���炵���͑傫���Ȃ�B

���̂悤�ɂ���΁A�Љ�I�A���_�I�A�o�ϓI���𓮂����Ă���u�m���v�������ɑ����邱�Ƃ��ł���ł͂Ȃ����B
���ꂪ���R�u�E�x���k�[�C�̌ւ�Ƃ������e�������̂ł���B
���R�u�͂���𒷂��g�߂Ă������A���S�������ďo�ł���@��Ȃ������������B
%�e�_��M���邪�䂦�ɁA�����i���̐�����M���āf���C���͂������낷�������@�ɂ͂��̕������c����Ă���B

\noindent
\Fig[�g�[�N���̎�������]{\textwidth}{5\baselineskip}

\noindent
{\gt ��V��}\, 
�O�͂̉ۑ�̉�@�@Solutio Problematis Praecedentis

���āA���悢��u�吔�̖@���v(Law of large numbers)�̏ؖ����������B

����m�����ۂ�1,\, 0 ��2�P�[�X���N���蓾��,���̊m����$r$��$s$�Ƃ���B
�ώ@��$n$��J��Ԃ��Ƃ��A����$n$�񒆂�1,\,0�̉�$x,\, y \,(x+y=n)$�́A
\begin{enumerate}
\item ���傤��$r,s$�̔�
\[
\text{�i��������H�j}
\]
�ƂȂ�m�����ő�ƂȂ��āA���̔�߂��ɏW����
\item �W���͊ώ@��$n$���傫���قǐ��m�ɂȂ�B
\end{enumerate}

���Ȃ݂ɁA�������d�݂�1000�񓊂���Ƃ��A�\�A���̉񐔂̊�����1�΂P�������i��������H�j�i�e500��j�ƂȂ�m�����ő�ƂȂ�݂̂Ȃ炸�A
���m�ɂ����łȂ��Ƃ������ނ�1��1�̂����ߕӂɏW������B
���̂��܂�ɂ��o���I�ɖ����Ȏ��������j��͂��߂ďؖ����ꂽ�̂ł���B

�ؖ��ł��邪�A�܂��A�t�F���}�[�A�p�X�J���A�x���k�[�C�̌ÓT�m���_�̏����ɂ́A
�u�m���v�͍�����1�ɋK�i�����ꂽ$p,\, q \, (p+q=1)$�ł͂Ȃ��A
�q���ɂ�����悤��2��$r,\, s$�̔�ŕ\�����ꂽ�̂Łi$2:1$�̂悤�ɐ������̔䂪�����j$r+s=t \ne 1$�ŁA
���̕\���͌�������s�K�v�ɍ��݂������`�ɂȂ��Ă���\footnote{
���v���X���A�m����$n/N$�̂悤�ɕ����ŕ\�������Ƃ��������i���ہA���̌����j�́A
�Ȍ�̊m���_�j�̓W�J�Ɍ���I�ɑ傫�Ȗ������ʂ������B}�B
�ȉ��ł́A�����̕K�v�ɉ����āA$r,\, s$��$p,\, q$�̂悤�ɋK�i�����ēǂݑւ�$t=1$�Ƃ���B

�x���k�[�C�́A�܂��ő�m���́A1.�̂悤��
\[
\text{�i��������H�j}
\]
�ł��邱�Ƃ��������B
����͍����I�ɂ́A$r,\, s$��$p,\, q$�ɓǂݑւ��āA�񍀕��z�i$x$�̓񍀊m���j�̍ő區��
\[
\text{�i��������H�j}
\]
�ł��邱�Ƃ��]���B
($np,\, nq=\text{����}$�̂悤��$n$���Ƃ��Ă���Ɖ���)�B
���ہA$x$�̓񍀊m��
\[
\text{�i��������H�j}
\]
��$x-1$�̂���Ŋ���A$\geqq 1$�Ƃ����Ĕ�r����ƁA�s�����𖞂����ő��$x$��
\[ x=np \]
�ł����āA����$x$���񍀊m���̍ő��^���邱�Ƃ��������B

�����āA�x���k�[�C��
\[
\text{�i��������H�j}
\]
�Ƒ����A�ŏI�I��
\[
\text{�i��������H�j}
\]
�ɒB����̂ł���B

\noindent
\Fig[����142���ʐ^]{\textwidth}{10\baselineskip}

�ł��邩��A��S�̍�ł��������̌��ʂ̓x���k�[�C�ɂƂ��Ă͍ŏI�I�ł͂Ȃ��A
�u�@���v���͂ނ���A�m���͌��ۂ��瑪��”\�ł��肻����u�����v(��conjectand)��
�L�Ӌ`�ɐ����������Ƃ������̂ł������炵���B
���ہA�u�吔�̖@���v�Ƃ������t����100�N�ȏ����la loi des grands nombres�Ƃ���
S. �|�A�\���i1837�j�ɂ����̂ł���B
�������A�����u�吔�̖@���v�̔��W�͒������A���́u��@���v��20���I�ɂȂ��Ă��烍�V�A�̃q���`��\footnote{
���A���N�T���h���E���R�u���r�b�`�E�q���`���F���V�A�A�\�A�̐��w�ҁB
�����̋Ɛт̓R�����S���t�̌����I�m���_(1935�N)���������B}
�i �@�|�u�{���p�~�t�� �`�{���r�|�u�r�y�� �V�y�~���y�~�j�ɂ��u���@���v�i�d���ΐ��ɂ‚�1924�N�j������B
���̏ؖ��͌��݂ł́u�m���ϐ��v��p���ē������̓X�}�[�g���ƒX�g���[�g�ɂȂ��Ă���A
���Ƃɋ��@���́u�G���S�[�h���_�v�̓���P�[�X�Ƃ��Ď��֐��_�̒藝�Ƃ������悤�B
�x���k�[�C�ɂ����̂́A�����̊m���_�̑f�p�����c�����킢�[�����i������B

\section{���ꂪ������������}

�Z���R�u�������v���‚������탈�n���ɂ���������A���邢�͒킪�Z�͉������������Ƃ��Ă���ƒm��
�����Ȃ炤�܂������邾�낤�Ƃ��A
�Ђ���Ƃ�����Z�͉����Ȃ��̂ł͂Ȃ����Ƃ��A�V�ˌZ��Ԃ͂����ނ˂��̂悤�Ȋԕ��ł������B
�����ނˌR�z�͒�ɏ��A�ΐ퐬�т͑��o�I�ɂ����Β��9��6�s�A
��̕���\underline{���w�I�ɂ�}brilliant�i�D�G�j�������Ƃ����̂��A�M�҂̈�ۂł���B

�������A�l���悤�ł͌Z���ꖇ��Ƃ������f�����肤��B
�Ȃ��Ȃ�A���̑�IV���͍ŏI�I��civil, moral and economic problem�A�‚܂�Љ�I�A���_�I�A�o�ϓI�ȓ��ɒ���ł��邩��ŁA
��������́u�m���v�ǂ���ł͂Ȃ��A�Љ�S�̂̊m�����������Ƃ��Ă���B
�x�C�Y�A���v���X�A�K�E�X�A�ȂLj̑�Ȋw�҂͂��ׂĂ���ɒ���ł���B
���������Љ�Ɂu�m���v�͂���̂��B
����ƍl����Ȃ� �e���ہA���̊m���͂����‚��B���̏ؖ��͏o����̂��f�ƒ��܂ꂽ���̓�����IV���Ȃ̂ł���B

\chapter{���n���E�x���k�[�C}

\section{���ϕ��w�̑�O�̊��胈�n���E�x���k�[�C}

���āA���悢�惈�n���E�x���k�[�C�̓o��ł���B
���n�� Johann�i�p���John�A�t�����X���Jean�j�̓��R�u�E�x���k�[�C�̒�A��q����_�j�G���E�x���k�[�C�̕��ł���B
�c�ȗF�B�I�C���[���7�ˁi�H�j��ŁA�w��̌p���̃��C����
\[
\text{���C�v�j�b�c} \, \Longrightarrow \, \text{���R�u�A���n���E�x���k�[�C} \, \Longrightarrow \, \text{�I�C���[}
\]
�Ƒ����Ă����B
�‚��łȂ���A�N��I�ɂ��̑O�́A
\[
\text{�K�����C}\, \Longrightarrow \, \text{�z�C�w���X}\, \Longrightarrow \, \text{�j���[�g��}
\]
��\footnote{
�K�����C��1640�N�v�œ��N�Ƀj���[�g�������܂�A�z�C�w���X��1629�N���܂�ŔN��I�ɗ��҂̒��Ԃł���B}�A
�j���[�g���ƃ��C�v�j�b�c�͂قړ�����ł���B
���̂��Ƃ�����A�z�C�w���X�ƃ��C�v�j�b�c�̊w�����x���k�[�C�Ƃ̊w��̔��W�̉��n��^�������Ƃ͏\���ɗ����ł���ł��낤�B
�w�����p�x�̑�I���̓z�C�w���X�̌��ʂ���ɂ��Ă���B

���������āA���n���E�x���k�[�C�͔��ϕ��w�̔��W��\kenten{�����}�Ƃ��āA
�j���[�g���A�z�C�w���X�ɑ�����O�̐l�ƂȂ����l�ł���B
�Z�Əd�Ȃ镔���͑������A�ƐтƂ���
\begin{enumerate}
\item �����̋t�Ƃ��Ắu�ϕ��v�̑n��
\item �����p�����Ȑ��_�̊􉽊w�i1.�̉��p�j
\item �ϕ��@�̔��z�i1.�A2.�̉��p�j�ƍŏ��ő剻
\item �����_
\item ���ϖ@�̃��X�g�i1.�A2.�̔��W�j
\end{enumerate}
�͔��ϕ��w�̒��S�����ł���A����p��Ƃ��Ă�
1.�͂�����u�����ϕ��w�̊�{�藝�v�A
2.�́u�����������v�ŁA���̗p��͓����͌����炸�A�܂��e�����f�Ƃ��킸�e�ϕ�����f�i���ϖ@�j�ƌ������B
3.�͈�ʉ�@�̓I�C���[�A���O�����W���Ń��[�y���e���C�̔��z����͊w�Ɏ�����u�������A
�x���k�[�C�͂��̑O��i�K�Ƃ��Ă����ɂ����
�����‚��̋Ȑ��i�֐��ł����邪�j���􉽊w�Ƃ��Ē�߂�͌^��������4.�́A
�t���I�戵���̌���ƈقȂ�A���̌���ɂ����Ă͓W�J�̃c�[���Ƃ��ďd�p�A�e���p�f���ꂽ�B
��ʓ񍀒藝�i�j���[�g���j���͂��߁A���X�̋����W�J��a�̌��������_�̎����S�����B
�x���k�[�C�ɂ������‚��̋����W�J�ɂ���ϕ�������B
5.�͍����́u������������̉�@�v�Ƃ���������̃��X�g�ɂȂ��Ă���B
���̂悤�ɁA���n���E�x���k�[�C���͌^�������A
�I�C���[�͂����̌n�����ēK�p�͈͂��g�債�c�����s�ɂ��̓V�˓I�\�͂𔭊������̂ł���B

�O�シ�邪�A���n���E�x���k�[�C��1667�N�o�[�[���ɐ��܂��B
�Z���R�u���13�Ή��ł������B
����������w�̓����w������邪�e���߂��A�Z�ɂ‚��Đ��w���w�ԁB
1694�N������Ԃ��Ȃ��I�����_�̃O���[�j���Q����w�ɐ��w�̐E�𓾂邪�A�{���]��ł����o�[�[����w�̐E���Z����΂ɏ���Ȃ��������߂Ɖ]����B
1705�N�o�[�[����w�̃M���V����̋����Ƃ��ĕ��C�̋A�r�ɌZ�����j�ŕa�v�Ƃ̕���󂯁A���̌�C�Ƃ��Đ��w�����ƂȂ�B
���łɃO���[�j���Q������ɖ{�i�I�ɓ������������ȂǑ����̗��j�I�Ɛт������Ă������A���̌�����͓I�ɋƐт������鐨���͐����Ȃ������B
���̊Ԏq�_�j�G���A����ŎႫ�I�C���[�ɐ��w���l�w���Ŏd���񂾂��Ƃ͂悭�m���Ă���B

�V�˔��̊w�҂𑽐��y�o�����x���k�[�C�Ƃł��邪�A
�Ƃ�킯���̃��n���̍U���I�Ƃ�����������S�Ǝ��M�Ƃ˂��݁i�����ƐS�j�͓��M�����B
�ŏ��͉����ȌZ���R�u�A��N���q�̃_�j�G���Ɍ�����ꂽ�B
�Ƃ͂����A�ϐl�A��l�������Ȃ��Ƃ���Ă��鐔�w�҂̒��ł͋��e�͈͓��Ǝv���A
���ǂ͋����‚‹��͂̌��ʂƂȂ������ƂŁA��Ƃ̂��̈̑�ȋƐт����j�Ɏc�邱�ƂɂȂ����̂ł��낤�B

\section{$dx, \, dy$�̓~�X�e���A�X}

���Z�̐��w�̃N���X�E���[���ł́A$\Delta x, \Delta y$�͏����ȍ��i�ω��ʁj�Ƃ��ċ������Ă���B
�������Ȃ���A$dx, \, dy$���̂͋����Ȃ����A�������Ȃ��B
����$dx/dy$�Ƃ��ϕ��L���̒��ɕ����Ƃ��ďo�Ă���̂𓪂���F�߂�A�Ƃ�����ɂȂ��Ă���B
���͂���ł����̂ł���B

$dx$�i$dy$���j�́e�����ɏ�����\kenten{��}�f�‚܂�
�e$\Delta x \to 0$�f��$\Delta x$��\�킷�A�Ɛ��������B
�������A���炩�ɁA�����ɏ�������Ε����ʂ�$dx \equiv 0$�ƂȂ��ĈӖ����Ȃ��Ȃ��B
�C�M���X�̐_�w�҃o�[�N���[�i���͂��Ƃɍ��K������ᔻ�����B
�����Ƃ��ł���B
�Ƃ���ŁA�ʂł͂Ȃ��e�����ɏ���������\kenten{����}�f�ł����āA
������$dx, \, dy$�����̂悤�ȑ���Ƃ��āA
��$dy/dx$���̂́i���x�́j���̂����‚��Ƃ�����Ƃ���΁A���̔ᔻ�͓��ʉ������A���ꂪ�����ł���B

�ł́A�P����$dx, \, dy$�Ə����Ă͂����Ȃ����Ƃ����^�₪�����B
�I�C���[�̓����́A�������ɖ����̓N�w�I�E�_�w�I��������ł��낤�A���_�͊��R�Ƃ��Ă��āA�u�����ɏ������v�܂܂ł悢�A�Ƃ����B
�I�C���[�Ɠ�����ŃI�C���[�̒n�ʁi�x�������E�A�J�f�~�[�j���Ŏ������_�����x�[���͒m�I���@�͂ɂ�����A
�I�C���[�ƈӌ��𓯂��ɂ����B
���̂悤�ɁA�x���k�[�C�̎��ӂ‚܂胉�C�v�j�b�c����A�Ȃ���̎��ӂł͖������͎󂯓�����Ă���A
���ꂪ�I�C���[�́w��������́x�Ɏ������񂾂̂ł���B

���O�����W���͖�����$dx$���߂��邱�̘_�c�ɂ͒�����ۂ��č����g�������A�u�����v�Ƃ������A
�V�����֐��������ɓ����ꂽ�Ƃ����Ӗ��Łu���֐��v\ruby{derivative}{�f���o�e�B�u}�A
���Ƃ̊֐����u���n�֐��v�Ƃ�񂾁B
�Ȃ��Ȃ����������@�ł��邪�A���̍l�����ɂ��������������̂�19���I��͊w�̃G�[�X�A�R�[�V�[�ł���A]
�e��w�����ꂵ�߂�f$\epsilon$--$\delta$�_�@���͂��߂Ƃ��āA
�֐��_�̐��ʊ֐��A�R�[�V�[�̐ϕ��藝�A�ϕ������Ȃǂ̐��������Ȑ��E���ł�������̂ł���B

\section{\underline{1690�N}�A���悢��u�ϕ��v���o��}

Sed exiis quae in method tangentium exposui, patet este d, 1/2 $xx=xdx$;\, 
ergo contra 1/2 $xx=\int xdx$ (ut enim potestates \& radices in vulgaribus calculis,
sic nobis summae \& differentiae seu $\int$ \& d. reciprocae sunt.\, p.130

���M��A���遄

���C�v�j�b�c�ɂ͍��i���邢�͍����c�c�j����јa�isumma�j�Ƃ����p�ꂵ���Ȃ������B
���̋L�@��d�����A�a�̋L�@�es�f�͓����̈�����V�ł�f�ƌ��������‚��Ȃ�����\footnote{
������񃉃e���ꂾ����Ƃ����̂łȂ��p��ł������ł���A���ꂳ�����Ӑ[�����ǂ���΁A18���I�p�ꕶ���ł��ǂ߂�B
������̓x�C�Y�̕����ł���B}�B
�����ňӖ��͂��̂܂܂Łes�f�͏㉺�Ɋg�債�āe$\int $�f�Ƃ���A���ꂪ���݂܂Ŏg���Ă���B
�Ȃ��Z�ʂނ���$\int $�̑ւ��$\sum$���g��ꂽ�B
�����s�̑Ή�

\noindent
\Fig[����81���}]{\textwidth}{5\baselineskip}

���āA�{�_�ɍs���ƁA�a�͋Ɍ��ł͂������e�ϕ��f�ƂȂ�̂����A����͒P�Ȃ�a�ł͂Ȃ�����A
���n���E�x���k�[�C�͂��̘a�Ɂu�ϕ��v�Ƃ��������p��ivocabulum integralis�j��^�����̂ł���\footnote{
�����ɗp����ꂽ�̂�1690�N�B}�B
�ϕ��͎��̂Ƃ��ĖʐρA�̐ρA�c�c���Ӗ����邪�A�����̎��̂Ȃ�΃A���L���f�X���n�߂Ƃ��āA
�J�����G���A�t�F���}�[�ȂǁA�����Â��v�Z�@�̓`�����������B
���ϕ��ɂ�����ʐόv�Z�͂����𒴉z���ċ����ׂ����̂ł���B
����́A���ϓI�ɂ͖��֌W�Ǝv������肾��
\underline{�ʐόv�Z�i�ϕ��j�͔����̋t���Z�ł���}�Ƃ������̂ł����āA
���ł́u���H�v�Ƃ������ƂȂ����Z���ł��m���Ă���A�^���}�G�̂��Ƃ���ł���B
�����͐��I�̑唭���ɂ��͂△�����ɂȂ��Ă���̂ł���B

���R�u�E�x���k�[�C�͂���Ƃ��Ȑ��̖����l���Ă����B

\begin{quote}
---���̂�����Ȑ�$y(x)$�i$x$�̊֐��j�ɉ����Ċ��藎���čs���Ƃ��A
���̉��������̑��x$dy/dt$���ǂ��ł��^����ꂽ���x$-b$�ɂȂ�悤�ɋȐ����߂�i���C�v�j�b�c�j---
\end{quote}

�͊w�̋����鏊����---�������K�����C�ɂ���Ċm���߂��Ă����悤��---
$v=\sqrt{-2gy}$�ł��邩��A$v^2$��$x, \, y$�����ɕ�������
\[
 \left( \dfrac{dx}{dt} \right)^2 
+\left( \dfrac{dy}{dt} \right)^2 =-2gy %\label{82}
\]
�����$(dy/dt)^2=b^2$�Ŋ�����$dt$�������A$dx/dy$���t���ɂƂ��
\[
\dfrac{dy}{dx}
=\dfrac{-1}{\sqrt{-1-2gy/b^2}} %\label{83}
\]
�𓾂�B
������񂱂�͌��㕗�ŁA�����x���k�[�C�͂ނ���
\[
dx=-\sqrt{-1-\dfrac{2gy}{b^2}} \, dy %\label{84}
\]
�ƌ��Ă����B
�����$dx, \, dy$�����ɂ���

\begin{quote}
����́A�����́i$x$�����$y$�́j\kenten{�ϕ�}�͓����� \\
Ergo \& horum integralia aequantur
\end{quote}
�Ƃ���\footnote{
$\text{ergo}=\text{����䂦}, \, \text{horum(hoc)}=\text{������}, \, \text{aequantur(aequo)}=\text{������}$�Ƃ����}�B
���ꂪ���j��͂��߂āu�ϕ��v���p��Ƃ��Ă��–{���̈Ӗ��ɂ‚��Ďg��ꂽ�ŏ���1690�N�̂��Ƃł���B
�u�ϕ��́`�v�Ƃ����Ƃ��낪�{���I�ŁA���㕗�ɂ́u���́`�v�Ƃ����̂��낤���A
�葱�����ꂽ�u�����������v�͖���������������ł���B
���̏q������R�u�A���n���E�x���k�[�C�̕����Ɍ��o���͓̂�����A�_�j�G���̒��ɂ͂͂�����Ƃ������B

���߂�$y$�͂ނ���t�֐��Ƃ���
\[
x=\dfrac{b^2}{3g} \left( -1-\dfrac{2gy}{b^2} \right)^{3/2} %\label{85}
\]
�ł���A���C�v�j�b�c�ɂ���āe3/2��̕������f�Ɩ��t�����Ă���B
�������āA1690�N�������ϕ��w�ɂ�����u�ϕ��v�̌��N�ƍl���邱�Ƃ��ł��悤�B

\section{�Ȑ��A���ƂɌ����ʂɂ�����u�ȗ��v�ɂ‚���}

�����A�S���⍂�����H�͎s�X�n���ʂ��đ����Ă���A��s�s���������H�ł͎��X�ƖڑO�ɔ���J�[�u�̘A���̏����ɑ����ŁA
�ߍx���痈��l�ł����Ă��ƂĂ��s��ē��W���Ȃnj���q�}�͂Ȃ��Ƃ����B
���ɓ����̓s�S����̓J�[�u���炯�łقƂ�ǒ����������Ȃ��B
���ł����l�ł��낤�B
�s�v�c�ɁA2�_�Ԃ̍ŒZ�����ł���\{ ���� \}�Ƃ����̂́A�������H�ł͒������̂ł���B

\noindent
\Fig[����86���ʐ^]{\textwidth}{10\baselineskip}

���āA�Ȑ��̖{�i�I�����͋ߑ�ɓ����Ă���̂��̂ŁA�����͉~���Ȑ��i�ȉ~�A�o�Ȑ��A�������j����A
�f�J���g�ɂ悭�􉽊w�Ɣ��ϕ��̔��W�ɂ��A
�T�C�N���C�h�A�J�e�J���[�i�������j�A�点��i�X�p�C�����j�ȂǂɌ������y�񂾁B
���R�u�A���n���E�x���k�[�C�Z��͂��܂��܂ȋȐ��̌���������B
���ƂɃ��R�u�E�x���k�[�C�͑����̉ۑ�̒񎦎҂ł���A�탈�n���́u�Z�v�ɋ��͂����苣������o����������ŁA
�Z���ߏ�Ɉӎ����Ă����B

�x���k�[�C�Z��̋Ȑ��_���q�ׂ�O��
1. ����A
2. �ȗ��~�A�ȗ����S�A�ȗ����a�A�ȗ��A
3. �k���A�L�J�����q�ׂĂ������B

{\gt ��ȗ���} �R�����^�]���Ă��āu�}�J�[�u�A�X�s�[�h�����v�Ƃ��u���̃J�[�u��$R=\cdots $�v�Ȃǂ̉^�]�҂ւ̒��ӁA
�x���̕W�������邱�Ƃ����邾�낤�B
�ǂ̂悤�ȋȐ����ׂ�������Ή~�ŋߎ�����邩��~�̔��a$R$�ŃJ�[�u�̂��‚���\�킷���Ƃ��ł���B
$R$���傫���قǃJ�[�u�͊ɂ��A�������قǂ��‚��B

�Ԃ��^�]���Ȃ��Ă��g�߂ȃP�[�X������B
�����R����ŃJ�[�u�̔��a���������
\begin{align*}
\text{���E�i��Ԃ̃J�[�u}   R&= \\
\text{�����E�L�y���̃J�[�u��} R&= 
\end{align*}
�������d�ԉ^�]�m�͂���$R$�ɂ�鑬�x�K����������Ǝ��Ȃ����
�E���]���̊댯��傫�����邱�Ƃ͂����܂ł��Ȃ��B
�S���t�@���Ȃ�m���Ă��邪�A�S�����[���͂��ׂĂ̒n�_�ŁA�^�s�̂��߂̗����W�A���z�W�A
�ȗ����a�i$R$�j�A�o�[�j�A�i�X�΁j�Ȃǂ̃f�[�^���̕\��������B

\noindent
\Fig[�ʐ^2��]{\textwidth}{10\baselineskip}

�R�z�V��������r�I�J�[�u�������A����$R=\cdots $�ł���Ƃ����B
�V�����̏ꍇ�A�����ł��邽�߂̃J�[�u��$R$��傫�����Ȃ���Ί댯�ł��邪�A
$R$��傫�����邱�Ƃ́A���n�v��ɑ傫�Ȑ��񂪂����邱�ƂɂȂ茚�݋Ƃɂ��e�����o��B

���̂悤�ɂӂ‚��A�Ȑ��̋ȗ��͂����2���܂Őڂ���~�̔��a�i�ȗ����a�j$R$�̋t��$1/R$�Œ�`����Ă���B
���Ƃ��΁A���_�𒆐S�Ƃ��锼�a$R$�̉~
\[
f(x)=\sqrt{R^2-x^2}
\]
��$x=0$��2���܂œW�J����ƁA�ߎ��I�ɕ�����
\[
f(x)=R-x^2/2R+\cdots 
\]
�ŁA��������
\[
f''(0)=-1/R
\]
�ƂȂ�A���̂��Ƃ���Ȑ��̒��_�ł̋ȗ��͂���2�K���֐�$f''$�ɂȂ�Ƃ̑z�����‚��B
�܂��Ȑ��̉��ʂ̗l�q��2�K���֐��ɂȂ�Ƌ������Ă���B
����͐��m�Ȓ�`�ł͂Ȃ��B
���ہA�����Ȃ�ƁA���a$R$�̉~��$x=0$�ȊO�ł͋ȗ����K������$1/R$�ł͂Ȃ��Ȃ�B
�Ȑ��̋ȗ��̒�`�͂���������ʓI�Ȃ��̂ł���A�����̒�`�ɂ͈�K���֐���������B

�Ȑ�$f(x)$��$x=c$�ł̋ȗ��Ƃ́A���̓_�ł̐ڐ��̌X����$x$���̐������ƂȂ��p�x$\theta$�Ƃ��āA
����$\theta$�̋Ȑ��̌ʒ�$s$�ɑ΂���u�ԓI�ω����i�u�ԓI��]���x�j
\[ d\theta/ds \]
�������B

�‚܂�A���̋Ȑ��ɉ����Đi�ނƂǂꂾ����������邩��\���B
����́A$\theta=\mathrm{arc} \tan f'(x)$�Ƃ�����
\[
d\theta/d��=d\theta/ds \cdot ds/dx
\]
����A
\[
\text{�i��������H�j}
\]
�����
\[
\text{�i��������H�j}
\]
����e�Ղɋ��߂��
\[
\text{�i��������H�j}
\]
�𓾂�B
���������āA�ɑ�_�A�ɏ��_�ł́����ƂȂ�B

\noindent
{\gt ��}\, �~�ł�
\[
\text{�i��������H�j}
\]
�ƂȂ��āA�������ɂ�����Ƃ���ȗ���$1/R$�ł���B

���̂��Ƃ���A�ȗ����a��
\[
\text{�i��������H�j}
\]
�܂��A�ȗ����S�͓_$(x_0, \, y_0)$�̖@����ł��̓_���狗�����������������_
\[
\text{�i��������H�j}
\]
�ł���B
���̋ȗ����S�𒆐S�Ƃ��ȗ����a�ʼn~��`���ƁA���̉~����$(x_0, \, y_0)$�ɂ����Ă��̋Ȑ��ɐڂ���B

�܂��A���΂��΂��邱�Ƃ����A�Ȑ�������ꎟ���p�����[�^�ŁA
\[
\text{�i��������H�j}
\]
�̂悤�ɕ\������Ă���ꍇ�́A�ȗ��A�ȗ����a��
\[
\text{�i��������H�j}
\]
�ƂȂ�B

{\gt ��k����} �Ȑ���̓_�ɑ΂���ȗ����S�́A���̌����Ȃ��_�𒆐S�ɁA
���̋Ȑ������񂾂�Ɓu�������܂�āv��荞��ł����悤�ȓ_�ŁA�Ȑ���œ_���ړ�����Ƌȗ����S���ړ������̋O�Ղ͈�‚̋Ȑ�����邪�A
���̋Ȑ����u�k���v�Ƃ�ԁB

�k���ɂ͂�����ʂ�̃X�g���[�g�ȋ��ߕ�������B
�����
\[
\text{�i��������H�j}
\]
�ƂȂ�B

\noindent
{\gt ��}\, �t�ɁA�k�����猳�̋Ȑ��������ł���킯�ŁA���̋Ȑ����k���ɂ������āu�L�J���v�Ƃ����B

{\gt ������} ���ϓI�ɂ́A�Ȑ����i�Ȑ��̏W�܂�j�������Ă���̈�̋��E�ɂȂ��Ă���Ȑ��ŁA
���̃C���[�W����e��݊܂ސ��f�ł��邪�A���m�ɂ͂��̋Ȑ����̋Ȑ����ׂĂɐڂ���Ȑ��ł���B
���Ƃ��΁A���_O�����苗��$d$�ɂ��钼���Q

\begin{align*}
&\ell_\alpha: x \cos \alpha+y \sin \alpha=d \\
&\text{�i}\alpha \text{�̓p�����[�^�ŁA���_����̐����̕����p�j} %\label{87}
\end{align*}
�̕���͂�����񌴓_�𒆐S�Ƃ��锼�a$d$�̉~$\mathrm{C}: \, x^2+y^2=d^2$�ł���B
�����$\ell_\alpha$�̎�
\begin{equation}
x \cos \alpha+y \sin \alpha -d=0 %\label{88}
\end{equation}
������$\alpha$�ł̔���
\begin{equation}
x (-\sin \alpha) +y \cos \alpha =0 %\label{89}
\end{equation}
�ƘA������$\alpha$����������Γ�����B
(1)��$d$���ڍ���A�����𕽕����ĉ������$\mathrm{C}$�𓾂�B

��ʂɋȐ���$\mathrm{C}_\alpha$�i$\alpha$�̓p�����[�^�j�̕����$\mathrm{C}_\alpha$�̎���
����$\alpha$�ł̔���$\text{�i�Δ����j}=\mathrm{O}$����$\alpha$���������ē�����B
���Ƃ���

---���������ʋ��ł���悤�ȏ�ɊJ���������i$\text{���a}=1$�j�ɁA
�㕔���畽�s�Ɉꕽ�ʓ��œ��˂�������Q�́A���ˌ�̌����Q�̕��---

���S����$a\, (0 \leqq a <1)$�������ꂽ���ˌ��̔��ˌ��̕�������
\begin{align*}
\ell_\alpha: 
&y=-b+\dfrac{1}{2}\left( \dfrac{b}{9}-\dfrac{a}{b} \right) (x-a), \\
&b\text{�͔��˓_�܂ł̐������B�����i}a=0 \text{�Ȃ�}b=1, \\
&a=1/2 \text{�Ȃ�}b=1/2, \, a \to 1 \text{�Ȃ�}b \to 0 \, \text{etc.�j} %\label{90}
\end{align*}
�����
\[
\partial y/\partial \alpha =0 %\label{91}
\]
��A�����āA$a$��������
\[
y=-\left( x^{2/3}+\dfrac{1}{2} \right) 
  \sqrt{1-x^{2/3}}, \qquad -1 \leqq x \leqq 1 %\label{92}
\]
��������i���n���E�x���k�[�C�C1691,2�j�B

---��C���C�e��������p�x�ŏ���$v_0=1$�őł��グ��Ƃ��̂��ׂĂ̒e���������̕��---

��C�̌��z��$a$�Ƃ���ƁA�e���̕�������
\[
\mathrm{C}_a: 
y=ax-\dfrac{1+a^2}{2} x^2 %\label{93}
\]
���̎���$a$�ŕΔ�������O�Ƃ����A�㎮�ƘA������$a$����������΁A���
\[
y=(1-x^2)/2
\]
�𓾂�i���n���E�x���k�[�C�j�B
�C�e�͋󒆂ł̗������Ƃ��Ĉ�ʉ�����邩��ԉ΂��l���邱�Ƃ��ł��悤�B

---����$x\text{�ؕ�}+y\text{�ؕ�}$���萔13�ɓ����������Q�̕��---

�����Q
\[
\ell_a: 
y=\dfrac{a-13}{a}(x-a)
\]
�̕�������$a$�Ŕ�������$\mathrm{O}$�Ƃ����A$\ell_\alpha$�ƘA������$a$�����������
\[
(y-x-13)^2=52x
\]
�ƂȂ�B
��]�������W
\[
u=(x+y)/\sqrt{2}, \quad 
v=(x-y)/\sqrt{2}
\]
�𓱓�����B
$x, \, y$�ɂ‚��ĉ����������΁A
\[
2v^2=26\sqrt{2}u-169
\]
�ŁA$u$���𒆐S���Ƃ���������ł���B

����͈�ʂɔ��`�̈�ۂ�^���A�ԉ΁A�����A�G��̍\�}�Ȃǂɑ������o�����B

{\gt ��ȗ��~�A�ȗ����S�A�ȗ����a�A�ȗ���}

{\gt �������d�������\�聙����}

\noindent
\Fig[������95�|105��������]{\textwidth}{10\baselineskip}

{\gt ��k���A�L�J����}

{\gt �������d�������\�聙����}

��ʂɂǂ̂悤�ȓ��H���邢�͓S���̃J�[�u�ł��P���łȂ��A���܂��܂ȋȗ��i�ȗ����a�j�̃J�[�u���ω����Ȃ���A�Ȃ��Ă���B
���R�̂��ƂȂ���h���C�o�[���邢�͉^�]�m�͂��̕ω��ɍ��킹�ĉ^�]���Ȃ��Ă͂Ȃ�Ȃ��B
�Ȑ��̊􉽊w�Ƃ��Č����ƁA�ȗ��~�Ƃ��̋ȗ����S�����X�ƈړ����邪�A
�ȗ����S�̓������O�Ղ��u�k���v(evolute)�Ƃ����B

�J�[�u����E�o���Ă����Ƃ��A�ȗ����a�͑傫���Ȃ�ȗ����S�����X�ɊO��ēW�J���čs������A
�p��ł́e�O�֓]����f�ievolute�̌ꊴ�j�ƂȂ�B
����ɑ΂��A���{���̓J�[�u�֐i������C���[�W�ŁA
�ȗ����a�͏������ȗ����S���Ȑ����̂ɋߊ��‚“����͓݂��Ȃ��āe�k�f�̌ꂪ�ӂ��킵���Ȃ�B

{\gt �������ȍ~�A�C���\�聙���� �i�ȉ~�ϕ����j}

\section{�x���k�[�C�Z��̋Ȑ��_}

�x���k�[�C�Z��́A���C�A�Η����‚A�S�̓I���ʂƂ��āA�Ȑ��̌����ɂ����āA�����ϕ��w�̊�b�T�O�̊m���A
�����w�̒���Ƃ��̉��p�A�ő�ŏ����A�ϕ��@�̗��O�̔��i�Ȃǐ��X�̋Ɛт��㐢�Ɏc�����B
���̂����‚������r���[���Ă݂悤�B

{\gt �ᓙ���~���Ȑ� iso chrone, 1690��}

$\text{iso}=\text{������}, \, \text{chrone}=\text{����}$�ł��邪�A
�e���ԁf�𑬂��ɓ]�`���Ă���B
���̕��͂͂��łɉ�������B

{\gt �ጜ���� catenary, 1691��}

$\text{catena}=\text{��}$�Avel funicularis�ŁA�����ɂ�linea catenaria(e) �ނ艺����ꂽ�����邢�̓��[�v�̌`�������B
�����K�����C�͂���͕������Ǝ咣�������A�z�C�w���X�͂���͌��ƌ������A�����Z���X�_���͌����ؖ����Ă���B
\[
\text{����106��������B������} %\label{106}
\]
���ꂩ��
\[
c \int \dfrac{dp}{\sqrt{1+p^2}}=\int dx %\label{106'}
\]
���ӂ�ϕ�����Ɓi�ĂсA���ӂ̐ϕ��͓������̂Łj
\[
\mathrm{arc}\sinh (p)=\dfrac{x-x_0}{c} %\label{107}
\]
�������A$\mathrm{arc} \sinh( \, \cdot \, )$�͑o�Ȑ����֐�
\[
\sinh u=\dfrac{e^u-e^{-u}}{2} %\label{108}
\]
�̋t�֐��ł���B
����������
\[
p=\sinh \left( \dfrac{x-x_0}{c} \right), %\label{109}
\]
�ŁA$p=dy/dx$�����猋��
\[
y=c \cdot \cosh \left( \dfrac{x-x_0}{c} \right)+y_0 \quad (y_0: \text{�ϕ��萔}), %\label{110}
\]
�ƂȂ�B�i���C�v�j�b�c�A���n���E�x���k�[�C,  1691�j

�Ȃ�
\[
\cosh (u)=\dfrac{e^u+e^{-u}}{2}=1+u^2+\cdots  %\label{111}
\]
�ł��邩��A�������͍ʼn��_���ӂł͕������ƍ�������B

{\gt ���\underline{�Z}�~���� Brachistochrone, 1696��}

����ɂ‚��Ă͊��ɉ�������B
�Ăы������Ă����ƁA�ۑ��$\text{�}=\text{����}$�Ƃ��čœK�����Ă���A�ۑ�Ƃ��Ă͓���ł͂��邪�A
���ꂪ�ϕ��@�̔��z�̗��j�I�N�_�ł��邱�Ƃ𒍈ӂ��Ă������B

{\gt �ᓙ����� isoperimetry��}

�j��n�߂Ă̖{�i�I�ȕϕ��w�̉ۑ�ł��邪�A
���{�ł͌ÓT�I�ȓ�_���v�w�ϕ��w�x���邢��Courant-Hilbert�ɂ̓x���k�[�C�ւ̌��y�͂Ȃ��B
��‚̕����ɂ��΃��R�u���ۑ��񎦂��A���n�������������̂������ł����āA����͋��Ŋ�g�w���w���T�x(��i���g��)�ɂ����f����Ă���B
��ʂɖ{���T�́u�x���k�C�Ɓv�̍��͔��ϕ��w�̐����ւ̈�Ԃ̍v���Ȃǖڔz��悭�q�ό����ɂ����Ă���B

������u���ꓙ�����v�Ƃ�
\begin{quote}
---���ʏ�ɂ����ė^����ꂽ����($L$)�̕‹Ȑ��ł��̈͂ޖʐ�($F$)���ő�ɂ���---
\end{quote}
�ł���A����ƖړI�֐�����ʉ�����΁u��ʓ������v�ƂȂ�B

�‹Ȑ��̕�������
\begin{align*}
&x=x(t), \, 
 y=y(t), \, 
 0 \leqq t \leqq 1, \\
&x(0)=x(1), \, 
 y(0)=y(1)
\end{align*}
�Ƃ���΁A$d/dt$��${}'$�Ƃ���
\[
L=\int_0^1 \sqrt{{x'}^2+{y'}^2} \, dt =\text{���} %\label{112}
\]
�̏�������
\[
F=\int_0^1 (xy'-x'y)dt /2 %\label{113}
\]
���ő剻������ƂȂ�B
����ɂ‚��ẮA�L���́u�����s�����v
\[
L^2 \geqq 4\pi F
\]
���m���Ă��邪�A���ϓI�ȉ��͂������~�ŁA���a$r$�Ƃ��ė��ӂƂ�$4\pi^2 r^2$�ɓ������B

�����œ���������ʓI�ɉ������Ƃ̓��x�����������Ƃ��瑼���ɏ��邪�A�I�C���[����ɂ̓��O�����W���ɂ���ʉ�@������\footnote{
�M�҂͑O�f��_�������Ă��邪���ɐ�łł���B�Ȃ�Courant-Hilbert��Hurwitz�ɂ�������̓I�ɉ�����Ă���B}�B

�ϕ��@�͓����Ȑ��̊֐������肷��􉽊w�̖�肩��X�^�[�g���A�I�C���[�A���O�����W���̌������o�āA
�n�~���g���A���R�r�ɂ���ĉ�͗͊w�̒��S�I���@�Ƃ��Ċm�������B
�������A�H�w�A�o�ϊw�ɂ����鐔���I���@�̓����ɂ��A�œK�֐��̌���̗L�͂ȕ��@�Ƃ��Ăӂ����яd�v�Ȗ������ʂ��Ă���B

�ϕ����͑����̋ɒl���ɒ��ړI�ɓK�p�����B�����ł��̂����‚����Љ�悤�i�����j�B
\begin{enumerate}
\item ���n��
\[
\text{�i���E���j}
\]
\item ���H
\[
\text{�i���j}
\]
\item �ő��~����
\[
\text{�i���j}
\]
\item �ŏ��\�ʐς̉�]�Ȗ�
\[
\text{�i���j}
\]
\item �ŏ��ȃA�t�B���I�����̋Ȑ�
\[
\text{�i���j}
\]
\item �ŏ���p�̌���
\[
\text{�i���j}
\]
\item ���ꓙ�����
\[
\text{�i���j}
\]
\item ���̕��t
\[
\text{�i���j}
\]
\item �f�B���N�����
\[
\text{�i���j}
\]
\item �ɏ��ǖʁi�v���g�[���j
\[
\text{�i���j}
\]
\end{enumerate}

\section{���߂��炵���֐��̐ϕ���������}

�񍀒藝
\[
(1+x)^n=\sum_{k=0}^n \binom{n}{k} x^k %\label{114}
\]
�ׂ̂��w��$n$�͂�����񐳐��������A���ꂪ���������邢�͔񐮐��ƂȂ���
\[
(1+x)^{-2}, \quad (1+x)^{2/3}, \, \text{etc.} %\label{115}
\]
�́u��ʓ񍀒藝�v�Ƃ����A����$\alpha$�ɑ΂�
\[
(1+x)^\alpha=\text{dummy} %\label{116}
\]
�Ƃ��Ēm���Ă���B

$\alpha=n$�i�������j�łȂ���΁A����͖��������ƂȂ邪�A�R���s���[�^�̂Ȃ�����ɂ��̖���������
�L���v�Z�@�Ƃ��ďd�󂳂ꂽ���Ƃ͂������A
���̖����������̂��������W���ł��������ϕ��̒��ŏd�����_�I�������ʂ��Ă������Ƃ͑z���ȏ�ł���B
���ہA�j���[�g������ʓ񍀒藝��
\[
\sqrt{1-x^2}=(1-x^2)^{1/2} %\label{117}
\]
�̐ϕ��v�Z�ɗp�������Ƃ͂悭�m���Ă���B
�Ƃ͂����A��ʓ񍀒藝�����L���A�e�C���[���邢�̓}�N���[�����W�J�̈��ł��邩��A
��ʂ̖��������ւ̗��_�j�[�Y�����������Ƃ����ׂ���������Ȃ��B

���̂悤�Ȓ��Ńx���k�[�C�̐l�X�����ɖ��������_�œ��L���ׂ����ʂ𑽂��������Ƃ��������͂Ȃ����A
����ł������‚��ʔ������ʂ𓾂Ă���B
���̈�‚�
\[ y=x^x \]
�Ƃ������Ȃ�߂��炵���֐��̐ϕ�
\[
S=\int_o^1 x^x \, dx %\label{118}
\]
�ł���B
$x \log x \to 0 \, (x \to 0)$����$0^0 \equiv 1$�ƋK�񂷂�΁A��ϕ��֐��͘A�����ϕ���ԂŗL�E������A
�L���Ȑϕ��l�����B
�Ƃ���ŁZ�Z�Z�c�Z�Z�Z�ł���A������$e^u$�̓W�J����p���A����$x^n(\log x)^n$�ɑ΂��A
$\log x$��������ɂ��������ϕ����s����
\[
\int_o^1 x^x \, dx
=1-\dfrac{1}{2^2}+\dfrac{1}{3^3}-\dfrac{1}{4^4}+\dfrac{1}{5^5}-\cdots %\label{119}
\]
�Ƃ�������܂��߂��炵�����ʂ𓾂�B
���ہA�E�ӂ͌�㋉���̔�������ɂ���������B�i���n���E�x���k�[�C�C1697�j

\section{�����������������Ƃ�}

���‚č��Z�̐��w�̐搶�Ƙb�����Ƃ��u�����͂�����񂢂����e�����������f�̓J���L�������O�Ń_���Ȃ�ł��v�Ƃ����
�i�]�k�����A$x^3$�̔����͂�����$x^4$�̓_���������ł����j�A
�����Ԃ�J���L�������R�c��̕��X�͖�̂킩��Ȃ���m���Ɓi���́j�Q�������̂ł���B

�����A���͂����l���Ă��Ȃ��B
���j�𒲂ׂĂ݂�Ɣ����������́u�ϕ��v�ɕ�ۂ���i���R�u�E�x���k�[�C�C1690�j�Ƃ藧�Ăāu�����������v���u�����v�Ƃ͂���Ȃ������B
���ׂĂ���͈͂ł́u�����������v�Ƃ����^�[���ɂ��o���Ȃ��B
�����Ȃ�ƁA������u���ϖ@�v�Ƃ����֗��ȗp���reclundant�i�]�v�Ȃ��́j�ƂȂ�̂��낤���B
�����̋��܂ň��ǂ����ꏼ�M�w��͊w�����x�̍����ɂ͕\�ꂸ�A�w��͊T�_�x�ɂ��o�����Ȃ��B

�����Ƃ��u���ϖ@�v�͈ꏼ�ł͖{�����Ɏg���Ă���A
\begin{quote}
�������̕ό`�A�ϐ��ϊ��A�s��ϕ���L���񂭂�Ԃ��ĉ�������
\end{quote}
�Ɩ��m�ɒ�߂��A�܂������u�ϕ��Ƃ���������̈Ӗ��Ɏg���K�������邪�A����͋�ʂ��ׂ��p��ł���v�Ƃ����Ă���B
�����Ȃ�΂��낢��Ƒ�ςł���B

�{���̓x���k�[�C�Ƃ̐��w�I���ւ𒲂ׂ邱�Ƃ��ړI�ł���̂ŁA���T�O�����̂悤�Ɍ���I�Ӗ��ɕ�������͈͂ɓ��݂��ނ��Ƃ͂��Ȃ��B
�������ł����Ă��A�x���k�[�C�Z�킨��у_�j�G���E�x���k�[�C�𒆐S�ɁA
�K�v�ŏ����ŃI�C���[�A���O�����W���܂ł��J�o�[����ɂƂǂ߂���̂Ƃ���B
�i$\Gamma$�֐��A$\zeta$�֐��̓x���k�[�C�����[���ѓ����Ă���̂ŁA���̌���ŐG��邱�ƂƂ����B�j
�����ŁA���̕ӂŃ��R�u�E�x���k�[�C�������悤�Ɂe����䂦�A�����̐ϕ��͑��������f�Ƃ�������ʼn���������������������Љ�čs����
\footnote{�n�C���[�ƃ��@���i�[�w��͋����x�ɏ]���B}�B

{\gt ��ϐ������`��} \quad
$y'=f(x) \cdot g(y)$

����͂悭�m����悤�ɁA���e�x���k�[�C�I�Ɂf
\[
\int \dfrac{dy}{g(y)}=\int f(x)dx +C 
\]
�Ƃ��ĉ�����B

{\gt ����`����������} \quad 
$y'=f(x) \cdot y$

��L��$g(y)=y$�̃P�[�X�ł����āA
\[
y=C \cdot \exp \left( \int f(x) dx \right) %\label{121}
\]

{\gt ����`�����������} \quad 
$y'=f(x)y+g(x)$

��L�ɔ�Ď���$g(x)$�����W�Ƃ��ĕt�����`�őΉ��Ď��������̉�$u(x)$���ό`����$u(x)\cdot v(x)$�ƂȂ�Ɨ\�z���A
�܂�$u(x)$�A�‚���$v(x)$�����߂���̂Ƃ���B
�܂�������
\[
u(x)=\exp \left( \int_0^u f(w) dw \right) %\label{122}
\]
�Ƌ��߂����ƁA�����
\[
y=C \cdot u(x) +u(x)\int_0^x \dfrac{g(w)}{u(w)} dw %\label{123}
\]
�̂悤�ɑ�2�����t�������ƂȂ�B

{\gt ��x���k�[�C�̔�����������} \quad 
$y'=f(x)y+g(x) \cdot y^n$

����͐��`�Ď��̉�$y$�̑ւ�ɂ���ɉ����đ�������$y^n$���ω��W��$g(x)$�ŕt�����ꂽ���W�`�ł���B
$y$��$y^n$�̓񍀂����Ȃ炱��͉�����̂ŁA���R�u�E�x���k�[�C�̖���t���Ă���B
�O���Ɠ����X�s���b�g��$y=u(x)\cdot v(x)$�Ƃ��āA
\[
u(x)=\exp \left( \int_0^x f(w) dw \right) %\label{124}
\]
�͑O�Ɠ����A�����p����
\begin{align*}
y=&u(x) \left\{ C \right. \\
  &\left. +(1-n) \int_0^x g(w) \cdot (u(w))^{n-1} dw \right\}^{1/(1-n)} %\label{125}
\end{align*}
�Ƃ��Ċ��S�ɉ�����B

�������A���̌`��$u=y^{1-n}$�𓱓�����$u$�Ɋւ�����`�������ɋA����������@������i�ꏼ�O�f�j

{\gt ��2�K������������} \quad 
$y''=f(x,y,y')$

2�K�������܂ޕ������͐ϕ��ʼn������@�͗�O�I�ɂ����Ȃ��B
�܂��A$f$��$y$���܂܂Ȃ���΁A���炩��
\[ p=y' \]
�Ƃ�����$p'=f(x, \, p)$�ƂȂ��ĊK����1����������B

$y$���܂ޏꍇ�ł�$x$���܂܂Ȃ����
\[ y''=f(y, \, y') \]
�ƂȂ邪�A����$y'=p(y)$�ƂȂ�悤��$p(y)$������΁A�����֐��̔�������$y''=p'p$�B

���������ĕ�������
\[ p'p=f(y,\, p) \]
�𓾂āA����Ŏ���悭$p(y)$�����܂�B
���Ƃ͂��̊֐�$p( \, \cdot \, )$�ɂ‚���\underline{$x$��}����������
\[ y'=p(y) \]
�������΂悢�B

����ɁA$y''=f(y)$�ƂȂ��Ă���ꍇ�͂܂��ʂ̍L���W�J������B

�����������̉��i���΂��ΐϕ��Ƃ���ꂽ���j�͂������R�u�A�����n��������q�ׂ‚������������邪�A
�_�j�G���͂��̔��W�Ƃ��āA���b�J�e�BRiccati\footnote{
���J�b�e�B�Ƃ�������������B}
�ƂЂ�ς�Ɍ�M���Ă���B
�u�����������v�̌�����ʂ���p�����Ă���B
���ہA���̃C�^���A�l���w�҂͓����Z���g�E�y�e���u���O�̃A�J�f�~�[�̉@���E�̃I�b�t�@�[���f�������͓I���w�҂ł������B
�֐S�̈�͎��w��@���_�ɋy�Ԃ��A���w�ł͖��m�֐��Ɋւ���2���̏�����������̈�ʌ`
\[
\text{�i��������H�j}
\]
�Œm����B
���̉���
\[
\text{�i��������H�j}
\]
�ł���B
���b�J�e�B�̓x���k�[�C�A�I�C���[�ɉe����^���A���ʂȃP�[�X�̓x���k�[�C�̔����������ɂȂ�B
�܂��A�x�N�g���E�s��^�ŕ\������ƁA���`�|2���K�E�X�^���䃂�f���̕������ƂȂ�ȂǁA
���݃V�X�e�����䗝�_��t�B���^�����O���_�̊�{���Ƃ��Ă̈Ӌ`������i���[�G���o�[�K�[�A�ؑ��j�B

\section{�U�q�̉^���Ƒȉ~�ϕ�}

�i�z�C�w���X�̔߈��j

�u�U�q�̓������v�̓K�����C���������A����N�ł��m���Ă��郆�j�o�[�T���Ȓm���ł��邪�A�������e�U�����������͈͓��Łf�ł���B
������t�p���ăz�C�w���X�A�j���[�g�������ԑ���ɗp�������Ƃ������͒m���Ă���B
�������A�U�����i��������܂Ȃ�����j�傫����΂ǂ������W�J�ƂȂ邩�B
�Ȃ��Ȃ���������������ꂾ���ɖʔ����ۑ�ł���B

�U�q�͎x�_�𒆐S�Ƃ����̒����ɓ��������a�̉~����i�����Ă��̏�ł̂݁j�^������B
�ʼn��_����̐U��p��$\theta$�Ƃ���ƁA�U�q�̈ʒu�͉~���̒�����$\ell \theta$�ł���B
����A�d�͂̉����x��$g \sin \theta$�ŁA$\theta$����������������ł���B
����ĉ^����������
\[
\ell \dfrac{d^2 \theta}{dt^2}=-g \sin \theta %\label{126}
\]
������$\ell/g=1$�Ƃ������Ƃ͖{���I���܂����ɂȂ�Ȃ��B
�����$\theta$��$y$�Ə�����
\[ y''+\sin y=0 \]
�ƂȂ���2�K������������������B

��ʂɗ͊w�ɂ悭�o������
\[ y''=f(y) \]
�̏ꍇ�A���ӂ�$2y'$���|���Đϕ�����ƈ�K����������
\[
{y'}^2=2F(y)+c, \quad F(y)=\int f(y)dy %\label{127}
\]
�ɒB����B
���̐ϕ��́A�܂��悪����̂ŁA�u���Ԑϕ��v�Ƃ�����B
����𕽕����ɊJ����
\[
y'=\sqrt{2F(y)+C} \quad \text{�i}C \text{�͐ϕ��萔} %\label{128}
\]
�Ƃ��悤�B
������$C$��Y����$\sqrt{}$�ɊJ���Ȃ����Ƃ�����̂Œ��ӂ���B

�������āA���̏ꍇ
\[
\dfrac{dy}{dt}=\sqrt{2\cos y -2\cos A} \qquad \text{�i}A\text{�͍ő�U���j}
\]
�ƂȂ邪�A�����܂ł�$y$����������$t$�̊֐��ƂȂ�Ȃ��B
�����ŕϐ��������āA$t$��$y$�̊֐��i�‚܂�A�֐��j�Ƃ���
\[
t=\int_0^y \dfrac{d\theta}{\sqrt{2\cos \theta -2\cos A}} %\label{130}
\]
�̂悤�ɋ��܂�B
���Ȃ݂ɍő�U���ɑ΂��������
\begin{align*}
T&=4\int_0^A \dfrac{d\theta}{\sqrt{2\cos \theta -2\cos A}} \\
 &=2\int_0^A \dfrac{dy}{\sqrt{\sin^2 \dfrac{A}{2} -\sin^2 \dfrac{y}{2}}} %\label{131}
\end{align*}
�ŗ^������B
���Ă̂Ƃ���A����$T$�́i�ő�j�U��$A$�Ɉ˂邱�ƂɂȂ邪�A���̊֐��́u�ȉ~�ϕ��v\footnote{
�u�ȉ~�ϕ��v�͑��̂ł����āA�ȉ~�̎����̊֐�����R���������A�����܂Ŗ����̗R�������ł���B
���ׂĂ��ȉ~�̎����ƂȂ�킯�ł͂Ȃ����Ƃɒ��ӂ��悤�B}�Ƃ�����ϕ��̈�‚ł���B

\noindent
\Fig[������131�|135��������]{\textwidth}{5\baselineskip}

�������U���ɂ��܂��܂��ɂȂ�̂͂�͂�܂����B
�Ȃ��Ȃ�A��莞�ԂŌJ��Ԃ���錻�ۂ����ׂČ����I�Ɂu���v�v�Ƃ��邩��ł���B
�z�C�w���X�͉~���^�������܂����̋Ȑ���̉^���ɉ��C���āA���̋Ȑ���̌ʒ�$s$�ɂ‚��āA
$s$�ɂ‚��Ă̒P�U���̕�����
\[ s''+Ks=O \]
�ɋA���ł��Ȃ����l�Ă����B
�����܂ł��Ȃ��A���́e�~�ɋ߂��f�‚܂�e�~���ۂ��f���́A�͂����ĕ����ʂ�T�C�N���C�h�ł������B
������u�T�C�N���C�h�U�q�v�Ƃ����B

���̌��ʂ́A��\underline{�Z}�~������肩��T�C�N���C�h���o�������n���E�x���k�[�C����΂����B
�������A���񂶂�̃T�C�N���C�h�U�q�̐���͂��܂��s�����A�H�w�Ƃ��Ă͎��s�����B
�@�B�̐��삪���܂��s������͂܂��������Ă��Ȃ������̂ł���B

{\gt �U��q�̗��_}
�@�U�����̏ꍇ�@�E�E�E�ς�

�@��ʂ̏ꍇ

{\gt �ȉ~�̎���}

{\gt �����j�X�P�[�g�̎���} ��ʂɂ́u�J�b�V�j�̃����j�X�P�[�g�v
\[
\text{�i��������H�j}
\]
�Ƃ����A���̂����u�x���k�[�C�̃����j�X�P�[�g�v(�A��`�j�̒�`�́������A���Ȃ킿
\[
\text{�i��������H�j}
\]
�ł���B�ɍ��W�ŕ\�������
\[
\text{�i��������H�j}
\]
�ƂȂ�B
���̌ʒ����v�Z���Ă݂悤�B
\[
\text{�i��������H�j}
\]
���Ȃ킿�A�ȉ~�ϕ��ɂȂ�B

����͋t�O�p�֐�
\[
\text{�i��������H�j}
\]
�̗ގ��`�Ƃ��ċ����[���B
�K�E�X�́i�t�j�O�p�֐��ɕ���āA���̐ϕ��i�����j�X�P�[�g�ϕ��j�̋t�֐�����
�����j�X�P�[�g�E�T�C��(sin.\, lemm,\,  $\mathit{sl}$)�����j�X�P�[�g�E�R�T�C��(cos.\ lemm,\ $\mathit{cl}$)���`���A
�t�O�p�֐��̒l$\pi$�ɑΉ�����$\omega$�i�{���́������‚��j���l���Ă����Ƃ����i���ؒ厡�w�ߐ����w�j�k�x�j�B
����烌���j�X�P�[�g�֐��͑����̊֐S�����‚߁A�����̘_�l���o�Ă���B
�����ł�
\[
\text{�i��������H�j}
\]
�������邾���Ƃ��悤

����Ŏ��̋����[���v�񂪂ł����B

\begin{table}[htb]
\begin{tabular}{c|l}
�Ȑ�           & �ʒ� \\ \hline
�񎟋Ȑ�       & \\
�~             & \\
������         & \\
�ȉ~           & \\
�o�Ȑ�         & \\
�T�C�N���C�h   & \\
�J�e�i���[     & \\
�����j�X�P�[�g & \\
\end{tabular}
\end{table}

\noindent
\Fig[����132���@�\�i�ȉ~�ϕ��j]{\textwidth}{8\baselineskip}

\chapter{�_�j�G���E�x���k�[�C}

\section{�_�j�G���N�̓o��}

���R�u�A���n���Ɨ��āA���̓_�j�G���ł���B
���n���̎q�ŁA�x���k�[�C�Ƃ�3�l�ڂɂȂ�B
���n���̎q�ɂ̓j�R���X�i�c���Ɠ����j2�������āA�������R�u��Ars Conjectandi���o�ł����i���ۂ́A�`�����j���A
�_�j�G���ɔ�ׂ�Ɖe��������A���R�u�A���n���A�_�j�G���c�c�͂�������悭�������[���b�p�l�̖��O�����A
�����Ă����ƃo�C�u���i�����j���̖��O�ł���B
���R�u�̓��_�������̑c�Ƃ����A�u���n���̓��ڂł���͈̂��B
�_�j�G���͑����a���ҁA����������񐹏�����ł���B
���n���͌����́e���n�l�f�ŁA���_�������̒��łӂ‚��̒j�q�̖��A���{���ɂ͑��Y�A���Y�Ƃ��������ŁA
�m����ʂ胈�n�l�͐V�񐹏��ɂ͂悭�o�Ă���B

�‚��łȂ���A��ő���������肪����̂����A�u�_�j�G���v�̓w�u����i���_�������̌���A�e���_����f�Ƃ͂���Ȃ��j�ŁA
�e���h�Ȕ����f���Ӗ�����B
���`�Ər���ȋK�����d�񂶂郆�_�������炵���l���ł���B
���������΃V�F�C�N�X�s�A�́w���F�j�X�̏��l�x�ŁA�e�������f---���́A��S�}�J�V�Ȃ̂���---���o���������_�j�G���ŁA
�����V���C���b�N���e���ꂼ�������_�j�G���l�I�f�Ɗ������ċ��ԏ�ʂ�����B
���̂��ƁA��ǂ�ł�Ԃ�������̂����A����͂������낤�B
���̃_�j�G���N�A�_�j�G�����N�������Ŋ�ʂ��傫���������������肵���ӎu�����������̖��ɒp���Ȃ��q�������悤�ł���B
���Ђ�����d���ȑ��݂����������i���R�u�j�A����p�p�i���n���j�̉��ł悭�T�o�C�o�����A
�x���k�[�C�Ƃ�3�Ԏ�̓V�˂Ƃ��č��������c���Ă���B

\noindent
\Fig[����133���ʐ^]{\textwidth}{8\baselineskip}

���i�̌����������n���ɏ������p���Ƃ����p���ɕ��������ɍ���Ȃ��B
���͈�w�����Ƃ�����B
�ƋƂ���폤�����牏���Ȃ��킯�ł͂Ȃ��A�e�F�s���q�Ƃ��Ă���͎����邪�A
���w������Ă悢�Ƃ���������t����B
�����̓\�c���Ȃ������ǂ��B
�x���k�[�C�Ƃ̏ё��������ƁA���ꂼ��̃L�����N�^�[���\��Ă���̂������[���A
�_�j�G���͂ǂ��������Ƃ肵�Ă��邪�A�X�}�[�g��������B
��F�֌W���ǂ������ł��낤�B
�ő�̗F�l�͂����܂ł��Ȃ��A������̃I�C���[�ł���B�i�_�j�G���̕���7�˔N���B�j

\section{�E���ƋƐ�}

�E�������������B
�_�j�G���E�x���k�[�C��1700�N�I�����_�̃o���[�j���Q������A5�΂̂��땃�ƂƂ��Ƀo�[�[���ɖ߂�i1705�N�j�B
�قǂȂ�1707�N���n�ŃI�C���[�������B
��������͐�ɏq�ׂ��ʂ肾���A���V�A�鍑�̃Z���g�E�y�e���u���N��w�̋����ɂȂ�B
�����̉ŃG�J�e���[�i���w��D���ňꗬ�w�҂����͂ɏW�߂Ă����̂ł���
�i�����̐ꐧ�N��ɂ͌[�֓I�ȌN������āu�[�֓I�ꐧ�N��v�ƌ���ꂽ�j�B
���������΁A�I�C���[���������Z���g�E�y�e���u���N��w��������������͒��������邪�A
�_�j�G���͂�����3�N�Ŏ�������B
���̓I�C���[�E�̓_�j�G���̏Љ�ŁA�Љ�҂̕�����Ɏ��߂��̂ł���B
�o�[�[���ɖ߂��Č�o�[�[����w�����ƂȂ邪�A�^�C�g���͈�w�E�N�w�̋����ł���B

�Ɛтɂ��‚낤�B
�x���k�[�C�Ƃ̋ƐтƂ‚������͖̂{���ɗ͋Ƃ��K�v�ł���B
�܂��ɕS��㇗������A�܂����e�S�ԁf�����ׂăt�H���[����킯�ɂ͂����Ȃ��B
�������A���̃x���k�[�C�Ƃ�3�l�ڃ_�j�G�����̋Ɛт𔲂����킯�ɂ͍s���Ȃ��̂́A���̌���I�Ӌ`�����邩��ł���B

\noindent
\Fig[����134���ʐ^hydro�c]{\textwidth}{10\baselineskip}

�_�j�G���̋Ɛт͉��Ƃ����Ă�1.���̗͊w�A������2.���p���_�ł���B
3. �C�̕��q�^���_������B
1.�́u�x���k�[�C�藝�v�Ƃ��č��Z�̕����ɂ��o�ė���݂̂Ȃ炸�A���㕶���Ɍ������Ȃ���s�@�̌����A
2.�͑�w�o�ϊw���̃~�N���̍u�`�ŏ����ɏo��u���p���_�v�̎n�܂�ł���B
1.�A2.�͑S���ʕ���̂��́i���n�ƕ��n�j�����A���ꂪ�����Ƃ���300�N�߂�����̍����Ɏc���Ă���̂�����A
���Q���ׂ����̂ł��낤�B
����͍����̑�w������300�N�߂������ɉ����c���邩�Ƃ�����Ɠ����ł���B

\section{���͊w�A���邢�͗��̗͊w}

�u���͊w�vhydraulics�͒m��l�͑����͂��Ȃ������m��Ȃ����A�d�v�Ȋw��ł���B
�����n���̋������C�n�ł���A�_�j�G���̏o�g�n�I�����_�̐������u�l�[�f�������h�v�iNetherland, �t�����X�� Pays bas�j�̈Ӗ���
���ƕ����ʂ�e��n�n���f(low countries)�ō��y�͂‚˂ɊC���̐Z���̍��ƓI�댯�ɔY��ł����B
�Z�����鐅�̔r�o�A����ɔ����͐�̊Ǘ��͑傫�ȉۑ�ł������B
�ߑ�ɂ����ẮA�s�s�ɂ����鐅�̈��S�A�m���ȋ����V�X�e���́A���m�҂̃K�o�i���X������鎊��ۑ�\footnote{
���{�ł��A����ƍN�������r��͂ĂĂ����]�˂ɓ��邵�Ĉȗ��A�����ɐ������͑�ۑ�ƂȂ�A
����Ȏ��ԁA�J�́A�������₵�Ă���A�ʐ�㐅��ʂ������Ƃ͒m����B}�ł���B
�኱�ׂ������A�X�y�C���̎�s�}�h���b�h�͍��n�ɂ��邪�A�����Z�p�ɂ���ČÂ�������S�ɐ��������Ȃ��ꂽ���Ƃ̓��m�̖{�ɏo�Ă���B

�Ƃ���ŁA���ƌ����Ă��ɂ߂‚��́A���Ȃ�ʃI�C���[�̎��s�ł���B
�I�C���[�͔ނ������]�������v���V�A�̌[�֓I�ꐧ�N��t���[�h���q�剤�i�H�j�ɏ��ւ�����ăx�������E�A�J�f�~�[�ɐV�����E�𓾂邪�A
��͂�f�l�̉ߓx�̊��҂͋��낵���B
�L���ȁu�T���E�X�[�V�{�v�iSanssouci �t�����X��Łu���J�{�v�j�ɒʂ���e���������f�̐���𖽂�����B
�@�B���͂��Ȃ�\footnote{
J. ���b�g�̏��C�@�ւ�1765�N���AR. �g�����B�V�b�N�̖{�i�I�ȏ��C�@�֎Ԃ�1804�N�B}�A
����Ɂu�i�v�@�ցv�̕s�”\�����A�ꕔ�̊w�҂̊Ԃł͂Ƃɂ������A�\���L���͒m���Ă��Ȃ�����ł������B
���̂̌����Ɏ��s���A�I�C���[�̕s�F�ȃx�������ߋ��̉����ƂȂ�i��C�̓��O�����W���j�B
�F�l�������_�j�G���E�x���k�[�C�ɂ́A���͊w�����ނ̍˔\�̌������������낤�B

\noindent
\Fig[�}����135��Sans souci�A�X�e�r��]{\textwidth}{5\baselineskip}

\section{���S���̂̃x���k�[�C�̒藝}

���āA���΂��󒆂ɓ�����ƁA�΂͕�������`���Ĕ�ї�������B
���̕������́u�_�v���������O�ՂŁA�_�́u���_�v�Ƃ���ꂻ���Ŏg���闝�_�́u���_�̗͊w�v�ł���B
���΂͎��ʂ��������_�ƌ��Ȃ���Ă���B
������x�̑傫���͈̔͂ł������A���΂���]���Ă��Ă��΂̏d�S�̉^���͕������ł��邱�Ƃɕς�͂Ȃ��B

�������A���΂��{�[���ł�������A�y�����‘傫�����̂ł���ƁA�K�炸�������m�ɂ͕������ł͂Ȃ��Ȃ�B
��C��R�������ł��Ȃ��Ȃ邩��ł���B
�{�[���ɉ�]�������ƋO���͔����ɕω����邪�A����Ɍ����Ȃ�A����s�@�Ȃ���͂�l���͈�ς���B
�����͂��ׂċ�C�ɂ��͂̉e�����x�z�I�ɂȂ邩��ł����āA�����ɋ�C�Ƃ������̗̂͊w�A
�‚܂�u���̗͊w�v���K�v�Ƃ���闝�R������B
������񗬑̂Ƃ����ǂ������̎��_�̏W���ł���A���͋ɂ߂đ����Ƃ����v�I�ɏW�v����΂悢�Ƃ����l�����A
���Ȃ킿�Ҍ���`������A����͂���Ƃ��ėL���ȗ��_�ł���i�C�̕��q�^���_�j�B
�������A�C�̈�ʂ̃}�N���Ȑ������������Ƃ���Ȃ�A��܂Ƃ܂�
---�Ƃ������A�ꑱ����---
�̑��݂Ƃ��čl���Ȃ��Ă͂Ȃ炸�A���̗͊w�̗�����u�A���̗̂͊w�v�Ƃ����B
���̂͘A���̂̈�‚ł���B

���̂��l����Ƃ��A
1. �e�_�ł̗���̑����ƕ������x�N�g��$u$�Ƃ��čl����u��v�̗���i�u�I�C���[�L�q�v�Ƃ����j��
2. ���̗��q�̍s���Ձi�O�Ձj���t�H���[���čs������i�u���O�����W���L�q�v�Ƃ����j��2�ʂ肪���邪�A
1. �̃I�C���[�L�q�����w�I�ɗe�Ղł���B
�����ŁA����̒��Ɉ�{�̋Ȑ����l���A���̏�̊e�_�ł̑��x�x�N�g��$u$���Ȑ��̐ڐ��ƂȂ�悤�ɂ���Ƃ��A
������u�����v(stream line)�Ƃ����B
����Ă͂��Ă������͎��ԂƂƂ��ɕς�Ȃ��Ƃ�������u��헬�v(stationary flow)�Ƃ����B
��헬�̘A���ʐ^�͂܂�ŐÎ~�摜�̂悤�Ɍ�����B

�}���ق̒I�ɕ���ł���2, 30���̈�܂Ƃ܂�̖{�̒��������̖{�����������ƁA
���C��R�ɂ���Ĉ����������������Ɏ��̖͂{������������Ă��܂����A
���C���Ȃ���΂���͋N��Ȃ��B
���̂̏ꍇ���A�����̑��i���ǁj�����̋��ڂ����R���󂯂錻�ۂ��u�S���vviscosity�‚܂�e�˂΂�f�ł���B
�ǂ̂悤�ȗ��̂ɂ��召�͕ʂƂ��ĔS��������A�����e�T���T���f�Ƃ͂��Ă��邪�A�S���������������Ń[���ł͂Ȃ��B
��C�����ɏ��������S�������������̂ŁA��ʂɗ��̂̉^���͋��E�̖ʂŒ�R���󂯁A�u�S�����́v�Ƃ�����B

\noindent
\Fig[����136���S���f�[�^]{\textwidth}{10\baselineskip}

���̔S�������̂悤�ɉ��x�ƂƂ��ɒቺ���A���T���T���ƂȂ�B
�Ƃ���ŁA�P�ꕨ�̗̂͊w�ł������ł��邪�A���C�͂���Ɨ��_�̎戵�͓���Ȃ�B
���̂̏ꍇ���S�����[���Ƃ������z��Ԃ��u���S���́vperfect fluid�ł���A
�_�j�G���E�x���k�[�C�A�I�C���[���z�肵�����̂����́u���S���́v�ł���B

���S���̂Ȃ爵���Ղ��B
�����̂���_�ɔ��ɏ����ȁi�p�����̂悤�ȁj�̐ς�z�肵�A���̉^�����l����B
���̂�$v$�A���������݂̂Ȃ炸�����ɂ������Ƃ��ė�����$h$�A
�d�͉����x��$g$�A���̂̑̐ϓ���̎��ʁi���x�j$\rho$�A
���̂ɂ͊O�����牽�炩�̈��́i�P�ʖʐϓ���j$p$�����‚Ă���Ƃ��悤�B
�����ɉ����āA�̐ϓ���
\begin{align*}
\text{�^���G�l���M�[}           &=(1/2)\rho v^2, \\
\text{�d�͂ɂ��ʒu�G�l���M�[} &=\rho gh, \\
\text{�O����̈��͂̂����d��}   &=p
\end{align*}
�ł���i�����ɂ́A���ǂ̈ꕔ�̗��[�ł��̂��̂̍����l���A���̘a��0�ɂȂ邱�Ƃ�����j�B
�����ŁA$\text{�d��}=(\text{����} \times \text{�ʐ�}) \times \text{����}=p \times \text{�̐�}=p$�Ƃ��āA
�����̑��a�̃G�l���M�[���ۑ�����邩��A�߂ł���
\begin{align*}
&\text{�x���k�[�C�̒藝} \\
&p+\dfrac{1}{2}\rho v^2+\rho gh=\text{���} %\label{137}
\end{align*}
�𓾂邱�ƂɂȂ����B
����ɗ������Ȃ��Ȃ�$h=0$�ŁA
\[
p+\dfrac{1}{2}\rho v^2=\text{���} %\label{138}
\]
�ł���B

�܂��́A�����ƈ��͂ƃ|�e���V�����ł���킵���I�C���[�̉^����������ϕ�����
\[
\text{�i��������j�i�P�j}
\]
�𓾂邪�A���ꂪ�u�x���k�[�C�̒藝�v�ł���B
�藝�͂��܂��܂ȓ����̌`�ɕ\�����B
\begin{align*}
&\text{�iwiki�j} \\
&\text{�i�Q�j} \\
&\text{�i�R�j}
\end{align*}

�u�x���k�[�C�̒藝�v�͂悭���ۂɂ��Ă͂܂�Ɖ]����B
���Ƃ��΁A��s�@�̗����͂̏�ʁA���ʂł͋�C�̗������قȂ�A
�x���k�[�C�̒藝���爳�͍��������ď�����̗g�͂�������B
�����A�g�͂̌����Ƃ��Ă����܂ł͐��������A�����͂̑��x���z�̗��_�͕ʌ•K�v�ł���A
������u�����̌����v�̌����n�ߌ��݂������̍���������i���ہA10����7���͌��������Ă���Ƃ����w�E������j�B

\noindent
{\gt ��} �v�Z���K�����������B
\[
\text{�i��������j}
\]

���̗͊w�ł���u���͊w�v�͗��̗͊whydrodynamics�̎�v����Ƃ���Ă��邪�A�����ł��̒藝�̉��p�͈͂͋ɂ߂čL���B
����Łu�����w�vhydraulics�Ƃ������������A�����炪���j���Â����قƂ�LjႢ�͂Ȃ��Ƃ����B
���邢�͕���ł̌ď̂̈Ⴂ�ł���Ƃ̐���������A���O�̂��̂ɂ͕�����ɂ����B
�������u���̗͊w�v�͋ߑ�ɂȂ��Ă��琔�w�I�ɂ����x�ɔ��B��������ł���A�������N������hydro-�i���́A�̈ӁB$\text{hydrogen}=\text{���f}$�j�������Ă��Ȃ���A
���k���̂��闬�́i�C�́j�������B
�{���I�ɐ��ʂ��� fluid mechanics�Ƃ������������B

����͐[���肵�Ȃ��Ƃ��āA�L���G�s�\�[�h������B
�_�j�G�����w���͊w�x�i���邢�͍����I�Ɂw���̗͊w�xhydrodynamics�j���o�ł���ƁA
�����ƁA�˂��݂��畃���n�������Ƃ���ɑ΍R���āw�����w�xhydraulics��---�o�œ���k����---�o�ł����B
�W����̂ł͂Ȃ��΍R���ďo���A�Ƃ��������܂��x���k�[�C�Ƃ炵���B

���̗͊w�̕���͐��w�I�Ƀn�[�h�ł���B
���发�͂������ɖ`���͐e���݈Ղ����A�����Ɋw��I�ɓ����Ă��܂��A
�x�N�g����́A�e���\����́A�֐��_�A�M�w�E���v�͊w�A�͊w�n�A�c�c�ƃt�H���[����̂���ςł���B
�����A��ʂ̗͊w���猩��΁A���̗͊w���̂��̂��•ʐ�啪��ł����āA����Ƃ����ǂ����I�ł���̂͂�ނ𓾂Ȃ��B
���Ԃ�������΁A����𐔊w�I�ɒ���ǂ݉����Ă䂯�΋����[�����E�������ɓW�J���邾�낤�B

\section{���������ŃI�C���[��ٌ�}

�I�C���[���t���[�h���q�剤�Ɍ����܂�A�T���E�X�[�V�{�̕�������𖽂���ꂽ���A���܂��s���Ȃ��������Ƃ͏q�ׂ��B
�I�C���[�͎��͂������Ă��Ȃ��A�x�������ɂƂǂ܂�A�e�剤�̐��w�ҁf�Ƃ��Ă̂ق���͎���Ȃ��������炢������A
���̒��x�̂��Ƃ̓x�������ދ��i�Z���g�E�y�e���u���N���A�j�̑傫�ȗ��R�ɂ͂Ȃ�Ȃ����낤�B
�ނ���A�I�C���[�͗ދH�Ɍ���h����ȃv���e�X�^���g�M�‚̕ێ��҂��������A
���������������t�����X�̖��_�_�I�[�֎v�z
---���H���e�[�����g�͖��_�_�ł͂Ȃ�������---
�̐�`�}���̃��H���e�[�����A�J�f�~�[�ɏ��ւ�����Ă��ăI�C���[��_��Ă������ƁA
�I�C���[�����ւ����㏂�ɂȂ��Ă����@�����[�y���e���C�������������ƂȂǂ��󐨂����������B

���̎����͂قƂ�ǒm���Ă��Ȃ��B
�����A�����̈ꌏ�ɂ͒��҂Ƃ��Ă�����肪����B
�����u�i�v�@�ցv�����Ǝw������Ă����̂Ȃ�A���ꂪ�s�”\�Ȃ��Ƃ͓����\���ɂ͒m���Ă��炸�i�����I��Clausius�܂Łj
���Ƃ��Ɩ������ł����������m��Ȃ��B
�ł̓I�C���[�͉ʂ����Ă����m���Ă����̂��낤���B
���{�ł́u���w�҃I�C���[�v���������@�I�ɋ���������Ď�l�̃A�C�h���ƂȂ�A�l�ԃI�C���[�̐l����v�z�E�M�‚ɂ͖��֐S�ł���A
�܂��Ă⎸�s����܂ɂ‚��Ă͒m��R���Ȃ��B
�܂��́A�t���[�h���b�q�剤�̕s���𕷂���

\begin{quote}
�����]�񂾂̂͒�̕����������B
�I�C���[�͒����r�֐���g����ɕK�v�Ȏԗւ̗͂��v�Z���A���͂������琅�H�𗎂��āA�Ō�͐����悭�T���X�[�V�ɗ���o��͂��ł������B
���̐��Ԃ͊􉽊w�I�ɂ͎����������A������������̐����킸��50���ł������r�֗g���邱�Ƃ��ł��Ȃ������B
������̋�\footnote{
���񐹏��w�`���̏��x��1�͖`���ɂ���L���Ȑ���
�u1 �_�r�f�̎q�A�G���T�����̉��ł���`���҂̌��t�B
2 �`���҂͌����A��̋�A��̋�A��؂͋�ł���B
�c�c6 ��݂͂ȁA�C�ɗ������B
�������C�͖����邱�Ƃ��Ȃ��B
��͂��̏o�Ă������ɂ܂��A���čs���B�v���w�������́B
�剤�͂������Ɍ[�֓I�ꐧ�N�傾�������āA�悭�ÓT�␹���ɒʂ��Ă���B
�M�[�������I�C���[�͂�����񂱂̐���ɂ͒ʂ��Ă������낤�B
�I�C���[���􉽊w�҂ƔF������Ă������Ƃ͒��ڂ����B}
�Ȃ邩�ȁI�@�􉽊w�̋�Ȃ邩�ȁI�@Vanity of vanities�I�@Vanity of�@geometry�I�v 
\end{quote}

\noindent
\Fig[�T���X�[�V]{\textwidth}{8\baselineskip}

�t���[�h���q�剤�́A18���I�[�֎v�z�̉e�����󂯎v�z�E�w��E�|�p�ɑ���Ȋ֐S���񂹂����E�j��́u�[�֓I�ꐧ�N��v
---�������ꐧ��E����ɂ͎���Ȃ�������---�̈�l�ł������B
�t���[�h���q�ɂƂ��ăT���X�[�V�ɕ����V�X�e�������‚��Ƃ͒��N�̖��ł���A���O�����߂��A���z�̔�p��������ꂽ�B
���̎��s�ȍ~�z�[�G���c�H���������Ƃ̑�X�̉������̊肢�ɂ�������炸�A�����V�X�e���͖�100�N���̊Ԓ��ق����܂܂ł������B

�I�C���[������𖽂���ꂽ���u�i�H�앨�j�͂ǂ��������̂��A�ǂ̂悤�ȏ����łǂ̂悤�ɐ݌v����A���삳��A
�ŏI�I�ɂ͑��u�͂ǂ��@�\�����i���Ȃ������j�̂��A���̗��R�⌴���͉��ł��������B
�ڂ�������͂킩��Ȃ����A�z������ɓ�‚̘_�_���v�������ԁB

���_�͗��̗͊w�i�����Ƃ��Ă͐��͊w�j�͂ق�̗c�t�i�K�ł������B
�������A�悤�₭�u�I�C���[�̕������v���m���Ă��邪�A���������H�앨�͗��w�I�m�������ł͐���ł����H�w�I�Z���X���K�v�ł���B
�����A�_�j�G���E�x���k�[�C�͂׃��k�[�C�Ƃ����Ă̌�����؊ώ@�h�ŃZ���X����}���Q�ł���i�����w�𐔊w�Ɠ����ȏ�̈ʒu�ɂ����Ă����j�A
�����w�҂Ƃ��ē����ŐV�̃n�[���F�C���̌��t�z�˜_�Ɍ���ɒʂ��Ă������痬�̗͊w�ɂ‚��Ă͈�����������ł��낤�B
�͂����ăI�C���[�ɍH�w�I�Ȋ�p����Z���X���������ۂ��B
���N�E�”N���ォ�狌�F�������I�C���[�ƃx���k�[�C�̃m�E�n�E�̌����͂Ȃ������̂��낤���B
�����ɂ‚��Ă͕s�K�ɂ��Ē��҂̂悭�m�鏊�ł͂Ȃ��B

���_�͂����������{�I�Ȃ��̂ŁA������i��1��́j�u�i�v�@�ցv�̂��Ƃł���B
�u�g����v�Ƃ���̂ŁA�����r�͐������琳�̍��x��\footnote{
���Ȃ݂ɁA�]�˖��{�̖��߂ō]�˂܂Ł����N�J�킳�ꂽ�u�ʐ�㐅�v�͑�����̉H��������������
�����̍]�ˎl�J��،ˁi�������s�V�h��l�J4���ڌ����_�j�܂ł̖�43km�ŁA���n�_�Ԃ̎��R�̗����͂킸��100���ł������B
�H���͒n���̖��������ē�q���ɂ߁A�\�莑���͒���‚��܂��Z�p�I���s�̌��ʂ̎��E�҂܂ŏo���B}
�ɂ������Ƒz�肳���B
�z�肪��������΁A���́i�����w�I�ɂ͊O������̎d���j���K�v�ł��邪�A���C�@�ւ͖������݂����A
�܂����͂Ȃǂ̎��R�͂Ɉ�ؗ���Ȃ������Ƃ���ƁA�����������̑��u���̂��u�i�v�@�ցv�ł����āA���������Ė{�����蓾�Ȃ��B

�I�C���[�M��҂ɂ́A�I�C���[���͂����ĉi�v�@�ւ̕s�”\����m���Ă������ۂ����C�ɂȂ邩������Ȃ��B
�M�҂͂��̊m��I�������������ʂ���ق��Ȃ��B
���E�G�w�����w�j�x�i��8�́j�ɂ��΁A�t�����X�w�m�@�����̖��̉�@�Ə̂�����̂̕s�󗝂����c�����̂�1775�N�ł���A
�i�v�@�ււ̓��B�s�”\��18���I�I��育��ɂ͊m�M�ƂȂ����B
�I�C���[�̌��̖��͐旧�‚��Ɛ��\�N�Ŕ����Ȏ����ł���B
����ɁA���Ƃ�������\�z���ȂĂ��Ă��A����@�ւ��i�v�@�ւł��邩�ǂ����̔��f�͈�ʂɂ͕��Ղł͂Ȃ�
---�����炱���A�w�m�@�ɐ\���̐��x������---���ꎩ�̈ꗝ�_�ɑ���������������i�����Q�Ɓj�B

\noindent
\Fig[���[�X]{\textwidth}{4\baselineskip}

�i�v�@�ւ̕s�”\���ɂ‚��āA�I�����_�̕����w�ҁA���w�ҁA�Z�p�ƂŃ��I�i���h�E�_�E���B���`�ɂ��䂹����S. �X�e���B���i�������j��
�u�X�e���B���̕��v�Œm����u�Ζʏ�̒ނ荇���̖@���v������B
�}�̂悤�ȃ��[�X�i���Ɋ|����ԗւ̍��j�̂��Ƃ��ŁA�E���̎Ζʂ������̎Ζʂ�蒷�����獽���d���A
���͎��v���ɉi���ɉ�]�������邾���낤�Ƒz������邪�A�X�e���B���́A
�Ζʏ�̗͂̒ނ荇����͊w�I�ɘ_�����ۂɂ͂���͋N����Ȃ��Ƃ���\footnote{
�_�j�G���͕����n���̕��C�n�I�����_�̃O���[�j���Q�����܂�ł���A���R�X�e���B���̖����ƒ����ȋƐт�m���Ă����͂��ł���A
�����I�C���[�͒|�n�̗F�ł���������A�I�C���[���X�e���B���̘_�ɂ͒ʂ��Ă����Ɛ��_����邪�A�����܂Ő��_�ł���B}
����ł́A�[������d�������܂�邱�ƂɂȂ�B
�������āA�G�l���M�[�̑��ʂ͈��ł���i�G�l���M�[�ۑ����j�A����ɂ���đ�1��̉i�v�@�ւ��Ȃ킿�u�����I�ɓ����ĊO���Ɏd�������A
�@�֎��g���O�E�����̎d���ȊO�͊��S�Ɍ��ɖ߂�@�ցv(����A�����A���o)�̉”\���͔ے肳���B

�����A���ꂪ�d�v�����A���͊w�̃G�N�X�p�[�g�̃_�j�G���E�x���k�[�C�̏����Ɗ֗^�������Ă��A
�ނ����1��̉i�v�@�ւ͕s�”\�ł��邱�Ƃ̍Ċm�F�ŏI���������ł��낤�B

\section{�x���k�[�C���j�܂ꂽ�u�o�����v�̐��E}

�����Ŋm�F���Ă������B
�‚����Ǘ������n�ɂ����Ă͑S�G�l���M�[�͈��ɕۂ���邪�A�n���ł̕ω��ɂ‚��Ă̓G�l���M�[�ۑ����͉������Ȃ�����肦�Ȃ��B
���̕ω��̌��������߂錴���̓G�l���M�[�̕ۑ����Ɩ����͂��Ȃ�������Ƃ͓Ɨ��Ȍ����ł���B
�M�G�l���M�[�̎󂯓n���ɂ‚��ẮA�G���g���s�[�̑���i�����ɂ͔񌸏��j$\Delta S=\Delta q/T>0$�̌����̂݉”\�ł���B
���ہA�M$\Delta q$��������$T_1$����ቷ��$T_2$�ֈڂ�Ƃ��A$T_1>T_2$���瑍�G���g���s�[�ω���$\Delta S>0$�ɂȂ邱�Ƃ݂͂₷��(Brillouin)�B

���́A����͏ؖ��ɂȂ��Ă��Ȃ��B
�Ȃ��Ȃ�A�ŏ�����u������$T_1$����ቷ��$T_2$�ֈڂ�Ƃ��v�Ƃ̑O��������Ă��āA������͓��`�����ł���B
�t�����̈ړ��̂��Ƃł�$\Delta S<0$�ƂȂ邪�A���ꂪ�ώ@���ꂽ���Ƃ͂Ȃ��̂ł���B
�u�G���g���s�[�̌����͎��R�E�ł͎�����S���ώ@�����Ȃ��v�i�����_�E�E���t�V�b�c�w���v�����w�x�M��E��j�B
�‚܂�A���́u�M�͊w�̑��@���v�͌o�����ł���͊w�̖����͖��p�ł���B

���̂悤�Ɍ����Ă���B

\begin{quote}
---���āA���C�@�ւ͔M����@�B�I�Ȏd���𔭐������邪�A����@�B�I�ȗ͂ɂ���Ă��A�M�𐶂ݏo�����Ƃ��ł���B
�ՓˁA���C�͂��ׂĂ��̂��Ƃ����Ă���̂ł���B
�n�������b�艮�́A�S�̞���Ȃőł‚����ŎܔM�����邱�Ƃ��ł���B
�ԗւ̎��́A�אS�ɖ����������Ƃɂ���āA���C�ɂ�锭�M�������Ȃ���΂Ȃ�Ȃ��B
�ߍ����̂��Ƃ���K�͂ɗ��p����Ă���B
���Ȃ킿�኱�̍H��̉ߏ�̐��͂��A����̓S�‚����̎��𒆐S�Ƃ��Đv���ɉ�]���Ă�����S�‚��݂ɖ��C����悤�ɗ��p����A
���̌��ʗ��҂͋������M���ꂽ�B
���̊l�����ꂽ�M��������g�߁A�R���̗v��ʃX�g�[�u������ꂽ�B
�Ƃ���œS�‚ɂ���Đ��ݏo���ꂽ�M�������ȏ��C�@�ւɉ��M����قǏ\���ł���A
���̏��C�@�ւ�����ɂ܂����̓S�‚𓮂��������邱�Ƃ��ł���悤�ɂȂ蓾�Ȃ����̂ł��낤���B
��������΂������i�v�@�ւ͔������ꂽ�ł��낤�B
���̖��͒�N����邱�Ƃ��ł������A�܂�\underline{
����͋����̐����@�B�w�m�͊w���w���Ǝv����n�̌���}\underline{�ł͌��肳�꓾�Ȃ�����}\footnote{
Sie(diese Frage)war durch die {\"a}lteren mathematisch-mechanischen Untersuchungen nicht zu entscheidon.}�B
�������N�ɏq�ׂ悤�Ǝv���Ă��镁�ՓI�@���́A���̖���{\gt �ے�}������̂ł��邱�Ƃ�O�����ĕt�����Ă�������\footnote{
H.�w�����z���c �w���R�͂̑��ݍ�p�ɂ‚��āx{\"U}ber die Wechselwirkung der Natur kr{\"a}fte. �i1854�N�A�P�[�j�q�X�x���N�ɂāj}---
\end{quote}

������񂱂��Ō����Ă���u�i�v�@�ցv�̕s�”\���͔M���ۂ���݂�����u��2��i�v�@�ցv�̂��Ƃł���A
������G�l���M�[�����o���u��1��i�v�@�ցv�ł͂Ȃ��A���������āA�͊w�̌��������荞�ޗ]�n�͂Ȃ��B
������({\"a}lteren)�Ƃ����Ɖ���狌�����ᔻ����Ă���悤�Ɋ������邪�A
�ʂ̂Ƃ���Ɂu�O���I�̈̑�Ȑ��w�҂����ɂ���Ĉ�ʓI�Ɋm�F���ꂽ�@���v
�c�cdieses Gesetz durch die gro��en Mathemamatiker des vorigen Jahrhundert allgemein war�Ƃ��āA
�_�j�G���E�x���k�[�C�ƃ_�����x�[�������߂���Ă���B
���������āA�����i�ҁj�̎��̕]���ł͂Ȃ��B

����́u�@���v�̘_���w��̒n�ʂ��q�ׂ����̂ŁA�ꌾ�ł����Ȃ�A�u�o�����v�͏ؖ��̌���ł͂Ȃ��B
���v���鎖���͖����Ɋώ@����Ă��邪�A����ɔ����鎖��́i�Ȃ��Əؖ��ł����킯�ł͂��Ȃ����j�����̈����Ȃ��A
�_���I�ؖ��͍���ŁA�{���I�ɂ����u�o���v�������ɂ���Đ������Ă���@���ł���B

\begin{quote}
�u��ɏq�ׂ��ȒP�Ȓ莮���m���v�͊w���ƂɃG���g���s�[�������n�������̎����ɑΉ����Ă��邱�Ƃɂ͉��̋^�����Ȃ��B
����͂����̓���̊ώ@�̂��ׂĂɂ���ė���������Ă���B
�������A���̖@�����̕����I�{���ƋN���ɂ‚��Ă̖��ɂ��������O����̍l�@��������i�ɂȂ�ƁA
�����ɂ������Ă��Ȃ�������x�͍�������Ȃ��ł��鍢������Ă���B�v\\
�i�����_�E�E���t�V�b�c�O�f�j
\end{quote}

\section{�C�̕��q�^���_��}

�������ɁA�M�҂͉Ǖ��ɂ��ďڂ����͒m��Ȃ��������A
�_�j�G���E�x���k�[�C�́w���̗͊w�x�̂Ȃ��Ń~�N���̋C�̕��q�̉^���_��_���Ă���
�i�M�͊w�̃e�L�X�g�͐l�������͋����Ă���j�B
�C�͈̂��k���̂��闬�̂ł���B
�ǂ��܂œ��B�����������̗]�n�����邪�A�Ƃɂ������̌����̗��j�I�Ӌ`�͑傫���B
�����������I�ɂ͐Î~������Ԃł��A���̍\�����q�͏���Ȍ����ɐ₦�ԂȂ��^�����Ă���B
���̗��G�ȉ^���̃G�l���M�[�́u�����G�l���M�[�v�Ɖ]���i�����j�A���̎��x���c�_����̂��M�w�ł���B

���Ȃ݂ɁA�C�̕��q�^���_�̋����鏊�ɂ��ƁA�{���c�}�����z�i���x�x�N�g�����Ɨ���3�������K���z�Ƃ���j�����ƂɌv�Z���ꂽ���q�̕��ϑ��x�́A
������2�摬�x\footnote{
���p�ꎮ�ɋL���̊O������ǂ񂾌ď̂ŁA�t�̓ǂ݁u2�敽�ϕ��������x�v(����)������B�����ł͕M�҂̊��ꂽ���V�ɂ�����B}
(root-mean-square velocity)
\[
\text{�i��������H�j}
\]
�Ƃ��ė^������i�N���E�W�E�X,\, 1857�j�B
�����$0\, {}^\circ \mathrm{C}$,\, 1�C���̕W����ԂŌv�Z���Ă݂悤�B
�C�̂̓w���E��He�Ƃ��A����������1�����Ōv�Z�����
\[ p=\qquad ,\quad V=\qquad ,\quad M= \qquad \]
����
\[
\text{�i��������H�j}
\]
�ƂȂ�B
�����1 km/�b�̃I�[�_�[�ŕs���R�ɑ傫�����A���ۂ͑����q�Ƃ̏Փ˂Ŏ��R�ɉ^���ł���򋗗��i���ώ��R�s��mean-free-path�j�͂����Ə������B
����������΁A�����q�ւ̏Փ˂͕��O��Ă���߂ĕp�ɂł���B

���̂悤�ɂ��āA�x���k�[�C�̒񏥂����C�̕��q�^���_�́u�����G�l���M�[�v�Ƃ����M�w�̊�b�T�O�֔��W�����B
���ہA�G�l���M�[�͎d��$W$�ƔM$Q$�Ƃ����`�ŕ��̂ɗ^�����邩��A�O��̓����G�l���M�[�ω��Ƃ���
\[
E_2-E_1=Q+W�@\qquad \text{(����E���oetc.)}
\]
���u�G�l���M�[�ۑ����v�̕\���ƂȂ�B
���ꂱ�����u�M�͊w�̑��@���v�ł���A�����܂ł͗͊w�I�l�@�œ��B�ł���̂ł���B

{\gt �����Fhttps://en.wikipedia.org/wiki/H-theorem}

\section{�s�‹t�����̂̓x���k�[�C���z�̎d�g�݂ŗ����ł���}

�������A��������悪�S�������Ƃ����킯�ł͂Ȃ��B

�u�M�͊w�̑��@���v�́A�M�͍�������ቷ�ֈڂ�G���g���s�[�͑�����(�����ɂ͔񌸏�)���̉ߒ��͕s�‹t�ł���Əq�ׂ�B
����ɔ����錻�ۂ͎��ۂ͋N���Ă��Ȃ��B
�l�ނ͏��Ȃ��Ƃ��L�j�ȗ��u�X�̏�ɒu�����₩��̐�����������̂��ώ@�������Ƃ͂Ȃ��v�B
�@���͌o�����ł����āA���������ނ̊m���_�I�E���v�I���m���̂���@���ł���B

�m���_�I�E���v�I���m���Ȃ�o�����łȂ��Ƃ��m�������̐��l��ŕ\���ł���B
�u�‹t�E�s�‹t���v�̃e�[�}�ɍ��킹�Ă͂��邪�A
�M���ۂ͉�݂������������Ė����u�M�͊w�̑��@���v�Ƃ͕ʕ��ł��邱�Ƃ𒍈ӂ��Ă������B

\underline{������薳������}\, ����$2n$�‚̃����_���ɉ^�����闱�q�����̎d�؂�ꂽ�������ɂ���B
������d�؂�������ĉ�����悤�B
�^���ɂ����\underline{�\���Ȏ��Ԃ̌�}$n$�‚���������
$n$�‚��E�����ɑ��݂���m���́A�e���q���E�����̔����ɂ���m���͊e1/2��---�x���k�[�C���z---�����āA
\[
P_1=\dfrac{2N!}{N!N!}\cdot (1/2)^{2N} %\label{143}
\]
�ł���B
���Ȃ݂�$n=1000$�Ƃ��Čv�Z�����
\[
P_1=
\]
�ƂȂ�B
�������ɑ��݂��闱�q��$x$�̊m�����z�͓񍀕��z$\mathrm{Bi}(2n, \, 1/2)$�ł��邪�A
���S�Ɍ��藝�ł�����ߎ������$N(n,\, n/4)$�ŁA1�V�O�}�͈͂�
\[
N \pm \sqrt{N/2} %\label{145}
\]
�ƂȂ�B
$x=n$�i�����j�̋ɑ�ɑ΂�����$(\sqrt{n}/2)/n$�ŁA�ɑ�͂���߂ĉs���m���炵���B

\underline{���������璁����}\, �^���ɂ���Ă���炷�ׂĂ̗��q�������ꂩ�����̕Б��ɂ��낢�W������m���́A
�������񍀕��z����
\[
P_2=(1/2)^{2n} \cdot 2=(1/2)^{2n-1}
\]
�ŁA������{\gt 0�ł͂Ȃ���}�ɂ߂ď������A������u�����Ċώ@����邱�Ƃ͂Ȃ��v\footnote{�{����}�B

\[
P_2/P_1=\dfrac{N!N!}{2 \cdot 2N!} %\label{147}
\]
���̐ݒ�̌���ł́A$\text{����}\, \Longrightarrow \, \text{������}$�̓]�ڂ�
\kenten{�ق�}�N���肦�Ȃ����Ƃ������I�Ɂi�m���_�I�Ɂj�����ł��悤�B

\noindent
\Fig[�}�ƕ\]{\textwidth}{10\baselineskip}

\section{�L���̂Ȃ��ɖ����Ɖi����ǂݍ���}

�n���͌Ǘ��n�ł͂Ȃ����A�S�F���́i��`�ɂ��j�Ǘ��n�ł���B
�G���g���s�[���傪�s�‹t�Ȃ�---�G���g���s�[�͑��傷�邵���Ȃ��Ȃ�---
�u�i���v�̒����Ԍ�ł͂��邪�A�S�F���̃G���g���s�[�͂��‚��ɑ�ɒB����B
�G���g���s�[�̓h�C�c�̕����w��R.�N���E�W�E�X�ɂ���Ē�`����(1865)�A
����͗L���Ș_���u�M�̈ړ��͂ɂ‚��āv{\"U}ber die bewegende Kraft der W{\"a}rme(1850)�Ɋ�Â��Ă����B
�G���g���s�[���ɑ�ɒB����ΑS�F���͂��͂�M�I���፷���������ăt���b�g��l�ɂȂ�A
��؂̕����I�d���i�ړ��j�̌����������āA�F���͊��S��~�i���j�̏�Ԃ��}����B
���̏I�ǃC���[�W�̓N���E�W�E�X�AW.�g���\���i�P�����B�����j�ɂ�邪�A�u�M�I���vheat death\footnote{
���̌�̐��m�ȑn�Ď҂ɂ‚��Ă͕s���B}�ƌ`�e���Ă悩�낤�B

�Ȋw�҂͎��R����w�Ԃ̂ł͂Ȃ��A���R�́u�@���v����w�сA�@�����l�ԂɂƂ��ĉ����Ӗ����邩����̓I�ɗ�������f�n�������Ă���B
 
\begin{quote}
---���̂悤�ɂ��āA�i�v�@�ւ̖���Nj������l�X���A�Ò��͍��I�ɖa���n�߂����́A
��������‚̕��ՓI�Ȏ��R�̍��{�@���ւƓ����A���̖@���́A�F���j�̏��߂ƏI��̉����Èłɂ܂Ō��𑗂��Ă���
(welches Lichtstrahlen in die fernen N{\"a}chte des Anfang und des Endes der Geschichte des Weltalls aussendet)�B
���̖@���́A�����l�ނɑ΂��Ă��A�����ł͂��邪�A�����������ĉi���ȑ���(ewig Bestehen)�����e������̂ł͂Ȃ��B
���̖@���́A�R���̓�\footnote{
�V�񐹏��u���n�l�َ��^�v�ȂǂɌ����郆�_���E�L���X�g���̏I���_�v�z�ł́A
���E�ɂ́u�I���̓��v����߂��Ă���A���`�̐_�ɂ��u�Ō�̐R���v���s����B
�\���͂���Ă��邪�ǂ̂悤�Ȍ`��󋵂ł��“������邩�͒m�炳��Ă��Ȃ��B
�������A���‚ł����Ă�����ɂ��Ȃ��Đ����������ϗ��ɐ����Ȃ���΂Ȃ�Ȃ��B
 �����ł́A�ے��I�ɁA�M�͊w�̑��@���̎w�������u�M�I���v���u���E�̏I���v�̋N������̈�‚ƌ����ĂĂ���B}
�������Đl�ނ����������̂ł͂��邪�A���̓��̗��鎞�́A�K���ɂ��܂��x�[���ɕ�܂�Ă���B
�e�l�Ɠ����悤�ɁA�l�ނ��܂����̎��ł̍l���Ɋ����Ȃ���΂Ȃ�Ȃ��B
�������l�ނ́A���Ɏ��ł������̐����`�Ԃ����D���āA�����ϗ��I�ۑ�(h{\"o}here sittliche Aufgaben)�����‚��̂ł���A
���̉ۑ��w�����A������������邱�Ƃɂ���Ă��̎g�����ʂ������̂ł���B
�i�w�����z���c�O�f�j---
\end{quote}

\section{�u�����I�v�łȂ����l��10�{�y���߂Ȃ�}

�����͂����Ō����̐��E�ɖ߂�B
����������$x,\, y,\, \ldots$�͐��w�I�Ȃ��̂łȂ��A�����ɕԂ����ʂł���B

�����̓I�J�l�⃂�m�̉��l�����̗�($x$)�ɔ�Ⴕ�Ȃ����Ƃ�m���Ă���B
�m���Ă���ǂ��납�A����ȑO�ɁA�����łȂ����ӎ��Ɋ����܂������łȂ����X�s�Ȃ��Ă���B
�����̓I�J�l�⃂�m�ɔ�Ⴕ�Ċy���߂�킯���Ȃ��B
�e����ς�f��������������Ƃ�����10�H�ׂ��10�{���������킯�ł͂Ȃ����A
�s�[�i�b�c������Ƃ��āA�ی��Ȃ��H�ׂ�ΐ�̕��ł͂��悻�����̂������Ȃ��Ȃ�A���������Ăǂ����ŃX�g�b�v���邾�낤�B
�e�����m��f���Ƃ͗ϗ��A������̖��ȑO�ŁA10�{�̃I�J�l�⃂�m�Łu10�{�y���ށv�Ƃ��u10�{�K���v�ɂȂ�킯�ł͂Ȃ��͓̂��R�ł���B
�����́A�����I�A�����I�ȗ�$x$�ɂ��̂܂܊�Â��Đ����Ă���킯�ł͂Ȃ��A
���Ƃ����Ĕ�Ⴗ���$ax$�A���邢��$ax+b$�Ő����Ă���킯�ł��Ȃ��B

�����A����͏d�v�Ȃ��Ƃ����A������x�͈̔͂ɂ����ĂȂ�A��葽���y���݂�葽���K���ɂȂ邱�Ƃ��炢�͌��������ł���B
�p���ċꂵ���Ȃ�����݂��߂ŕs�K�ɂȂ邱�Ƃ͂Ȃ��B
���ہA$2>1$�ł���ȏ�2�‚̃����S��1�‚�艿�l�������������A
$10>5$�ł���ȏ�10���~��5���~��肢�낢��Ȃ��Ƃ��ł��邩��A���l���������Ƃ��m���ł���B
�����I��$x$�͌����Ė��֌W�ł͂Ȃ������Ă���Ƃ͌�����B
�ł́A$x$��\underline{�ǂ�}�����̂��B

\section{�����ΐ��_}

�x���k�[�C�Ƃ̐l�X�́A�N�w�̃g���[�j���O��ς�ł����B
�����̒m������l�X�̖񑩎��̂悤�Ȃ��̂ł���������A�����I�ɍl���邱�Ƃɂ͊���Ă����B
�܂��Ă�_�j�G���͐����w�����ɂ��Ă����B
�����ɏq�ׂ����������ǂ�������t���Ă������킩��Ȃ�
---�킩��؂��Ă��邪�䂦�ɂ������ē��---���ɂ�����قNj�J�͂��Ȃ������B

�e�l�Ԃɂ͐S��������A���l�͐S���ɂ��ƂÂ��f�Ƃ����΂��Ȃ�߂��̂����A�S���w�Ȃ���̂͂܂��Ȃ������B
�����A���Ȃ��Ƃ��A���炩�ɂ���́u���_�v�̌��ۂł���B
�����I�A�����I���݂����łȂ��A�u���_�v�̑��݂�����A
�I�J�l�⃂�m��\kenten{����}�����ŁA����ɉ��l�����o���̂́u���_�v�ł���B
���Ȃ킿�I�J�l�����m�Ɍ��o����鉿�l�̓_�j�G���E�x���k�[�C�́u���_�I���l�v�Ȃ̂ł���B

�u���_�I���l�vmoral value��moral�ɂ͂������u�����E�_���́v�Ƃ��������I�Ӗ������邪�A�����Ёw��p�a���T�x�ł́A
��3�ԖڂɁu�����I�Ȃ��̂ɑ΂��Đ��_�I�v�Ƃ�����ꂪ����B
�����͂��́e���_�I�f�Ƃ����`�e���͍L���p�����A�e�l�ԓI�f����Ɂe�Љ�I�f�܂ŋy��ł���B
���R�u�E�x���k�[�C��Ars Conjectandi�̋��ɖړI�́A�m���̉��p���Љ��l�Ԃ̒��Ɍ��o���\�z�ł������B
���̑�IV���́e�Љ�̒��ɂ���m���f
---�p�X�J����t�F���}�[�A�z�C�w���X�͂������낵�����ɂȂ�����---��ڂ������̂ŁA
�����Ƃ��Ă͌^�j��ɑ�_�ȍ\�z�������낤�B
���̍\�z�͂����̃_�j�G���ɂ����Ă��������m�Ȍ`���Ƃ����B
�I�J�l�̐��_�I���l�ł���B
�����炩�ɂ���͌o�ϊw�����ē��v�w�̎n�܂����������̂ł��������B

�e���n�f�̐l�ɂ́A�����ϕ��i���R�u�A���n���E�x���k�[�C�j�◬�̗͊w�A�M�͊w�i�_�j�G���E�x���k�[�C�j�Ɨ��āA
�������A�C�}�C�Ń~�X�e���A�X�Ȃ��̂ɘA�ꍞ�܂ꂽ�Ɗ�����l�͑����B
�u���_�v�͏��Ȃ��Ƃ��u�����v�ł͂Ȃ��A���m�ł����m�ł��Ȃ��B
����͐l�ԂɊւ��邱�Ƃł�����蓖�R�ł���̂����A���́A
���P�C�ɂ��킩��A���_�I���l���l���Ȃ���΂Ȃ�Ȃ����ۂ����݂���B
���ꂪ�_�j�G���E�x���k�[�C�����ɂ����u�Z���g�E�y�e���X�u���O�̋t��(St. Petersburg's paradox)�v�ł���B

\begin{quote}
---���܁A���z�I�ȍd�݁i$\text{�\�̊m��}=p=1/2$�j���A�\���o��܂œ����‚Â���B
1��ڂŏo���2�~�A2��ڂȂ��4�~�A3��ڂȂ�8�~�A$\ldots $�A$n$��ڂȂ�$2^n$�~���󂯎��B
�q�̎Q������$f$�~�Ƃ���B
���̓q���ɎQ�����ׂ����B---
\end{quote}

���̓q���瓾����z�̊��Ғl���v�Z���悤�B
$n$��ڂɂ͂��߂ĕ\���o��Ƃ����m����$(1/2)^n$�Ȃ̂ŁA$\text{�m��}\times \text{���z}$�̘a���Ƃ�΁A
\[
(1/2)   \times 2   +
(1/2)^2 \times 2^2 +\cdots +
(1/2)^n \times 2^n +\cdots =\infty
\]
\noindent
\Fig[����139��]{\textwidth}{5\baselineskip}
�ŁA������ƂȂ�B
���������āA$f$�������獂���Ă��q�ɎQ�������������ƂȂ邪�A�Ƃ��Ă������ł��ʃo�J�o�J�������_�ł���B
���ہA$f$�������Ȃ�΁A���̓q�ɎQ������l�̐��͌�������ł��낤�B
�Ⴆ��$f=1000$�~�Ƃ����$n \geqq 10$�łȂ���΂��̓q�͑��ƂȂ�A�e�܂��߂Ɂf�q����l�Ȃ�΂��̓q�ɂ͂܂��Q�����Ȃ�
�i�h�X�g�G�t�X�L�[�w�q���ҁx�ɂ������a�I�q���҂ɂ͒ʂ��Ȃ��j�B
�ǂ����Ă��̂悤�Ȍ�������w�I���ʂɂȂ�̂ł��낤���B
����𐳂����C��������@�͂Ȃ��̂��낤���B

�j�R���X�E�x���k�[�C��1713�N9��9����P. R. de�������[���ɂ��ĂĂ�������N�����B
���ꂪ��́u�Z���g�E�y�e���u���O�̋t���v�ł���B
1728�NG. Cramer��$\sqrt{x}$�֐���p���Ă���������i��q�j�A5��21���Ƀj�R���X�ɓ`�����B
�����āA�j�R���X��1732�N4��5���A�_�j�G���ɂ����`���Ă���B
�_�j�G���͌��ʂ�s�\���ƍl���A1738�N���̗L���Ș_���ƂȂ����B
�_���̓��e����i�ꕔ�t�����X��j��
\begin{quote}
Specimen theoriae novae de mensura sortis \\
{\it Commentarii Academiae Scientiarum Imperialis Petropolitanae}
\end{quote}
�����Ė󂹂΁w�q���̑���Ɋւ���V���_�̌��{�x�ƂȂ낤�B
specimen�i���{�j�́A�����������̂��쐬���܂����̂ň�‚������������A���炢�̈Ӗ��ŁA
�����ɂ͒P�ɋt���������Ƃ��������A�����ƑO�i���ĐV���_���Ă���C�T������\footnote{
���\�́u�Z���g�E�؃e���u���N�鍑�Ȋw�@�_�����v�i���邢�͋I�v�j��ŁA
Petropolitan�̓��V�A�鍑�̃y�e���u���N�̃M���V���������e����󂵂����́B}�B

���āA���������u�q���v�isors,\,$\text{�p}=\text{lottery}$�j�͂��̋��z�Œ��ڕ]���ł�����̂ł͂Ȃ��A
����K�؂ȉ��l�֐��i�ΐ��֐��j�Ɋ��Z���������ŕ��ρi���Ғl�j���Ƃ��āA
���̎������l�imoral value�S���I�F����̉��l�j���߂�ׂ��ł���B
���̖���͉��l�̖{���ɔ������I�Ȃ��̂ŁA���̎��Ɛ}�Ő��������B

$C,\, D,\, E,\, F,\, \ldots$�͓�����”\���̂�����z�A$m : n : p : q : \cdots$�͂���炪������”\���̔䗦�Ƃ���B
�K�؂Ȋ֐���p�ӂ��A���̊֐��ŋ��z�̎������l�]����$CG,\, DH,\, EL,\, FM,\, \ldots$�ƂȂ����Ƃ���΁A
���̓q���̎������l�͂��̊��Ғl
\[
\text{�i��������H�j}
\]
�ŕ\�����B
�����āA����͗L���Ȓl�ɂ��邱�Ƃ��ؖ��ł��邵�A���یv�Z���”\�ł���B

\noindent
\Fig[�}]{\textwidth}{10\baselineskip}

���ݎ��Y�z��$B$�_�i�傫��$AB$�j�Ȃ�΋��z�͂���Ŋ����Ă����Ă��悢�B
�������̉��l�֐���ΐ��֐�$\log x$�i��ʂ�$b \log x$�B������$b$�͌��ʓI�Ɍ����Ȃ��j����Ȃ�΁A
\[
\text{�i��������H�j}
\]
�ł���B
��������z�ɖ߂��Čv�Z�������Ȃ�
\[
\text{�i��������H�j}
\]
�ƂȂ�A���̏ꍇ�͂���ʼn��������B
�Ȃ����l�֐��͐��w�I�ɂ͑ΐ��֐��Ɍ��闝�R�͂Ȃ��A$\sqrt{}$�֐��ł������x���Ȃ��B
�������A���ݎ��Y���ʂ���肭����Ȃ��Ȃǂ̌��_������B

���̂悤�ȉ��l�֐�����ꂽ���Ȃ��Ȃ�A���̂悤�ɍl���Ă��悢�B
�t���̐ݒ�ł́A�q���̒񋟎ҁi�n�E�X�j�͖����̎��Y��ۗL����Ɖ��肳��Ă��邪�A����͌����I�łȂ��B
���ɑ傫�����L���̎��Y��ۗL���A�Q���҂͂�������x�Ƃ��ēq���̏܋��𓾂�ƍl����΁A
�q���̊��Ғl�͗L���ƂȂ�A�t���͉�������B
���ہA
\[
\text{�i��������H�j}
\]
�Ƃ����v�Z���ʂ�����B

�o�ϊw��P. �T�~���G���\���́A�s�ꃁ�J�j�Y���̂��Ƃł͂���قǂ̋���ȑ���������Z���i�͂���������������Ȃ�����A
���Ƃ��Ƌt���͐����Ȃ��Ƃ����B
�@
\section{�u���p�v�i�R�E���E�j�Ƃ́H}

���āA���_���ɓ����Ē��ׂ��̂����A��������߂��Ă݂悤�B
�ݕ��̉��l�Ƃ����̂�����܂��͌o�ϊw�̖��ł���B
�����Ȃ肾���A�u���p�Ƃ͉����낤���B�v
�_�j�G���E�x���k�[�C�́e���_�I���l�f�ƌ����Ă��鉿�l���A�����ł͌o�ϊw�Łu���p�vutility�ƍL���������炳��Ă���T�O�ł���B
���Ƃ��A�e���l�f�Ƃ́A�����P�����ƁA�ǂ����ƁA�C���������ƁA���ɗ��‚��ƁA�����������ƁA�K���Ȃ��Ɓc�c�Ȃǂ�
���̂ŁA��������炩�̌����߂����邱�ƂƂ��āu���p�v�Ɖ��Ɍ������Ƃɂ��悤
�i����Ȃ炻������Ȃ��Ă������̂����c�c�j�B
����ƁA�����܂ŏq�ׂ����Ƃ͂܂����̖���ƂȂ�B
$x$��$10x$�ɂȂ��Ă��V�A���Z�̌����ڂ��P�����I��10�{�ɂȂ�킯�ł͂Ȃ����A�����I�ɃV�A���Z���L�т�킯�ł͂Ȃ��B

\begin{quote}
---���m��I�J�l�̗ʂ�$x$�Ƃ���ƁA$x$�̌��p$u(x)$��$x$���邢��$ax$���邢��$ax+b$�Ƃ͕\�킳�ꂸ�A����֐�$u(x)$�ł���---
\end{quote}

�������Ƃ���ƁA$u(x)$�́e���_�I�Ȃ��́f��\�킷�䂦�A�����A�C�}�C�ŃG�^�C�̒m��Ȃ��������ƂȂ낤�B
�m���ɂ��������ʂ͂���A���ꂪ�u���p�֐��v$u(x)$���ƈ�ʂ莦�����Ƃ͂ł��Ȃ��B
���������l�ɂ��ꍇ�ɂ��܂�$x$�����ł��邩�ɂ�邾�낤�B
���Ȃ��Ƃ�
\[
x>y \text{�Ȃ�} \quad u(x)>u(y)
\]
�łȂ��Ă͂Ȃ�Ȃ����낤�B
����ȊO�Ƀn�b�L�����邱�Ƃ͂Ȃ��̂��낤���B

\section{�Z���g�E�y�e���u���O�̋t���̉���---�o�ϊw�̃X�^�[�g}

���������āA���_�I���l�̋Ȑ��́u���K�̌��p�Ȑ��v�ƂȂ�B
�}��������ł���B

�d�v�Ȃ��Ƃ́A���������z�i���Ƃ���5�~�j����1�~�����������ꍇ�ƁA
�傫�����z�i�Ⴆ��1000000�~�j����1�~�����������ꍇ�Ƃł́A����1�~�ɑ΂�����p�̑�����
---���E���p�Ƃ���---����҂̕��������Ə������B
�������قǁu�����v�̂��肪���݂��킩��Ȃ��B
����͓���̌o�����ł��邪�A�d���o���I�����Ōo�ϊw�Łu���E���p����\footnote{
�u�����v�Ƃ͂��񂾂񌸂�A�̈ӁB}
�̖@���v�Ƃ����Ă���B
�������A�x���k�[�C�̌��p�Ȑ�
\begin{align*}
& u(x)=k \log x \quad (k>0) \\
& \text{�i���R�ΐ��Ƃ��Ē��}e=2.71828\cdots \text{�j}
\end{align*}
�ł����̖@���͎�������Ă���B

�ȒP�̂��߂�$\mathrm{AB}=1, \, k=1$�Ƃ���ƁA$x\text{�~�̌��p}=\log x$�ƂȂ邪�A
\[
\log 2.72 x=\log 2.72+\log x=1+\log x
\]
���K����2.72�{��\kenten{�Ȃ邲�Ƃ�}�A���p��1���‚ӂ���B
\[
\log 10 x = \log 10 +\log x=2.3+\log x
\]
����āA10�{�ɂȂ��Ă���2.3���x�ӂ��邾���ł���B
�����āA10�{�V�A���Z�ɂȂ邱�Ƃ͂Ȃ��I

�Z���g�E�y�e���u���O�̖����A�q�ɂ���ċ��K���̂��̂𓾂�̂łȂ����K�̌��p---���_�I���l---�𓾂Ă���ƍl���A
���̉��l�̊��Ғl�Ƃ��āA���x��$\infty$�ł͂Ȃ�
\begin{align*}
&(1/2)\log 2+(1/2)^2 \log 2^2+(1/2)^3 \log 2^3+\cdots \\
&=(\log 2)\{ 1\times (1/2)+2\times (1/2)^2+3\times (1/2)^3+\cdots \} \\
&=\log 4
\end{align*}
�𓾂�Ƃ������ƂŁA�߂ł��������i�����j����i$\{ \quad \}\text{��}=2$�ɒ��Ӂj�B

�������Ƃ���������ς��A���̓q���̉��l�����K���Z����ƁA�e�m���ȁf4�~
---���̊��Z�l���u�m�����l�z�vcertainty (monetary) eqivalent�Ƃ���---
���^������p�ɓ������B
���p�ő��������̓q�̊��Z���Ғl�͂킸����4�~�ŁA�Q������4�~�ȉ��̂Ƃ������q�ɎQ������΂悢���A
�܂Ƃ��Ȍ��_�ł��낤�B
�Ɠ����ɁA�u���p�v�T�O�̗L�����A�Ó���������ŗ��j��͂��߂Ċm�������̂ł���B

\begin{itembox}[l]{�Ȃ��A�u�Z���g�E�y�e���u���N�v���B}
�_�j�G���E�x���k�[�C�̓Z���g�E�y�e���u���N��w�̋����ŁA�����ł��̉ۑ��m��A
�_���������̊w�m�@�I�v�ɏo�����B
�u�I�v�v�Ƃ͊w�Ғ��Ԃ̘_���W�ŁA�ӂ‚��̐l�X�͎��l���Ԃ̓��l�����v�������ׂ邩������Ȃ��B
�������A�w��̂��̕��ɂ��N�I���e�B�̐R���Ōf�ڋ��‚�����̂ŁA�Ȃ��Ȃ����т����B

�Ȃ��A���̒n���͎���I�ϑJ������A
\begin{quote}
1. �Z���g�E�y�e���u���N $\Longrightarrow$
2. �y�g���O���[�h�i�g�j \\
$\Longrightarrow$ 
3. ���j���O���[�h�i�g�j   $\Longrightarrow$
4. �T���N�g�E�y�e���u���N
\end{quote}
�ƕς����B
1.�Z���g�́u���v�ŃC�G�X�E�L���X�g�̒�q�y�e���𐹐l�Ƃ������A
���邢�͗��j��̌����҃s���[�g�����Ɉ��ނƂ����Ă��悢�B
2.��1.�̃��V�A���A�u�u���N�v�u�O���[�h�v�́c�s�A
3.�̓��V�A�v����A4.�̓\�A�����ŁA�e�T���N�g�f�e�u���N�f��1.�̉������ʂ������̂ŁA���ݖ��ł���B
�Ȃ��A�p�ꖼPetersburg�ł�s�𔭉����邪�A���V�A���ł͌�������̂ŁA
�����s���R�����As����ꂸ�������Ȃ���u�Z���g�v����ꂽ�n���ď̂Ƃ����B
\end{itembox}

\section{�u���p�v�̐؂ꖡ}

���ꂾ���L�p�Ȃ炻�̐؂ꖡ��m�肽�����̂����A�T�^�I�Ȍ���I�p�������Љ�悤�B���҂̒���������p����B

���X�N���ɂ�����l�ԍs�����݂Ă���ƁA���p�̊T�O�Ȃ��ɁA�ʏ�̋��K�̊��Ҋz
---���ҋ��z(expected monetary value)---�����ł́A
���ꖞ���ł���������ł��Ȃ����Ƃ́A�Z���g�E�y�e���u���O�̋t���ł݂��ʂ�ł���B
���ꂾ���ł͂Ȃ��B
�l�͂Ȃ��A���i�R�⃉�X�x�K�X�Ŗׂ���̂��H�@
�������A�J�W�m�̌o�c�҂͂߂����ɕ����Ĕj�Y���邱�Ƃ͂Ȃ��B
�΍Еی��Ƃ����e�q�f�ł̓v���~�A���i�ی����j�̓��v�́A�����̏ꍇ�A�x������ł��낤�ی�����荂�����A
�����łȂ���Εی���Ђ͓|�Y���Ă��܂��B
����ƁA�l�͂Ȃ��ی��ɓ���̂��H�@
�����̖��́A���p�̋Ȑ��̗l�q---����\ruby{����}{�����Ƃ�}---�ɂ���āA��������B

���p�Ȑ���\kenten{���^}�̂Ƃ��ɁA�\���ł���킳���悤�ȓq�i�����j���l���悤�B
�����A���s��50--50�Ƃ������X�N�̂��铊���ł����āA���������200���~�������A���s�����200���~��������B
�܂��A���������Ȃ��ꍇ�̌��݊z��300���~�ł���B

\noindent
\Fig[�\����140��]{\textwidth}{5\baselineskip}

�܂��A�����ɂ����ҋ��z�ōl����΁A
\[
\frac{1}{2}\times 500+\frac{1}{2}\times 100=300 \, \text{�i���~�j}
\]
�ƂȂ�A���������Ȃ�����z�ɓ������B
�䂦�Ɋ��ҋ��z�̊�́A��ʂ������A�����̉”ۂɂ‚��ĉ��������Ȃ��B
����A���K�z�łȂ����p�𓱓����āA���p�̊��Ғl�ōl���悤�B
�q�ɓ������Ƃ���ƁA
\[
u(500\text{���~})=\mathrm{QG}, \quad 
u(100\text{���~})=\mathrm{PF}
\]
�ł��邩��A��`$\mathrm{PFSQ}$�̒��_�A���藝�ɂ��A���p�̊��Ғl��
\[
\frac{1}{2} u(500\text{���~})+\frac{1}{2} u(100\text{���~})=\mathrm{S'M}
\]
�ƂȂ�B
����A�q�ɓ���Ȃ���΁A���p��
\[
u(300\text{���~})=\mathrm{SM}
\]
�����ŋȐ������^�ł��邩��A�K��
\[ \mathrm{S'M}< \mathrm{SM} \]
���Ȃ킿�A�q�i�����j��\kenten{���炸}�A���X�N��������邱�ƂɂȂ�B
���邢�́A�������Ƃ����K�z��---���p�ł͂Ȃ��A���̋��K��---���̂悤�ɂ����Ă��悢�B

���p�Ȑ����$\mathrm{R}$��$\mathrm{RS'}\parallel x$���Ȃ�悤�ɂƂ�A
$x$���ɉ����������̑���$\mathrm{N}$�A���̓_��$x$�̒l��$x_0$�Ƃ���B
�����������$\mathrm{RS'}=\mathrm{RN}$�Ȃ邱�Ƃ��l����ƁA�\���̓q�ɓ��邱�Ƃ́A
�z$x_0$�̋��K�i���i�}�j���m���Ɂ\�q�ɋ������Ɂ\�ێ����Ă��邱�ƂƁA����̌��p��^����z����̋��z�ł��邪�A
���p���l���ɓ��ꂽ�ꍇ�ɁA���X�N�̂��Ƃɂ��铊�����Y�̎����I
---���X�N�̂Ȃ����S�ȏꍇ�Ɋ��Z����---���l�ł���B
����$x_0$���A�u�m�����l�z�v�Ƃ������Ƃ͐�ɏq�ׂ��B
�������́A$x_0< 300$�A�‚܂�
\[
\text{�m�����l�z}<\text{����z}
\]
�Ɠ������Ƃ������Ă���B

���Ȃ݂ɁA�_�j�G���E�x���k�[�C�̌��p�֐��̂Ƃ��i�������A$k=1$�j�ɂ́A
\[
\mathrm{QG}=\log(5 \times 10^6), \qquad 
\mathrm{PF}=\log(1 \times 10^6)
\]
�ł��邩��A
\begin{align*}
\mathrm{RN}
&=\mathrm{S'M}
 =\dfrac{1}{2}(\mathrm{QG}+\mathrm{PF})
 =\log (\sqrt{5 \times 1 \times 10^{12}}) \\
x_0&=\sqrt{5}\times 10^6 
\fallingdotseq 223\text{��}6\text{��~} %\label{141}
\end{align*}
�‚܂�A���̓����̎����I���l�́A���������Ȃ�����z300���~���66���~����邱�ƂɂȂ�A���R�A
���X�N��������ē������T���邱�ƂɂȂ�̂ł���B

\section{�S�[���h�o�b�n�̖��}

\subsection{�u���͕�������vresurgo}

�o�[�[���́u�~�����X�^�[\footnote{
�����̏C���@�ɗR�������‘吹���B�ق��Ƀ{���A�R���X�^���X�A�G�b�Z���A�t���C�u���O�A�E�����Ȃǂɂ���B}�v�́A
������藬��郉�C��������Ɍ��鏬�����u�ɓV���Ղ��ė����Ă���B
�o�[�[����w�u�x���k�[�C���Ɂv�̃t���b�c���̊��߂ł��̑吹����K�ꂽ�B

\noindent
\Fig[�ʐ^]{\textwidth}{10\baselineskip}

���R�u�E�x���k�[�C�̕��͂��̑吹���̈�p�̒��ɍ��܂�A����̉��ɁA�ΐ��点��ƂƂ��ɁA�W��
\begin{quote}
{\it Eadem mutata resurgo}\footnote{
�Ăяオ��(rise again)�B
�����p��ŁA�����痧���オ��i��������j�̈ӁB} \\
(Though changed I shall rise the same.) \\
�ς��Ƃ����ǂ������`�Ŏ��͕�������
\end{quote}
���̐l�ւ���ƂƂ��ɏ������܂�Ă������B
�u��������v�i�p�Fresuurect�j�́A
�u������̕����v�‚܂�u\ruby{�S}{��݂���}��v�Ɖi���̐������Ӗ����L���X�g���M�‚̒��S���`�ł���B
���͂���ɂ͐��w�I�\�����^������B
�������ɕ��c�ȗ��̔M�S�ȐV���k�i���v�h����j�̖ʖږ��@������̂�����B

�����ŁA����������߂��B
\begin{quote}
IACOBUS BERNOULLI                         \\
MATHEMATICUS INCOMPARABILIS               \\
ACAD. BASIL.                              \\
VLTRA XVIII ANNOS PROF.                   \\
ACADEM. ITEM REGIAE PARIS. ET BEROLIN.    \\
SOCIUS                                    \\
EDITIS LUCUBRAT. INLUSTRIS.               \\
MORBO CHRONICO                            \\
MENTE AD EXTREMUM INTEGRA                 \\
ANNO SALUT. MDCCV. D. XVI. AUGUSTI        \\
AETATIS L. M. VII                         \\
EXTINCTUS                                 \\
RESURRECT.  PIOR. HIC  PRAESTOLATUR\footnote{
RESURRECT.\, PIOR. HIC PRAESTOLATUR 
���͐��߂��A�����Ă����Łi����ɂ���āj������҂��]�݁m���@�I�ɂ͎󓮑ԁn�c�c�̈ӁB}\\
IUDITHA STUPANA                           \\
XX ANNOR. UXOR                            \\
CUM DUOBUS LIBERIS                        \\
MARITO ET PARENTI                         \\
EHEU DESIDERATISS.                        \\
H.M.P.\footnote{
hoc monumentum posuit \,�u��������v�i���V�j�B���̌������L���ŏI���B} \\
���M�󒆁�
\end{quote}

\noindent
\Fig[�i�ʐ^�j]{\textwidth}{10\baselineskip}

\begin{itembox}[l]{�ΐ��点��}
�m����悤�ɁA���ʋȐ��̂点��ɂ͂��낢�날�邪�A
\noindent
\Fig[�\]{\textwidth}{5\baselineskip}

���R�u�E�x���k�[�C�͂��̑ΐ��点��
\[
\text{�i�ɕ\���Ŏ�������H�j}
\]
���C�ɓ����Ă����B
���̒�����������
\[
\text{�i��������H�j}
\]
�ł����āA�x���k�[�C�ɂ́u�����`�ōĐ�����v�Ƃ����@���I�e�[�}���Î����Ă����B
�܂��A���̂点��́u���p�点��vequiangular spiral�Ƃ������A
�e�_�ł̐ڐ������a�ƈ��p$\phi$���Ȃ��Ƃ������������邪�A�����$\theta,\, \theta +d\theta$�̓��a��
���O�p�`����e�Ղɏؖ������B
\end{itembox}

\noindent
\Fig[�}]{\textwidth}{10\baselineskip}

\noindent
\Fig[�x���k�[�C�ƔN���F�쐬��]{\textwidth}{10\baselineskip}


\include{Chap02}
\include{Chap03}
\include{Chap04}
\include{Chap05}
\end{document}



\chapter{�x���k�[�C���}

\section{�x���k�[�C�E�t�@�~���[�Ɓe�o�[�[���g�f}

�X�C�X�ό��Ƃ����ƃ}�b�^�[�z�����A�����O�t���E�A�����u�����A�s�s�Ɖ]���Ύ�s�x�����A
�o�ς̒��S�`���[���b�q�A���������c�F�������v�������ԁB
�������A�o�ϊw�ҁA���w�҂Ȃ�W���l�[�u�A
�����Đ��w�҂Ȃ�A�c�c�������o�[�[���ł���B
�����̔��ϕ��w�̑̌n�̔��z�����������΁e�o�[�[���Y�f�ł���B
�o�[�[���ɂ͂����̊w��I���т��L�O���āu�x���k�[�C�ʂ�v(Bernoullistrasse)��
�u�I�C���[�ʂ�v(Eulerstrasse)������B
���炭�����Ȃ����ł����������������A�I�C���[�ɂ����
\[
\dfrac{1}{1^2}+\dfrac{1}{2^2}+\dfrac{1}{3^2}+\cdots=\dfrac{\pi^2}{6}
\]
�Ƃ��ĉ�����\footnote{
�I�C���[�̍ŏ��́e�㐔�I��@�f���܂߂ĕ��@�͍����܂�3,\, 4�ʂ肠��B}
���A���͏ے��I�Ɂu�o�[�[�����v�ƌĂ΂�Ă���B
�����ϕ��w�̓��C�v�j�b�c�ƃj���[�g���ɂ���Ĉ�܂ꂽ���A
���ꂩ��q�ׂ�x���k�[�C�Ƃ���уI�C���[�́e�o�[�[���g�f�ɂ���đ̌n�I�A�Z�p�I�ɕҐ�����A
���i�������ނˊ��������B
����ǂ��납�A���Z�̐��w�ł����x�����グ��
\begin{align*}
&1^4+2^4+3^4+4^4+\cdots +n^4
\intertext{�‚��ł�}
&1^{10}+2^{10}+3^{10}+4^{10}+\cdots +n^{10}
\end{align*}
�͂ǂ����낤���B
�������e�o�[�[���Y�f�̌����ʼn�����B

�o�[�[���̓X�C�X�恜�̓s��ł���B
�`���[���b�q�ɂ͓��{���璼�s�ւ͂��邪�A�o�[�[���ւ̓h�C�c�̃t�����N�t���g�Ȃǂ��o�R����B
��ԗ��s���y���݂�����΃p���E�������w����ł��悢�B
���̃o�[�[���̒��S�X�̈�pFreie Strasse 20�Ԓn�ɁA
�u�x���k�[�C��ǁv(����Goldene Apotheke M.Bernoulli)���X���\���Ă���B
�u�x���k�[�C�Ɓv�i�����j�͊w�҂̓V�ˉƌn�Ƃ��ÂƂɒm���Ă��邪�A
�ƋƂ͑�X��폤�ł��̖����ł���B
�V�ˉƌn�̍ŏ��̃o�b�^�[�ł��郄�R�u�E�x���k�[�C�����܂ꂽ�̂́A
���傤�ǂ��̗L���ȃp�X�J�����t�F���}�[�̊m���_�̉������Ȃ�1654�N�Ƃ�������A
�{�N��361�N�ڂɂȂ邪�A
��폤�͈ȑO��肾������ۂɂ͂���ɒ����B

�u��폤�v(�papothecary, ��Apotheke)�Ƃ͐���܎t�̂���u��ǁv�ł���B
�����Ƃ�'pharmacy', \, 'chemist'�Ƃ̐��x��̈Ⴂ�͂悭�킩��Ȃ����A
�Ƃɂ������w�E�����w�̐��ȊO�Ɉ�w�A�����w�A�A���w�������ł������̂����̂������ł��낤�B
���E�G���[�g�ł��邱�Ƃɉ����A�����炭�͖�폤�̕t�����l�͑傫���A
���‚��ꂾ���̘V�܂ł���Ώ��H��c���̗L�͎҂ł����������Ƃ��z���ɓ�Ȃ��B
���j�㑽���̐��w�ҁE�����w�҂������ɋ��X�Ƃ����邢�͑�w�|�X�g�ŋ�J�������ƂƂ���ׂ�ƁA
�܂��ɗ]�T�̐����œ��փP���J�⒇�Ⴂ�̗]�n�����������킯�ł���B
�t�Ɍ����΁A�x�M�Ɉ����������̐��E�j���̊w�҂�y�o���A
�ƋƂ������̒����ɏ‚��葱�����͈̂̋ƂƂ������Ȃ�\footnote{
���Ƃ����{�̗{�q���x�͂Ȃ��������Ƃɒ��ӂ���B}�B

���āA���w�╨���w���w��ł��āA���́e�x���k�[�C�f�̖��ɏo���Ȃ����Ƃ͂Ȃ����낤�B
���Ƃ���
\begin{itemize}
\item �x���k�[�C���s�A�x���k�[�C�̓񍀕��z
\item �x���k�[�C��
\item �x���k�[�C�̔���������
\item �x���k�[�C�̒藝
\item �u�Z���g�E�y�e���u���N�̋t���v�̃x���k�[�C�ɂ�����
\end{itemize}
�Ȃǂ́A�݂ȃx���k�[�C�Ƃ̊w�҂����̖��O�������Ă���B
�܂��A���R�u�E�x���k�[�C�i�����j�A���̒탈�n���E�x���k�[�C�i�����j�A
���̎q�_�j�G���E�x���k�[�C��3�l�������Ȃ��Ă͂Ȃ�Ȃ��B
�����ŁA�_�j�G���̏]�Z��j�R���E�X(II)�E�x���k�[�C�i�����j�������Ă������B
���̌�w�҂̓`���͈����p�����w�A�N�w�A�_�w���邢�͌��z�w�Ȃǂ̊e����Ɋ��􂷂�l�ނ��y�o���A���݂Ɏ����Ă���B
�������h�C�c�̕��w�҃w���}���E�w�b�Z�v�l�Ȃǂ�y�o���Ă���B
���̉ؗ�Ȋw�҉ƌn�́A���y�ƃo�b�n�Ƃƕ���ň�`�w�̃e�L�X�g�ɂ��o������B
%�܂����v�w�̑n�n�҂̈�lF�E�S���g���i�����j�́w�����x�Ŏ��グ���Ă��邭�炢�ł���B

�ł́A�V�ˉƌn�̈�`�𓝌v�I�ɕ��͂�������F.�S���g���w��`�I�V�ˁx
(Heredetary Genius,����)�ɂ���x���k�[�C�ƕ]�����Ă݂悤�B
�������A�ƕ�\footnote{
���S�l�����猩�āAB�͌Z��AN�͉��Ȃǂ�\���B}
�̘A���ɒ��ڂ��A����������]���S�Ŋw��I�ɂ͕s�\���A�s�ύt�ȓ_�����邪�A
���i�`�ʂɓ��ݍ���ł���_�́i�q�ϐ��͂Ƃɂ������j���ڂ������B
\[
\text{��������܂����H}
\]

{\gt ���R�u�E�x���k�[�C}�@��̑�ɕ��O�ꂽ���̍����Ȑ��w�҂ƉȊw�҂����𐶂񂾃X�C�X�̉ƌn�̂����A�����̍ŏ�������B
��Ƃ͑S�ʓI�ɂ��񂩑������Œ݂��ɒ������������B
�����Ȃ�ґ����A80�΂𒴂���3�l�𐔂���B
���R�u�͋���E�҂̗\��Ȃ���A�������琔�w�ɐ�S���邪���̓��@�͂��Ƃ͋��R�ł������B
���̋C���͒_�`���i�q�|�N���e�X��4�̉t���ސ��ŁA����I�œ{����ۂ��U���I�j�ŗJ�T�^�ł���B
���������s���͂ނ���x���B
��������邪���̑���ȑԓx�͋ɓx�ɒ��������B
���񂩂Ƒ΍R�S�͕K�R�B
���w�҂Ƃ��ẴI���W�i���e�B�[�ƍ˔\�͍ō����x���B
�t�����X�̃A�J�f�~�[����B

B {\gt ���n���E�x���k�[�C}�@�����͏��Ƃ̓��̗\��ł��������A
�i�H��ύX���m���R�n�Ȋw�Ɖ��w�ɐi�ށB
�t�����X�E�A�J�f�~�[����i�_�����x�[���ɂ��̐l�^������j�B
�ȉ�5�l�̑c�ƂȂ�B
�m���F�������w�̋Ɛт͊֐S�̑Ώۂ���O��Ă���B�n

N.{\gt �j�R���X�E�x���k�[�C}�i���N31�΁j�@��͂葽��Ȑ��w�I�˔\�Ɍb�܂�邪�A�Z���g�E�y�e���u���O�ɂĚ�܁B
���n�̐V���A�J�f�~�[�Ɍ��ʂ�^������l�B

N.{\gt �_�j�G���E�x���k�[�C}�@���Ȉ�A�A���w�ҁA��U�w�҂ŁA���̗͊w�̗L������������B
���ɑ��n�B
5��̎�܂����邪�A1��͕����n���������A�q�̐����Ɏ��i���������B
�t�����X�E�A�J�f�~�[����i�R���h���Z�ɂ��̐l�^������j�B�@(�ȉ����j�B

\noindent
\Fig[�q�n�}�r�q�ʐ^�r]{\textwidth}{10\baselineskip}

\section{�ƋƂ͖�폤�A�����ė���鐸�_�͉��v�h}

�x���k�[�C�Ƃ̂��������̃��[�c�́A�k������ł́A
17���I�̃x���M�[�i�����Ƃ��Ă̓X�y�C���̃l�[�f�������h�j�̃A���g���[�v�݂̖�폤�ł�����
�i������ȑO�ɁA�I�����_�̃A���X�e���_���o�g�Ƃ̏��������邪�A�m���łȂ��j�B
�������d���ꂽ�����̓��[���b�p�ł͎Y�o���ꂸ�A���m�f�՗R���̖��[���i�͂���߂č����t�����l���傫�������B
��Ɨǂ��䗦�ŗL���ɓ�����������Ă������炢�ł���B

�ƋƂ���폤�ł������w��ɂ́A���E�j�̋��ȏ��ɂ��o�Ă��郈�[���b�p�Ɠ��m�Ƃ̍����f�Ղ̗��j���������B
���E�j�Ŋw�Ԃ悤�ɁA1453�N�r�U���`���鍑�i�����[�}�鍑�j�̎�s�R���X�^���`�m�[�v�����C�X�����̎x�z�Ɋח����A
�I�X�}���E�g���R�鍑�����m�Ƃ̌��՘H�̊Ԃɓ��������ʁA
���m�f�Ղ͒n���C���S����V�q�H���J���Ɛ肵���X�y�C���A�|���g�K���̎�Ɉڂ����B
���n�A���g���[�v�ɂ̓I�[�X�g���A����уX�y�C���̗L���ȃn�v�X�u���N�Ƃ��N�Ղ��Ă�������A
�A���g���[�v�͂��̖f�ՏW�ϒn�Ƃ��Ĕɉh��搉̂����B

�u�n�v�X�u���O�Ɓv�Ƃ����΁A����͉��邪���̃}���[�E�A���g���l�b�g�𐶂݁A
�܂����[�c�@���g�A�x�[�g�[�x���A�V���[�x���g�̌����ȁA
���n���E�V���g���E�X�̃����c��A�z���邠�̉؂₩�ȃE�B�[�����v���‚����A
�����n�v�X�u���O�鍑�̎x�z�͍L�͂��‹���ŁA�����ȗ��������[���b�p�̑啕���̎�u���S�[�j����Ƃ̍�����ʂ��āA
���̃I�����_�E�x���M�[�̒n��܂ł����̗L���Ă����̂ł���B
�����ցA���łɑO���I�Ɏn�܂����@�����v�́A�@���A�����A�Љ�A�����ɑ傫�ȉe���������炵���B
���ƂɁA�I�����_�A�x���M�[����уh�C�c�ł́A
�����ȗ��̃��[�}�E�J�g���b�N���i�����j�ɑ΍R���ĉ��v�h�v���e�X�^���g�i�V���j�̐Z���͑傫���A
�l�X�̐��_�̊o����ʂ��āA�����琭���A�Љ�𓮂����G�l���M�[�ƂȂ��Ă����B
����ɑ΂��ăn�v�X�u���N�Ƃ̑�X�N�傽���̓��[�}�E�J�g���b�N�̌싳�҂�C���āA���܂��܂Ȉ������������̂ł���B

�x���k�[�C�Ƃ͐V���k�Ƃ��ĉ��v�I�Ȑi��̋C�ۂ������Ă�������A
���n�̎x�z�҂ɂ��V���i�J�����@���h�j�����͋����I�A�ꐧ�I�Ɗ�����ꂽ�B
���łɈ��@�������u�O�\�N�푈�v�i1618--1648�j�͏I�����@���n�}�͈ꉞ�̈�������Ă����̂ŁA
�x���k�[�C�Ƃ�����17���I�ȍ~�X�C�X�̃o�[�[���ɐV�������Z�̒n�����o�����B
�X�C�X�͒����ȗ��̒����Ɨ��̐킢�̖��A
�O�\�N�푈�̍u�a�i�E�F�X�g�t�@���A���j��Ɨ��������Ƃ����ւ荂�����‚�����Ƃ����V�����ł������B
�悭�m����悤�ɃX�C�X�̉i�������͂���ȍ~�����܂ň�x������ꂽ���Ƃ͂Ȃ��B

�o�[�[���s���ɂ����Ă��x���k�[�C�Ƃ͏d�����Ȃ��A
���S�̎s�X���ɂ��u�x���k�[�C�ʂ�v(Bernoullistrasse)�����邱�Ƃ͂��łɏq�ׂ��B
�X�C�X��{���Ƃ����@�����v�҃J�����@�����ꎞ�o�[�[��������̒n�̈�‚Ƃ������A
���̃o�[�[���̒m�I�A���_�I�`���͐V���k�x���k�[�C�Ƃɂ͂ӂ��킵�����̂ł������B
�J�����@���̋֗~�I�ȉ��v��`�͑����ɉ՗�ŁA
�u���e�v�̐��_���Ȃđ΍R�����G���X���X�͐��Ɂu�J�����@���΃G���X���X�v�̑Ό��ł��m���邪�A�������o�[�[���Ő��U���߂����B
���̂悤�ɏd�w���Ȃ��_�w�I���͋C�̒��ŎႫ���ɏC����ς񂾂̂��x���k�[�C�Ƃ̐l�X�ł���A
����͏I���ς�邱�ƂȂ����̐l�X�̐��_�̔w���ƂȂ����B

\noindent
\Fig[���@���ʐl���W�@��]{\textwidth}{10\baselineskip}

\section{�Ȃ�$dx=0$�ł͂Ȃ��̂�}

�{���̖ړI�ł͂Ȃ����A�o�[�[���̂�����l�̃r�b�O�E�l�[���̓I�C���[�ł���B
�I�C���[�͏��N���ォ��x���k�[�C�Ƃɏo���肵�A�_�j�G���E�x���k�[�C�Ƃ͂��ꂱ���e�|�n�̗F�f�ł������B
���n�����ƒ닳�t�Ƃ��A�_�j�G���Ɗ�����ׂ��B
���ƂɘA�Ȃ�҂Ƃ��ăI�C���[�͏I���v���[�����p�Ɖ]����قǐM�‚ɓĂ��h����Ȑl�ł������B
�w�L���X�g���l�����T�x�ɂ��u�I�C���[�v�̍�������B

�l�̏@����̐M�O�͊w��ɂ��e����^����B
������$dx$�͖{�����C�v�j�b�c�䂸��ł������B
���w�j�Ɖ����Б��Y�̓��C�v�j�b�c�̐��w���u�_�I���w�v�ƕ]�������A�������A
��̃C�M���X�̊ϔO�I�o����`�N�w�҃o�[�N���[�ɉ]�킹��Ȃ�A
�����ɏ�����$dx$�Ƃ͂ǂ��l���Ă�����0�ɑ��Ȃ�Ȃ��̂ł͂Ȃ����B
���Ƃɓ�K���֐�$d(dx)$�‚܂�$(dx)^2$�Ƃ͉����Ƌ򂢉�����o�[�N���[��
�I�C���[�͏������������i�_���ɂ��Ȃ炸�j�_�w�I�ɑނ��Ă���B
�l�̗L���v�l�Łu�����v�܂Ŋ��S�ɉ��؂邱�Ƃ͂ł��Ȃ��B
�������̊T�O�ɂ͐l�Ԃ̊ϔO�v�l����͋ߕt����_�I�[��������߂��Ă���B
�������ɐl�Ԃ̊ϔO�ɂ��L���v�l�ɏ]���Ȃ�o�[�N���[�͐������A$dx=0$�ƂȂ��āA
���Z�̐��w�̐搶������������ʂ݂̂Ȃ炸�A���݂̔����ϕ��w�͂̂�������j�]���Ă��܂��B
�I�C���[�����҂͂��̕s�Žv�c�ɂ܂��܂��������邾�낤�B
�S�����ϕ��w��_�w�I�x�_�݂̂ŋ~�ς����I�C���[�̗͗ʂɂ͋��Q����B
�l�̗L���v�l�ɍ����I�Ȕ��f��~�𖽂���@���͈�ʂɍl������قǕs�����Ȃ��̂ł͂Ȃ��B

\section{���[�y���e���C}

�x���k�[�C�ƂƐe�D������I�C���[�ɂ��e����^�����̂́A���������l�I�ɂȂ邪�A
�������o�[�[���̕����w�ҁA���w�҂Łu�ŏ���p�̌����v�̃��[�y���e���C�ł���B
���[�y���e���C�́A�K�����C�A�z�C�w���X�A�j���[�g���A���C�v�j�b�c�A�x���k�[�C�ƁA
�I�C���[�ɋ��ʂɌ����鎞��v�z��T�^�I�ɑ̌������l�ŁA�_�́u�ۗ��v�Ƃ��Đ��E��
�őP�A�ŗǂ̏�Ԃɑn�������ƋK�肵���i���E�G�w�����w�j�x���Q�Ɓj�B
���ہA���ꂪ�ϕ��@�̎v�z�I���^�ƂȂ����B
���̔��z�͂��܂�ɂ��_��I�ŁA���j�I�ɂ͖Y�p���ꂪ���ł���B
���ہA�͊w�̃e�L�X�g�ɂ��ꕔ���������͂�o�����Ȃ����A
�傫�Ȕ��z�̎n�܂�͂����܂��ł͂����肵�Ȃ����Ƃ����΂��΂ł���B
���n���E�x���k�[�C�͍ŏ��̕ϕ��@���̈�‚Ƃ��āu�ő��~�����v�̉ۑ��^�����̉��̓��o�ɍv���������A
���܂��������Ƃ͂����A���[�y���e���C�A�����ăx���k�[�C�ƁA
�I�C���[�̃o�[�[���g�ɂ���Ďn�܂����ϕ��@�̓��O�����W���Ɏ󂯌p����A
��͗͊w��͊w�n�̌����Ƃ��ĕ����w�̑S����ł̗L�͂ȕ��@�Ƃ��āA
����ɂ͐����o�ϊw�ɂ����Ď��ԍœK���̕W���I���_�Ƃ��Č��݂Ɏ����Ă���B

�������A�u���O�����W���֐��v�A�u�n�~���g���֐��v���������ɂȂ�����͗͊w�ł́A
�{�ƍŏ���p�̌����͈�藝�܂ŗ��Ƃ����߈��𖡂���Ă���B
����ɂ́A���O�����W�������[�y���e���C������������Ƃ��`�����邪�A����΂���ł͂Ȃ��낤�B
���[�y���e���C�̔ӔN�̏Z�����܂��͍������o�[�[���̒��S�s�X�̈�p�Ɍ�������B

\noindent
\Fig[�q�o�[�[���ʐ^�r�q�n�}�r]{\textwidth}{10\baselineskip}

\section{�x���k�[�C�Ƃ𐶂ݏo��������}

���������Ă��̂��Ƃ͐�����B
�M�҂͐��w�̎g�p�҂ł��舤�D�҂ł��邪�A���w���߂��邢�����͋C��y����������Ă�������������̂́A
�G�z�ł��邪�A�R���ė���Ƃ��낪�Y����Ă��邩��ł͂Ȃ����Ɗ�����B
��㐔�w�̎w���I�n�ʂɂ�������i���g���w���㐔�w�̊�b�T�O�x�i1944�j�ɏq�ׂĂ���Ƃ�����Љ�悤�B

\begin{quote}
---�T�O�̌`���I������^����̂ł͂Ȃ��A���̗R���ė���Ƃ���A���̐��w�S�̂ɂ�����݂肩�������A
�����Ắu�w��v�ɂ����邻�̒n�ʂ������炩�ɂ���̂ɖ𗧂‚̂��ڕW�ł���B
����͏��Ȃ��Ƃ����w���̂��̗̂��j���Ȃ݂Ȃ���Ȃ���Ȃ���΂Ȃ�Ȃ��B
���j�̔��Ȃ͔���ȊՎ��Ƃł͌����ĂȂ��B
�w��̖{���̔c���̂��߂ɂނ���K�R�I�Ȏ�i�ł���B
���‚͌ÓT�ɂ����đn���҂̌��ɐڂ���̂͐S�y�����Ƃł���B---
\end{quote}

���w�҂łȂ��Ƃ������̊֐S�����������n���҂̌��ɐڂ���̂͊y�����B
�n���҂����̎���ɂ͂��̎�����L�̕��͋C�����肻�̒��ɑn���҂������������Ƃɂ́A���䂩�����C����������B

�ł́A�x���k�[�C�Ƃ̐l�X��I�C���[����ł������͋C�͂ǂ̂悤�Ȃ��̂ł������낤���B

---�����ʂɊ��‚��~�߂��Ƃ��Ă������҂̏@���I�����͉����̐��ł��ނ�̕����w��̊����ɉe����^�����B
������񌤋��Ҍl�̌����͌����̋���̋��`�Ƃ͂������ɕʌ‚̂��̂ł���A
�ނ���N�w�I�ɉe�����ꂽ�Ƃ݂�ׂ����̂ł��낤�B
�P�v���[�A�f�J���g�A���C�v�j�b�c�A�j���[�g���݂͂Ȃ��̎����𕠑��Ȃ��q�ׂĂ���B
�����18���I�Ɂi���[�y���e���C�́j�ŏ���p�̌����̂���������������B
���̎����ɃJ���g�̓N�w���Ȋw�I�F���Ə@���I�M�‚̊��S�ȓƗ�����錾����
�i�w���������ᔻ�x�͉Ȋw�҂̂��߂ɏ����ꂽ�ʂ�ے�ł��Ȃ��j�B
����ł��̎������ɂȂ�ƕ����w�̒���̒��ɏ@���I�Ȃ��̂͑S�R��������Ȃ��Ȃ�B

������Ƃ����Č����Č㐢�̎��R�Ȋw�҂̌����������S�̉��[���ނ�̏@���S��
���т‚��Ă��Ȃ������Ƃ������_�ɂ͂Ȃ�Ȃ��B
�w���̐^���̑̌������牻�̈Ӗ��ɂ�����theoria���Ȃ킿�u�_�̐ۗ��v�ł���Ƃ�������́A
�܂��ɂ����̂Ȃ���{\gt �ō��̂��̂ɑ΂���S����̌Ăт���}�ɈႢ�Ȃ��B
�m���Ɍ������Ă��ꂪ���p�����Ƃ����l�����邱�ƂȂ��ɁA�Ђ��ނ��ɓw�͂��邱�Ƃ�
�u����N��ʂ��Đl�Ԃ̖{���I�X���ł���A�l�Ԃ̍��M�Ȗ{�������̏ے��v�iK�D���X�p�[�X�j�ł���
�iv.���E�G\footnote{
1953�NWatson, Crick�����̊j�_�̓�d�点��̍\���𔭌������Ƃ�X�������wXray-
Crystallography�̑傫�ȏ��������������A����X�������w��1914�N���E�G�̌������ȂĎn�܂�B
���E�G�������w�j���u������x����̌����̂��ƂɁv���������̂��w�����w�j�x(1947)�ł��邪�A
�����ɏq�ׂ�ꂽ�Ȋw�҂̐��_�̓x���k�[�C�Ƃɂ����o�����B}
�w�����w�j�x�j�B

\section{����͎n�܂�Ȃ��n�܂�i�o�����{�C���j}

�����͉]���Ă��A�̑�Ȃ��Ƃ���̎n�܂�͂��΂��΃n�b�L�����Ȃ��B
���̂��Ƃ̌`���͌`������悤�ł��Ă���Ƃ͂킩��Ȃ��B
�n�܂�̕s���S�����p���Ă��ꂩ��N���邱�Ƃ̈̑傳�𘺂߂����Ă���B

�x�[�g�[�x����\ruby{������}{�V���t�H�j�[}�����ׂđ�5�ԁw�^���x�̂悤��
�e�h�A�h�A�h�A�h�[���f�Ƃ����������‚킯�ł͂Ȃ��B
��9�ԁw�����t���xmit Schlusschor�̖`���ł́A�ǂ�����Ƃ��Ȃ��A�����ł������ȁA�ɂ��������炸
�������܂��΂͂�����Ɛ����ɗL�@�I�Ɍ������錷�̂����߂��ŁA���������n�܂��Ă��邫�������_��I�ɂق̂߂������B
It begins without beginning. \, 
����w���҂͂������B

��ȉƂł���]�_�Ƃł�����������O�Y�̕]�F

\begin{quote}
---�����̒掦�ɐ旧���Č���鏘�́A�����ɑ΂�����߂ėL�@�I�֘A�������A
���‚܂����I�Ȑ��i���I�݂ɕ\�����Ă���_�ɂ����Ă����ꂽ���̂ł���B
����͂܂�A����E���Ƃ̍�銮�S�ܓx�ɂ���Ďn�߂��邪
�a���Ƃ��đ�O���������Ă��邽�߂ɂ͂Ȃ͂��s����ȁi���������ď��I�ȁj�C�������o���Ă���B
���̌ܓx�������s�A�j�V���łȂ��Ă��邤���ɁA�����`���̍ł��d�v�ȉ��`����������Ă���B---
\end{quote}

\noindent
\Fig[�y��]{\textwidth}{10\baselineskip}

�ق�Ƃ��������΁A���ϕ��̗��j�̓x���k�[�C�Ƃ��ȂĎn�܂�A�قǂȂ��o�g�̃I�C���[�����������������B
�j���[�g���ƃ��C�v�j�b�c
---�ǂ���ł��邩�͂Ƃɂ�����---
�ŁA\kenten{�Ƃɂ�����}���ϕ����n�܂������Ƃ͌����̗��j�N�\��̎����ł͂��邪�A
����𕶎��ʂ�́u�n�܂�v�ƌ���ɂ͂��܂�ɂ��ّ��ł���A
���҂Ƃ͂�������Ă���B
�j���[�g���́w�v�����L�s�A�E�}�e�}�e�B�J�x�͔��ϕ��w�̋��񐹏��u�n���L�v�Ƃ������悤���A
�������Ȃ���f�J���g�̊v���I�ȑ㐔�L�@�‚܂�u�����v�͂܂��قڊF���A����Җ{�l�̋Ό������A
�ĉ�ȃL�����N�^�[���������̂悤�Ɂw�􉽊w���{�x�̘_�ؗ��V�����̂܂ܓ��P���Ă���B
�����̓��e����A���ʐl�ɂƂ��Ă��ǂ߂���̂ł͂Ȃ��A���ꂾ���ɔ��ϕ��w�e�L�X�g�Ƃ������̂ł͂Ȃ��B
��i�͂ނ��둑�����ɖ����Ă���A�n���[�i���̖��̕t�����d���̔����ҁj�ɂ��u�̐l�^���v�Ɍ���悤�ɁA
�_�̑n���̒����A���̐ۗ��ƌo�ς�̂�����̂Ƃ��āA���j�コ��R�ƋP���Ă���B


�����A�j���[�g���ɑ΂��D�搫�𑈂����C�v�j�b�c���w�����x�ɂ���Ĕ��ϕ��w�̓�����\���������B
�����A��i�͑S�̂Ƃ��Ė��m���������A
���_�̌`���Ȃ��ɂ͌p���ҁi�����j�x���k�[�C�A�I�C���[�̍˔\�������˂΂Ȃ�Ȃ������B
�����A�x���k�[�C�ɂƂ��Ắu�����Ȃ��̂悤�Ȃ��́v�Ɍ������Ƃ����B
�ɂ�������炸�A����̔��ϕ��ɂ‚Ȃ�����Ƃ��ẮA�ϕ��i���͘a�j�L��$\int $\footnote{
s�̕ό`�B���e����\�L��s�͏㉺�Ɉُ�ɒ����L�сAf�Ƃ̌�����������ł������B}
�̓����i���C�v�j�b�c�j�A�p��u�ϕ��v�̓����i�x���k�[�C�j�A�����������̋��ρi���j�A
�̌n�w��������́x�̊����i�I�C���[�j�ɂ���āA
���`�Ƃ��Ă͖����Ɂe���C�v�j�b�c�̐��f�̏�Ɍ��オ����Ƃ����Ă悢�B

�w�҂̃p�[�\�i���e�B�[�������̔g���v�f�����A�����͈̑�Ȏn�܂�̒��ł̓G�s�\�[�h�ǂ܂�ł���B
�j���[�g���������u���i�������l�v�Łi�z�[�L���O�j�A�u���ɐS����j�����v�i�j���[�g���B���C�v�j�b�c�f�̐܁j�Ȃǂ�
���₨��Ƃ����v�킹��B
���̃j���[�g�����A�u�C�M���X�l�̏����͕K�v�Ƃ��Ȃ��v�Ƃ����x���k�[�C�Ƃ͂ǂ����B
�������̎��i�ƓG�ΓI�ȑ΍R�S�Ɠ��փP���J�����ʓI�ɂ͌�葐�ɂ����Ȃ��B
���オ����΁A�x���k�[�C�ƃj���[�g���̊Ԃɂ���M��K�₪�������悤�ŁA��͂�w�҂̐����͂ӂ‚��l�̊֌W����͐����ʂꂸ�A
����ł��ĕ\�ʏ���͕����u�Ă��Ȃ��_�́A�w�҂Ƃ����l�ނ̂��肪�������T�ł���B
���ہA�w�҂���J�͑������A���̍K�����Ȃ���Έ̑�Ȏ��т͎c���Ȃ��B

\chapter{���R�u�E�x���k�[�C}

\section{���R�u�E�x���k�[�C}

���R�u�E�x���k�[�C�i�����j���ȂāA�x���k�[�C�Ƃ̓V�˗��͎n�܂�B
���j�R���X�E�x���k�[�C�i�����j�̓��R�u�ɐ_�w�҂ɂȂ邱�Ƃ�]��ł����B
�u�_�w�v�itheology�j�͓��{�l�ɂ͑z���ł��Ȃ����A�u�_�v�ɂ‚��Ă̊w�A
���ƂɃL���X�g���̋��{�A���j�A���`�A�M�����̘_���ɂ‚��g�D�I�Ɍ�������w��ł���B
�L���X�g���͌Ñ�A�����ȗ��A�������[���b�p�̐��_�A�����̊�Ղł���������A
�w��̑̌n�ɂ����Ă����ׂĂ̊w��̕M���i�̏d���ƌ��Ђ������Ă����B
���E�҂����O�̐M�‚̗ǂ�������A����Ƃ��đ��h���W�߁A�Љ�I�АM�ɂ��������̂�����A�E�ƂƂ��Ă����肵���I���ł������B
�ߑ㏉���̑����̐��w�҂͐”N����ɐ_�w�̏C�����󂯁A���̉F���_�͔��z�̊�b���{�ƂȂ�����͑����B
���Ƃɐe���_�w���E�҂̃L�����A��]��Ŋw����^���A�������Ȃ���{�l�̊֐S���ӂ���ݐ_�w�ɖ����ł��Ȃ��Ƃ����i�H�������܂�ł������B
�����Ƃ����w�҂ƂȂ��Ă�����_�w�_���̒�����c�����̂̓I�C���[�A�����Ăقړ�����̊m���_�̃x�C�Y�i�����j�ł������B
���Ƀx�C�Y�͎��g���V���i�v���e�X�^���g�j�̖q�t�ł���A�I�C���[���q�t�̎q�ł����ďI���M�[���l���𑗂����B
���R�u�͐��E�҂������w�̓���I�ԁB

���āA���R�u�E�x���k�[�C�Ƃ����΁u�A���X�E�R���C�F�N�^���f�B�vArs Conjectandi�i�󂵂āw�����p�x�j��
�����m��l���m��m���_�̈��ÓT�����A���̑�삨��т��̑�II���Ɋ܂܂��_��I��
---�����͌����Ȃ���---
���Ȃ����w�̂����������Ɏp��������u�x���k�[�C���v�͌�ł܂Ƃ߂Ă������q�ׂ邱�Ƃɂ��A
�����ł́A���R�u�E�x���k�[�C�̍L�ĂȋƐяЉ�̈�[�Ƃ��āA
�ނ̔��z���ȂĎn�܂�탈�n���A����ɃI�C���[�A���O�����W���Ɏ󂯌p����č����́u�ϕ��@�v�ƂȂ����A
�ŏ��̉ۑ�u\ruby{�ő��i�Z�j�~����}{�u���L�X�g�N���[�l}�v�����������茩�Ă������B

�ϕ��@�͌��݂́u��͗͊w�v�u�͊w�n�v���x���鐔�w�I���@�ƂȂ��Ă��邪�A
�ŏ��͊􉽊w�̋Ȑ��_�Ƃ��ďo�������B���̒a���ɂ͎��̂悤�Ȋ􉽌��w�̑O�j������B

\begin{quote}
---���͕���A���̓_$\mathrm{P}(0, \, 1)$����o������A�̒����E���ɐi�݁A
����B�Ƃ̋��E$x$����̓_$(x, \, 0)$�ŕ���B�֓��˂��A
����B���ł͋��܂��ē_$\mathrm{Q}(1, \, -1)$�֒B������̂Ƃ���B
���܁A���E�֓��˂���p�i���ˊp�j��$i$�A
���E������܂��ďo�čs���p�i���܊p�j��$r$�A
����A���AB���ł̑��������ꂼ��$v_\mathrm{A},\, v_\mathrm{B}$�Ƃ���Ƃ��A
���̐i�HP��X��Q�����肹��---
\end{quote}

����X�i���Ȃ킿$x$�j���߂�@���͊􉽌��w�Łu�X�l���̖@���v�Ƃ����A
\[
\frac{\sin i}{\sin r}=\frac{v_\mathrm{A}}{v_\mathrm{B}} %\label{1}
\]
�ł���B
���Ƃ���$v_\mathrm{A} > v_\mathrm{B}$�̂Ƃ��͐}�̔@������PQ���㑤�ɋ��Ȃ���
�i$v_\mathrm{A} < v_\mathrm{B}$�̂Ƃ��́A�t�Ɍo�H�͉����֋��Ȃ���j

\noindent
\Fig[����2��]{\textwidth}{10\baselineskip}

�v�񂷂�ƁA���͑����̑傫�������֋��Ȃ��A���A�o�H�̕����i�p$i, \, r$�j��$\sin$�͑����ɔ�Ⴕ�Ă���B

�z�C�w���X�ƃt�F���}�[�̓X�l���̖@���͌o�H$\mathrm{P} \to \mathrm{Q}$�̓��B����
\[
 \frac{\mathrm{PX}}{v_\mathrm{A}}
+\frac{\mathrm{XQ}}{v_\mathrm{B}} %\label{3}
\]
���ŒZ�ƂȂ�悤��P��X��Q�����܂�Ƃ���΁A������邱�Ƃ��������B
���ہA�㎮��$x$��
\[
 \frac{\sqrt{1+ x^2}}{v_\mathrm{A}}
+\frac{\sqrt{1+(1-x)^2}}{v_\mathrm{B}} %\label{4}
\]
�ƕ\�킵�A�����$x$�Ŕ��������
\[
 \frac{  x}{v_\mathrm{A} \sqrt{1+ x^2}}
-\frac{1-x}{v_\mathrm{B} \sqrt{1+ (1-x)^2}}=0 %\label{5}
\]
����A��������
\[
\frac{\sin i}{v_\mathrm{A}}-\frac{\sin r}{v_\mathrm{B}}=0 %\label{6}
\]
���o��B

���Ƃ���A����C�AB�����Ȃ�$v_\mathrm{A} : v_\mathrm{B}=1.333:1$�ł����āA
\begin{align*}
x     &= \\
\sin i&=\qquad (i= \qquad ) \\
\sin r&=\qquad (r= \qquad )
\end{align*}

���̂悤�ɁA�����ɂ���Čo�H�i���邢�͊֐��̃O���t�j���߂���@���u�ϕ��@�v�ł���B
���̗�ł́A��͌o�H�ɂ����鎞�Ԃł���B
J�E�x���k�[�C�͂��̌��̐i�H����Ɋւ���z�C�w���X�ƃt�F���}�[�̌��ʂɃq���g�𓾂āA
���̂�����Ȑ��ɉ����ė�������Ƃ��́u�ŒZ���Ԃ̖��v���������B

�ŒZ���Ԃ̖��͏]�O���~�ʂ̂悤�Ȍ�������i�K�����I�j���o����Ă������ł���B
�������Ɍ��̋��Ȃɂ����āA�o�H�������̑傫�����֋��Ȃ��邱�ƂŁA�o�ߎ��Ԃ�Z�����邱�Ƃ��ł���悤��
---���傤�ǁA���H�ʼn^�]����Ƃ��A���������I�ɉ����ł��������H
---�n�C�E�F�C�Ƃ��^�[���p�C�N�ȂǂƉ]����---
�ɏ�邱�ƂŁA���Ԃ�Z�k�ł���悤��---
���̗����̊􉽊w�ɂ����Ă��A�d�͉����x�ɂ���ĉ����ɂ����鑬�����傫���̂ŁA�����o�H�Ƃ��Ď΂߂ɐ��`�ɗ�����������A
����ɉ����ɋ��Ȃ��邱�ƂŁA�������Ԃ��ŒZ�ɂł���B
���炽�߂āA���̍ŒZ���Ԃ̋Ȑ��ł���u�ŒZ�~�����v�𐳂������߂邱�Ƃ��ۑ�ƂȂ�B

���́A�탈�n���E�x���k�[�C����Â����e�R���y�����f�Ō������ꂽ��
�i�Z�ɑ΂��钧��̓��S�̈Ӑ}���������j�A���R�u�����n�������������ɒB���Ă����B
���̍ŒZ�~�����̗��j�I�������㕗�ɉ����Ă݂悤�B
�t�F���}�[�̌����̃A�i���W�[����A�����ɔ��������‚̋��ܑw������ƍl����ƁA

\noindent
\Fig[����7�C�}���@Vanner p.165]{\textwidth}{5\baselineskip}

\[
\frac{v}{\sin \alpha}=K
\]
�ƂȂ��Ă���B
$\alpha$�͋Ȑ��̕����i�����ƂȂ��p�j�A$v$�͂��̓_�ł̑����ł���B
���̗̂͊w����A$g$���d�͉����x�Ƃ���
\[
v=\sqrt{2gy} \qquad (y\text{�͐�����������})
\]

�܂��A$y'$������W���Ƃ���ƁA
\[
\sin \alpha =1 \Big/ \sqrt{1+{y'}^2} %\label{8}
\]
����������
\[
\sqrt{1+{y'}^2} \cdot \sqrt{2gy}=K %\label{9}
\]
���ꂩ��A$y'=dy/dx$�������o���ƁA����������
\[
dx=\sqrt{\dfrac{y}{c-y}} dy \qquad
\left( c=\dfrac{K^2}{2g} \right) %\label{10}
\]
�𓾂邪�A$\sqrt{}$ ���̕�����������邽��
\[
y=c \cdot \sin^2 s = (c/2)(1-\cos^2 s)
\]
�ƕϊ�����ƁA�e�Ղɐϕ��i�����������̉��j�͋��߂���
\[
x-x_0=cs-(c/2)\sin^2 s
\]
�ƂȂ�B
$s$�����ϐ��Ƃ���_$(c(s), \, y(s))$�̋O�Ղ́u�T�C�N���C�h�v�icycloid�j�Ƃ�΂�A
���ꂪ�ŒZ�~�����̉��ł���B
���́u�T�C�N���C�h�v�́e�[�~�f�i"-oid"�͋[�`�j���Ӗ�����B
����$x_0\equiv 0$�Ȃ�A
\[
(n-cs)^2+(y-c/2)^2=(c/2)^2
\]
����A$(x(s),\, y(s))$�͒��S��$c/2$�̍��������ԂƂƂ��ɑ���$c$�Ő����ɕ��s�ړ�����~����̓_�ł�����
���]�Ԃ̗ւɔ��˔‚�t���A����𓮉�Ƃ��ĎB�e����Ɠ�����B
$X$�Ɉړ���$cs$���Ȃ���Ή~�ɂȂ�B
������~�ƌ��ԈႦ���K�����C�̌��_���S���I�O��ł͂Ȃ������B

���́A�Z���R�u�����������������Ă���A���n���̒���ɂ���čēx�̎��݂ł��̐����ɒB�������̂ł���B
���ۂ����ɏq�ׂ��V�˓I��@�͒탈�n���ɂ����̂ŁA
�탈�n���̒��z�ɂ������Z���܂����Ă����x���k�[�C�Z��Ԃ̎��i�Ƌ����S�������ɂ�������Ă���B
�ϕ��@�ɂ‚��Ă͂��̒i�K�ł͂����܂łƂ��A��ɔ��W���q�ׂ邱�Ƃɂ��悤�B

\section{�w�����p�x�̐��E}

���R�u�E�x���k�[�C(Jacob Bernoulli, 1654--1705)�Ƃ����΁A
���������́eArs Conjectandi�f(1713)�̑��������Ȃ��Ă͂Ȃ�Ȃ��B
Ars�͉p���(art)���Ȃ킿�u�p�v�Aconjectandi\footnote{
conjectandi �ɂ‚��Ē��A����H}
��of conjecturing���Ȃ킿�u�\���́v�ł���B
�����Ƃ��A�������w�ł́econjecture�f�́u�\�z�v�Ƃ����Ă���B
������ɂ���u�\���p�v�Ƃ��󂵂��邪�A�‚܂�́u�m���_�v�̚���ł���B

���̑��͒��҂̎���i1705�v�j�̏o�łł���A�̐l�̌��e�������Ă������̃j�R���X�E�x���k�[�C�����͂��犩�߂��A
�Z��̔����Ȋ֌W�̒��ł��̘J���Ƃ����Ƃ܂������ɏq�ׂ��Ă���B
�����Ƃ��A���ł��邱�Ƃ͊ԈႢ�Ȃ����̂́A����̍ŏ��̑��ł��邱�Ƃɂ͈٘_������A
�o�Ŏ������߂������������[��(����)�́w���R�̃Q�[�����͗��_�x�̃��x���̍������w�E�����������B
�܂�����قǎ����o�����ďo�ł��ꂽ�h�E���A�u���́eThe Doctrine of Chances�f(�w���R�_�c�c�x)��
�\�����e�̏ڍׁA�������ɂ����āA����i��ł��邪�A���ꂾ����Ars Conjectandi�̐�쐫�͂ނ��낢���������������ł��낤�B
����4�����琬�钘�삪�A�旧�ƒz�C�w���X�́w���R�̃Q�[���ɂ�����v�Z�ɂ‚��āx
De Ratiociniis In Ludo Aleae�ɐG������Ă͂��邪�A
���̌�̊m���_�̔��W�̑b�΂ƂȂ�d�v�T�O�������Ă��邱�Ƃ́A�����������邱�Ƃ͂Ȃ����낤�B
\begin{enumerate}
\item[1.]�u���Ғl�v�̊T�O�𓱂���
\item[2.] ����������ނɁA�u�g�ݍ��킹���w�vcombinatorics�ɂ�鐔���̊�b�������
\item[3.]�u�x���k�[�C���v���`����
\item[4.] �����ł����u�吔�́i��j�@���v���ŏ��ɓ�����
\item[5.] �Љ�I�A�o�ϓI�Ȏ��ۂ̕���
\end{enumerate}

\section{�w�����p�x��I��}


�܂��́A4����̕M���ł���A����͕����ʂ莟�̒ʂ�F

\begin{quote}
---�z�C�w���X�̘_���w���R�̃Q�[���ɂ�����v�Z�ɂ‚��āx���܂ޑ�I���A
���R�u�E�x���k�[�C�ɂ�钍�߂‚�\, 
complectens�@Tractatum Hugenii De Ratiociniis In Ludo Aleae, Cum Annotationibus, Jacobi Bernoullj---
\end{quote}

ludo�́u�Q�[���v�Aalea�͕����ʂ�u��������v(aleae�͂��̑��i�A���i�Łu��������́v)�ł��邪�A
�����ł́u��������v�́u���R�v�Ǝ����I�ɂ�---�J�[�h������---
���`�ŁA���ۃt�����X��aleatoire�́u���R�I�v�u�m���I�v�Ƃ����M�󂪒蒅���Ă���B
�m���_�̎v�z�I���[�c�Ƃ��Ēm���Ă����ׂ��ł��낤�B
�‚��łȂ���A�u�Q�[���v�͍�����von Neumann���̂�����헪�I�u�Q�[�����_�v�̓��e�͊܂܂��A
�قڂ��̊m�������̖ʂ��w���B
de�͑O�u���i�D�i�x�z�j�ő��`�ł��邪�w��薼�ł́u�`�ɂ‚��āA�ւ��āv���Ӗ����A���΂��΁u�`�_�v\footnote{
����͐��w�A���R�Ȋw�Ɍ������󋵂ł͂Ȃ��A�l���A�Љ�Ȋw�̕���ł��c�c�B}
�Ɩ�o�����B
ratiocinium�i�����ł͒D�i�j��ratio�Ɠ������u�v�Z�v�����A�ނ���L���u�l�@�v�u�_���v�̖ʂ��܂ށB

\noindent
\Fig[�\���ʐ^]{\textwidth}{10\baselineskip}

���̑�I���͐�y�i�̓V�˃z�C�w���X���������Q�[���̌v�Z���Љ�icompectens�u�܂ށv�j�A
���̂����Łe���Ȃ炱������f�Ɖ������X�}�[�g�ŋZ�I�I��@��掦�����͂ŁA�Q�[���������@�W�ł���B
���҃p�X�J���ƃt�F���}�[�̖�����i�ƃ��x���͍����A��@���i�������Œn���I�ɂȂ��Ă���B
�e���ɂ͂��ꂼ��m��I�Ɋm������̌`���œ��Ƃ��ė^������_���炵�āA
�������Ɂu�\���v�̖��ɒp���Ȃ����e�ł���B
�ȉ��A���ڂ����f����B

\begin{enumerate}
\item[����I]   ���Ғl�̑㐔���i$a,\, b$�����ɉ”\�ȂƂ��j
\item[����II]  ���Ғl�̑㐔���i$a,\, b, \, c$�����ɉ”\�ȂƂ��j
\item[����III] ���Ғl�̑㐔���i$a,\, b$�e$p,\, q$�̓����ɉ”\���̂Ƃ��j
\end{enumerate}
�ȏ�͏��l�̗��V�ɂ�銨��ŁA���������玩�R�ɓ�����A�����̃e�L�X�g�ɂ���悤�ȓV����̒�`�ł͂Ȃ����������[���B
�܂��u�”\���v��$p:q$�̂悤�ɔ�ŗ^�����A������1�ɋK�i�����ꂽ�m���̊T�O�͈�ʓI�ł͂Ȃ������B
\begin{enumerate}
\item[����IV]  3�Q�[�����ŏ��ƒ��[���ŁA�e2���A1���ŃQ�[�������f����Ƃ��̌����ȓq�������z�̕��@
\item[����V]   ��1�̃v���[���[��1�Q�[���A��2��3�Q�[����v����ꍇ
\item[����VI]  ��1�̃v���[���[��2�Q�[���A��2�̃v���[���[��3�Q�[����v����ꍇ
\item[����VII] ��1�̃v���[���[��2�Q�[���A��2�̃v���[���[��4�Q�[����v����ꍇ \\
\end{enumerate}
����͂��܂�ɂ��L���ȃp�X�J�����t�F���}�[�̉������ȂŘ_����ꂽ�u���z���v�i���邢�́u�q���̒��f���v�j�ł���B
�����ł���͂蕪�z�̌��������L�[���[�h�ƂȂ��Ă���B
���l�Ԃ̎���ł͍ő�̗��Q�֐S�ł������̂��낤�B

���̋@��Ɉ�‚̃G�s�\�[�h���q�ׂĂ������B
�p�X�J�����t�F���}�[�̏��Ȃ�1654�N�̂��Ƃł��������A���傤�ǂ��̔N�Ƀ��R�u�E�x���k�[�C�͐���Ă���B

�ȍ~�͂�������\footnote{
�Ȃ��A�u��������v���O���̂悤�Ɂu�T�C�R���v�Ə����̂͏��߂��Ȃ��B
����͖{�����{��ł���u\ruby{��}{����}�v�ɗR������B
���{�̌Ñ�̖������͏�c����ΐ�̐��A�o�Z���΁A�R�@�t��V���̎O��s�@�ӂƂ����}�b�͂悭�m���Ă���B}
�̃Q�[���ł���B
\begin{enumerate}
\item[����VIII] ��1�A��2�̃v���[���[��1�Q�[���A��3�̃v���[���[��2�Q�[����v����ꍇ
\item[����IX]   ��1�A��2�̃v���[���[�̓x�X�g��1���邢��2�Q�[����v���A��3�̃v���[���[��5�܂ł̃Q�[����v����ꍇ
\end{enumerate}
����͂����܂ł̈�ʉ��ł���ȊO�A�Ƃ藧�ĂăR�����g���ׂ���ނ͂Ȃ��B

\noindent
\Fig[�\1.(2��)]{\textwidth}{10\baselineskip}

\vspace{5\baselineskip}

\begin{enumerate}
\item[����X]    �q����$a$�́A��������𓊂��������߂�6���o���Ƃ��Ɏ擾�ł���B
��������̊e��܂łɎ擾�ł���z�����߂�i���ۂɂ́A$1,\, 2,\, 3,\, 4$��ڂ܂Łj
\item[����XI]   �������A2�‚̂�������𓊂��������߂Ęa12�i6--6�̃y�A�j���o���Ƃ��A�Ƃ���
\item[����XII]  ���������2���6�𓾂邽�߂Ɂm����ɓq���邱�Ƃ��L���A���Ȃ킿�m����1/2�ȏ�Łn�K�v�ȉ�
\item[����XIII] 2�‚̂�������𓊂��A2�l�̃v���[���[���a�����ꂼ��$7, \, 10$�̂Ƃ��q�����𓾁A����ȊO�͕����z������
\item[����XIV]  ���肪���ŁA����͐��6���o�����Ƃ��A������͐��7���o�����Ƃ��A�q�����𓾂�
\end{enumerate}

\begin{enumerate}
\item[���I]   $\mathrm{A},\, \mathrm{B}$��2�‚̂�������Ńv���[���A�a��6�Ȃ�A�̏����A7�Ȃ�B�̏����Ƃ���B
�܂�A�������A����B��������2�񓊂���B
�‚��ŁAA��2�񓊂��A�ȉ����l�Ƃ��A�O�҂��邢�͌�҂����҂ƂȂ�܂ő�����B
A�̌����ݑ�B�̌����݂̔�͂ǂ��Ȃ邩�B
���F10,577��12,276�B
\item[���II]  3�l�̃v���[���[���A12���̃g�[�N���������A����4���͔��A8���͍��ł���A���̏����Ńv���[����F���̂����N�ł��ډB���������܂ܔ���I�񂾎҂������Ƃ��āA�܂��A�ŏ���A�A2�Ԗڂ�B�A3�Ԗڂ�C�������B
3�l�̌����݂͂ǂ��Ȃ邩�B
\item[���III]  A��B�Ƌ����Ă��āA���ꂼ��10���̃J�[�h��4�ʂ�̑g47������A���ꂼ��1����4���I�Ԃ��Ƃ�錾���Ă���B
A�̌����݂�B�̌����݂̔��1000��8139�ƌv�Z�����B
\item[���IV]   �ȑO�Ɠ��l�A��4���A��8���̃g�[�N��12���������AA��B�ɑ΂��A�ډB��������7���̃g�[�N���̂���3�������ł���悤�ɑI�Ԃ��Ƃ�q���Ă���B
A�̌����݂�B�̌����݂ɑ΂����قǂ̔�ƂȂ邩�B
\item[���V]    A��B�͂��ꂼ��12���̃R�C���������A3�‚̂�������Řa��11�Ȃ��A��B�ɃR�C��1����n���A14�Ȃ�B��A�ɃR�C��1����n�������Ńv���[����B
���ׂẴR�C���𓾂��������B
A�̌����݂�B�̌����݂̔��
\[ 244, \, 140, \, 625 \, \text{��}\,  282, \, 429, \, 536, \, 481 \]
�ƌv�Z�����B

�����́A�z�C�w���X����͂Ƃ��������x���k�[�C������̕��@�ʼn���^�������̂ł���B
\end{enumerate}

%%%%%%%%%%%%%%%%%%%%%
\section{�w�����p�x��II��}

\begin{comment}
��1���͐�y�i�œV�˔��̃z�C�w���X���������Q�[���̌v�Z�ɑ΂��A
�e���Ȃ炱������f�Ɖ������X�}�[�g�ŋZ�I�I��@��掦�����͂ŁA�Q�[���������@�W�ł���B
���҃p�X�J���ƃt�F���}�[�̖�����i�ƃ��x���͍����A��@���i�������Œn���I�ɂȂ��Ă���B
�e���ɂ͂��ꂼ��m��I�Ɋm��---�����I�p�@�łȂ��A��̌`����---�����Ƃ��ė^������_���炵�āA
�������Ɂu�\���v�̖��ɒp���Ȃ����e�ł���B
\end{comment}

�Z�J���h�E�p�[�g�́A�����薼��
\begin{quote}
---����Ƒg�ݍ��킹�̏����_���܂ޑ�II��\, Doctrinam De Permutationibus \& Combinationibus---
\end{quote}
�ł���B
���̑�II���́A��I���̔��W�Ƃ������ނ���t�ɁA�\���̊�b�Ƃ��Ắe����Ƒg�ݍ��킹�f�̊�b�����_�̉���ł���B
���ہA���ʂ���u�����_�vdoctrinam�idoctrina�̕����Ίi�j�Ɩ��ł��Ă��āA�܂��͐��������m�ɐ����邱�Ƃ��甭�W���āA
�����鏇��iP�j�A�g�ݍ��킹�iC�j�̕��“I�����𒴂��A�}�`���A�p�X�J���̎Z�p�O�p�`�A���R���̗ݏ�a�A
�x���k�[�C���Ȃǐ��w�҂ɂ͋����[����i�ƌ@�艺������ނ����ԁB
�Z�I�I�Ƃ�����肻�̊�b�I���e�́A�h�E���A�u���i���S�Ɍ��藝�j�A���v���X�i�m���̒�`�A�����t�m���A��֐��j��
�c���Ɋ��􂷂���ݒ肷����̂Ƃ��č����]�����^������B
�����Ƃ��A����͐����I�ʂ̍����I�]���ł����āA���Ƃ��ƍ�҂̍ŏI�ړI�͎Љ��������ɓ��ꂽ���@�v���ł������B
�c�O�Ȃ���A����͍�҂̎��ɂ���ē����΂ɏI������i���O�o�łł͂Ȃ��ꗝ�R�Ɛ��������j�B

���āA�܂��ꌩ�e�m���_�f�I�ɂ͌����Ȃ����A���������̊�b�Ƃ��āA���R���̗񂩂�n�߂�B
\begin{align}
&1,1,1,1,1,1,\ldots 
\intertext{����A�����A�����a}
&1,2,3,4,5,6,\ldots 
\intertext{����ɂ��̕����a}
&1,3,6,10,15,21,\ldots 
\intertext{�����A����𓯗l��}
&1,4,10,20,35,56,\ldots \\
&1,5,15,35,70,126,\ldots \\
&\ldots \ldots \ldots \notag
\end{align}
�̂悤�ɂ���Ԃ��B
��6�i�ȉ��͗�����2�A������
\[
\text{��}n+1\text{�i�̑�}k\text{��}=\text{��}n\text{�i�̑�}k\text{���܂ł̘a}
\]
�̂悤�ɕ��ԁB
�x���k�[�C�́A�������c�ɁA���������炵��

\noindent
\Fig[�}�}��]{\textwidth}{10\baselineskip}

\[ \ldots \ldots \ldots \]
�̂悤�ɎO�p�`�ɕ��ׂ�B
����́A�܂��Ɂu�p�X�J���̎Z�p�O�p�`�v�ɑ��Ȃ炸�A�ʖ��u�񍀌W���v���邢�͑g�ݍ��킹�̐�${}_n\mathrm{C}_r$�̕\�ł���B
�����āA��̕����a�̊֌W�́A����ǂ�
\[
\text{��}n+1\text{��̑�}k+1\text{��}=\text{��}n\text{��̑�}k\text{���܂ł̕����a}
\]
�ƂȂ��Ă���B
���邢��${}_n\mathrm{C}_r$�ł́A����͎��͓���
\[ {}_{n+1}\mathrm{C}_{k+1}={}_k\mathrm{C}_k+{}_{k+1}\mathrm{C}_k+\cdots+{}_n\mathrm{C}_k \]
�ł���A$n=5, \, k=3$�Ȃ�
\[
20=1+3+6+10
\]
�̂��Ƃ��ł���B

���̓����̏ؖ��͓���Ȃ��B
\begin{align*}
a&=\text{dummy} \\
b&=\text{dummy} \\
c&=\text{dummy} \\
d&=\text{dummy} %\label{11}
\end{align*}

���̂悤�Ɋ֐���p���ČW���𔭐���������@�͈�ʂɁu��֐��vgenerating function�Ƃ����A
����e�򓹋�Ƃ��āf�悭�p������i�Ⴆ�΁A���v���X�j�B

�e��I,\, II,\, III,\, IV, $\ldots$�͐}�`�I�ɂ����܂��֌W�ƂȂ��Ă���B
II�̐������X�ƕt�������Ă����ƁA�O�p�`�̗�

\noindent
\Fig[����12���@�iIII�j]{\textwidth}{3\baselineskip}

\noindent
�����������̂ŁAIII�́u�O�p���v�Ƃ�΂��B

�����O�p���iIII�j�����X��1�i�A2�i�A3�i�A4�i�́e�s���~�b�h�f��ɗ��̓I�ɐς�ōs���ƁA���R�A����3������

\noindent
{\gt ������I��} ���̌����͍��Z���w�I�ɂ���Ȃ�������B
�悭�m��ꂽ�֌W��
\[
{}_n\mathrm{C}_k+{}_n\mathrm{C}_{k+1}={}_{n+1}\mathrm{C}_{k+1}
\]
��ό`�����A�����i�Q�����j�̊֌W
\[
{}_{n+1}\mathrm{C}_{k+1}-{}_n\mathrm{C}_{k+1}={}_n\mathrm{C}_k
\]
�𗘗p����B
�����ŁA$n \to n-1, \, n+1 \to n$�̓Y������������$n=k$�ƂȂ�܂ŌJ��Ԃ�
�i������${}_k\mathrm{C}_{k+1}=\circ $�Ƃ���j�A���������ׂĕӁX�������
\[
{}_{n+1}\mathrm{C}_{k+1}={}_n\mathrm{C}_{k}+{}_{n-1}\mathrm{C}_k+\cdots+{}_k\mathrm{C}_k
\]
�𓾂�B
\begin{equation}
1, \, 4,\, 10,\, 20,\, 35,\ldots \text{�i�j} %\label{(IV)}
\end{equation}
�ςݏオ�������l�ʑ̂��������������B
�����IV�́u�l�ʑ̐��v�u�s���~�b�h���v�ƌĂ΂��B
���Ɏl�ʑ̐�IV�̐ςݏグ�ŁAV����������邪�A�����͏������I�Ȃ��̂ŁA���ϓI�ɔF���͂ł��Ȃ��B

�����S�̂Ƃ��āAI,\, II,\, III,\, IV,$\ldots$���e�����́u�}�`���v(figurate numbers)�Ƃ����B
�}�`���͌×�����m���邪�A�x���k�[�C������炪�}�`���ł��邱�Ƃ��w�E���Ă���B
���Ȃ킿�A�u�p�X�J���̎O�p�`�v�̂�������ɂ͐}�`�����_��I�ɉB����Ă���B
������ɂ���A�}�`��V,\,VI,$\ldots$�͋�ԓI�ɐ����ł��Ȃ��̂ŁA���̂���肵�Ȃ��B

\begin{table}[htb]
\begin{tabular}{c|l}
    & �ʏ�̖���               \\ \hline 
I   & �i�萔�j                 \\
II  & ���R���i��������j       \\
III & �O�p��                   \\
IV  & �l�ʑ̐��i�s���~�b�h���j \\
V   & �s���~�b�h�I�l�ʑ̐�     \\
\end{tabular}
\end{table}

�ȏ�̍\�����琔��A���̘a�Ƃ��Ă̐���������B
II�͓�������ł��邩��A���̘a�͂悭�m����悤�ɁA�e�U�ς݂̌����f
\[
\dfrac{n(n+1)}{2}, \quad n=1,2,3,\ldots
\]
�ł����āA���̌����͕����ʂ�O�p���i�U�ςݎ��j�ɑ��Ȃ�Ȃ��B
����${}_{n+1}\mathrm{C}_2$�ł��邩���̕����a�֌W����
\[
{}_{n+1}\mathrm{C}_3+{}_2\mathrm{C}_2+{}_3\mathrm{C}_2+\cdots+{}_n\mathrm{C}_2\quad (n \geqq 2)
\]
�‚܂�A�P����
\[
\frac{(n+1)n(n-1)}{6}
=\sum_{1}^n \frac{k(k-1)}{2} %\label{13}
\]
���o�āA���ӂ̑��삩��$\sum k^2$��
\[
\sum k^2 
=\frac{n(n+1)(2n+1)}{3} %\label{14}
\]
���e�ꔭ�Łf���o�����B
�x���k�[�C�͂��̗v�̂ŏ���$p=3,\, 4,\, 5,\ldots$�ɑ΂��āA�֌W��
\[
{}_{n+1}\mathrm{C}_{p+1}
={}_p\mathrm{C}_p+{}_{p+1}\mathrm{C}_p + \cdots + {}_n\mathrm{C}_p \quad (n \geqq p)
\]
�̍���
\[
\sum_{k=1}^n \frac{k(k-1)(k-2)\cdots (k-p+1)}{1 \cdot 2 \cdots p} %\label{15}
\]
�𕪉����A���R���̗ݏ�a$s_p$�̈�ʎ�
\[
\sum_{k=1}^n k^p=(n\text{��}p+1\text{���̑�����})
\]
�ɒB�����B
�ȉ��͂��̌��ʂł���B

\noindent
\begin{table}[tbh]
\caption{������$\psi$�̌W��}
{\tiny
  \begin{tabular}{rrrrrrrrrrrr}
���� &                  &                &          &           &           &           &       &       &       &        &        \\
$p$  & 1                & 2              & 3        & 4         & 5         & 6         & 7     & 8     & 9     & 10     & 11     \\ \hline
1    & $(1/2)^{\, *}$   & (1/2)          &          &           &           &           &       &       &       &        &        \\
2    & $(1/6)^{\, *}$   & (1/2)          & (1/3)    &           &           &           &       &       &       &        &        \\
3    &                  & (1/4)          & (1/2)    & (1/4)     &           &           &       &       &       &        &        \\
4    & $(-1/30)^{\, *}$ &                & (1/3)    & (1/2)     & (1/5)     &           &       &       &       &        &        \\
5    &                  & $(-1/12)$      &          & (5/12)    & (1/2)     & (1/6)     &       &       &       &        &        \\
6    & $(1/42)^{\, *}$  &                & $(-1/6)$ &           & (1/2)     & (1/2)     & (1/7) &       &       &        &        \\
7    &                  & (1/12)         &          & $(-7/24)$ &           & (7/12)    & (1/2) & (1/8) &       &        &        \\
8    & $(-1/30)^{\, *}$ &                & (2/9)    &           & $(-7/15)$ &           & (2/3) & (1/2) & (1/9) &        &        \\
9    &                  & $(-3/20)^{**}$ &          & (1/2)     &           & $(-7/10)$ &       & (3/4) & (1/2) & (1/10) &        \\
10   & $(5/66)^{\, *}$  &                & $(-1/2)$ &           & 1         &           & $-1$  &       & (5/6) & (1/2)  & (1/11) \\
\end{tabular}}
\noindent
*�F�x���k�[�C���A\qquad **�F������
\end{table}

\begin{align*}
\sum_{k=1}^n 1   &=n, \\
\sum_{k=1}^n k   &=\dfrac{n(n+1)}{2}, \\
\sum_{k=1}^n k^2 &=\dfrac{(2n+1)n(n+1)}{6}, \\
\sum_{k=1}^n k^3 &=\dfrac{n^2(n+1)^2}{4}, \\
\sum_{k=1}^n k^4 &=\dfrac{(2n+1)n(n+1)(3n^2+3n-1)}{30}, \\
\sum_{k=1}^n k^5 &=\dfrac{n^2(n+1)^2(2n^2+2n-1)}{12}, \\
\sum_{k=1}^n k^6 &=\dfrac{(2n+1)n(n+1)(3n^4+6n^3-3n+1)}{42}, \\
\sum_{k=1}^n k^7 &=\dfrac{n^2(n+1)^2(3n^4+6n^3-n^2-4n+2)}{24}, \\
\sum_{k=1}^n k^8 &=\dfrac{n(n+1)(2n+1)(5n^6+15n^5+5n^4-15n^3-n^2+9n-3)}{90}
\end{align*}

���́A�x���k�[�C�̌v�Z�ɂ͌�肪����B
�����‚��̃`�F�b�N�E�|�C���g��
\begin{enumerate}
\item $n$�܂ł̗ݏ�a�ł��邩��A$n=1$�ɑ΂��Ă�1
\item ���o�̏�ŁA$n(n+1)$�͂��‚��ێ�����邩��A����$n(n+1)$�������Ƃ��Ċ܂ށB
����������$n=-1$��0
\item ���L�̗אڔ�̃p�^�[���ȏォ��A$p=9$�ň�J���̌������o�����B
\end{enumerate}

\section{���ܖ͗l�̃t�H�����̕s�v�c}

���̌W���̕\�ɂ͒����������Əd�v�Ȑ��w�I�^���i�x���k�[�C���A���[�}����$\zeta$�֐��Ȃǁj���B����Ă���B
�܂��ꌩ���Ă킩�邱�Ƃ́G

$n$�܂ł�$p$��a$s_p$��
\begin{enumerate}
\item[(1)] $n$��$p+1$���������ł���
\item[(2)] �ō���$p+1$���̌W����$1/(p�{1)$�A$p$���̌W���͂‚˂�1/2
\item[(3)] $p-1$���̍��������Ȃ�
\end{enumerate}
�����$p+1, \, p, \, p-1$����3�����܂Ƃ܂�����p�^�[���ƂȂ��Ă���B
����Ɏ����᎟�̍���
\begin{enumerate}
\item[(4)] $p-2, \, p-4, \, p-6, \ldots$���̍�������
\item[(5)] $p-3, \, p-5, \, p-7, \ldots$���̍��͎c��я�̃X�b�L�������e���ܖ͗l�f�̃t�H�����ƂȂ��Ă���B
\end{enumerate}
%%%%%%%%%%%%%%%%%%
���āA�e�W���ɂ‚��Č�������B
�Ίp���ɂ�$1/2$��$p$���̍��Ƃ��ĕ��сA���̈�i��͍ō���$(p�{1)$���̍���$1/2, \, 1/3, \, 1/4, \ldots$�Ƃ��ĕ��Ԃ��A
���̗אڂ��鍀�̔��
\[
2/3, \, 3/4, \, 4/5, \, \ldots
\]
�ƂȂ��Ă���B
�O�̂��ߑΊp����$1/2$�̍��̗אڍ��̔�́A�������
\[
1, \, 1, \, 1, \ldots
\]
�ł���B
���ɁA�Ίp���̈�i����$p-1$���̍��̕��тł́A�אڍ��̔�́A����ǂ�
\[
3/2, \, 4/3, \, 5/4, \, \ldots
\]
�ƂȂ��Ă��邱�ƂɋC�Â����낤�B
���đz�肳���
$4/2\, (=2), \, 5/3, \, 6/4, \ldots$�ɑ�������$p-2$���̍��͌����A
���i��$p-3$���֍s���ƁA�אڍ��̔�
\[
5/2, \, 6/3, \, 7/4, \ldots
\]
���\����B
����$6/2\, (=3), \, 7/3, \, 8/4, \ldots$�̔�͌����邪�A��i���ł�$p-5$����
\[
7/2, \, 8/3, \, 9/4, \ldots
\]
�������ė���B

\section{�x���k�[�C���̗R��}

���̂悤�ɍ��ォ��E���֌W���̕��т��t�H���[���čs���ƁA�������������P���ȗאڔ�̃p�^�[���������Ă���B
���������āA�e���тɂ����Ă����אڔ�����X�ƘA�悵�A����\kenten{����}��^����΂����̌W���͊��S�Ɍ��肳���B

\noindent
{\gt ���၄} $p-3$���̌W���̕��сi�Ίp����3�i���j�����悤�B
�אڔ�$5/2, \, 6/3,$ $7/4, \ldots$��4�Ԗڂ܂ł̘A���
\[
\frac{5}{2} \cdot 
\frac{6}{3} \cdot 
\frac{7}{4} \cdot 
\frac{8}{5} %\label{17}
\]
����ɏ���$-1/20$���悶�Ď���悭$-7/15 \, (p=8)$�𓾂�B
���l��$p-1$���ɂ‚��Ă��m���߂�Ƃ悢�B

��ʂ�$p-3$���̌W���́A������$B=-1/30$�Ƃ���
\[
B\cdot \frac{p(p-1)(p-2)}{1\cdot 2\cdot 3\cdot 4}
\]
�ł���A���l��$p-5$���̍��̌W���́A������$B=1/42$�Ƃ���
\[
C \cdot \frac{p(p-1)(p-2)(p-3)(p-4)}{1�E2�E3�E4�E5�E6}
\]
�ȂǂŁA�ȉ��A���l�ł���B
�Ȃ��A�����̃��[���́A�Ίp���̒����i��$p-1$���̌W���ɂ��ʂ��A�ȒP�ɁA$A=1/6$�������Ƃ���
\[ A \cdot \frac{p}{2} \]
�ŗ^������B
���̂悤�ɁA�W�������ォ��E���ɕ��ׂ��Ƃ��̗אڔ�Ƀ��[��������ȏ�A
�����W�������肷��̂́A�W���\�̑�1��ɕ���
\[
A=1/6, \, B=-1/20, \, C=1/42, \, D=-1/30, \, E=5/66, \ldots
\]
�i����$F=691/2730, \, G=7/6, \ldots$�j�ł��邱�Ƃ��킩��B
������---�����āA�x���k�[�C�ɂ��΁A�����̐������𕄍����܂߂�---�u�x���k�[�C���v�Ƃ����B
�ŏ��̗�O�������āA������$p$�ɑΉ����Ă̂ݒ�`�����B
\begin{align*}
1^p+2^p+\cdots +n^p
=& \frac{n^{p+1}}{p+1}+\frac{n^p}{2}+\frac{p}{2}                   An^{p-1}
  +\frac{p(p-1)(p-2)}{2 \cdot 3 \cdot 4}                           Bn^{p-3} \\
 &+\frac{p(p-1)(p-2)(p-3)(p-4)}{2 \cdot 3 \cdot 4 \cdot 5 \cdot 6} Cn^{p-5}+\cdots %\label{18}
\end{align*}

�I�C���[���v�Z�����������A���̃x���k�[�C���������Ă����ƁA
\begin{align*}
B_0   &=1, \quad 
B_1    =-\frac{1}{2}�C\text{�i�ȏ�̓I�C���[�ɂ��j} \\
B_2   &= \frac{1}{6}, \quad 
B_4    =-\frac{1}{30},\quad 
B_6    = \frac{1}{42},\quad 
B_8    =-\frac{1}{30}, \\
B_{10}&= \frac{5}{66},\quad 
B_{12} =-\frac{691}{2730},\quad 
B_{14} = \frac{6}{7},     \quad 
B_{16} =-\frac{3617}{510},\quad 
B_{18} = \frac{43867}{798}, \\
B_{20}&=-\frac{174611}{330},     \quad 
B_{22} = \frac{854513}{138},     \quad 
B_{24} =-\frac{236364091}{2730}, \\
B_{26}&= \frac{8553103}{6},      \quad 
B_{28} =-\frac{23749461029}{870},\quad 
B_{30} = \frac{8615841276005}{14322} %\label{19}
\end{align*}

\noindent
{\gt ������II��}

�x���k�[�C���́A�������i���ہA���z���j�ł���$\pi, \, e$�ȂǂƈقȂ�L�����ł��邱�Ƃ������ł��邩��A
�܂����̕���A���q�ɂ͓��ʂ̊֐S����������B���Ƃɕ��q�ɂ‚���
\[
\text{�i��������H�j}
\]
���m������B

�x���k�[�C���̕s�v�c�ȈЗ́i���́j�́A��͊w�i�����ϕ��j�A�����_�ɏo�����邱�ƂŁA
���̂ق�̈�[�́A��ɂ��Љ��
\[
\sum_{k=1}^{\infty} \frac{1}{k^2}=\frac{\pi^2}{6}\qquad \text{�i�o�[�[�����j} %\label{20}
\]
�ŁA����$1/6$���ŏ��̃x���k�[�C��$A$�ł���B
����ȊO�ł�
%%%%%%%%%%%%%%%%%%
\begin{enumerate}
\item $\sum_{k=1}^{\infty} \dfrac{1}{k^{2q}}$�i��ʂ̐������j
\item $\sum_{k=1}^{\infty} \dfrac{1}{k^s}$�i$s=\text{���f��}$�j�m���[�}��$\zeta$�֐��n%\label{21}
\item �t�O�p�֐�$\arctan$�Ȃǂ̋����W�J
\item �I�C���[���}�N���[�����̘a����
\end{enumerate}
�Ȃǂő劈�􂷂邪�A�X�y�[�X�̓s���Ŗ{�͂ł͈���Ȃ��B

�x���k�[�C���͓����x���k�[�C�̒�`��肸���Əd�v�Ȑ��ł����Đ��w�̕������������Ƃ������Ă���B
�u�A���X�E�R���C�F�N�^���f�B�v�͂ӂ‚��m���_�̏��Ƃ���Ă��邪�A���w�̗��j�I��{���ł�����A
��������R�u�E�x���k�[�C�̑傫�Ȍ��тł���B
�o�[�[���ɂ����āA���N���ォ��x���k�[�C�Ƃɏo�����Ă����I�C���[�̈��ɂ�������ΉB�ꂪ���ł��邪�A
�I�C���[�̌����ォ�牟�����̂̓x���k�[�C�E�t�@�~���[�̐l�X�ł���B
���m�̂��Ƃ킴�̂��Ƃ��u������Ė�����v�ł���B

���āA�x���k�[�C���𐶐����郋�[���͂ǂ̂悤�Ȃ��̂ł��낤���B
�I�C���[�̌v�Z�ł́A���łɏq�ׂ��悤��
\[
\mathit{1}, \, \mathit{-1/2}, \, 1/6, \, -1/30, \, 0, \, 1/42, \, 0, \, -1/30, \, \ldots
\]
�ł����āA�x���k�[�C�̒�`�͑�3���ȉ��A$1, \, -1/2$�̓I�C���[�ɂ����̂ł���B
������ɂ���A���ʂɁu�x���k�[�C���v�Ƃ͂������̗̂L�����ł��邱�ƁA��������シ�邱�ƂȂǂ𒘂������F�Ƃ���B
�����Ƃ��A�����̃e�L�X�g�ɂ́A�������Ȃ�����A�����0���J�b�g���ĕt�Ԃ���\����������A�����̍����������Ă���B
�����ňꉞ�A�ȉ��ł͍��ؒ厡�w��͊T�_�x�ɂ��������B

�������ɂ�炸�d�v�Ȃ̂͑�2����$-1/2$�ł���B
����$-1/2$�����Ƃ��đ�W�J���N��̂ŁA�x���k�[�C�̃I���W�i���̒�`�͑�3���ȉ�������ǂ��ł��������̂́A
�����$1/2$�Ƃ���̂ł̓h���}�͌����Ȃ��B

\noindent
\Fig[�f�g�̂�4�s���f]{\textwidth}{4\baselineskip}

\section{�x���k�[�C���̐���}

�x���k�[�C���͗ݏ�a$s_p=\sum k^p$�̕\������R�����邪�A����炪�������钼�ړI�֌W���͂ǂ̂悤�Ȃ��̂��B
���ꂪ�킩��΁A�߂�ǂ��ȗݏ�a�̎��ɋ�J���邱�Ƃ͂Ȃ����A���́A�x���k�[�C���𐶐������֐��͂���B
�����A����𓱂��o�����Ƃ͂Ȃ��Ȃ�---�ł�������---���܂��ł��Ȃ��B
�w��͊T�_�x����ۂ悭������^���Ă���i��64�́j�B
�������A���؂͕��������������̂�$B$�A���ꂽ���̂�$b$�ƕ\�L���Ă���B
�ȉ��ł́A$b$��$B$�ƕ\�L���邱�ƂƂ���B

�����ŁA�����V����I�����A���̂悤�ȍ��ӂ̊֐��������W�J����
$B_0, \, B_1,$ $B_2, \ldots$������ꂽ�Ƃ��悤�B
\[
(\sharp) \qquad 
\frac{z}{e^z-1}
=B_0+B_1 z+\frac{B_2}{2!}z^2+\frac{B_3}{3!}z^3+\cdots %\label{22}
\]
$e^z-1$��W�J���A����
\begin{align*}
&\left( z+\frac{1}{2!}z^2+\frac{1}{3!}z^3+\cdots \right) \\
& \times \left( 
        B_0+B_1 z+\frac{B_2}{2!}z^2+\frac{B_3}{3!}z^3+\cdots \right)=z %\label{23}
\end{align*}
�Ɏ������݁A$z$�A�‚���$z^2, \, z^3, \ldots$�ƌW�������킹�čs���΁A���ۂ�
\[
B_1=1, \, B_2=-1/2, \, B_3=1/3, \, B_4=-1/30, \, B_5=1, \, B_6=-1/42, \ldots
\]
������������B
�����$B_0, \, B_1, \, B_2, \, B_3, \ldots$��
\[
\sum k^p=1^p+2^p+\cdots+n^p
\]
�𐶐����邱�Ƃ��ؖ����悤�B
�i�Ȃ��A��֐��̓W�J�́A�����ς�㐔�I�����̖ړI�̂��߂ł����āA���̎����͖��Ȃ��̂��ʏ�ł���B
���Ȃ��Ƃ��A���v���X�܂ł͂����ł������B
�����������ɗv�������̂̓R�[�V�[�A���C�G���V���g���X����ł����āA���̌�͕��f�֐��_�̒��֎��e����邱�ƂƂȂ����B
�܂������ł����Ă����A�u$\zeta$�֐��v�ɂ����ăx���k�[�C���ɏd�v�Ȗ������^������̂ł���B�j

���������āA����$(\sharp)$���u�x���k�[�C���̕�֐��v�Ƃ����B�i���i�j

������ؖ����邽�߂ɁA���̕�֐����g�������i$x$���܂ށj���̐V���ȕ�֐����l���悤�B
$B_0, \, B_1, \, B_2, \ldots$�́A���R�A�萔�łȂ�$x$�̊֐��ƂȂ�B
���Ȃ킿
\begin{align*}
\frac{ze^{zx}}{e^z-1}
=&B_0(x)+B_1(x)z+\frac{B_2(x)}{2!}z^2 \\
 &+\frac{B_3(x)}{3!}z^3+\cdots %\label{24}
\end{align*}
�����$B_0(x), \, B_2(x), \, B_3(x), \ldots$���u�x���k�[�C�������v�Ƃ����B
���炩�ɁA$x=0$�Ƃ������ƂŁA�x���k�[�C����
$B_0(0)=B_0, \, B_1(0)=B_1, \, B_2(0)=B_2, \, B_3(0)=B_3, \ldots$�Ƃ��ċ��߂���B

���̍��ӂ�
\begin{align*}
&\left( B_0+B_1 z+\frac{B_2}{2!}z^2+\frac{B_3}{3!}z^3+\cdots \right) \\
&\times 
 \left( 1  +  x z+\frac{x^2}{2!}z^2+\frac{x^3}{3!}z^3+\cdots \right) %\label{25}
\end{align*}
�ŁA������E�ӂɓ������Ƃ����ƁA��ʂ�
\[
B_n(x)= B_0x^n 
       +B_1\binom{n}{1}x^{n-1}
       +B_2\binom{n}{2}x^{n-2}+\cdots +B_n %\label{26}
\]
���Ƃ���
\[
B_0(x)=1, \, B_1(x)=x-1/2, \ldots
\]
�Ȃǂ��o��B�i���i�j

�Ƃ����
\begin{align*}
\frac{ze^{z(x+1)}}{e^z-1}
&=\frac{ze^{zx}\cdot e^z}{e^z-1} \\
&=\frac{ze^{zx}}{e^z-1} +z \cdot e^{zx} %\label{27}
\end{align*}
���A$z^p$�̍����r����
\[
B_p(x�{1)-B_p(x)=px^{p-1}
\]
������$p$��$p+1$�Ƃ��āA�e���ꂢ�Ȋ֌W�f
\[
B_{p�{1}(x�{1)-B_{p�{1}(x)=(p�{1)x^p
\]
������ꂽ�B
�����܂ŗ���΁A���Ƃ�$x=0, \, 1, \, 2, \ldots , n$�Ƃ��ĉ�����΂悢�B
���Ȃ킿�A
\begin{align*}
1^p+2^p+\cdots +n^p
=&\frac{1}{p+1} \{ B_{p+1}(n)-B_{p+1}(0) \}+n^p \\
=&\frac{n^{p+1}}{p+1}
 +\frac{n^p}{2}
 +\binom{p}{1}\frac{B_1}{2} n^{p-1}
 -\binom{p}{3}\frac{B_3}{4} n^{p-3}+\cdots %\label{28}
\end{align*}
�ƂȂ�A����悭�x���k�[�C�̗^�������ɓ��B�����B�i�ؖ��I�j

�ȏ�A�u�x���k�[�C���v���g�������u�x���k�[�C�̑������v���o�R����`�ƂȂ������A�I���ȑ������^�p���܂܂�Ă���_�A
���̑������Ȃ��Ō��ʂɒB���邱�Ƃ͔ے�I�Ǝv����B
�܂��A�x���k�[�C���������Ȃ��֐��ŗ^�������Ƃɂ������̈����ڂ���������B
�Ƃ͂����A�ȏ�̏ؖ����ŋ߂̃e�L�X�g���Ɍ��o���̂͋H�Ȃ̂ŁA������₳����������邱�Ƃɂ͈�[�̉��l�͂���ł��낤�B

\section{���̂���}

�����܂ł́u�x���k�[�C���v��V���莮�ɕ�֐��ɂ���Ē�`���A���ꂩ��ݏ�a�𓱂��o�����B
���ہA�قƂ�ǂ̃e�L�X�g���x���k�[�C�����֐��Œ�`���Ă���B
�������A�{���́u�x���k�[�C���v�͂ނ���ݏ�a�ɂ���Ē�`���ꂽ����A����ł͋t�ő����Ɉ�a�����c��B
�����ŁA�i�{���́j�x���k�[�C�����A���̕�֐������R�ɓ������Ƃ��������B

$(k�{1)^{p+1}$�̓񍀓W�J���
\[
(k+1)^{p+1}-k^{p+1}
=\sum_{j=0}^p \binom{p+1}{j} k^j %\label{29}
\]
������$k=n, \, n-1, \, \ldots, \, 1$�������ĉ������
\[
(\sharp) \qquad 
(n+1)^{p+1} -1
=\sum_{j=0}^p \binom{p+1}{j} s_j %\label{30}
\]
�������A$s_j$��$j$��̘a��
\[
s_j=\sum_{k=1}^n k^j \qquad(j=0, \, 1,\, \ldots,\, p)
\]
�𓾂�B
$s_j$��$n$�̑������ŁA$s_0=n$�A�����
$s_1=n(n�{1)/2, \, 
 s_2=n(n�{1) \times (2n�{1)/6,\ldots $�͊��Ɍ����B
�����Ł���$n$�̈ꎟ�̍��ɒ��ڂ���ƁA���̌W��$B_j$�ɂ‚�
\[
p+1=\sum_{j=0}^p \binom{p+1}{j} B_j %\label{31}
\]

������$B_0=1,\, B_1=1/2$�ł���A�܂�$B_2, \, B_3, \, \ldots $��
�i�x���k�[�C�̒�`�ɂ��j�x���k�[�C���ł���B
���ӂ��E�ӂ�$j=1$�̍��ɌJ�����$B_1-1=-1/2$�����炽�ɂ�����
\[
\sum_{j=0}^p \binom{p+1}{j} B_j=0 
\qquad (B_j=-1/2) %\label{32}
\]

{\gt ����A�����ւ��y�[�W����}

\newpage

\section{�x���k�[�C�̓ƒd��}

�u���[�}���\�z�v�ł��܂˂��m����u$\zeta$�֐��v
\[
\zeta(s)=\sum_{k=1}^{\infty} \frac{1}{k^s}, \quad \mathrm{Re}(s)>1
\]
��$\Gamma$�֐���
\[
\zeta(s)=\frac{1}{\Gamma(s)} \int_0^\infty \frac{u^{s-1}}{e^u-1} du %\label{33}
\]
�Ƃ��ĕ\�킳��邱�Ƃ͂悭�m���Ă���B
���͂��ꂪ�x���k�[�C���̊���̏���J���̂ł��邪�A�܂�����������Ă������B

����͈ӊO�ɏ����ŁA�܂�
\begin{align*}
\Gamma(s)
&=\int_0^\infty e^{-t}t^{s-1} dt \\
&=k^s \int_0^\infty e^{-ku} u^{s-1} du 
\qquad (t=ku) %\label{34}
\end{align*}
�ł��邩��A������g����
\[
\sum_{k=1}^N \frac{1}{k^s}
=\frac{1}{\Gamma(s)} \int_0^\infty \sum_{k=1}^N e^{-ku} \cdot u^{s-1} du %\label{35}
\]
�𓾂�B
���̐ϕ����̓��䐔��̘a�́A�������
\[
\sum_{k=1}^N e^{-ku}
=\frac{1}{e^u-1}-\frac{e^{-Nu}}{e^u-1} %\label{36}
\]
�ƂȂ�B
�̂ɁA�ϕ���2�‚ɕ�����
\[
 \int_0^\infty \frac{u^{s-1}}{e^u-1} du 
-\int_0^\infty \frac{e^{-Nu}}{e^u-1} u^{s-1} du %\label{37}
\]
�ŁA���ʂ͂��������Ă��邪�A��2���́A
$e^{-Nu} \to 0\, (N \to \infty)$�ɒ��ڂ���ƁA$N \to \infty$�̂Ƃ�$\to 0$�ƂȂ�B��

�������A���̋c�_�͖{����$\mathrm{Re}(s)>1$�̌��Ō����ȋc�_��K�v�Ƃ���̂����A
��܂��Ȍ��ʂł������肢����\footnote{������i�w���f��́x������w�o�ʼn�ȂǎQ�ƁB}�B

\newpage
{\gt �Q�y�[�W����������ւ�}

\newpage
{\gt �Q�y�[�W����������ւ�}

\newpage
�u�R�^���v�̐��E�̓x���k�[�C���Ɛ[���ʂ��Ă���B
���łɌ������ł���B
���́u�R�^���̐��E�v�̓[�[�^�֐�$\zeta(s)$�Ƃ��ʂ��Ă��āA���ꂪ�����Ŏ��������ŏ��̂��Ƃ���ł���A
���ꂩ��u���[�}���\�z�v�ɏo������L���ȁu���[�}���̊֐������v���������B

�����ł͂������Ɂu���[�}���\�z�v�̖{�_�܂œ��邱�Ƃ͂��Ȃ����A�R�^���̐��E��ʂ��āA
���͍��܂łɂ��܂��Ė{�i�I�ȈӖ��ɂ�����$\zeta$�֐����x���k�[�C���ɍ��������Ă��邱�ƁA
�u$\zeta$�֐��̗�_�v�Ƃ������[�}�������̌��֌��܂ōs���Ă݂悤�B

\noindent
\Fig[����38��]{\textwidth}{5\baselineskip}

�܂��A������͂���
\begin{align*}
&\frac{1}{2i} \int_{\Gamma_n} \frac{\cot \pi z}{z^s} dz
 =\sum_{k=1}^n \frac{1}{k^s} \\
&\text{�������A}\Gamma_n \text{�͗̈�} %\label{39}
\end{align*}
\[
G_n : \, |z| \leqq n+\frac{1}{2}, \, \mathrm{Re}(z)>a \, (0<a<1)\text{�̋��E} %\label{40}
\]
���������Ƃ��ł���B
���ہA��ϕ��֐���$z=1, \, 2, \, \ldots,\, n$�Ɉ�ʂ̋ɂ����—L���^�֐��ŁA
$z=k$�ɂ����闯����$1/k^s$����B

�����ŁA�u$n$�͉~�ʕ���$k_n$�Ƌ��$[a-iy_n, \, a+iy_n]$���琬�邪�A
$\int k_n \to 0 \, (n \to \infty)$���������Ƃ��ł���B

\noindent
\Fig[����41���}]{\textwidth}{5\baselineskip}

�䂦��
\begin{align*}
\frac{1}{2i} \int_{a+iy_n}^{a-iy_n} \frac{\cot \pi z}{z^s} dz
=-\frac{1}{2i} \int_{a}^{a+iy_n} \frac{\cot \pi z}{z^s} dz
 +\frac{1}{2i} \int_{a}^{a-iy_n} \frac{\cot \pi z}{z^s} dz %\label{42}
\end{align*}
���������B
�E�ӂ̐ϕ��ɂ����Ă͂��ꂼ��
\begin{align*}
\frac{\cot \pi z}{z^s}
=-\frac{1}{2}-\frac{1}{e^{-2\pi iz }-1}, \quad 
  \frac{1}{2}+\frac{1}{e^{ 2\pi iz }-1} %\label{43}
\end{align*}
�ł��邩��A�ŏ��̐ϕ���
\begin{align*}
-\frac{1}{2i} \int_{a}^{a+iy_n} \frac{\cot \pi z}{z^s} dz
&=\int_{a}^{a+iy_n} \left( \frac{z^{-s}}{2} + \frac{z^{-s}}{e^{-2\pi iz }-1} \right) dz \\
&= \frac{1}{2} \frac{a^{1-s}}{s-1}
  +\frac{1}{2} \frac{(a+iy_n)^{1-s}}{1-s}
  +\int_{a}^{a+iy_n} \frac{z^{-s}}{e^{-2\pi iz }-1} dz %\label{44}
\end{align*}
�ƂȂ�B
������$1-\mathrm{Re}(s)<0$������$\text{��}2\text{��}\to 0 \, (n \to \infty)$�B

2�Ԗڂ����l�ł���A2�‚𕹂��āA$n \to \infty$�̂Ƃ��A
\begin{align*}
\zeta(s)
&=\sum_{k=1}^\infty \frac{1}{k^s} \\
&=\frac{a^{1-s}}{s-1} 
  +\int_{a}^{a+iy_n} \frac{z^{-s}}{e^{-2\pi iz }-1} dz
  +\int_{a}^{a-iy_n} \frac{z^{-s}}{e^{ 2\pi iz }-1} dz
  +o \left( \dfrac{1}{n} \right) %\label{45}
\end{align*}
�𓾂��B
�ȏ�A���f�̈�ł̐ϕ������A�Ӗ����鏊�͂����ނˏ����ɗ����ł��悤�B

$0<a<1$�ł������$a$�͖��֌W������$a \to 0$�Ƃ���
\begin{align*}
\lim_{a \to 0}
  \int_{a}^{a-i\infty} \frac{z^{-s}}{e^{ 2\pi iz }-1} dz
=i\int_{0}^{ - \infty} \frac{|y|^{-s} e^{i\pi s/2}}{e^{-2\pi y }-1} dy \\
\lim_{a \to 0}
  \int_{a}^{a+i\infty} \frac{z^{-s}}{e^{-2\pi iz }-1} dz
=i\int_{0}^{   \infty} \frac{  y^{-s} e^{-i\pi s/2}}{e^{2\pi y }-1} dy %\label{46}
\end{align*}
����āA
\begin{align*}
\zeta(s)
&=ie^{-i\pi s/2}
   \int_{0}^{ \infty} \frac{y^{-s}}{e^{2\pi y}-1} dy
  +ie^{ i\pi s/2}
   \int_{0}^{-\infty} \frac{|y|^{-s}}{e^{-2\pi y}-1} dy \\
&=\frac{1}{i} (e^{ i\pi s/2}-e^{-i\pi s/2})
   \int_{0}^{ \infty} \frac{y^{-s}}{e^{2\pi y}-1} dy \\
&=2 \sin \frac{\pi s}{2}
   \int_{0}^{ \infty} \frac{y^{-s}}{e^{2\pi y}-1} dy \\
&=2 (2\pi)^{s-1} \sin \frac{\pi s}{2}
   \int_{0}^{ \infty} \frac{x^{-s}}{e^x -1} dx %\label{47}
\end{align*}
�Ƃ����܂ł͂悢���A�Ō�̐ϕ�����ł���B
�����ōH�v������B
���Ȃ킿�A���łɒm���Ă���

\noindent
\Fig[����47�f���}]{\textwidth}{5\baselineskip}

\[
\zeta(1-s)
=\frac{1}{\Gamma(1-s)}
   \int_{0}^{ \infty} \frac{x^{-s}}{e^x -1} dx %\label{48}
\]
�𗘗p����΁A���̐ϕ����������邱�Ƃ��ł�
\[
\zeta(1-s)
=\frac{1}{\Gamma(1-s)}
 \frac{\zeta(s)}{2 (2\pi)^{s-1} \sin \dfrac{\pi s}{2}} %\label{49}
\]
���Ȃ킿�A�����̋��߂�$\zeta$�֐��̊֌W��
\[
\zeta(s)
=2 (2\pi)^{s-1} \sin \frac{\pi s}{2} \Gamma(1-s) \zeta(1-s) %\label{50}
\]
�𓾂��B
���ꂪ�悭�m����u���[�}���̊֐������v�idie Riemanische Funktional gleichnung�j�ł���B

\section{�o�[�[����� I}

�Ƃ���œ����́u�o�[�[�����v�Ƃ́A��������
\[
\zeta(2)
=\dfrac{1}{1^2}+\dfrac{1}{2^2}+\dfrac{1}{3^2}+\cdots +\dfrac{1}{n^2}+\cdots %\label{50'}
\]
�����߂���ł������B

���̔w�i�ɂ͂�����e�ՂȖ�������
\[
\dfrac{1}{1}+\dfrac{1}{2}+\dfrac{1}{3}+\cdots +\dfrac{1}{n}+\cdots %\label{50''}
\]
�͗L���Șa�������Ȃ����Ƃ����炩�ɂ���Ă������Ƃ�����B
���ہA���̘a��$\log n$�Ɋ֌W��
\[
\text{����51��������} %\label{51}
\]
�ł����āA$\gamma$�͂�����u�I�C���[�萔�v�i$\gamma=\cdots$�j�ł���B
$\gamma$��$e,$���ƕ���ł悭�m����d�v�萔�ł��邪�A���̐����ɂ‚��Ă͂قƂ�ǂ킩���Ă��Ȃ��B
���Ȃ݂ɁA�u�I�C���[���}�N���[�����̘a�����v�ɂ���
\[
\text{����52��������} %\label{52}
\]
�ŁA$n \to \infty$�ł̔��U�͂���߂Ēx�����A����ł��a�͔��U�ł���i���܂���Ă͂����Ȃ��j�B

���āA�{���$\zeta(2)$�͗L����$<2$�ł��邱�Ƃ͂킩���Ă����B
���Ȃ킿�A���̎���ɂ����Ă��łɖ��������̘a�̗L���A�����̋�ʂ͏\���Ɉӎ�����Ă����B
���̉ۑ�ɑ΂��A�I�C���[�̉��́e�h�̂𔲂��f�����ŁA����Ȃ��Ƃ��������̂��A
�Ƃ����㐢����̃l�K�e�B�u�ȕ]���i18���I�̐��w�́u�����v�łȂ��j�������������̂ł���B
$\sin x$��$x$�̑������i�����ł���ƌ��Ȃ��A������$\sin x=0$�̎��́j����
\[
x=0, \, \pm \pi, \, \pm 2\pi, \, \pm 3\pi, \, \ldots
\]
�ł��邩��A$x$�ȊO�Ɉ�������
\begin{align*}
&\left( 1-\dfrac{x}{ \pi} \right) \left( 1+\dfrac{x}{ \pi} \right)
 \left( 1-\dfrac{x}{2\pi} \right) \left( 1+\dfrac{x}{2\pi} \right) \cdots \\
&=\left( 1-\dfrac{x^2}{ \pi^2} \right) \left( 1-\dfrac{x^2}{4\pi^2} \right) 
  \left( 1-\dfrac{x^2}{9\pi^2} \right) \cdots \qquad (\sharp) %\label{53}
\end{align*}
�����‚͂��ł���B
����
\[
\sin x/x=1-\frac{1}{6} x^2+\cdots
\]
�ł��邩��A���Ɗ֐��̊֌W����
\[
\dfrac{1}{\pi^2} \left( 
\dfrac{1}{1^2}+\dfrac{1}{2^2}+\dfrac{1}{3^2}+\cdots \right)
=\dfrac{1}{6} %\label{54}
\]
���ꂩ��A���]��
\[
\dfrac{1}{1^2}+\dfrac{1}{2^2}+\dfrac{1}{3^2}+\cdots 
=\dfrac{\pi^2}{6} %\label{55}
\]
�𓾂邱�ƂɂȂ�̂ł���B
���̌��ʂ݂̂Ȃ炸�A$(\sharp)$��$\sin x/x$�́u������ϓW�J�v�Ƃ����A���ې������B

����͐����������Ƃ��������łȂ��A18���I�̐��w�͌����łȂ������A�Ƃ����ᔻ�́A
�K�������������Ȃ��Ƃ������_�͐��w�j�Ƃ̖₢������B
����ǂ��납�A���Z���w�Ƒ�w���w�̊Ԃ̓��e�̒������M���b�v�́A�{���A
���w�̒���18���I�̐��w�ɑ΂���K�؂ȗ������������Ă��邩��ł��낤�B
���w��3K���̉������헪���l�����ł��̔F���͏d�v�ł���B

�����Ƃ��A�I�C���[�͂������ɋC�ɂȂ����炵���A�������������ƂȂ�
\[
\text{����55'��������} %\label{55'}
\]
���l����ʉ���^���Ă��āA������V�˂̗]�T�̖ʖږ��@���鏊�ł���B

�ł͗]���������āA����$p \geqq 3$�ɑ΂�
\[
\zeta(p)
=\dfrac{1}{1^p}+\dfrac{1}{2^p}+\dfrac{1}{3^p}+\cdots +\dfrac{1}{n^p}+\cdots %\label{56}
\]
�͂ǂ��Ȃ邩�Ƃ����ۑ肪����B
�����̓x���k�[�C��$B_1$��p����$\zeta(2)=B_1 \pi^2$�𓾂����A
��ʂɋ�����$p=2m$�ɑ΂��Ă��A�Ăуx���k�[�C����
\[
\zeta(2m)=\text{dummy} %\label{58}
\]
�Ƃ���������^���Ă����B
���Ȃ킿�A�o�[�[�����̈�ʉ�
\[
\text{����59��������} %\label{59}
\]
�ɍs���������B

���̓x���k�[�C���̊���͂���ǂ���ł͂Ȃ��B
����ɂ‚��Ă͖{���ɏq�ׂ邱�ƂƂ��悤�B

\section{�o�[�[����� II}

�I�C���[�́A�e�����̕��䂩���э~��Č�����f�悤�ȕ��@��
\[
\dfrac{1}{1^2}+\dfrac{1}{2^2}+\dfrac{1}{3^2}+\cdots 
=\dfrac{\pi^2}{6} %\label{60}
\]
�������ւ�ɁA��w�̔��ϕ��̃N���X�E���[���ɂӂ��킵���e�ӂ‚��f�̂���������B
�����̃e�L�X�g�ɏ�������Ă���t�[���G�����̕��@�ŁA���̋Z�@�ɏ]���A�����֐�
\[
f(x)=\text{dummy} %\label{61}
\]
���O�p�����ɓW�J���悤�B
�t�[���G�W����
\begin{align*}
a_n&=\text{dummy} \\
b_n&=\text{dummy} %\label{62}
\end{align*}
�ł��邩��A�t�[���G������
\[
\text{����63��������} %\label{63}
\]
�ƂȂ�B
���ꂪ�͂����Ď���������$f(x)$�ɓ������Ȃ邩�͂܂��ʖ��ł���B
�����̈ʑ��i���ώ����A��l�����j�̊m�F�A���ʐϕ��̕ۏ؂����߂��A�u���[�}�������x�[�O�̒藝�v�������ł͖������ʂ����A
�����ł͂ӂ�Ȃ����ƂƂ��A�t�[���G������$f(x)$�Ɏ������邱�Ƃ������邩��A������$x=\cdots$�Ƃ����ƁA
\[
\text{����64��������} %\label{64}
\]
�𓾂ďI��B
�����̐l�X�ɂƂ��āA���̂�����$\pi^2/6$�𓾂���@���܂��͕��Ղł��낤�B

\noindent
\Fig[����65���@�}]{\textwidth}{8\baselineskip}

\section{�Ȃ��e���[�}���́f$\zeta$�֐���}

$\zeta$�֐���
\[
\zeta(s)
=\dfrac{1}{1^s}+\dfrac{1}{2^s}+\dfrac{1}{3^s}+\cdots %\label{66}
\]
�ƒ�`����Ă��邪�A���̒�`�̓��[�}���ɂ����̂ŁA�I�C���[�̌��藝�͂���ƈقȂ��Ă����B
��̒�`�ŗႦ��
\[
(\sharp) \qquad 120^{-s}=(2^{-s})^3 \cdots 3^{-s} \cdot 5^{-s}
\]
�́A�W�J��
\begin{align*}
\dfrac{1}{1-2^{-s}}=1+2^{-s}+(2^{-s})^2+(2^{-s})^3+\cdots \\
\dfrac{1}{1-3^{-s}}=1+3^{-s}+(3^{-s})^2+(3^{-s})^3+\cdots \\
\dfrac{1}{1-5^{-s}}=1+5^{-s}+(5^{-s})^2+(5^{-s})^3+\cdots %\label{67}
\end{align*}
�̎�ɂ�����x�����o������B
���̂��Ƃ���A�I�C���[�ɂ��Ζ������
\begin{align*}
\zeta(s)
=&\dfrac{1}{1-2^{-s}} \cdot \dfrac{1}{1-3^{-s}} \cdot 
  \dfrac{1}{1-5^{-s}} \cdot \dfrac{1}{1-7^{-s}} \cdots \\
 &\text{�i�ȉ�}(\text{�f��})^{-s}\text{�ɂ���ρj} %\label{68}
\end{align*}
��$\zeta$�֐��i���͕̂ʂƂ��āj�̒�`�ł���B
$\zeta$�֐����f���ɐ[��������肪���邱�Ƃ́A���コ��ɂ͂����肷�邾�낤�B

\section{��͓I�����Ȃ����͉�͐ڑ��̃t�V�M}

%\footnote{�w��͊T�_�x�͂������p����B}

�u�J�C�Z�L�E�Z�c�]�N�v�͊֐��_---�������͉�͊֐��_---�̍ł��ؖ��̉s���L�p�Ȓ藝�ł���B
���ϐ��̔��ϕ��ł͏o���Ȃ��v�������Ȃ��A���́e�ǂ����Ă���Ȃ��Ƃ�������́H�f�Ƃ����Ƃ܂ǂ���^���A
�������Ă��̐������e���킩��Ȃ��Ȃ邭�炢�ł���B

���Ƃ��΁A$s>0$�i���͕��f���ł��悭$\mathrm{Re}(s)>0$�j�ɑ΂�
\[
\Gamma(s)
=\int_0^\infty e^{-x}x^{s-1} dx \qquad \text{�i�K���}�֐��j} %\label{69}
\]
���A�֐�����
\[
\Gamma(s+1)=s\Gamma(s), \quad s>0
\]
�𖞂����Ƃ͂悭�m���邪�A���̊֐��������e�t�p�f�����Q���֌W
\[
(\sharp) \qquad \Gamma(s)=\Gamma(s+1)/s
\]
�͑S�R�ʂ̓���������B
���Ȃ킿$s+1>0$�Ő��藧�‚���A$0>s>-1$�ɑ΂��Ă����ӂ͒�`�����B

����{��$(\sharp)$����`���ƂȂ�A���͂�ϕ��ɂ���`�͗p�����Ȃ��B
���ہA������$s$�ɑ΂��Ă͉���$\Gamma(s)<0$�ł���B
$\Gamma( \, \cdot \, )$�̒�`�悪$0>s>-1$�܂ʼn������ꂽ�̂ŁA
$(\sharp)$��$0>s+1>-1$�ł悭�A����Ē�`���$-1>s>-2$�܂ł���ɉ��������B
���̂悤��$\Gamma$�֐��͕��̐������������ׂĂ�$s$�ɑ΂��Ē�`����邱�ƂɂȂ�B

���ꂾ���ł͂Ȃ��B
����$s>0$�ɑ΂��Đϕ��Œ�`���ꂽ���֐��͂��̒�`�𒴂��Ă��ׂĂ�$s$�ɑ΂��Ē�`���ꂽ���A
���͂��̉����͂��ׂĂ̕��f��$s$�i���f���ʁj�܂Ŏ��邱�Ƃ��ؖ��ł���B

���̐܁A�����̒�`�͌`�𗯂߂Ȃ��̂ŁA��u�Ƃ܂ǂ��Ȃ����͍�����������B
�����͊֐������ɂ���čs���邪�A���ꂾ���ł͂Ȃ��B
�u��͓I�����v�Ƃ����悤�ɕ��f��$s$�̊֐��Ƃ���
\[
\Gamma(s)
=\int_0^\infty e^{-x}x^{s-1} dx, \qquad 
\mathrm{Re}(s)>0 %\label{70}
\]
����͊֐��ł��邱�Ƃ����������Ă���B
���Ȃ݂ɂ��̒l�������‚��^���Ă������B

\noindent
\Fig[����71���i�����j�i�}�j]{\textwidth}{10\baselineskip}

\section{$\zeta$�֐��̉�͓I����}

$\zeta$�֐�$\zeta(s)$������
\[
\zeta(s)
=\dfrac{1}{1^s}+\dfrac{1}{2^s}+\dfrac{1}{3^s}+\cdots, \qquad 
\mathrm{Re}(s)>1 %\label{72}
\]
�Œ�`���ꂽ�B
$\zeta(2)=\pi^2/6, \, 
 \zeta(3)=\cdots, \, 
 \zeta(4)=\cdots $�Ȃǂ��m���Ă��āA$\zeta(2m)$�̓x���k�[�C���ł���킳�ꂽ�B
�������A$\zeta(-1), \, \zeta(-2), \, \ldots$�͂ǂ����낤���B
�ꌩ���Ē�`�ɍ���Ȃ��ǂ��납�i���Z���X�Ƃ������ƂɂȂ�B
�������A����͓����̒�`�����Ƃ��āi���̌���ŁA�Y��j�A���[�}���̊֐�����
\[
\text{����73��������} %\label{73}
\]
��ʂ̐V������`���Ƃ���
---�������A�����̒�`�Ƃ�$\mathrm{Re}(s)>1$�ł͈�v��---��͓I�������s���B
���̂Ƃ��͓�����$\zeta(s)$�̒�`�͓K�p����Ȃ����Ƃɒ��ӂ��悤�B

�ł͌v�Z���Ă݂悤�B
���R�����ł��x���k�[�C����������
\begin{align*}
\zeta(1-2m)
&=2(2\pi)^{-2m} \text{���Ǖs�\} 
  \dfrac{\pi (1-2m)}{2} \Gamma(2m) \zeta(2m) \\
&=2(2\pi)^{-2m} 
  (-1)^m (2m-1)! \dfrac{(2\pi)^{2m}B_m}{2 \cdot (2m)!} \\
&=\dfrac{(-1)^m B_m}{2m}, \qquad m \geqq 1, %\label{74}
\end{align*}
�ȂǂŁA�ڗ��‚͕̂��̋���$-2m$�ɑ΂��ẮA�֐���������
\[
\zeta(-2m)=0, \quad k=1, \, 2, \ldots
\]
�ł��邱�Ƃł���B
������$\zeta$�֐��́u�����ȗ�_�v�Ƃ����B
���f���ʂ̍������ʂ̗�_�͂����Ɍ���B

�ʗp���Ȃ��Ƃ͂����A$\zeta(s)$�̒�`����0�ɂȂ�Ƃ͑z�����ł��Ȃ��������A���������ϐ��̔��ϕ��Ƃ͕ʐ��E
---�������A�������牄�����ׂ荇�����i�ڑ������j���E---����䂦��ł���B
�Ƃ���ŁA�����ȗ�_�ȊO��
\begin{equation*}
\zeta(s)=0 %\label{75}
\end{equation*}
�ƂȂ�$s$�͂Ȃ����낤���B
����͂����āA���f���ʂ̒���
\[
s=\dfrac{1}{2}+ti \quad (t\text{�͎���})
\]

�‚܂�R���s���[�^�E�T�[�`��$\mathrm{Re}(s)=1/2$�̏�ɋɂ߂đ������o����邱�Ƃ��킩���Ă���B
������ƌ����āA�e$\mathrm{Re}(s)=1/2$��$s$�Ɍ�����f�Ƃ͌��_�ł����A�����܂Łu�\�z�v�ɂƂǂ܂�B
������u���[�}���\�z�v�iRiemann's hypothesis�j�Ƃ������A���󂷂�΁u���[�}�������v�ł���B
���[�}�������͂��܂��ؖ�����Ă��Ȃ����㐔�w�̈����ł���B

���[�}���������ؖ������΁A�f���̕��z�֐�
\[
\pi(x)=x \text{�܂ł̑f���̌�}
\]
�ɑ傫�Ȑi����^����B
$\zeta$�֐����f���̕��z�Ɋ֌W���邱�Ƃ́A���[�}�������ނ���I�C���[�́i�f����p�����j��`������[���ł��鏊�ł��낤�B

�Ƃ͂����A$\zeta$�֐��̒m��ꂽ�����Ȓl�̓x���k�[�C���ƃ΂ŕ\�킳��Ă��邱�Ƃ͒m���Ă����������悢�B
�e�[�[�^�̐��E�̓x���k�[�C���ŕ~���‚߂��Ă���f�Ƃ����������Ƃ��ł��悤�B

\section{�x���k�[�C���̐��E�͑���}

���R�u�E�x���k�[�C�͎����̑n�o������������قNj���ȑ��݂ɂȂ�Ƃ͎v���Ă��݂Ȃ������ł��낤�B
�������A���z��$\pi$��$e$�Ƃ͈���ăx���k�[�C���͗L�����ł���A���ꂪ�������ăt�V�M�ł���B
���̌v�Z�������A���҂��ǂ񂾁w��͊T�_�x�i19�����N�Łj�ɂ́A�����Ԗڂ܂Ōv�Z���ꕪ�q�́c���A����́c���A�ƋL���Ă���B
���ł͂���ɐ�܂ő����Ă���ł��낤�B
�����ł͉��������Ȃ��B

���̗��_�I�˒����܂��悪���邪�A���Ƃ��΁A
�u�t�F���}�[�̍ŏI�藝�v�̏ؖ��̒��ɏo������i���ǂ͏ؖ��͂ł��Ȃ��������j�u�N���}�[�̗��z���v�́A
�x���k�[�C�������Ȃ�n�[�h�ɗ��p����B
�܂��A�ʑ��􉽊w�ɂ����Ă��A�x���k�[�C���̊��p������悤�����A���͂�����g�s�b�N�ł���A���҂��֐S���񂹂Ă���B

���������ƁA�x���k�[�C���͂��Ȃ艏�������Y�J�V�C�ِ��l�̃I�n�i�V�Ǝv�������m��Ȃ��B
�������A�u�I�C���[���}�N���[�����̘a�����v�̂悤�ɁA���ϕ��̃N���X�E���[���Ő������Ă������L�p�Ȍ����ɂ�
�x���k�[�C����x���k�[�C���������o������B
���̌����͂悭�m����}�N���[�����W�J�i�e�C���[�W�J�j�̕ό`�ł��邪�A�ނ��뉞�p�Ƃ��Ă͕ʕ��ƍl���������悢�B
�܂��A
\[
\text{����76��������} %\label{76}
\]
�ł��邪�A$f'$�̂Ƃ���ɂ��̎����̂����q�̂悤�Ɏg���A���ŒW���̐ςݏd�Ȃ�����肷��ƁA�܂�
\[
\text{����77��������} %\label{77}
\]
��������B
���̌`�ł����Ȃ�̌��p�����邪�A���͏o���Ƃ��Ă͐��ς��ŁA�������Ȍ��ʂ�h���؂�Ȃ��B
���Ƃ���
\[
\text{����78��������} %\label{78}
\]
�ł���B
�I�C���[���g���u$\cdots \cdots $�v�i���e�����$\cdots \cdots $�j�ƒ��ӂ��āA����ɏ�]������������ƕt��
\[
\text{����79��������} %\label{79}
\]
�̂悤�ɍ��グ�Ă���B
���ꂪ�����̌����ł���B

���p�Ƃ���
\[
\text{����80��������} %\label{80}
\]
�Ȃǂ��l������B

\section{��������ɐ��ރx���k�[�C�A���E�X�s���b�g}

���j�I�ɂ͂��ׂĂ͂�������Ɏn�܂�Ɖ]���Ă������߂��ł͂Ȃ��B
�x���k�[�C����������̖ڂ̘a�̌���

$N$�‚̂�������ɂ‚���
\[
n\text{�‚̂�������̖ڂ̘a}=k
\]
�̊m�����z�i���ۂɂ͏ꍇ�̐��j�����܂��v�Z�������Ƃ���A
���̋Ɛт��܂߂āA�����Ԃ�Ђ����ڂ����邪�A18���I�̑����̐��w��̔��W���n�܂�A
���ꂪ���ʂ̔g��̂悤�ɔg�y�����B
���ہA�悭�����Ă���B
�������Ȃ���A�m���_�ɂƂ��Ċ̗v�ł���ɂ�������炸�ӊO�Ɖ�͓I�ɋl�߂��Ă��Ȃ��B
�x���k�[�C��$n=10$�܂Ŏ������̂̍Ō�̕��͕����I�ł���B
�h�E���A�u��(de Moivre)�A�����̃t�F���[(W. Feller)�͎w�j�͎������A���ʂ܂ŒB���Ă��Ȃ��B
��莩�̂͊ȒP�����A��֐��A�d���g�ݍ��킹�A�}�`���A�x���k�[�C���A�������p���鎮�������K�v�ł���B
���̌���
%\begin{enumerate}
%\item 

(1) �����Ȃ�$n$�ɑ΂��Ă��ŏ���6��$k=n, \, n+1, \, \ldots, \, n+6$�ɂ‚��Ă͉��������̂͂Ȃ��A
$1,\, 1,\, 1,\, 1,\, \ldots $����a���d���I�ɐς݂������i$n$�d�a���j������u�}�`���v
\[
\text{�i��������H�j}
\]
�ƂȂ�B
�������u�}�`���vfigurate numbers�ɂ‚��Ă͎�X�̌ď̂�����B


(2) ��ʂ�$k$�ɂ‚��ẮA$n$�‚̂�������̖ڂ�$\text{�a}=k$�ƂȂ�ꍇ�̐��́A��֐�
\[
(t^1+ t^2+ t^3+ t^4+ t^5+ t^6)^n
=t^n\cdot (1-t^6)^n\times (1-t)^{-n}
\]
��$t^k\, (k=n,\, n+1,\,\ldots, \, 6n)$�̌W���ł���B
���̕�֐���

\begin{enumerate}
\item $(1-t)^{-n}\, \diamond$
\item $t^n \cdot(1-t^6)^n \, \diamond$
\end{enumerate}
��2���q���琬��A1.��$n$�d�a���̕�֐��ɂ���Đ}�`�����Q���I�ɗݐς��鉉�Z�ŁA
�g�ݍ��킹�_�I�Ɂu�d���g�ݍ��킹�̐��v${}_n\mathrm{H}_r$�i�����̓񍀌W���j�ŕ\������A
2.�́A�㔼��1.�ɑ΂���C���ŁA
\[
\text{�i��������H�j}
\]
����A$\Delta k=6$���ɁA1.�ɂ��}�`������������������ŎZ�����邢�͍����������Z���i��֐��Łj�\���Ă���B
�Ȃ�$t^n$�͂�������̖ڂ̘a�̕��z�̍ŏ�$\text{�i�ŏ��j}=n$���������ƂŁA���z�̍����w�b�_�[�̈ʒu�����߂�B
�ȏ�̌��ʂ���A�񍀌W���ȏォ��A�ŏI�I�Ȍv�Z�̉�͓I�\����
\[
\text{�i��������H�j}
\]
�ƂȂ�B
%\end{enumerate}

\noindent
{\gt �q��r}\, $n=16$�̃P�[�X

\vspace{10\baselineskip}

\noindent
{\gt �q��r}\, ��ʂ̏ꍇ
\vspace{10\baselineskip}

\section{�w�����p�x��III��}

��III���́A���H�E���p�ʂł���ɐi�݁A�q���ɂ�����m���ɂ������ȕ��z�@�A���Ҋz�̌v�Z�ւƓW�J����B
�����薼��

\begin{quote}
---���܂��܂ȓq���̕��z����ы��R�̃Q�[���Ɋւ���O�q�̗��_�̉��p����������III��\, 
Usum Praecedentis Doctrinae In Variis Sortitionibus \& Ludis Aleae---
\end{quote}
�ł���B
sortitio��sors�u�q���v�̕��z���Ӗ�����B
���̂��߁A��II���̑g�ݍ��킹���w����Ăъm���̉ۑ�֖߂��āA��III���͎���24�₩�琬��B
����߂ďڍׂł��邪�A����]���΁A�����̃f�J���g���̑㐔�I���L�@�̔��B���\���łȂ����߁A�璷�̈�ۂ��ے�ł��Ȃ��B
�܂��A�����Q���҂������v���[����ǖʂ������̃Q�[�����_�ł����u�W�J�`�Q�[���v�̌`���ł���A
������u�c���[�v�̕\����p����΂�薾�m�ɂȂ�B
�����Ƃ��A����͖{���I�ł͂Ȃ����낤�B

�����ł͍ŏ���10��݂̂��Љ��B

\begin{enumerate}
\item[��I] 
��I������11�̓��ʃP�[�X�m���A���̃g�[�N��\footnote{
���̑�p�d�݁B�����ł́A�w���̌��ʋ@�ւ̗��p�����̎x�����ȂǂɎg����B}
�e1���������Ă���‚ڂ���A3�l�������񕜌��Œ��o���A�����������҂������Ƃ���B����́u�����v�ɂ�����n

\noindent
\Fig[�}]{0.8\textwidth}{5\baselineskip}

\item[��II] 
���A3�l�Ƃ����������Ȃ��ꍇ�͊l���z�m�{���n�E�X�ɑ�����n��3�l�ɕ��z�����.

\item[��III] 
�����@�m���������n���K�p�����B
A,\, B,\, C,\, D,\, E,\, F�̂����A�܂�A,\, B�̏��҂�C�Ƒΐ�A���̏��҂�D�Ƒΐ�A�c�c�Ƃ���B

\item[��IV] 
���A���҂ɂ͂��̓s�x2�{�̏��P�[�X�����蓖�Ă���
�m$1:1, \, 2:1,\, 4:1,\, \ldots $�̂悤�ɏ��������i����n\\
���F

\noindent
\Fig[�}]{0.8\textwidth}{5\baselineskip}

\item[��V] 
�z�C�w���X�̕t�͑�3��ő�I���ɂ��邪�A����i�H�j�ʼn����B

\item[��VI] ����4��B
\item[��VII] 
I�Ɠ��A�������G�D1�����܂ރJ�[�h�E�f�b�N��������B
�m��ӂł̓J�[�h�̖����ɂ͒�߂��Ȃ��A�K�v�ɉ����āu�J�[�h�E�Q�[���v�̃J�[�h�����Z�b�g���p�ӂ���B�n
\item[��VIII] 
���A�G�D�͕������Ƃ���B
\item[��IX] 
���A�G�D�̍ő喇���𓾂��҂������Ƃ��A�^�C�̏ꍇ�͕�����������B���ʖ����ɂ͕��z���Ȃ��B
�m���z�͏ꍇ�̐��𕪔z������̂Ƃ���B�n
\item[��X] 
A,B,C,D4�l�ɁA�G�D16������ȊO�̃J�[�h36�����܂ރJ�[�h�����̏��Ԃɔz��B
23���z��I��������_�ŁAA,\,B,\,C,\,D�͂��̂���4,\,3,\,2,\,1���̊G�D�𓾂��B
D�͒f�O���āA���̌�����A,\,B,\,C�̂����ꂩ�ɔ�����̂Ƃ���B
���̉��i�͂�����ɂȂ邩�B
�����̊m���͂ǂ��Ȃ邩�B
\end{enumerate}

\section{�w�����p�x��IV��}

��҂��^�ɂ߂������ɖړI���W�Ԃ���Ă���ŏI���ł���A�薼�������ʂ�
\begin{quote}
---�Љ�I�A���_�I�A�o�ϓI(���Ƃ���)�Ɋւ���O�q�̗��_�̉��p���������IV�� \, 
Usum \& Applicationem Praecedentis Doctorina�@In Civilibus, Moralibus \& Oeconomicis---
\end{quote}
�Ƃ����܂łƂ��Ȃ苿�����قȂ�B�������A�����́u������v�œ��e���Љ�I�ȗL�p�������‚��Ƃ��������Ă���B

���w���_�̓L�`���Ƃ�邪�A���ꂾ�����ړI�����ł͂Ȃ��B
���ہAcivilus�i�����ł͒D�i�j�́u�����v�u���Ɓv�u�Љ�v�u�s���v���Ӗ�����B
�����́A���ꂳ�ꂽ���Ƃ̌`���悤�₭�`������‚‚������ߑ�̓�����̎���ŁA
���̋C�^���u�Љ�I�v�u�s���I�v�i���̌���Łu�S�̓I�v�j�Ƃ������Ƃ΂ɍ��߂��Ă���B
��������́u���n�v�Ȃ̂ł���B
moralia�����݂́u�ϗ��I�v�Ƃ͑����قȂ�
---�͂�����A�ʕ��ƍl���Ă悢---
�u�����I�v�łȂ��Ƃ����Ӗ��Łu���_�I�v�u�S���I�v�A��荡�����ɂ́u�F�m�I�v�Ƃ����悤���B
economicus�����Ƃ́i�ƌv�́j�؂萷��A��肭��ł��邱�Ƃ͑����̒m��Ƃ���ł��낤�B
���ꂪ�Љ��ɂȂ��āu�o�ρveconomics�ɂȂ����B
�����ł́u�o�ς̌v�Z�v�Ƃ����Ă������B

���̂悤�ɑ�IV�����߂������͎Љ�́u�O�����h�E�Z�I���[�v�ł���A�������匾�s��ł͂Ȃ��A
���̕��@�̊�b�����ɂ̌��ł����ւ�B
���̈�Ԏ肪�u�吔�̖@���v�ŁA���ꂾ���ł��̑�Ȗ@���ł��邪�A��҂͂��̐���s���O�ɖv�����B
�����̓��v�w�́w�����p�x�̊�{���_�̕����ł���B

���̑�IV����5�͂ɕ�����Ď����ɏ������ꂽ�Y��ȍ�i�ŁA�܂��Љ�ɂ�����u�m���v�̃R�g�o�g������n�߂��Ă���B

\noindent
{\gt ��I��}\, ���Ƃ���̊m�����A�m���炵���A�K�R������ъW�R���ɂ‚��Ă̂����‚��̊�{�v�f

Praeliminaria Quaedam De Certitudine, Probabilitate, Necessitate \&�@Contingentia Rerum

\newpage

\begin{table}[ht]
\caption{���{��A�p��A�������e����ɂ��u�m���v�֘A��b}
  \begin{tabular}{lll} \hline
�m���炵���m�m���炵���n & probable                  & probabile \\
���m���炵������       & more probable             &           \\
�”\                     & possible                  & possibile \\
�F����m��               & morally certain           & moraliter certum \\
�F����s�”\             & morally impossible        & moraliter impossibile \\
�K�R�m�I�n               & necessary�@�@�@�@�@�@     & necessitate \\
������K�R               & physically necessary      & necessitate vel physica \\
������K�R               & hypothetically necessary  & necessitate vel hypothetica \\
�_���A���x��K�R       & contractually,            & necessitate pacti seu instituti \\
                         & institutionally necessary &           \\
�W�R�m�I�n               & contingent                & contingens \\
                         &                           & liberum,\,fortuitum, casuale \\
�m�C�Ӂm�I�n���邢��     & free                      & \\
�F��m�I�n�A���R�I�n     &                           &
 \end{tabular}
\end{table}

\noindent
{\gt ��II��}\, �m���Ɛ����ɂ‚��āA�����p�ɂ‚��āA�����p�̋c�_�ɂ‚��āA�֘A���邢���‚��̈�ʓI����

De Scieintia \& Conjectura, De Arte Conjectandi. De Argumentis Conjecturarum. 
Axiomata Quaedam Generalia Huc Pertinentia

����Љ�ł��\���ɒʗp����ӎv����̏��������ʂȂ��K�؂ɏq�ׂ��Ă���A�����ɂ̓X�g�A��`�I�ȗϗ�������������B

1.���_�ɓ��B�ł���ꍇ�͐����̗]�n�͂Ȃ�

2.���ꂱ��̋c�_�̏d�v�x�m�E�F�C�g\footnote{
�����I�Ӗ��ɂ�����”\���̊m���i���Ƃ���0.6�Ȃǁj���l����΂悢�B}�n��
�i���ꂼ��j�]������݂̂ł͕s�\���ŁA�����𑍍����Ēm���𓾂��Ƃ���̏؋��Ƃ��ׂ��ł���B

3.�������̋c�_�݂̂Ȃ炸���Ε����̋c�_���l�����ׂ��ł����āA
�����̏d�v�x�𐳂����ʂ�΂ǂ���̋c�_���D�邩�m���ɂȂ�ł��낤�B

4.���ՓI�Ȃ��Ƃ���Ɋւ��锻�f�ɂ͊ԐړI�ŕ��ՓI�ȋc�_�ŏ\���ł��邪�A
�•ʂɊւ��鐄���ɂ͒��ړI�Ō•ʓI�ȋc�_���A�”\�Ȕ͈͓��ŁA�K�v�ł���B

5.�s�m���ŋ^�₪���邱�Ƃ���ɂ‚��Ă͏�񂪓�����܂Ŕ��f��ۗ����ׂ��ł��邪�A
�P�\��������Ȃ��ꍇ�ɂ����ẮA��҂̂����ǂ�����ϋɓI�ɗǂ��Ȃ��ꍇ�ł����Ă��A
��͂���K�؂ŁA���S�ŁA�v���[���A�m���炵���s�ׂ�I�����ׂ��ł���B

6.�v���Ȃ��Ƃ��Q�̂Ȃ����肪�]�܂����B

7.���ʂ݂̂Ől�Ԃ̍s�ׂ𔻒f���Ă͂Ȃ�Ȃ��B

8.�t�^���ׂ��d�v�x�m�E�F�C�g�n�̕]��������Ă͂Ȃ�Ȃ��B
���ΓI�Ɋm�����ΓI�ƌ���Ă͂Ȃ�Ȃ����A����𑼐l�Ɏ咣���Ă͂Ȃ�Ȃ��B

9. ���ׂĂ̈Ӗ��ƕK�v���ƗL�p���ɂ����Ċ��S�Ȋm�����𓾂邱�Ƃ͋H�ł���ȏ�A
�F����m�S�̒��ł���́n�m�����ΓI�m���ƒ�߂邪�悢�B

\noindent
{\gt ��III��}\, ���܂��܂ȋc�_�@����т��Ƃ���̊m���炵�����Z�o���邽�߂̏d�v�x���]���������@�ɂ‚���

De Variis Argumentorum Generibus \& Quomodo Eorum Pondera
Aestimentur Ad Supputandas Rerum Probabilitates

�������琔���I�i�K�ɓ�����H�I�������Ȃ����B
quomodo�́u�����ɂ��āv�̈ӂŌ���I�ɂ�how to�ƍl����΂悢�B
�ȉ��͊e���v��ł���B

\begin{enumerate}
\item ���R�I�ɑ��݂��邪����ɕK�R�����������Ƃ���̌v�Z�m�ȉ��v�Z���n
\item �K�R�I�ɑ��݂��邪����ɋ��R�����������Ƃ���̌v�Z
\item 1,\, 2�𑍍������v�Z
\item ����̂��Ƃ���ɂ‚��A�����̂��Ƃ��炪���݂���ꍇ�ւ̈�ʉ�
\item ���A���ꂼ��̋c�_�����ׂĈقȂ邱�Ƃ���̏ꍇ
\item ���A���ꂼ��̋c�_��4,5����������ꍇ
\item �؋��A���Ώ؋�����������Ƃ��A���ꂼ��̊m���炵���̌v�Z
\end{enumerate}

\noindent
{\gt ��IV��}\, �ꍇ�̐������o����‚̕��@�ɂ‚��āB
����ɂ‚��ĉ���m��ׂ����B�ώ@�Ɋ�Â�������߂��邩�B
���̕��@�Ɋւ�����ʂ̉ۑ�B

De Duplici Modo Investigandi Numeros Casuum.Quid Sentiendum De Illo, 
Qui Instituitur Per Experimenta. Problema Singulare Eam In Rem Propositum, \& c.

�\�z���n�߂ċ�̓I�ɖ��炩�ɂ����B
20�N�����߂Ă����\�z�ł���Ƃ����B

�l�̎����l���Ă݂悤�B
�����̌����ɂ���Đl�͎��ʂ��炻�̌�������l�̎��́u�m���v��m�邱�Ƃ͌����Ăł��Ȃ��B
�������A�����ɍ���‚̕��@������B
�ώ@�ɂ����
��������@�ŁA����͕ʒi�V�������̂ł͂Ȃ�\footnote{
�|�[���E�����C�����_���w�̎w����A.�A���m�[�́w�v�l�p�xArs Cognitandi�B
���ہA�x���k�[�C�́w���_�p�xArs Conjectandi�͂��̏����ɕ�����Ɖ]���Ă���B}�B
���܂Ő_��I�������u�m���v�͊ώ@����v�l�ɂ���đ��邱�Ƃ��ł���B

����𗝉����邽�߂Ɏ��̎������l���Ă݂悤�B

�|�‚ڂ������āA�����ɔ��̃g�[�N���m�F�̕t�����d�݂͍l�����Ȃ��̂Łn3000���ƍ��̃g�[�N��2000���������Ă���B
��������ƈ���Ă����ł͒P���ɔ���������2�ʂ肵���Ȃ��B
��������1�����‚Ƃ�o���B
�����������ώ@�������ƋL�^���A�g�[�N���͂��Ƃ֖߂��B
����āA�‚ڂ̂Ȃ��̃g�[�N���̖����͕s�ςɕۂ����B
����𑽐��񂭂�Ԃ��ƁA���A���̐��̊ώ@�L�^��������B---

�����Œ��ӂ������B
���̔��A���̗����m���ɂȂ�킯�ł͂Ȃ��B
���ł�\kenten{�m��}�͂��łɒ�܂��Ă��āA����ꂽ���A���̊ώ@���ꂽ�䗦�͂���ɋ߂��A
������킸���ɑ傫�����Ƃ�����킸���ɏ��������Ƃ̊Ԃɑ��݂���A���Ƃ��m�F�������̂ł���B
�������A�Ƃ�o��������������ƁA���̊Ԃ̊Ԋu�͋����Ȃ�A���̊O���ɑ��݂��邱�Ƃ͂܂��܂��N����ɂ����Ȃ�B

���Ƃ��΁A���̖�����
\begin{align*}
 500\text{���̎��_�ŁA}\quad &  299 \text{����}  301 \text{���̊�} \\
1000\text{���̎��_�ŁA}\quad & 2999 \text{����} 3001 \text{���̊�}
\end{align*}
�ƂȂ���\footnote{
�R���s���[�^�E�V�~�����[�V�����ł͎��ۂɂ͂���قǐ��m�ɂ͂Ȃ�Ȃ����A�_�Ƃ��Ă͐�������B}
����������Ƃ̔�$3 : 2$�ɋ߂��A���̔���킸���ɑ傫����Ə�������̊Ԃɓ���B
���Ȃ킿�A3:2�̔䂪������邱�Ƃ��m���炵���Ȃ�A�������ώ@�������傫���Ȃ�ɂ��������A
�܂��܂����̊m���炵���͑傫���Ȃ�B

���̂悤�ɂ���΁A�Љ�I�A���_�I�A�o�ϓI���𓮂����Ă���u�m���v�������ɑ����邱�Ƃ��ł���ł͂Ȃ����B
���ꂪ���R�u�E�x���k�[�C�̌ւ�Ƃ������e�������̂ł���B
���R�u�͂���𒷂��g�߂Ă������A���S�������ďo�ł���@��Ȃ������������B
%�e�_��M���邪�䂦�ɁA�����i���̐�����M���āf���C���͂������낷�������@�ɂ͂��̕������c����Ă���B

\noindent
\Fig[�g�[�N���̎�������]{\textwidth}{5\baselineskip}

\noindent
{\gt ��V��}\, 
�O�͂̉ۑ�̉�@�@Solutio Problematis Praecedentis

���āA���悢��u�吔�̖@���v(Law of large numbers)�̏ؖ����������B

����m�����ۂ�1,\, 0 ��2�P�[�X���N���蓾��,���̊m����$r$��$s$�Ƃ���B
�ώ@��$n$��J��Ԃ��Ƃ��A����$n$�񒆂�1,\,0�̉�$x,\, y \,(x+y=n)$�́A
\begin{enumerate}
\item ���傤��$r,s$�̔�
\[
\text{�i��������H�j}
\]
�ƂȂ�m�����ő�ƂȂ��āA���̔�߂��ɏW����
\item �W���͊ώ@��$n$���傫���قǐ��m�ɂȂ�B
\end{enumerate}

���Ȃ݂ɁA�������d�݂�1000�񓊂���Ƃ��A�\�A���̉񐔂̊�����1�΂P�������i��������H�j�i�e500��j�ƂȂ�m�����ő�ƂȂ�݂̂Ȃ炸�A
���m�ɂ����łȂ��Ƃ������ނ�1��1�̂����ߕӂɏW������B
���̂��܂�ɂ��o���I�ɖ����Ȏ��������j��͂��߂ďؖ����ꂽ�̂ł���B

�ؖ��ł��邪�A�܂��A�t�F���}�[�A�p�X�J���A�x���k�[�C�̌ÓT�m���_�̏����ɂ́A
�u�m���v�͍�����1�ɋK�i�����ꂽ$p,\, q \, (p+q=1)$�ł͂Ȃ��A
�q���ɂ�����悤��2��$r,\, s$�̔�ŕ\�����ꂽ�̂Łi$2:1$�̂悤�ɐ������̔䂪�����j$r+s=t \ne 1$�ŁA
���̕\���͌�������s�K�v�ɍ��݂������`�ɂȂ��Ă���\footnote{
���v���X���A�m����$n/N$�̂悤�ɕ����ŕ\�������Ƃ��������i���ہA���̌����j�́A
�Ȍ�̊m���_�j�̓W�J�Ɍ���I�ɑ傫�Ȗ������ʂ������B}�B
�ȉ��ł́A�����̕K�v�ɉ����āA$r,\, s$��$p,\, q$�̂悤�ɋK�i�����ēǂݑւ�$t=1$�Ƃ���B

�x���k�[�C�́A�܂��ő�m���́A1.�̂悤��
\[
\text{�i��������H�j}
\]
�ł��邱�Ƃ��������B
����͍����I�ɂ́A$r,\, s$��$p,\, q$�ɓǂݑւ��āA�񍀕��z�i$x$�̓񍀊m���j�̍ő區��
\[
\text{�i��������H�j}
\]
�ł��邱�Ƃ��]���B
($np,\, nq=\text{����}$�̂悤��$n$���Ƃ��Ă���Ɖ���)�B
���ہA$x$�̓񍀊m��
\[
\text{�i��������H�j}
\]
��$x-1$�̂���Ŋ���A$\geqq 1$�Ƃ����Ĕ�r����ƁA�s�����𖞂����ő��$x$��
\[ x=np \]
�ł����āA����$x$���񍀊m���̍ő��^���邱�Ƃ��������B

�����āA�x���k�[�C��
\[
\text{�i��������H�j}
\]
�Ƒ����A�ŏI�I��
\[
\text{�i��������H�j}
\]
�ɒB����̂ł���B

\noindent
\Fig[����142���ʐ^]{\textwidth}{10\baselineskip}

�ł��邩��A��S�̍�ł��������̌��ʂ̓x���k�[�C�ɂƂ��Ă͍ŏI�I�ł͂Ȃ��A
�u�@���v���͂ނ���A�m���͌��ۂ��瑪��”\�ł��肻����u�����v(��conjectand)��
�L�Ӌ`�ɐ����������Ƃ������̂ł������炵���B
���ہA�u�吔�̖@���v�Ƃ������t����100�N�ȏ����la loi des grands nombres�Ƃ���
S. �|�A�\���i1837�j�ɂ����̂ł���B
�������A�����u�吔�̖@���v�̔��W�͒������A���́u��@���v��20���I�ɂȂ��Ă��烍�V�A�̃q���`��\footnote{
���A���N�T���h���E���R�u���r�b�`�E�q���`���F���V�A�A�\�A�̐��w�ҁB
�����̋Ɛт̓R�����S���t�̌����I�m���_(1935�N)���������B}
�i �@�|�u�{���p�~�t�� �`�{���r�|�u�r�y�� �V�y�~���y�~�j�ɂ��u���@���v�i�d���ΐ��ɂ‚�1924�N�j������B
���̏ؖ��͌��݂ł́u�m���ϐ��v��p���ē������̓X�}�[�g���ƒX�g���[�g�ɂȂ��Ă���A
���Ƃɋ��@���́u�G���S�[�h���_�v�̓���P�[�X�Ƃ��Ď��֐��_�̒藝�Ƃ������悤�B
�x���k�[�C�ɂ����̂́A�����̊m���_�̑f�p�����c�����킢�[�����i������B

\section{���ꂪ������������}

�Z���R�u�������v���‚������탈�n���ɂ���������A���邢�͒킪�Z�͉������������Ƃ��Ă���ƒm��
�����Ȃ炤�܂������邾�낤�Ƃ��A
�Ђ���Ƃ�����Z�͉����Ȃ��̂ł͂Ȃ����Ƃ��A�V�ˌZ��Ԃ͂����ނ˂��̂悤�Ȋԕ��ł������B
�����ނˌR�z�͒�ɏ��A�ΐ퐬�т͑��o�I�ɂ����Β��9��6�s�A
��̕���\underline{���w�I�ɂ�}brilliant�i�D�G�j�������Ƃ����̂��A�M�҂̈�ۂł���B

�������A�l���悤�ł͌Z���ꖇ��Ƃ������f�����肤��B
�Ȃ��Ȃ�A���̑�IV���͍ŏI�I��civil, moral and economic problem�A�‚܂�Љ�I�A���_�I�A�o�ϓI�ȓ��ɒ���ł��邩��ŁA
��������́u�m���v�ǂ���ł͂Ȃ��A�Љ�S�̂̊m�����������Ƃ��Ă���B
�x�C�Y�A���v���X�A�K�E�X�A�ȂLj̑�Ȋw�҂͂��ׂĂ���ɒ���ł���B
���������Љ�Ɂu�m���v�͂���̂��B
����ƍl����Ȃ� �e���ہA���̊m���͂����‚��B���̏ؖ��͏o����̂��f�ƒ��܂ꂽ���̓�����IV���Ȃ̂ł���B

\chapter{���n���E�x���k�[�C}

\section{���ϕ��w�̑�O�̊��胈�n���E�x���k�[�C}

���āA���悢�惈�n���E�x���k�[�C�̓o��ł���B
���n�� Johann�i�p���John�A�t�����X���Jean�j�̓��R�u�E�x���k�[�C�̒�A��q����_�j�G���E�x���k�[�C�̕��ł���B
�c�ȗF�B�I�C���[���7�ˁi�H�j��ŁA�w��̌p���̃��C����
\[
\text{���C�v�j�b�c} \, \Longrightarrow \, \text{���R�u�A���n���E�x���k�[�C} \, \Longrightarrow \, \text{�I�C���[}
\]
�Ƒ����Ă����B
�‚��łȂ���A�N��I�ɂ��̑O�́A
\[
\text{�K�����C}\, \Longrightarrow \, \text{�z�C�w���X}\, \Longrightarrow \, \text{�j���[�g��}
\]
��\footnote{
�K�����C��1640�N�v�œ��N�Ƀj���[�g�������܂�A�z�C�w���X��1629�N���܂�ŔN��I�ɗ��҂̒��Ԃł���B}�A
�j���[�g���ƃ��C�v�j�b�c�͂قړ�����ł���B
���̂��Ƃ�����A�z�C�w���X�ƃ��C�v�j�b�c�̊w�����x���k�[�C�Ƃ̊w��̔��W�̉��n��^�������Ƃ͏\���ɗ����ł���ł��낤�B
�w�����p�x�̑�I���̓z�C�w���X�̌��ʂ���ɂ��Ă���B

���������āA���n���E�x���k�[�C�͔��ϕ��w�̔��W��\kenten{�����}�Ƃ��āA
�j���[�g���A�z�C�w���X�ɑ�����O�̐l�ƂȂ����l�ł���B
�Z�Əd�Ȃ镔���͑������A�ƐтƂ���
\begin{enumerate}
\item �����̋t�Ƃ��Ắu�ϕ��v�̑n��
\item �����p�����Ȑ��_�̊􉽊w�i1.�̉��p�j
\item �ϕ��@�̔��z�i1.�A2.�̉��p�j�ƍŏ��ő剻
\item �����_
\item ���ϖ@�̃��X�g�i1.�A2.�̔��W�j
\end{enumerate}
�͔��ϕ��w�̒��S�����ł���A����p��Ƃ��Ă�
1.�͂�����u�����ϕ��w�̊�{�藝�v�A
2.�́u�����������v�ŁA���̗p��͓����͌����炸�A�܂��e�����f�Ƃ��킸�e�ϕ�����f�i���ϖ@�j�ƌ������B
3.�͈�ʉ�@�̓I�C���[�A���O�����W���Ń��[�y���e���C�̔��z����͊w�Ɏ�����u�������A
�x���k�[�C�͂��̑O��i�K�Ƃ��Ă����ɂ����
�����‚��̋Ȑ��i�֐��ł����邪�j���􉽊w�Ƃ��Ē�߂�͌^��������4.�́A
�t���I�戵���̌���ƈقȂ�A���̌���ɂ����Ă͓W�J�̃c�[���Ƃ��ďd�p�A�e���p�f���ꂽ�B
��ʓ񍀒藝�i�j���[�g���j���͂��߁A���X�̋����W�J��a�̌��������_�̎����S�����B
�x���k�[�C�ɂ������‚��̋����W�J�ɂ���ϕ�������B
5.�͍����́u������������̉�@�v�Ƃ���������̃��X�g�ɂȂ��Ă���B
���̂悤�ɁA���n���E�x���k�[�C���͌^�������A
�I�C���[�͂����̌n�����ēK�p�͈͂��g�債�c�����s�ɂ��̓V�˓I�\�͂𔭊������̂ł���B

�O�シ�邪�A���n���E�x���k�[�C��1667�N�o�[�[���ɐ��܂��B
�Z���R�u���13�Ή��ł������B
����������w�̓����w������邪�e���߂��A�Z�ɂ‚��Đ��w���w�ԁB
1694�N������Ԃ��Ȃ��I�����_�̃O���[�j���Q����w�ɐ��w�̐E�𓾂邪�A�{���]��ł����o�[�[����w�̐E���Z����΂ɏ���Ȃ��������߂Ɖ]����B
1705�N�o�[�[����w�̃M���V����̋����Ƃ��ĕ��C�̋A�r�ɌZ�����j�ŕa�v�Ƃ̕���󂯁A���̌�C�Ƃ��Đ��w�����ƂȂ�B
���łɃO���[�j���Q������ɖ{�i�I�ɓ������������ȂǑ����̗��j�I�Ɛт������Ă������A���̌�����͓I�ɋƐт������鐨���͐����Ȃ������B
���̊Ԏq�_�j�G���A����ŎႫ�I�C���[�ɐ��w���l�w���Ŏd���񂾂��Ƃ͂悭�m���Ă���B

�V�˔��̊w�҂𑽐��y�o�����x���k�[�C�Ƃł��邪�A
�Ƃ�킯���̃��n���̍U���I�Ƃ�����������S�Ǝ��M�Ƃ˂��݁i�����ƐS�j�͓��M�����B
�ŏ��͉����ȌZ���R�u�A��N���q�̃_�j�G���Ɍ�����ꂽ�B
�Ƃ͂����A�ϐl�A��l�������Ȃ��Ƃ���Ă��鐔�w�҂̒��ł͋��e�͈͓��Ǝv���A
���ǂ͋����‚‹��͂̌��ʂƂȂ������ƂŁA��Ƃ̂��̈̑�ȋƐт����j�Ɏc�邱�ƂɂȂ����̂ł��낤�B

\section{$dx, \, dy$�̓~�X�e���A�X}

���Z�̐��w�̃N���X�E���[���ł́A$\Delta x, \Delta y$�͏����ȍ��i�ω��ʁj�Ƃ��ċ������Ă���B
�������Ȃ���A$dx, \, dy$���̂͋����Ȃ����A�������Ȃ��B
����$dx/dy$�Ƃ��ϕ��L���̒��ɕ����Ƃ��ďo�Ă���̂𓪂���F�߂�A�Ƃ�����ɂȂ��Ă���B
���͂���ł����̂ł���B

$dx$�i$dy$���j�́e�����ɏ�����\kenten{��}�f�‚܂�
�e$\Delta x \to 0$�f��$\Delta x$��\�킷�A�Ɛ��������B
�������A���炩�ɁA�����ɏ�������Ε����ʂ�$dx \equiv 0$�ƂȂ��ĈӖ����Ȃ��Ȃ��B
�C�M���X�̐_�w�҃o�[�N���[�i���͂��Ƃɍ��K������ᔻ�����B
�����Ƃ��ł���B
�Ƃ���ŁA�ʂł͂Ȃ��e�����ɏ���������\kenten{����}�f�ł����āA
������$dx, \, dy$�����̂悤�ȑ���Ƃ��āA
��$dy/dx$���̂́i���x�́j���̂����‚��Ƃ�����Ƃ���΁A���̔ᔻ�͓��ʉ������A���ꂪ�����ł���B

�ł́A�P����$dx, \, dy$�Ə����Ă͂����Ȃ����Ƃ����^�₪�����B
�I�C���[�̓����́A�������ɖ����̓N�w�I�E�_�w�I��������ł��낤�A���_�͊��R�Ƃ��Ă��āA�u�����ɏ������v�܂܂ł悢�A�Ƃ����B
�I�C���[�Ɠ�����ŃI�C���[�̒n�ʁi�x�������E�A�J�f�~�[�j���Ŏ������_�����x�[���͒m�I���@�͂ɂ�����A
�I�C���[�ƈӌ��𓯂��ɂ����B
���̂悤�ɁA�x���k�[�C�̎��ӂ‚܂胉�C�v�j�b�c����A�Ȃ���̎��ӂł͖������͎󂯓�����Ă���A
���ꂪ�I�C���[�́w��������́x�Ɏ������񂾂̂ł���B

���O�����W���͖�����$dx$���߂��邱�̘_�c�ɂ͒�����ۂ��č����g�������A�u�����v�Ƃ������A
�V�����֐��������ɓ����ꂽ�Ƃ����Ӗ��Łu���֐��v\ruby{derivative}{�f���o�e�B�u}�A
���Ƃ̊֐����u���n�֐��v�Ƃ�񂾁B
�Ȃ��Ȃ����������@�ł��邪�A���̍l�����ɂ��������������̂�19���I��͊w�̃G�[�X�A�R�[�V�[�ł���A]
�e��w�����ꂵ�߂�f$\epsilon$--$\delta$�_�@���͂��߂Ƃ��āA
�֐��_�̐��ʊ֐��A�R�[�V�[�̐ϕ��藝�A�ϕ������Ȃǂ̐��������Ȑ��E���ł�������̂ł���B

\section{\underline{1690�N}�A���悢��u�ϕ��v���o��}

Sed exiis quae in method tangentium exposui, patet este d, 1/2 $xx=xdx$;\, 
ergo contra 1/2 $xx=\int xdx$ (ut enim potestates \& radices in vulgaribus calculis,
sic nobis summae \& differentiae seu $\int$ \& d. reciprocae sunt.\, p.130

���M��A���遄

���C�v�j�b�c�ɂ͍��i���邢�͍����c�c�j����јa�isumma�j�Ƃ����p�ꂵ���Ȃ������B
���̋L�@��d�����A�a�̋L�@�es�f�͓����̈�����V�ł�f�ƌ��������‚��Ȃ�����\footnote{
������񃉃e���ꂾ����Ƃ����̂łȂ��p��ł������ł���A���ꂳ�����Ӑ[�����ǂ���΁A18���I�p�ꕶ���ł��ǂ߂�B
������̓x�C�Y�̕����ł���B}�B
�����ňӖ��͂��̂܂܂Łes�f�͏㉺�Ɋg�債�āe$\int $�f�Ƃ���A���ꂪ���݂܂Ŏg���Ă���B
�Ȃ��Z�ʂނ���$\int $�̑ւ��$\sum$���g��ꂽ�B
�����s�̑Ή�

\noindent
\Fig[����81���}]{\textwidth}{5\baselineskip}

���āA�{�_�ɍs���ƁA�a�͋Ɍ��ł͂������e�ϕ��f�ƂȂ�̂����A����͒P�Ȃ�a�ł͂Ȃ�����A
���n���E�x���k�[�C�͂��̘a�Ɂu�ϕ��v�Ƃ��������p��ivocabulum integralis�j��^�����̂ł���\footnote{
�����ɗp����ꂽ�̂�1690�N�B}�B
�ϕ��͎��̂Ƃ��ĖʐρA�̐ρA�c�c���Ӗ����邪�A�����̎��̂Ȃ�΃A���L���f�X���n�߂Ƃ��āA
�J�����G���A�t�F���}�[�ȂǁA�����Â��v�Z�@�̓`�����������B
���ϕ��ɂ�����ʐόv�Z�͂����𒴉z���ċ����ׂ����̂ł���B
����́A���ϓI�ɂ͖��֌W�Ǝv������肾��
\underline{�ʐόv�Z�i�ϕ��j�͔����̋t���Z�ł���}�Ƃ������̂ł����āA
���ł́u���H�v�Ƃ������ƂȂ����Z���ł��m���Ă���A�^���}�G�̂��Ƃ���ł���B
�����͐��I�̑唭���ɂ��͂△�����ɂȂ��Ă���̂ł���B

���R�u�E�x���k�[�C�͂���Ƃ��Ȑ��̖����l���Ă����B

\begin{quote}
---���̂�����Ȑ�$y(x)$�i$x$�̊֐��j�ɉ����Ċ��藎���čs���Ƃ��A
���̉��������̑��x$dy/dt$���ǂ��ł��^����ꂽ���x$-b$�ɂȂ�悤�ɋȐ����߂�i���C�v�j�b�c�j---
\end{quote}

�͊w�̋����鏊����---�������K�����C�ɂ���Ċm���߂��Ă����悤��---
$v=\sqrt{-2gy}$�ł��邩��A$v^2$��$x, \, y$�����ɕ�������
\[
 \left( \dfrac{dx}{dt} \right)^2 
+\left( \dfrac{dy}{dt} \right)^2 =-2gy %\label{82}
\]
�����$(dy/dt)^2=b^2$�Ŋ�����$dt$�������A$dx/dy$���t���ɂƂ��
\[
\dfrac{dy}{dx}
=\dfrac{-1}{\sqrt{-1-2gy/b^2}} %\label{83}
\]
�𓾂�B
������񂱂�͌��㕗�ŁA�����x���k�[�C�͂ނ���
\[
dx=-\sqrt{-1-\dfrac{2gy}{b^2}} \, dy %\label{84}
\]
�ƌ��Ă����B
�����$dx, \, dy$�����ɂ���

\begin{quote}
����́A�����́i$x$�����$y$�́j\kenten{�ϕ�}�͓����� \\
Ergo \& horum integralia aequantur
\end{quote}
�Ƃ���\footnote{
$\text{ergo}=\text{����䂦}, \, \text{horum(hoc)}=\text{������}, \, \text{aequantur(aequo)}=\text{������}$�Ƃ����}�B
���ꂪ���j��͂��߂āu�ϕ��v���p��Ƃ��Ă��–{���̈Ӗ��ɂ‚��Ďg��ꂽ�ŏ���1690�N�̂��Ƃł���B
�u�ϕ��́`�v�Ƃ����Ƃ��낪�{���I�ŁA���㕗�ɂ́u���́`�v�Ƃ����̂��낤���A
�葱�����ꂽ�u�����������v�͖���������������ł���B
���̏q������R�u�A���n���E�x���k�[�C�̕����Ɍ��o���͓̂�����A�_�j�G���̒��ɂ͂͂�����Ƃ������B

���߂�$y$�͂ނ���t�֐��Ƃ���
\[
x=\dfrac{b^2}{3g} \left( -1-\dfrac{2gy}{b^2} \right)^{3/2} %\label{85}
\]
�ł���A���C�v�j�b�c�ɂ���āe3/2��̕������f�Ɩ��t�����Ă���B
�������āA1690�N�������ϕ��w�ɂ�����u�ϕ��v�̌��N�ƍl���邱�Ƃ��ł��悤�B

\section{�Ȑ��A���ƂɌ����ʂɂ�����u�ȗ��v�ɂ‚���}

�����A�S���⍂�����H�͎s�X�n���ʂ��đ����Ă���A��s�s���������H�ł͎��X�ƖڑO�ɔ���J�[�u�̘A���̏����ɑ����ŁA
�ߍx���痈��l�ł����Ă��ƂĂ��s��ē��W���Ȃnj���q�}�͂Ȃ��Ƃ����B
���ɓ����̓s�S����̓J�[�u���炯�łقƂ�ǒ����������Ȃ��B
���ł����l�ł��낤�B
�s�v�c�ɁA2�_�Ԃ̍ŒZ�����ł���\{ ���� \}�Ƃ����̂́A�������H�ł͒������̂ł���B

\noindent
\Fig[����86���ʐ^]{\textwidth}{10\baselineskip}

���āA�Ȑ��̖{�i�I�����͋ߑ�ɓ����Ă���̂��̂ŁA�����͉~���Ȑ��i�ȉ~�A�o�Ȑ��A�������j����A
�f�J���g�ɂ悭�􉽊w�Ɣ��ϕ��̔��W�ɂ��A
�T�C�N���C�h�A�J�e�J���[�i�������j�A�点��i�X�p�C�����j�ȂǂɌ������y�񂾁B
���R�u�A���n���E�x���k�[�C�Z��͂��܂��܂ȋȐ��̌���������B
���ƂɃ��R�u�E�x���k�[�C�͑����̉ۑ�̒񎦎҂ł���A�탈�n���́u�Z�v�ɋ��͂����苣������o����������ŁA
�Z���ߏ�Ɉӎ����Ă����B

�x���k�[�C�Z��̋Ȑ��_���q�ׂ�O��
1. ����A
2. �ȗ��~�A�ȗ����S�A�ȗ����a�A�ȗ��A
3. �k���A�L�J�����q�ׂĂ������B

{\gt ��ȗ���} �R�����^�]���Ă��āu�}�J�[�u�A�X�s�[�h�����v�Ƃ��u���̃J�[�u��$R=\cdots $�v�Ȃǂ̉^�]�҂ւ̒��ӁA
�x���̕W�������邱�Ƃ����邾�낤�B
�ǂ̂悤�ȋȐ����ׂ�������Ή~�ŋߎ�����邩��~�̔��a$R$�ŃJ�[�u�̂��‚���\�킷���Ƃ��ł���B
$R$���傫���قǃJ�[�u�͊ɂ��A�������قǂ��‚��B

�Ԃ��^�]���Ȃ��Ă��g�߂ȃP�[�X������B
�����R����ŃJ�[�u�̔��a���������
\begin{align*}
\text{���E�i��Ԃ̃J�[�u}   R&= \\
\text{�����E�L�y���̃J�[�u��} R&= 
\end{align*}
�������d�ԉ^�]�m�͂���$R$�ɂ�鑬�x�K����������Ǝ��Ȃ����
�E���]���̊댯��傫�����邱�Ƃ͂����܂ł��Ȃ��B
�S���t�@���Ȃ�m���Ă��邪�A�S�����[���͂��ׂĂ̒n�_�ŁA�^�s�̂��߂̗����W�A���z�W�A
�ȗ����a�i$R$�j�A�o�[�j�A�i�X�΁j�Ȃǂ̃f�[�^���̕\��������B

\noindent
\Fig[�ʐ^2��]{\textwidth}{10\baselineskip}

�R�z�V��������r�I�J�[�u�������A����$R=\cdots $�ł���Ƃ����B
�V�����̏ꍇ�A�����ł��邽�߂̃J�[�u��$R$��傫�����Ȃ���Ί댯�ł��邪�A
$R$��傫�����邱�Ƃ́A���n�v��ɑ傫�Ȑ��񂪂����邱�ƂɂȂ茚�݋Ƃɂ��e�����o��B

���̂悤�ɂӂ‚��A�Ȑ��̋ȗ��͂����2���܂Őڂ���~�̔��a�i�ȗ����a�j$R$�̋t��$1/R$�Œ�`����Ă���B
���Ƃ��΁A���_�𒆐S�Ƃ��锼�a$R$�̉~
\[
f(x)=\sqrt{R^2-x^2}
\]
��$x=0$��2���܂œW�J����ƁA�ߎ��I�ɕ�����
\[
f(x)=R-x^2/2R+\cdots 
\]
�ŁA��������
\[
f''(0)=-1/R
\]
�ƂȂ�A���̂��Ƃ���Ȑ��̒��_�ł̋ȗ��͂���2�K���֐�$f''$�ɂȂ�Ƃ̑z�����‚��B
�܂��Ȑ��̉��ʂ̗l�q��2�K���֐��ɂȂ�Ƌ������Ă���B
����͐��m�Ȓ�`�ł͂Ȃ��B
���ہA�����Ȃ�ƁA���a$R$�̉~��$x=0$�ȊO�ł͋ȗ����K������$1/R$�ł͂Ȃ��Ȃ�B
�Ȑ��̋ȗ��̒�`�͂���������ʓI�Ȃ��̂ł���A�����̒�`�ɂ͈�K���֐���������B

�Ȑ�$f(x)$��$x=c$�ł̋ȗ��Ƃ́A���̓_�ł̐ڐ��̌X����$x$���̐������ƂȂ��p�x$\theta$�Ƃ��āA
����$\theta$�̋Ȑ��̌ʒ�$s$�ɑ΂���u�ԓI�ω����i�u�ԓI��]���x�j
\[ d\theta/ds \]
�������B

�‚܂�A���̋Ȑ��ɉ����Đi�ނƂǂꂾ����������邩��\���B
����́A$\theta=\mathrm{arc} \tan f'(x)$�Ƃ�����
\[
d\theta/d��=d\theta/ds \cdot ds/dx
\]
����A
\[
\text{�i��������H�j}
\]
�����
\[
\text{�i��������H�j}
\]
����e�Ղɋ��߂��
\[
\text{�i��������H�j}
\]
�𓾂�B
���������āA�ɑ�_�A�ɏ��_�ł́����ƂȂ�B

\noindent
{\gt ��}\, �~�ł�
\[
\text{�i��������H�j}
\]
�ƂȂ��āA�������ɂ�����Ƃ���ȗ���$1/R$�ł���B

���̂��Ƃ���A�ȗ����a��
\[
\text{�i��������H�j}
\]
�܂��A�ȗ����S�͓_$(x_0, \, y_0)$�̖@����ł��̓_���狗�����������������_
\[
\text{�i��������H�j}
\]
�ł���B
���̋ȗ����S�𒆐S�Ƃ��ȗ����a�ʼn~��`���ƁA���̉~����$(x_0, \, y_0)$�ɂ����Ă��̋Ȑ��ɐڂ���B

�܂��A���΂��΂��邱�Ƃ����A�Ȑ�������ꎟ���p�����[�^�ŁA
\[
\text{�i��������H�j}
\]
�̂悤�ɕ\������Ă���ꍇ�́A�ȗ��A�ȗ����a��
\[
\text{�i��������H�j}
\]
�ƂȂ�B

{\gt ��k����} �Ȑ���̓_�ɑ΂���ȗ����S�́A���̌����Ȃ��_�𒆐S�ɁA
���̋Ȑ������񂾂�Ɓu�������܂�āv��荞��ł����悤�ȓ_�ŁA�Ȑ���œ_���ړ�����Ƌȗ����S���ړ������̋O�Ղ͈�‚̋Ȑ�����邪�A
���̋Ȑ����u�k���v�Ƃ�ԁB

�k���ɂ͂�����ʂ�̃X�g���[�g�ȋ��ߕ�������B
�����
\[
\text{�i��������H�j}
\]
�ƂȂ�B

\noindent
{\gt ��}\, �t�ɁA�k�����猳�̋Ȑ��������ł���킯�ŁA���̋Ȑ����k���ɂ������āu�L�J���v�Ƃ����B

{\gt ������} ���ϓI�ɂ́A�Ȑ����i�Ȑ��̏W�܂�j�������Ă���̈�̋��E�ɂȂ��Ă���Ȑ��ŁA
���̃C���[�W����e��݊܂ސ��f�ł��邪�A���m�ɂ͂��̋Ȑ����̋Ȑ����ׂĂɐڂ���Ȑ��ł���B
���Ƃ��΁A���_O�����苗��$d$�ɂ��钼���Q

\begin{align*}
&\ell_\alpha: x \cos \alpha+y \sin \alpha=d \\
&\text{�i}\alpha \text{�̓p�����[�^�ŁA���_����̐����̕����p�j} %\label{87}
\end{align*}
�̕���͂�����񌴓_�𒆐S�Ƃ��锼�a$d$�̉~$\mathrm{C}: \, x^2+y^2=d^2$�ł���B
�����$\ell_\alpha$�̎�
\begin{equation}
x \cos \alpha+y \sin \alpha -d=0 %\label{88}
\end{equation}
������$\alpha$�ł̔���
\begin{equation}
x (-\sin \alpha) +y \cos \alpha =0 %\label{89}
\end{equation}
�ƘA������$\alpha$����������Γ�����B
(1)��$d$���ڍ���A�����𕽕����ĉ������$\mathrm{C}$�𓾂�B

��ʂɋȐ���$\mathrm{C}_\alpha$�i$\alpha$�̓p�����[�^�j�̕����$\mathrm{C}_\alpha$�̎���
����$\alpha$�ł̔���$\text{�i�Δ����j}=\mathrm{O}$����$\alpha$���������ē�����B
���Ƃ���

---���������ʋ��ł���悤�ȏ�ɊJ���������i$\text{���a}=1$�j�ɁA
�㕔���畽�s�Ɉꕽ�ʓ��œ��˂�������Q�́A���ˌ�̌����Q�̕��---

���S����$a\, (0 \leqq a <1)$�������ꂽ���ˌ��̔��ˌ��̕�������
\begin{align*}
\ell_\alpha: 
&y=-b+\dfrac{1}{2}\left( \dfrac{b}{9}-\dfrac{a}{b} \right) (x-a), \\
&b\text{�͔��˓_�܂ł̐������B�����i}a=0 \text{�Ȃ�}b=1, \\
&a=1/2 \text{�Ȃ�}b=1/2, \, a \to 1 \text{�Ȃ�}b \to 0 \, \text{etc.�j} %\label{90}
\end{align*}
�����
\[
\partial y/\partial \alpha =0 %\label{91}
\]
��A�����āA$a$��������
\[
y=-\left( x^{2/3}+\dfrac{1}{2} \right) 
  \sqrt{1-x^{2/3}}, \qquad -1 \leqq x \leqq 1 %\label{92}
\]
��������i���n���E�x���k�[�C�C1691,2�j�B

---��C���C�e��������p�x�ŏ���$v_0=1$�őł��グ��Ƃ��̂��ׂĂ̒e���������̕��---

��C�̌��z��$a$�Ƃ���ƁA�e���̕�������
\[
\mathrm{C}_a: 
y=ax-\dfrac{1+a^2}{2} x^2 %\label{93}
\]
���̎���$a$�ŕΔ�������O�Ƃ����A�㎮�ƘA������$a$����������΁A���
\[
y=(1-x^2)/2
\]
�𓾂�i���n���E�x���k�[�C�j�B
�C�e�͋󒆂ł̗������Ƃ��Ĉ�ʉ�����邩��ԉ΂��l���邱�Ƃ��ł��悤�B

---����$x\text{�ؕ�}+y\text{�ؕ�}$���萔13�ɓ����������Q�̕��---

�����Q
\[
\ell_a: 
y=\dfrac{a-13}{a}(x-a)
\]
�̕�������$a$�Ŕ�������$\mathrm{O}$�Ƃ����A$\ell_\alpha$�ƘA������$a$�����������
\[
(y-x-13)^2=52x
\]
�ƂȂ�B
��]�������W
\[
u=(x+y)/\sqrt{2}, \quad 
v=(x-y)/\sqrt{2}
\]
�𓱓�����B
$x, \, y$�ɂ‚��ĉ����������΁A
\[
2v^2=26\sqrt{2}u-169
\]
�ŁA$u$���𒆐S���Ƃ���������ł���B

����͈�ʂɔ��`�̈�ۂ�^���A�ԉ΁A�����A�G��̍\�}�Ȃǂɑ������o�����B

{\gt ��ȗ��~�A�ȗ����S�A�ȗ����a�A�ȗ���}

{\gt �������d�������\�聙����}

\noindent
\Fig[������95�|105��������]{\textwidth}{10\baselineskip}

{\gt ��k���A�L�J����}

{\gt �������d�������\�聙����}

��ʂɂǂ̂悤�ȓ��H���邢�͓S���̃J�[�u�ł��P���łȂ��A���܂��܂ȋȗ��i�ȗ����a�j�̃J�[�u���ω����Ȃ���A�Ȃ��Ă���B
���R�̂��ƂȂ���h���C�o�[���邢�͉^�]�m�͂��̕ω��ɍ��킹�ĉ^�]���Ȃ��Ă͂Ȃ�Ȃ��B
�Ȑ��̊􉽊w�Ƃ��Č����ƁA�ȗ��~�Ƃ��̋ȗ����S�����X�ƈړ����邪�A
�ȗ����S�̓������O�Ղ��u�k���v(evolute)�Ƃ����B

�J�[�u����E�o���Ă����Ƃ��A�ȗ����a�͑傫���Ȃ�ȗ����S�����X�ɊO��ēW�J���čs������A
�p��ł́e�O�֓]����f�ievolute�̌ꊴ�j�ƂȂ�B
����ɑ΂��A���{���̓J�[�u�֐i������C���[�W�ŁA
�ȗ����a�͏������ȗ����S���Ȑ����̂ɋߊ��‚“����͓݂��Ȃ��āe�k�f�̌ꂪ�ӂ��킵���Ȃ�B

{\gt �������ȍ~�A�C���\�聙���� �i�ȉ~�ϕ����j}

\section{�x���k�[�C�Z��̋Ȑ��_}

�x���k�[�C�Z��́A���C�A�Η����‚A�S�̓I���ʂƂ��āA�Ȑ��̌����ɂ����āA�����ϕ��w�̊�b�T�O�̊m���A
�����w�̒���Ƃ��̉��p�A�ő�ŏ����A�ϕ��@�̗��O�̔��i�Ȃǐ��X�̋Ɛт��㐢�Ɏc�����B
���̂����‚������r���[���Ă݂悤�B

{\gt �ᓙ���~���Ȑ� iso chrone, 1690��}

$\text{iso}=\text{������}, \, \text{chrone}=\text{����}$�ł��邪�A
�e���ԁf�𑬂��ɓ]�`���Ă���B
���̕��͂͂��łɉ�������B

{\gt �ጜ���� catenary, 1691��}

$\text{catena}=\text{��}$�Avel funicularis�ŁA�����ɂ�linea catenaria(e) �ނ艺����ꂽ�����邢�̓��[�v�̌`�������B
�����K�����C�͂���͕������Ǝ咣�������A�z�C�w���X�͂���͌��ƌ������A�����Z���X�_���͌����ؖ����Ă���B
\[
\text{����106��������B������} %\label{106}
\]
���ꂩ��
\[
c \int \dfrac{dp}{\sqrt{1+p^2}}=\int dx %\label{106'}
\]
���ӂ�ϕ�����Ɓi�ĂсA���ӂ̐ϕ��͓������̂Łj
\[
\mathrm{arc}\sinh (p)=\dfrac{x-x_0}{c} %\label{107}
\]
�������A$\mathrm{arc} \sinh( \, \cdot \, )$�͑o�Ȑ����֐�
\[
\sinh u=\dfrac{e^u-e^{-u}}{2} %\label{108}
\]
�̋t�֐��ł���B
����������
\[
p=\sinh \left( \dfrac{x-x_0}{c} \right), %\label{109}
\]
�ŁA$p=dy/dx$�����猋��
\[
y=c \cdot \cosh \left( \dfrac{x-x_0}{c} \right)+y_0 \quad (y_0: \text{�ϕ��萔}), %\label{110}
\]
�ƂȂ�B�i���C�v�j�b�c�A���n���E�x���k�[�C,  1691�j

�Ȃ�
\[
\cosh (u)=\dfrac{e^u+e^{-u}}{2}=1+u^2+\cdots  %\label{111}
\]
�ł��邩��A�������͍ʼn��_���ӂł͕������ƍ�������B

{\gt ���\underline{�Z}�~���� Brachistochrone, 1696��}

����ɂ‚��Ă͊��ɉ�������B
�Ăы������Ă����ƁA�ۑ��$\text{�}=\text{����}$�Ƃ��čœK�����Ă���A�ۑ�Ƃ��Ă͓���ł͂��邪�A
���ꂪ�ϕ��@�̔��z�̗��j�I�N�_�ł��邱�Ƃ𒍈ӂ��Ă������B

{\gt �ᓙ����� isoperimetry��}

�j��n�߂Ă̖{�i�I�ȕϕ��w�̉ۑ�ł��邪�A
���{�ł͌ÓT�I�ȓ�_���v�w�ϕ��w�x���邢��Courant-Hilbert�ɂ̓x���k�[�C�ւ̌��y�͂Ȃ��B
��‚̕����ɂ��΃��R�u���ۑ��񎦂��A���n�������������̂������ł����āA����͋��Ŋ�g�w���w���T�x(��i���g��)�ɂ����f����Ă���B
��ʂɖ{���T�́u�x���k�C�Ɓv�̍��͔��ϕ��w�̐����ւ̈�Ԃ̍v���Ȃǖڔz��悭�q�ό����ɂ����Ă���B

������u���ꓙ�����v�Ƃ�
\begin{quote}
---���ʏ�ɂ����ė^����ꂽ����($L$)�̕‹Ȑ��ł��̈͂ޖʐ�($F$)���ő�ɂ���---
\end{quote}
�ł���A����ƖړI�֐�����ʉ�����΁u��ʓ������v�ƂȂ�B

�‹Ȑ��̕�������
\begin{align*}
&x=x(t), \, 
 y=y(t), \, 
 0 \leqq t \leqq 1, \\
&x(0)=x(1), \, 
 y(0)=y(1)
\end{align*}
�Ƃ���΁A$d/dt$��${}'$�Ƃ���
\[
L=\int_0^1 \sqrt{{x'}^2+{y'}^2} \, dt =\text{���} %\label{112}
\]
�̏�������
\[
F=\int_0^1 (xy'-x'y)dt /2 %\label{113}
\]
���ő剻������ƂȂ�B
����ɂ‚��ẮA�L���́u�����s�����v
\[
L^2 \geqq 4\pi F
\]
���m���Ă��邪�A���ϓI�ȉ��͂������~�ŁA���a$r$�Ƃ��ė��ӂƂ�$4\pi^2 r^2$�ɓ������B

�����œ���������ʓI�ɉ������Ƃ̓��x�����������Ƃ��瑼���ɏ��邪�A�I�C���[����ɂ̓��O�����W���ɂ���ʉ�@������\footnote{
�M�҂͑O�f��_�������Ă��邪���ɐ�łł���B�Ȃ�Courant-Hilbert��Hurwitz�ɂ�������̓I�ɉ�����Ă���B}�B

�ϕ��@�͓����Ȑ��̊֐������肷��􉽊w�̖�肩��X�^�[�g���A�I�C���[�A���O�����W���̌������o�āA
�n�~���g���A���R�r�ɂ���ĉ�͗͊w�̒��S�I���@�Ƃ��Ċm�������B
�������A�H�w�A�o�ϊw�ɂ����鐔���I���@�̓����ɂ��A�œK�֐��̌���̗L�͂ȕ��@�Ƃ��Ăӂ����яd�v�Ȗ������ʂ��Ă���B

�ϕ����͑����̋ɒl���ɒ��ړI�ɓK�p�����B�����ł��̂����‚����Љ�悤�i�����j�B
\begin{enumerate}
\item ���n��
\[
\text{�i���E���j}
\]
\item ���H
\[
\text{�i���j}
\]
\item �ő��~����
\[
\text{�i���j}
\]
\item �ŏ��\�ʐς̉�]�Ȗ�
\[
\text{�i���j}
\]
\item �ŏ��ȃA�t�B���I�����̋Ȑ�
\[
\text{�i���j}
\]
\item �ŏ���p�̌���
\[
\text{�i���j}
\]
\item ���ꓙ�����
\[
\text{�i���j}
\]
\item ���̕��t
\[
\text{�i���j}
\]
\item �f�B���N�����
\[
\text{�i���j}
\]
\item �ɏ��ǖʁi�v���g�[���j
\[
\text{�i���j}
\]
\end{enumerate}

\section{���߂��炵���֐��̐ϕ���������}

�񍀒藝
\[
(1+x)^n=\sum_{k=0}^n \binom{n}{k} x^k %\label{114}
\]
�ׂ̂��w��$n$�͂�����񐳐��������A���ꂪ���������邢�͔񐮐��ƂȂ���
\[
(1+x)^{-2}, \quad (1+x)^{2/3}, \, \text{etc.} %\label{115}
\]
�́u��ʓ񍀒藝�v�Ƃ����A����$\alpha$�ɑ΂�
\[
(1+x)^\alpha=\text{dummy} %\label{116}
\]
�Ƃ��Ēm���Ă���B

$\alpha=n$�i�������j�łȂ���΁A����͖��������ƂȂ邪�A�R���s���[�^�̂Ȃ�����ɂ��̖���������
�L���v�Z�@�Ƃ��ďd�󂳂ꂽ���Ƃ͂������A
���̖����������̂��������W���ł��������ϕ��̒��ŏd�����_�I�������ʂ��Ă������Ƃ͑z���ȏ�ł���B
���ہA�j���[�g������ʓ񍀒藝��
\[
\sqrt{1-x^2}=(1-x^2)^{1/2} %\label{117}
\]
�̐ϕ��v�Z�ɗp�������Ƃ͂悭�m���Ă���B
�Ƃ͂����A��ʓ񍀒藝�����L���A�e�C���[���邢�̓}�N���[�����W�J�̈��ł��邩��A
��ʂ̖��������ւ̗��_�j�[�Y�����������Ƃ����ׂ���������Ȃ��B

���̂悤�Ȓ��Ńx���k�[�C�̐l�X�����ɖ��������_�œ��L���ׂ����ʂ𑽂��������Ƃ��������͂Ȃ����A
����ł������‚��ʔ������ʂ𓾂Ă���B
���̈�‚�
\[ y=x^x \]
�Ƃ������Ȃ�߂��炵���֐��̐ϕ�
\[
S=\int_o^1 x^x \, dx %\label{118}
\]
�ł���B
$x \log x \to 0 \, (x \to 0)$����$0^0 \equiv 1$�ƋK�񂷂�΁A��ϕ��֐��͘A�����ϕ���ԂŗL�E������A
�L���Ȑϕ��l�����B
�Ƃ���ŁZ�Z�Z�c�Z�Z�Z�ł���A������$e^u$�̓W�J����p���A����$x^n(\log x)^n$�ɑ΂��A
$\log x$��������ɂ��������ϕ����s����
\[
\int_o^1 x^x \, dx
=1-\dfrac{1}{2^2}+\dfrac{1}{3^3}-\dfrac{1}{4^4}+\dfrac{1}{5^5}-\cdots %\label{119}
\]
�Ƃ�������܂��߂��炵�����ʂ𓾂�B
���ہA�E�ӂ͌�㋉���̔�������ɂ���������B�i���n���E�x���k�[�C�C1697�j

\section{�����������������Ƃ�}

���‚č��Z�̐��w�̐搶�Ƙb�����Ƃ��u�����͂�����񂢂����e�����������f�̓J���L�������O�Ń_���Ȃ�ł��v�Ƃ����
�i�]�k�����A$x^3$�̔����͂�����$x^4$�̓_���������ł����j�A
�����Ԃ�J���L�������R�c��̕��X�͖�̂킩��Ȃ���m���Ɓi���́j�Q�������̂ł���B

�����A���͂����l���Ă��Ȃ��B
���j�𒲂ׂĂ݂�Ɣ����������́u�ϕ��v�ɕ�ۂ���i���R�u�E�x���k�[�C�C1690�j�Ƃ藧�Ăāu�����������v���u�����v�Ƃ͂���Ȃ������B
���ׂĂ���͈͂ł́u�����������v�Ƃ����^�[���ɂ��o���Ȃ��B
�����Ȃ�ƁA������u���ϖ@�v�Ƃ����֗��ȗp���reclundant�i�]�v�Ȃ��́j�ƂȂ�̂��낤���B
�����̋��܂ň��ǂ����ꏼ�M�w��͊w�����x�̍����ɂ͕\�ꂸ�A�w��͊T�_�x�ɂ��o�����Ȃ��B

�����Ƃ��u���ϖ@�v�͈ꏼ�ł͖{�����Ɏg���Ă���A
\begin{quote}
�������̕ό`�A�ϐ��ϊ��A�s��ϕ���L���񂭂�Ԃ��ĉ�������
\end{quote}
�Ɩ��m�ɒ�߂��A�܂������u�ϕ��Ƃ���������̈Ӗ��Ɏg���K�������邪�A����͋�ʂ��ׂ��p��ł���v�Ƃ����Ă���B
�����Ȃ�΂��낢��Ƒ�ςł���B

�{���̓x���k�[�C�Ƃ̐��w�I���ւ𒲂ׂ邱�Ƃ��ړI�ł���̂ŁA���T�O�����̂悤�Ɍ���I�Ӗ��ɕ�������͈͂ɓ��݂��ނ��Ƃ͂��Ȃ��B
�������ł����Ă��A�x���k�[�C�Z�킨��у_�j�G���E�x���k�[�C�𒆐S�ɁA
�K�v�ŏ����ŃI�C���[�A���O�����W���܂ł��J�o�[����ɂƂǂ߂���̂Ƃ���B
�i$\Gamma$�֐��A$\zeta$�֐��̓x���k�[�C�����[���ѓ����Ă���̂ŁA���̌���ŐG��邱�ƂƂ����B�j
�����ŁA���̕ӂŃ��R�u�E�x���k�[�C�������悤�Ɂe����䂦�A�����̐ϕ��͑��������f�Ƃ�������ʼn���������������������Љ�čs����
\footnote{�n�C���[�ƃ��@���i�[�w��͋����x�ɏ]���B}�B

{\gt ��ϐ������`��} \quad
$y'=f(x) \cdot g(y)$

����͂悭�m����悤�ɁA���e�x���k�[�C�I�Ɂf
\[
\int \dfrac{dy}{g(y)}=\int f(x)dx +C 
\]
�Ƃ��ĉ�����B

{\gt ����`����������} \quad 
$y'=f(x) \cdot y$

��L��$g(y)=y$�̃P�[�X�ł����āA
\[
y=C \cdot \exp \left( \int f(x) dx \right) %\label{121}
\]

{\gt ����`�����������} \quad 
$y'=f(x)y+g(x)$

��L�ɔ�Ď���$g(x)$�����W�Ƃ��ĕt�����`�őΉ��Ď��������̉�$u(x)$���ό`����$u(x)\cdot v(x)$�ƂȂ�Ɨ\�z���A
�܂�$u(x)$�A�‚���$v(x)$�����߂���̂Ƃ���B
�܂�������
\[
u(x)=\exp \left( \int_0^u f(w) dw \right) %\label{122}
\]
�Ƌ��߂����ƁA�����
\[
y=C \cdot u(x) +u(x)\int_0^x \dfrac{g(w)}{u(w)} dw %\label{123}
\]
�̂悤�ɑ�2�����t�������ƂȂ�B

{\gt ��x���k�[�C�̔�����������} \quad 
$y'=f(x)y+g(x) \cdot y^n$

����͐��`�Ď��̉�$y$�̑ւ�ɂ���ɉ����đ�������$y^n$���ω��W��$g(x)$�ŕt�����ꂽ���W�`�ł���B
$y$��$y^n$�̓񍀂����Ȃ炱��͉�����̂ŁA���R�u�E�x���k�[�C�̖���t���Ă���B
�O���Ɠ����X�s���b�g��$y=u(x)\cdot v(x)$�Ƃ��āA
\[
u(x)=\exp \left( \int_0^x f(w) dw \right) %\label{124}
\]
�͑O�Ɠ����A�����p����
\begin{align*}
y=&u(x) \left\{ C \right. \\
  &\left. +(1-n) \int_0^x g(w) \cdot (u(w))^{n-1} dw \right\}^{1/(1-n)} %\label{125}
\end{align*}
�Ƃ��Ċ��S�ɉ�����B

�������A���̌`��$u=y^{1-n}$�𓱓�����$u$�Ɋւ�����`�������ɋA����������@������i�ꏼ�O�f�j

{\gt ��2�K������������} \quad 
$y''=f(x,y,y')$

2�K�������܂ޕ������͐ϕ��ʼn������@�͗�O�I�ɂ����Ȃ��B
�܂��A$f$��$y$���܂܂Ȃ���΁A���炩��
\[ p=y' \]
�Ƃ�����$p'=f(x, \, p)$�ƂȂ��ĊK����1����������B

$y$���܂ޏꍇ�ł�$x$���܂܂Ȃ����
\[ y''=f(y, \, y') \]
�ƂȂ邪�A����$y'=p(y)$�ƂȂ�悤��$p(y)$������΁A�����֐��̔�������$y''=p'p$�B

���������ĕ�������
\[ p'p=f(y,\, p) \]
�𓾂āA����Ŏ���悭$p(y)$�����܂�B
���Ƃ͂��̊֐�$p( \, \cdot \, )$�ɂ‚���\underline{$x$��}����������
\[ y'=p(y) \]
�������΂悢�B

����ɁA$y''=f(y)$�ƂȂ��Ă���ꍇ�͂܂��ʂ̍L���W�J������B

�����������̉��i���΂��ΐϕ��Ƃ���ꂽ���j�͂������R�u�A�����n��������q�ׂ‚������������邪�A
�_�j�G���͂��̔��W�Ƃ��āA���b�J�e�BRiccati\footnote{
���J�b�e�B�Ƃ�������������B}
�ƂЂ�ς�Ɍ�M���Ă���B
�u�����������v�̌�����ʂ���p�����Ă���B
���ہA���̃C�^���A�l���w�҂͓����Z���g�E�y�e���u���O�̃A�J�f�~�[�̉@���E�̃I�b�t�@�[���f�������͓I���w�҂ł������B
�֐S�̈�͎��w��@���_�ɋy�Ԃ��A���w�ł͖��m�֐��Ɋւ���2���̏�����������̈�ʌ`
\[
\text{�i��������H�j}
\]
�Œm����B
���̉���
\[
\text{�i��������H�j}
\]
�ł���B
���b�J�e�B�̓x���k�[�C�A�I�C���[�ɉe����^���A���ʂȃP�[�X�̓x���k�[�C�̔����������ɂȂ�B
�܂��A�x�N�g���E�s��^�ŕ\������ƁA���`�|2���K�E�X�^���䃂�f���̕������ƂȂ�ȂǁA
���݃V�X�e�����䗝�_��t�B���^�����O���_�̊�{���Ƃ��Ă̈Ӌ`������i���[�G���o�[�K�[�A�ؑ��j�B

\section{�U�q�̉^���Ƒȉ~�ϕ�}

�i�z�C�w���X�̔߈��j

�u�U�q�̓������v�̓K�����C���������A����N�ł��m���Ă��郆�j�o�[�T���Ȓm���ł��邪�A�������e�U�����������͈͓��Łf�ł���B
������t�p���ăz�C�w���X�A�j���[�g�������ԑ���ɗp�������Ƃ������͒m���Ă���B
�������A�U�����i��������܂Ȃ�����j�傫����΂ǂ������W�J�ƂȂ邩�B
�Ȃ��Ȃ���������������ꂾ���ɖʔ����ۑ�ł���B

�U�q�͎x�_�𒆐S�Ƃ����̒����ɓ��������a�̉~����i�����Ă��̏�ł̂݁j�^������B
�ʼn��_����̐U��p��$\theta$�Ƃ���ƁA�U�q�̈ʒu�͉~���̒�����$\ell \theta$�ł���B
����A�d�͂̉����x��$g \sin \theta$�ŁA$\theta$����������������ł���B
����ĉ^����������
\[
\ell \dfrac{d^2 \theta}{dt^2}=-g \sin \theta %\label{126}
\]
������$\ell/g=1$�Ƃ������Ƃ͖{���I���܂����ɂȂ�Ȃ��B
�����$\theta$��$y$�Ə�����
\[ y''+\sin y=0 \]
�ƂȂ���2�K������������������B

��ʂɗ͊w�ɂ悭�o������
\[ y''=f(y) \]
�̏ꍇ�A���ӂ�$2y'$���|���Đϕ�����ƈ�K����������
\[
{y'}^2=2F(y)+c, \quad F(y)=\int f(y)dy %\label{127}
\]
�ɒB����B
���̐ϕ��́A�܂��悪����̂ŁA�u���Ԑϕ��v�Ƃ�����B
����𕽕����ɊJ����
\[
y'=\sqrt{2F(y)+C} \quad \text{�i}C \text{�͐ϕ��萔} %\label{128}
\]
�Ƃ��悤�B
������$C$��Y����$\sqrt{}$�ɊJ���Ȃ����Ƃ�����̂Œ��ӂ���B

�������āA���̏ꍇ
\[
\dfrac{dy}{dt}=\sqrt{2\cos y -2\cos A} \qquad \text{�i}A\text{�͍ő�U���j}
\]
�ƂȂ邪�A�����܂ł�$y$����������$t$�̊֐��ƂȂ�Ȃ��B
�����ŕϐ��������āA$t$��$y$�̊֐��i�‚܂�A�֐��j�Ƃ���
\[
t=\int_0^y \dfrac{d\theta}{\sqrt{2\cos \theta -2\cos A}} %\label{130}
\]
�̂悤�ɋ��܂�B
���Ȃ݂ɍő�U���ɑ΂��������
\begin{align*}
T&=4\int_0^A \dfrac{d\theta}{\sqrt{2\cos \theta -2\cos A}} \\
 &=2\int_0^A \dfrac{dy}{\sqrt{\sin^2 \dfrac{A}{2} -\sin^2 \dfrac{y}{2}}} %\label{131}
\end{align*}
�ŗ^������B
���Ă̂Ƃ���A����$T$�́i�ő�j�U��$A$�Ɉ˂邱�ƂɂȂ邪�A���̊֐��́u�ȉ~�ϕ��v\footnote{
�u�ȉ~�ϕ��v�͑��̂ł����āA�ȉ~�̎����̊֐�����R���������A�����܂Ŗ����̗R�������ł���B
���ׂĂ��ȉ~�̎����ƂȂ�킯�ł͂Ȃ����Ƃɒ��ӂ��悤�B}�Ƃ�����ϕ��̈�‚ł���B

\noindent
\Fig[������131�|135��������]{\textwidth}{5\baselineskip}

�������U���ɂ��܂��܂��ɂȂ�̂͂�͂�܂����B
�Ȃ��Ȃ�A��莞�ԂŌJ��Ԃ���錻�ۂ����ׂČ����I�Ɂu���v�v�Ƃ��邩��ł���B
�z�C�w���X�͉~���^�������܂����̋Ȑ���̉^���ɉ��C���āA���̋Ȑ���̌ʒ�$s$�ɂ‚��āA
$s$�ɂ‚��Ă̒P�U���̕�����
\[ s''+Ks=O \]
�ɋA���ł��Ȃ����l�Ă����B
�����܂ł��Ȃ��A���́e�~�ɋ߂��f�‚܂�e�~���ۂ��f���́A�͂����ĕ����ʂ�T�C�N���C�h�ł������B
������u�T�C�N���C�h�U�q�v�Ƃ����B

���̌��ʂ́A��\underline{�Z}�~������肩��T�C�N���C�h���o�������n���E�x���k�[�C����΂����B
�������A���񂶂�̃T�C�N���C�h�U�q�̐���͂��܂��s�����A�H�w�Ƃ��Ă͎��s�����B
�@�B�̐��삪���܂��s������͂܂��������Ă��Ȃ������̂ł���B

{\gt �U��q�̗��_}
�@�U�����̏ꍇ�@�E�E�E�ς�

�@��ʂ̏ꍇ

{\gt �ȉ~�̎���}

{\gt �����j�X�P�[�g�̎���} ��ʂɂ́u�J�b�V�j�̃����j�X�P�[�g�v
\[
\text{�i��������H�j}
\]
�Ƃ����A���̂����u�x���k�[�C�̃����j�X�P�[�g�v(�A��`�j�̒�`�́������A���Ȃ킿
\[
\text{�i��������H�j}
\]
�ł���B�ɍ��W�ŕ\�������
\[
\text{�i��������H�j}
\]
�ƂȂ�B
���̌ʒ����v�Z���Ă݂悤�B
\[
\text{�i��������H�j}
\]
���Ȃ킿�A�ȉ~�ϕ��ɂȂ�B

����͋t�O�p�֐�
\[
\text{�i��������H�j}
\]
�̗ގ��`�Ƃ��ċ����[���B
�K�E�X�́i�t�j�O�p�֐��ɕ���āA���̐ϕ��i�����j�X�P�[�g�ϕ��j�̋t�֐�����
�����j�X�P�[�g�E�T�C��(sin.\, lemm,\,  $\mathit{sl}$)�����j�X�P�[�g�E�R�T�C��(cos.\ lemm,\ $\mathit{cl}$)���`���A
�t�O�p�֐��̒l$\pi$�ɑΉ�����$\omega$�i�{���́������‚��j���l���Ă����Ƃ����i���ؒ厡�w�ߐ����w�j�k�x�j�B
����烌���j�X�P�[�g�֐��͑����̊֐S�����‚߁A�����̘_�l���o�Ă���B
�����ł�
\[
\text{�i��������H�j}
\]
�������邾���Ƃ��悤

����Ŏ��̋����[���v�񂪂ł����B

\begin{table}[htb]
\begin{tabular}{c|l}
�Ȑ�           & �ʒ� \\ \hline
�񎟋Ȑ�       & \\
�~             & \\
������         & \\
�ȉ~           & \\
�o�Ȑ�         & \\
�T�C�N���C�h   & \\
�J�e�i���[     & \\
�����j�X�P�[�g & \\
\end{tabular}
\end{table}

\noindent
\Fig[����132���@�\�i�ȉ~�ϕ��j]{\textwidth}{8\baselineskip}

\chapter{�_�j�G���E�x���k�[�C}

\section{�_�j�G���N�̓o��}

���R�u�A���n���Ɨ��āA���̓_�j�G���ł���B
���n���̎q�ŁA�x���k�[�C�Ƃ�3�l�ڂɂȂ�B
���n���̎q�ɂ̓j�R���X�i�c���Ɠ����j2�������āA�������R�u��Ars Conjectandi���o�ł����i���ۂ́A�`�����j���A
�_�j�G���ɔ�ׂ�Ɖe��������A���R�u�A���n���A�_�j�G���c�c�͂�������悭�������[���b�p�l�̖��O�����A
�����Ă����ƃo�C�u���i�����j���̖��O�ł���B
���R�u�̓��_�������̑c�Ƃ����A�u���n���̓��ڂł���͈̂��B
�_�j�G���͑����a���ҁA����������񐹏�����ł���B
���n���͌����́e���n�l�f�ŁA���_�������̒��łӂ‚��̒j�q�̖��A���{���ɂ͑��Y�A���Y�Ƃ��������ŁA
�m����ʂ胈�n�l�͐V�񐹏��ɂ͂悭�o�Ă���B

�‚��łȂ���A��ő���������肪����̂����A�u�_�j�G���v�̓w�u����i���_�������̌���A�e���_����f�Ƃ͂���Ȃ��j�ŁA
�e���h�Ȕ����f���Ӗ�����B
���`�Ər���ȋK�����d�񂶂郆�_�������炵���l���ł���B
���������΃V�F�C�N�X�s�A�́w���F�j�X�̏��l�x�ŁA�e�������f---���́A��S�}�J�V�Ȃ̂���---���o���������_�j�G���ŁA
�����V���C���b�N���e���ꂼ�������_�j�G���l�I�f�Ɗ������ċ��ԏ�ʂ�����B
���̂��ƁA��ǂ�ł�Ԃ�������̂����A����͂������낤�B
���̃_�j�G���N�A�_�j�G�����N�������Ŋ�ʂ��傫���������������肵���ӎu�����������̖��ɒp���Ȃ��q�������悤�ł���B
���Ђ�����d���ȑ��݂����������i���R�u�j�A����p�p�i���n���j�̉��ł悭�T�o�C�o�����A
�x���k�[�C�Ƃ�3�Ԏ�̓V�˂Ƃ��č��������c���Ă���B

\noindent
\Fig[����133���ʐ^]{\textwidth}{8\baselineskip}

���i�̌����������n���ɏ������p���Ƃ����p���ɕ��������ɍ���Ȃ��B
���͈�w�����Ƃ�����B
�ƋƂ���폤�����牏���Ȃ��킯�ł͂Ȃ��A�e�F�s���q�Ƃ��Ă���͎����邪�A
���w������Ă悢�Ƃ���������t����B
�����̓\�c���Ȃ������ǂ��B
�x���k�[�C�Ƃ̏ё��������ƁA���ꂼ��̃L�����N�^�[���\��Ă���̂������[���A
�_�j�G���͂ǂ��������Ƃ肵�Ă��邪�A�X�}�[�g��������B
��F�֌W���ǂ������ł��낤�B
�ő�̗F�l�͂����܂ł��Ȃ��A������̃I�C���[�ł���B�i�_�j�G���̕���7�˔N���B�j

\section{�E���ƋƐ�}

�E�������������B
�_�j�G���E�x���k�[�C��1700�N�I�����_�̃o���[�j���Q������A5�΂̂��땃�ƂƂ��Ƀo�[�[���ɖ߂�i1705�N�j�B
�قǂȂ�1707�N���n�ŃI�C���[�������B
��������͐�ɏq�ׂ��ʂ肾���A���V�A�鍑�̃Z���g�E�y�e���u���N��w�̋����ɂȂ�B
�����̉ŃG�J�e���[�i���w��D���ňꗬ�w�҂����͂ɏW�߂Ă����̂ł���
�i�����̐ꐧ�N��ɂ͌[�֓I�ȌN������āu�[�֓I�ꐧ�N��v�ƌ���ꂽ�j�B
���������΁A�I�C���[���������Z���g�E�y�e���u���N��w��������������͒��������邪�A
�_�j�G���͂�����3�N�Ŏ�������B
���̓I�C���[�E�̓_�j�G���̏Љ�ŁA�Љ�҂̕�����Ɏ��߂��̂ł���B
�o�[�[���ɖ߂��Č�o�[�[����w�����ƂȂ邪�A�^�C�g���͈�w�E�N�w�̋����ł���B

�Ɛтɂ��‚낤�B
�x���k�[�C�Ƃ̋ƐтƂ‚������͖̂{���ɗ͋Ƃ��K�v�ł���B
�܂��ɕS��㇗������A�܂����e�S�ԁf�����ׂăt�H���[����킯�ɂ͂����Ȃ��B
�������A���̃x���k�[�C�Ƃ�3�l�ڃ_�j�G�����̋Ɛт𔲂����킯�ɂ͍s���Ȃ��̂́A���̌���I�Ӌ`�����邩��ł���B

\noindent
\Fig[����134���ʐ^hydro�c]{\textwidth}{10\baselineskip}

�_�j�G���̋Ɛт͉��Ƃ����Ă�1.���̗͊w�A������2.���p���_�ł���B
3. �C�̕��q�^���_������B
1.�́u�x���k�[�C�藝�v�Ƃ��č��Z�̕����ɂ��o�ė���݂̂Ȃ炸�A���㕶���Ɍ������Ȃ���s�@�̌����A
2.�͑�w�o�ϊw���̃~�N���̍u�`�ŏ����ɏo��u���p���_�v�̎n�܂�ł���B
1.�A2.�͑S���ʕ���̂��́i���n�ƕ��n�j�����A���ꂪ�����Ƃ���300�N�߂�����̍����Ɏc���Ă���̂�����A
���Q���ׂ����̂ł��낤�B
����͍����̑�w������300�N�߂������ɉ����c���邩�Ƃ�����Ɠ����ł���B

\section{���͊w�A���邢�͗��̗͊w}

�u���͊w�vhydraulics�͒m��l�͑����͂��Ȃ������m��Ȃ����A�d�v�Ȋw��ł���B
�����n���̋������C�n�ł���A�_�j�G���̏o�g�n�I�����_�̐������u�l�[�f�������h�v�iNetherland, �t�����X�� Pays bas�j�̈Ӗ���
���ƕ����ʂ�e��n�n���f(low countries)�ō��y�͂‚˂ɊC���̐Z���̍��ƓI�댯�ɔY��ł����B
�Z�����鐅�̔r�o�A����ɔ����͐�̊Ǘ��͑傫�ȉۑ�ł������B
�ߑ�ɂ����ẮA�s�s�ɂ����鐅�̈��S�A�m���ȋ����V�X�e���́A���m�҂̃K�o�i���X������鎊��ۑ�\footnote{
���{�ł��A����ƍN�������r��͂ĂĂ����]�˂ɓ��邵�Ĉȗ��A�����ɐ������͑�ۑ�ƂȂ�A
����Ȏ��ԁA�J�́A�������₵�Ă���A�ʐ�㐅��ʂ������Ƃ͒m����B}�ł���B
�኱�ׂ������A�X�y�C���̎�s�}�h���b�h�͍��n�ɂ��邪�A�����Z�p�ɂ���ČÂ�������S�ɐ��������Ȃ��ꂽ���Ƃ̓��m�̖{�ɏo�Ă���B

�Ƃ���ŁA���ƌ����Ă��ɂ߂‚��́A���Ȃ�ʃI�C���[�̎��s�ł���B
�I�C���[�͔ނ������]�������v���V�A�̌[�֓I�ꐧ�N��t���[�h���q�剤�i�H�j�ɏ��ւ�����ăx�������E�A�J�f�~�[�ɐV�����E�𓾂邪�A
��͂�f�l�̉ߓx�̊��҂͋��낵���B
�L���ȁu�T���E�X�[�V�{�v�iSanssouci �t�����X��Łu���J�{�v�j�ɒʂ���e���������f�̐���𖽂�����B
�@�B���͂��Ȃ�\footnote{
J. ���b�g�̏��C�@�ւ�1765�N���AR. �g�����B�V�b�N�̖{�i�I�ȏ��C�@�֎Ԃ�1804�N�B}�A
����Ɂu�i�v�@�ցv�̕s�”\�����A�ꕔ�̊w�҂̊Ԃł͂Ƃɂ������A�\���L���͒m���Ă��Ȃ�����ł������B
���̂̌����Ɏ��s���A�I�C���[�̕s�F�ȃx�������ߋ��̉����ƂȂ�i��C�̓��O�����W���j�B
�F�l�������_�j�G���E�x���k�[�C�ɂ́A���͊w�����ނ̍˔\�̌������������낤�B

\noindent
\Fig[�}����135��Sans souci�A�X�e�r��]{\textwidth}{5\baselineskip}

\section{���S���̂̃x���k�[�C�̒藝}

���āA���΂��󒆂ɓ�����ƁA�΂͕�������`���Ĕ�ї�������B
���̕������́u�_�v���������O�ՂŁA�_�́u���_�v�Ƃ���ꂻ���Ŏg���闝�_�́u���_�̗͊w�v�ł���B
���΂͎��ʂ��������_�ƌ��Ȃ���Ă���B
������x�̑傫���͈̔͂ł������A���΂���]���Ă��Ă��΂̏d�S�̉^���͕������ł��邱�Ƃɕς�͂Ȃ��B

�������A���΂��{�[���ł�������A�y�����‘傫�����̂ł���ƁA�K�炸�������m�ɂ͕������ł͂Ȃ��Ȃ�B
��C��R�������ł��Ȃ��Ȃ邩��ł���B
�{�[���ɉ�]�������ƋO���͔����ɕω����邪�A����Ɍ����Ȃ�A����s�@�Ȃ���͂�l���͈�ς���B
�����͂��ׂċ�C�ɂ��͂̉e�����x�z�I�ɂȂ邩��ł����āA�����ɋ�C�Ƃ������̗̂͊w�A
�‚܂�u���̗͊w�v���K�v�Ƃ���闝�R������B
������񗬑̂Ƃ����ǂ������̎��_�̏W���ł���A���͋ɂ߂đ����Ƃ����v�I�ɏW�v����΂悢�Ƃ����l�����A
���Ȃ킿�Ҍ���`������A����͂���Ƃ��ėL���ȗ��_�ł���i�C�̕��q�^���_�j�B
�������A�C�̈�ʂ̃}�N���Ȑ������������Ƃ���Ȃ�A��܂Ƃ܂�
---�Ƃ������A�ꑱ����---
�̑��݂Ƃ��čl���Ȃ��Ă͂Ȃ炸�A���̗͊w�̗�����u�A���̗̂͊w�v�Ƃ����B
���̂͘A���̂̈�‚ł���B

���̂��l����Ƃ��A
1. �e�_�ł̗���̑����ƕ������x�N�g��$u$�Ƃ��čl����u��v�̗���i�u�I�C���[�L�q�v�Ƃ����j��
2. ���̗��q�̍s���Ձi�O�Ձj���t�H���[���čs������i�u���O�����W���L�q�v�Ƃ����j��2�ʂ肪���邪�A
1. �̃I�C���[�L�q�����w�I�ɗe�Ղł���B
�����ŁA����̒��Ɉ�{�̋Ȑ����l���A���̏�̊e�_�ł̑��x�x�N�g��$u$���Ȑ��̐ڐ��ƂȂ�悤�ɂ���Ƃ��A
������u�����v(stream line)�Ƃ����B
����Ă͂��Ă������͎��ԂƂƂ��ɕς�Ȃ��Ƃ�������u��헬�v(stationary flow)�Ƃ����B
��헬�̘A���ʐ^�͂܂�ŐÎ~�摜�̂悤�Ɍ�����B

�}���ق̒I�ɕ���ł���2, 30���̈�܂Ƃ܂�̖{�̒��������̖{�����������ƁA
���C��R�ɂ���Ĉ����������������Ɏ��̖͂{������������Ă��܂����A
���C���Ȃ���΂���͋N��Ȃ��B
���̂̏ꍇ���A�����̑��i���ǁj�����̋��ڂ����R���󂯂錻�ۂ��u�S���vviscosity�‚܂�e�˂΂�f�ł���B
�ǂ̂悤�ȗ��̂ɂ��召�͕ʂƂ��ĔS��������A�����e�T���T���f�Ƃ͂��Ă��邪�A�S���������������Ń[���ł͂Ȃ��B
��C�����ɏ��������S�������������̂ŁA��ʂɗ��̂̉^���͋��E�̖ʂŒ�R���󂯁A�u�S�����́v�Ƃ�����B

\noindent
\Fig[����136���S���f�[�^]{\textwidth}{10\baselineskip}

���̔S�������̂悤�ɉ��x�ƂƂ��ɒቺ���A���T���T���ƂȂ�B
�Ƃ���ŁA�P�ꕨ�̗̂͊w�ł������ł��邪�A���C�͂���Ɨ��_�̎戵�͓���Ȃ�B
���̂̏ꍇ���S�����[���Ƃ������z��Ԃ��u���S���́vperfect fluid�ł���A
�_�j�G���E�x���k�[�C�A�I�C���[���z�肵�����̂����́u���S���́v�ł���B

���S���̂Ȃ爵���Ղ��B
�����̂���_�ɔ��ɏ����ȁi�p�����̂悤�ȁj�̐ς�z�肵�A���̉^�����l����B
���̂�$v$�A���������݂̂Ȃ炸�����ɂ������Ƃ��ė�����$h$�A
�d�͉����x��$g$�A���̂̑̐ϓ���̎��ʁi���x�j$\rho$�A
���̂ɂ͊O�����牽�炩�̈��́i�P�ʖʐϓ���j$p$�����‚Ă���Ƃ��悤�B
�����ɉ����āA�̐ϓ���
\begin{align*}
\text{�^���G�l���M�[}           &=(1/2)\rho v^2, \\
\text{�d�͂ɂ��ʒu�G�l���M�[} &=\rho gh, \\
\text{�O����̈��͂̂����d��}   &=p
\end{align*}
�ł���i�����ɂ́A���ǂ̈ꕔ�̗��[�ł��̂��̂̍����l���A���̘a��0�ɂȂ邱�Ƃ�����j�B
�����ŁA$\text{�d��}=(\text{����} \times \text{�ʐ�}) \times \text{����}=p \times \text{�̐�}=p$�Ƃ��āA
�����̑��a�̃G�l���M�[���ۑ�����邩��A�߂ł���
\begin{align*}
&\text{�x���k�[�C�̒藝} \\
&p+\dfrac{1}{2}\rho v^2+\rho gh=\text{���} %\label{137}
\end{align*}
�𓾂邱�ƂɂȂ����B
����ɗ������Ȃ��Ȃ�$h=0$�ŁA
\[
p+\dfrac{1}{2}\rho v^2=\text{���} %\label{138}
\]
�ł���B

�܂��́A�����ƈ��͂ƃ|�e���V�����ł���킵���I�C���[�̉^����������ϕ�����
\[
\text{�i��������j�i�P�j}
\]
�𓾂邪�A���ꂪ�u�x���k�[�C�̒藝�v�ł���B
�藝�͂��܂��܂ȓ����̌`�ɕ\�����B
\begin{align*}
&\text{�iwiki�j} \\
&\text{�i�Q�j} \\
&\text{�i�R�j}
\end{align*}

�u�x���k�[�C�̒藝�v�͂悭���ۂɂ��Ă͂܂�Ɖ]����B
���Ƃ��΁A��s�@�̗����͂̏�ʁA���ʂł͋�C�̗������قȂ�A
�x���k�[�C�̒藝���爳�͍��������ď�����̗g�͂�������B
�����A�g�͂̌����Ƃ��Ă����܂ł͐��������A�����͂̑��x���z�̗��_�͕ʌ•K�v�ł���A
������u�����̌����v�̌����n�ߌ��݂������̍���������i���ہA10����7���͌��������Ă���Ƃ����w�E������j�B

\noindent
{\gt ��} �v�Z���K�����������B
\[
\text{�i��������j}
\]

���̗͊w�ł���u���͊w�v�͗��̗͊whydrodynamics�̎�v����Ƃ���Ă��邪�A�����ł��̒藝�̉��p�͈͂͋ɂ߂čL���B
����Łu�����w�vhydraulics�Ƃ������������A�����炪���j���Â����قƂ�LjႢ�͂Ȃ��Ƃ����B
���邢�͕���ł̌ď̂̈Ⴂ�ł���Ƃ̐���������A���O�̂��̂ɂ͕�����ɂ����B
�������u���̗͊w�v�͋ߑ�ɂȂ��Ă��琔�w�I�ɂ����x�ɔ��B��������ł���A�������N������hydro-�i���́A�̈ӁB$\text{hydrogen}=\text{���f}$�j�������Ă��Ȃ���A
���k���̂��闬�́i�C�́j�������B
�{���I�ɐ��ʂ��� fluid mechanics�Ƃ������������B

����͐[���肵�Ȃ��Ƃ��āA�L���G�s�\�[�h������B
�_�j�G�����w���͊w�x�i���邢�͍����I�Ɂw���̗͊w�xhydrodynamics�j���o�ł���ƁA
�����ƁA�˂��݂��畃���n�������Ƃ���ɑ΍R���āw�����w�xhydraulics��---�o�œ���k����---�o�ł����B
�W����̂ł͂Ȃ��΍R���ďo���A�Ƃ��������܂��x���k�[�C�Ƃ炵���B

���̗͊w�̕���͐��w�I�Ƀn�[�h�ł���B
���发�͂������ɖ`���͐e���݈Ղ����A�����Ɋw��I�ɓ����Ă��܂��A
�x�N�g����́A�e���\����́A�֐��_�A�M�w�E���v�͊w�A�͊w�n�A�c�c�ƃt�H���[����̂���ςł���B
�����A��ʂ̗͊w���猩��΁A���̗͊w���̂��̂��•ʐ�啪��ł����āA����Ƃ����ǂ����I�ł���̂͂�ނ𓾂Ȃ��B
���Ԃ�������΁A����𐔊w�I�ɒ���ǂ݉����Ă䂯�΋����[�����E�������ɓW�J���邾�낤�B

\section{���������ŃI�C���[��ٌ�}

�I�C���[���t���[�h���q�剤�Ɍ����܂�A�T���E�X�[�V�{�̕�������𖽂���ꂽ���A���܂��s���Ȃ��������Ƃ͏q�ׂ��B
�I�C���[�͎��͂������Ă��Ȃ��A�x�������ɂƂǂ܂�A�e�剤�̐��w�ҁf�Ƃ��Ă̂ق���͎���Ȃ��������炢������A
���̒��x�̂��Ƃ̓x�������ދ��i�Z���g�E�y�e���u���N���A�j�̑傫�ȗ��R�ɂ͂Ȃ�Ȃ����낤�B
�ނ���A�I�C���[�͗ދH�Ɍ���h����ȃv���e�X�^���g�M�‚̕ێ��҂��������A
���������������t�����X�̖��_�_�I�[�֎v�z
---���H���e�[�����g�͖��_�_�ł͂Ȃ�������---
�̐�`�}���̃��H���e�[�����A�J�f�~�[�ɏ��ւ�����Ă��ăI�C���[��_��Ă������ƁA
�I�C���[�����ւ����㏂�ɂȂ��Ă����@�����[�y���e���C�������������ƂȂǂ��󐨂����������B

���̎����͂قƂ�ǒm���Ă��Ȃ��B
�����A�����̈ꌏ�ɂ͒��҂Ƃ��Ă�����肪����B
�����u�i�v�@�ցv�����Ǝw������Ă����̂Ȃ�A���ꂪ�s�”\�Ȃ��Ƃ͓����\���ɂ͒m���Ă��炸�i�����I��Clausius�܂Łj
���Ƃ��Ɩ������ł����������m��Ȃ��B
�ł̓I�C���[�͉ʂ����Ă����m���Ă����̂��낤���B
���{�ł́u���w�҃I�C���[�v���������@�I�ɋ���������Ď�l�̃A�C�h���ƂȂ�A�l�ԃI�C���[�̐l����v�z�E�M�‚ɂ͖��֐S�ł���A
�܂��Ă⎸�s����܂ɂ‚��Ă͒m��R���Ȃ��B
�܂��́A�t���[�h���b�q�剤�̕s���𕷂���

\begin{quote}
�����]�񂾂̂͒�̕����������B
�I�C���[�͒����r�֐���g����ɕK�v�Ȏԗւ̗͂��v�Z���A���͂������琅�H�𗎂��āA�Ō�͐����悭�T���X�[�V�ɗ���o��͂��ł������B
���̐��Ԃ͊􉽊w�I�ɂ͎����������A������������̐����킸��50���ł������r�֗g���邱�Ƃ��ł��Ȃ������B
������̋�\footnote{
���񐹏��w�`���̏��x��1�͖`���ɂ���L���Ȑ���
�u1 �_�r�f�̎q�A�G���T�����̉��ł���`���҂̌��t�B
2 �`���҂͌����A��̋�A��̋�A��؂͋�ł���B
�c�c6 ��݂͂ȁA�C�ɗ������B
�������C�͖����邱�Ƃ��Ȃ��B
��͂��̏o�Ă������ɂ܂��A���čs���B�v���w�������́B
�剤�͂������Ɍ[�֓I�ꐧ�N�傾�������āA�悭�ÓT�␹���ɒʂ��Ă���B
�M�[�������I�C���[�͂�����񂱂̐���ɂ͒ʂ��Ă������낤�B
�I�C���[���􉽊w�҂ƔF������Ă������Ƃ͒��ڂ����B}
�Ȃ邩�ȁI�@�􉽊w�̋�Ȃ邩�ȁI�@Vanity of vanities�I�@Vanity of�@geometry�I�v 
\end{quote}

\noindent
\Fig[�T���X�[�V]{\textwidth}{8\baselineskip}

�t���[�h���q�剤�́A18���I�[�֎v�z�̉e�����󂯎v�z�E�w��E�|�p�ɑ���Ȋ֐S���񂹂����E�j��́u�[�֓I�ꐧ�N��v
---�������ꐧ��E����ɂ͎���Ȃ�������---�̈�l�ł������B
�t���[�h���q�ɂƂ��ăT���X�[�V�ɕ����V�X�e�������‚��Ƃ͒��N�̖��ł���A���O�����߂��A���z�̔�p��������ꂽ�B
���̎��s�ȍ~�z�[�G���c�H���������Ƃ̑�X�̉������̊肢�ɂ�������炸�A�����V�X�e���͖�100�N���̊Ԓ��ق����܂܂ł������B

�I�C���[������𖽂���ꂽ���u�i�H�앨�j�͂ǂ��������̂��A�ǂ̂悤�ȏ����łǂ̂悤�ɐ݌v����A���삳��A
�ŏI�I�ɂ͑��u�͂ǂ��@�\�����i���Ȃ������j�̂��A���̗��R�⌴���͉��ł��������B
�ڂ�������͂킩��Ȃ����A�z������ɓ�‚̘_�_���v�������ԁB

���_�͗��̗͊w�i�����Ƃ��Ă͐��͊w�j�͂ق�̗c�t�i�K�ł������B
�������A�悤�₭�u�I�C���[�̕������v���m���Ă��邪�A���������H�앨�͗��w�I�m�������ł͐���ł����H�w�I�Z���X���K�v�ł���B
�����A�_�j�G���E�x���k�[�C�͂׃��k�[�C�Ƃ����Ă̌�����؊ώ@�h�ŃZ���X����}���Q�ł���i�����w�𐔊w�Ɠ����ȏ�̈ʒu�ɂ����Ă����j�A
�����w�҂Ƃ��ē����ŐV�̃n�[���F�C���̌��t�z�˜_�Ɍ���ɒʂ��Ă������痬�̗͊w�ɂ‚��Ă͈�����������ł��낤�B
�͂����ăI�C���[�ɍH�w�I�Ȋ�p����Z���X���������ۂ��B
���N�E�”N���ォ�狌�F�������I�C���[�ƃx���k�[�C�̃m�E�n�E�̌����͂Ȃ������̂��낤���B
�����ɂ‚��Ă͕s�K�ɂ��Ē��҂̂悭�m�鏊�ł͂Ȃ��B

���_�͂����������{�I�Ȃ��̂ŁA������i��1��́j�u�i�v�@�ցv�̂��Ƃł���B
�u�g����v�Ƃ���̂ŁA�����r�͐������琳�̍��x��\footnote{
���Ȃ݂ɁA�]�˖��{�̖��߂ō]�˂܂Ł����N�J�킳�ꂽ�u�ʐ�㐅�v�͑�����̉H��������������
�����̍]�ˎl�J��،ˁi�������s�V�h��l�J4���ڌ����_�j�܂ł̖�43km�ŁA���n�_�Ԃ̎��R�̗����͂킸��100���ł������B
�H���͒n���̖��������ē�q���ɂ߁A�\�莑���͒���‚��܂��Z�p�I���s�̌��ʂ̎��E�҂܂ŏo���B}
�ɂ������Ƒz�肳���B
�z�肪��������΁A���́i�����w�I�ɂ͊O������̎d���j���K�v�ł��邪�A���C�@�ւ͖������݂����A
�܂����͂Ȃǂ̎��R�͂Ɉ�ؗ���Ȃ������Ƃ���ƁA�����������̑��u���̂��u�i�v�@�ցv�ł����āA���������Ė{�����蓾�Ȃ��B

�I�C���[�M��҂ɂ́A�I�C���[���͂����ĉi�v�@�ւ̕s�”\����m���Ă������ۂ����C�ɂȂ邩������Ȃ��B
�M�҂͂��̊m��I�������������ʂ���ق��Ȃ��B
���E�G�w�����w�j�x�i��8�́j�ɂ��΁A�t�����X�w�m�@�����̖��̉�@�Ə̂�����̂̕s�󗝂����c�����̂�1775�N�ł���A
�i�v�@�ււ̓��B�s�”\��18���I�I��育��ɂ͊m�M�ƂȂ����B
�I�C���[�̌��̖��͐旧�‚��Ɛ��\�N�Ŕ����Ȏ����ł���B
����ɁA���Ƃ�������\�z���ȂĂ��Ă��A����@�ւ��i�v�@�ւł��邩�ǂ����̔��f�͈�ʂɂ͕��Ղł͂Ȃ�
---�����炱���A�w�m�@�ɐ\���̐��x������---���ꎩ�̈ꗝ�_�ɑ���������������i�����Q�Ɓj�B

\noindent
\Fig[���[�X]{\textwidth}{4\baselineskip}

�i�v�@�ւ̕s�”\���ɂ‚��āA�I�����_�̕����w�ҁA���w�ҁA�Z�p�ƂŃ��I�i���h�E�_�E���B���`�ɂ��䂹����S. �X�e���B���i�������j��
�u�X�e���B���̕��v�Œm����u�Ζʏ�̒ނ荇���̖@���v������B
�}�̂悤�ȃ��[�X�i���Ɋ|����ԗւ̍��j�̂��Ƃ��ŁA�E���̎Ζʂ������̎Ζʂ�蒷�����獽���d���A
���͎��v���ɉi���ɉ�]�������邾���낤�Ƒz������邪�A�X�e���B���́A
�Ζʏ�̗͂̒ނ荇����͊w�I�ɘ_�����ۂɂ͂���͋N����Ȃ��Ƃ���\footnote{
�_�j�G���͕����n���̕��C�n�I�����_�̃O���[�j���Q�����܂�ł���A���R�X�e���B���̖����ƒ����ȋƐт�m���Ă����͂��ł���A
�����I�C���[�͒|�n�̗F�ł���������A�I�C���[���X�e���B���̘_�ɂ͒ʂ��Ă����Ɛ��_����邪�A�����܂Ő��_�ł���B}
����ł́A�[������d�������܂�邱�ƂɂȂ�B
�������āA�G�l���M�[�̑��ʂ͈��ł���i�G�l���M�[�ۑ����j�A����ɂ���đ�1��̉i�v�@�ւ��Ȃ킿�u�����I�ɓ����ĊO���Ɏd�������A
�@�֎��g���O�E�����̎d���ȊO�͊��S�Ɍ��ɖ߂�@�ցv(����A�����A���o)�̉”\���͔ے肳���B

�����A���ꂪ�d�v�����A���͊w�̃G�N�X�p�[�g�̃_�j�G���E�x���k�[�C�̏����Ɗ֗^�������Ă��A
�ނ����1��̉i�v�@�ւ͕s�”\�ł��邱�Ƃ̍Ċm�F�ŏI���������ł��낤�B

\section{�x���k�[�C���j�܂ꂽ�u�o�����v�̐��E}

�����Ŋm�F���Ă������B
�‚����Ǘ������n�ɂ����Ă͑S�G�l���M�[�͈��ɕۂ���邪�A�n���ł̕ω��ɂ‚��Ă̓G�l���M�[�ۑ����͉������Ȃ�����肦�Ȃ��B
���̕ω��̌��������߂錴���̓G�l���M�[�̕ۑ����Ɩ����͂��Ȃ�������Ƃ͓Ɨ��Ȍ����ł���B
�M�G�l���M�[�̎󂯓n���ɂ‚��ẮA�G���g���s�[�̑���i�����ɂ͔񌸏��j$\Delta S=\Delta q/T>0$�̌����̂݉”\�ł���B
���ہA�M$\Delta q$��������$T_1$����ቷ��$T_2$�ֈڂ�Ƃ��A$T_1>T_2$���瑍�G���g���s�[�ω���$\Delta S>0$�ɂȂ邱�Ƃ݂͂₷��(Brillouin)�B

���́A����͏ؖ��ɂȂ��Ă��Ȃ��B
�Ȃ��Ȃ�A�ŏ�����u������$T_1$����ቷ��$T_2$�ֈڂ�Ƃ��v�Ƃ̑O��������Ă��āA������͓��`�����ł���B
�t�����̈ړ��̂��Ƃł�$\Delta S<0$�ƂȂ邪�A���ꂪ�ώ@���ꂽ���Ƃ͂Ȃ��̂ł���B
�u�G���g���s�[�̌����͎��R�E�ł͎�����S���ώ@�����Ȃ��v�i�����_�E�E���t�V�b�c�w���v�����w�x�M��E��j�B
�‚܂�A���́u�M�͊w�̑��@���v�͌o�����ł���͊w�̖����͖��p�ł���B

���̂悤�Ɍ����Ă���B

\begin{quote}
---���āA���C�@�ւ͔M����@�B�I�Ȏd���𔭐������邪�A����@�B�I�ȗ͂ɂ���Ă��A�M�𐶂ݏo�����Ƃ��ł���B
�ՓˁA���C�͂��ׂĂ��̂��Ƃ����Ă���̂ł���B
�n�������b�艮�́A�S�̞���Ȃőł‚����ŎܔM�����邱�Ƃ��ł���B
�ԗւ̎��́A�אS�ɖ����������Ƃɂ���āA���C�ɂ�锭�M�������Ȃ���΂Ȃ�Ȃ��B
�ߍ����̂��Ƃ���K�͂ɗ��p����Ă���B
���Ȃ킿�኱�̍H��̉ߏ�̐��͂��A����̓S�‚����̎��𒆐S�Ƃ��Đv���ɉ�]���Ă�����S�‚��݂ɖ��C����悤�ɗ��p����A
���̌��ʗ��҂͋������M���ꂽ�B
���̊l�����ꂽ�M��������g�߁A�R���̗v��ʃX�g�[�u������ꂽ�B
�Ƃ���œS�‚ɂ���Đ��ݏo���ꂽ�M�������ȏ��C�@�ւɉ��M����قǏ\���ł���A
���̏��C�@�ւ�����ɂ܂����̓S�‚𓮂��������邱�Ƃ��ł���悤�ɂȂ蓾�Ȃ����̂ł��낤���B
��������΂������i�v�@�ւ͔������ꂽ�ł��낤�B
���̖��͒�N����邱�Ƃ��ł������A�܂�\underline{
����͋����̐����@�B�w�m�͊w���w���Ǝv����n�̌���}\underline{�ł͌��肳�꓾�Ȃ�����}\footnote{
Sie(diese Frage)war durch die {\"a}lteren mathematisch-mechanischen Untersuchungen nicht zu entscheidon.}�B
�������N�ɏq�ׂ悤�Ǝv���Ă��镁�ՓI�@���́A���̖���{\gt �ے�}������̂ł��邱�Ƃ�O�����ĕt�����Ă�������\footnote{
H.�w�����z���c �w���R�͂̑��ݍ�p�ɂ‚��āx{\"U}ber die Wechselwirkung der Natur kr{\"a}fte. �i1854�N�A�P�[�j�q�X�x���N�ɂāj}---
\end{quote}

������񂱂��Ō����Ă���u�i�v�@�ցv�̕s�”\���͔M���ۂ���݂�����u��2��i�v�@�ցv�̂��Ƃł���A
������G�l���M�[�����o���u��1��i�v�@�ցv�ł͂Ȃ��A���������āA�͊w�̌��������荞�ޗ]�n�͂Ȃ��B
������({\"a}lteren)�Ƃ����Ɖ���狌�����ᔻ����Ă���悤�Ɋ������邪�A
�ʂ̂Ƃ���Ɂu�O���I�̈̑�Ȑ��w�҂����ɂ���Ĉ�ʓI�Ɋm�F���ꂽ�@���v
�c�cdieses Gesetz durch die gro��en Mathemamatiker des vorigen Jahrhundert allgemein war�Ƃ��āA
�_�j�G���E�x���k�[�C�ƃ_�����x�[�������߂���Ă���B
���������āA�����i�ҁj�̎��̕]���ł͂Ȃ��B

����́u�@���v�̘_���w��̒n�ʂ��q�ׂ����̂ŁA�ꌾ�ł����Ȃ�A�u�o�����v�͏ؖ��̌���ł͂Ȃ��B
���v���鎖���͖����Ɋώ@����Ă��邪�A����ɔ����鎖��́i�Ȃ��Əؖ��ł����킯�ł͂��Ȃ����j�����̈����Ȃ��A
�_���I�ؖ��͍���ŁA�{���I�ɂ����u�o���v�������ɂ���Đ������Ă���@���ł���B

\begin{quote}
�u��ɏq�ׂ��ȒP�Ȓ莮���m���v�͊w���ƂɃG���g���s�[�������n�������̎����ɑΉ����Ă��邱�Ƃɂ͉��̋^�����Ȃ��B
����͂����̓���̊ώ@�̂��ׂĂɂ���ė���������Ă���B
�������A���̖@�����̕����I�{���ƋN���ɂ‚��Ă̖��ɂ��������O����̍l�@��������i�ɂȂ�ƁA
�����ɂ������Ă��Ȃ�������x�͍�������Ȃ��ł��鍢������Ă���B�v\\
�i�����_�E�E���t�V�b�c�O�f�j
\end{quote}

\section{�C�̕��q�^���_��}

�������ɁA�M�҂͉Ǖ��ɂ��ďڂ����͒m��Ȃ��������A
�_�j�G���E�x���k�[�C�́w���̗͊w�x�̂Ȃ��Ń~�N���̋C�̕��q�̉^���_��_���Ă���
�i�M�͊w�̃e�L�X�g�͐l�������͋����Ă���j�B
�C�͈̂��k���̂��闬�̂ł���B
�ǂ��܂œ��B�����������̗]�n�����邪�A�Ƃɂ������̌����̗��j�I�Ӌ`�͑傫���B
�����������I�ɂ͐Î~������Ԃł��A���̍\�����q�͏���Ȍ����ɐ₦�ԂȂ��^�����Ă���B
���̗��G�ȉ^���̃G�l���M�[�́u�����G�l���M�[�v�Ɖ]���i�����j�A���̎��x���c�_����̂��M�w�ł���B

���Ȃ݂ɁA�C�̕��q�^���_�̋����鏊�ɂ��ƁA�{���c�}�����z�i���x�x�N�g�����Ɨ���3�������K���z�Ƃ���j�����ƂɌv�Z���ꂽ���q�̕��ϑ��x�́A
������2�摬�x\footnote{
���p�ꎮ�ɋL���̊O������ǂ񂾌ď̂ŁA�t�̓ǂ݁u2�敽�ϕ��������x�v(����)������B�����ł͕M�҂̊��ꂽ���V�ɂ�����B}
(root-mean-square velocity)
\[
\text{�i��������H�j}
\]
�Ƃ��ė^������i�N���E�W�E�X,\, 1857�j�B
�����$0\, {}^\circ \mathrm{C}$,\, 1�C���̕W����ԂŌv�Z���Ă݂悤�B
�C�̂̓w���E��He�Ƃ��A����������1�����Ōv�Z�����
\[ p=\qquad ,\quad V=\qquad ,\quad M= \qquad \]
����
\[
\text{�i��������H�j}
\]
�ƂȂ�B
�����1 km/�b�̃I�[�_�[�ŕs���R�ɑ傫�����A���ۂ͑����q�Ƃ̏Փ˂Ŏ��R�ɉ^���ł���򋗗��i���ώ��R�s��mean-free-path�j�͂����Ə������B
����������΁A�����q�ւ̏Փ˂͕��O��Ă���߂ĕp�ɂł���B

���̂悤�ɂ��āA�x���k�[�C�̒񏥂����C�̕��q�^���_�́u�����G�l���M�[�v�Ƃ����M�w�̊�b�T�O�֔��W�����B
���ہA�G�l���M�[�͎d��$W$�ƔM$Q$�Ƃ����`�ŕ��̂ɗ^�����邩��A�O��̓����G�l���M�[�ω��Ƃ���
\[
E_2-E_1=Q+W�@\qquad \text{(����E���oetc.)}
\]
���u�G�l���M�[�ۑ����v�̕\���ƂȂ�B
���ꂱ�����u�M�͊w�̑��@���v�ł���A�����܂ł͗͊w�I�l�@�œ��B�ł���̂ł���B

{\gt �����Fhttps://en.wikipedia.org/wiki/H-theorem}

\section{�s�‹t�����̂̓x���k�[�C���z�̎d�g�݂ŗ����ł���}

�������A��������悪�S�������Ƃ����킯�ł͂Ȃ��B

�u�M�͊w�̑��@���v�́A�M�͍�������ቷ�ֈڂ�G���g���s�[�͑�����(�����ɂ͔񌸏�)���̉ߒ��͕s�‹t�ł���Əq�ׂ�B
����ɔ����錻�ۂ͎��ۂ͋N���Ă��Ȃ��B
�l�ނ͏��Ȃ��Ƃ��L�j�ȗ��u�X�̏�ɒu�����₩��̐�����������̂��ώ@�������Ƃ͂Ȃ��v�B
�@���͌o�����ł����āA���������ނ̊m���_�I�E���v�I���m���̂���@���ł���B

�m���_�I�E���v�I���m���Ȃ�o�����łȂ��Ƃ��m�������̐��l��ŕ\���ł���B
�u�‹t�E�s�‹t���v�̃e�[�}�ɍ��킹�Ă͂��邪�A
�M���ۂ͉�݂������������Ė����u�M�͊w�̑��@���v�Ƃ͕ʕ��ł��邱�Ƃ𒍈ӂ��Ă������B

\underline{������薳������}\, ����$2n$�‚̃����_���ɉ^�����闱�q�����̎d�؂�ꂽ�������ɂ���B
������d�؂�������ĉ�����悤�B
�^���ɂ����\underline{�\���Ȏ��Ԃ̌�}$n$�‚���������
$n$�‚��E�����ɑ��݂���m���́A�e���q���E�����̔����ɂ���m���͊e1/2��---�x���k�[�C���z---�����āA
\[
P_1=\dfrac{2N!}{N!N!}\cdot (1/2)^{2N} %\label{143}
\]
�ł���B
���Ȃ݂�$n=1000$�Ƃ��Čv�Z�����
\[
P_1=
\]
�ƂȂ�B
�������ɑ��݂��闱�q��$x$�̊m�����z�͓񍀕��z$\mathrm{Bi}(2n, \, 1/2)$�ł��邪�A
���S�Ɍ��藝�ł�����ߎ������$N(n,\, n/4)$�ŁA1�V�O�}�͈͂�
\[
N \pm \sqrt{N/2} %\label{145}
\]
�ƂȂ�B
$x=n$�i�����j�̋ɑ�ɑ΂�����$(\sqrt{n}/2)/n$�ŁA�ɑ�͂���߂ĉs���m���炵���B

\underline{���������璁����}\, �^���ɂ���Ă���炷�ׂĂ̗��q�������ꂩ�����̕Б��ɂ��낢�W������m���́A
�������񍀕��z����
\[
P_2=(1/2)^{2n} \cdot 2=(1/2)^{2n-1}
\]
�ŁA������{\gt 0�ł͂Ȃ���}�ɂ߂ď������A������u�����Ċώ@����邱�Ƃ͂Ȃ��v\footnote{�{����}�B

\[
P_2/P_1=\dfrac{N!N!}{2 \cdot 2N!} %\label{147}
\]
���̐ݒ�̌���ł́A$\text{����}\, \Longrightarrow \, \text{������}$�̓]�ڂ�
\kenten{�ق�}�N���肦�Ȃ����Ƃ������I�Ɂi�m���_�I�Ɂj�����ł��悤�B

\noindent
\Fig[�}�ƕ\]{\textwidth}{10\baselineskip}

\section{�L���̂Ȃ��ɖ����Ɖi����ǂݍ���}

�n���͌Ǘ��n�ł͂Ȃ����A�S�F���́i��`�ɂ��j�Ǘ��n�ł���B
�G���g���s�[���傪�s�‹t�Ȃ�---�G���g���s�[�͑��傷�邵���Ȃ��Ȃ�---
�u�i���v�̒����Ԍ�ł͂��邪�A�S�F���̃G���g���s�[�͂��‚��ɑ�ɒB����B
�G���g���s�[�̓h�C�c�̕����w��R.�N���E�W�E�X�ɂ���Ē�`����(1865)�A
����͗L���Ș_���u�M�̈ړ��͂ɂ‚��āv{\"U}ber die bewegende Kraft der W{\"a}rme(1850)�Ɋ�Â��Ă����B
�G���g���s�[���ɑ�ɒB����ΑS�F���͂��͂�M�I���፷���������ăt���b�g��l�ɂȂ�A
��؂̕����I�d���i�ړ��j�̌����������āA�F���͊��S��~�i���j�̏�Ԃ��}����B
���̏I�ǃC���[�W�̓N���E�W�E�X�AW.�g���\���i�P�����B�����j�ɂ�邪�A�u�M�I���vheat death\footnote{
���̌�̐��m�ȑn�Ď҂ɂ‚��Ă͕s���B}�ƌ`�e���Ă悩�낤�B

�Ȋw�҂͎��R����w�Ԃ̂ł͂Ȃ��A���R�́u�@���v����w�сA�@�����l�ԂɂƂ��ĉ����Ӗ����邩����̓I�ɗ�������f�n�������Ă���B
 
\begin{quote}
---���̂悤�ɂ��āA�i�v�@�ւ̖���Nj������l�X���A�Ò��͍��I�ɖa���n�߂����́A
��������‚̕��ՓI�Ȏ��R�̍��{�@���ւƓ����A���̖@���́A�F���j�̏��߂ƏI��̉����Èłɂ܂Ō��𑗂��Ă���
(welches Lichtstrahlen in die fernen N{\"a}chte des Anfang und des Endes der Geschichte des Weltalls aussendet)�B
���̖@���́A�����l�ނɑ΂��Ă��A�����ł͂��邪�A�����������ĉi���ȑ���(ewig Bestehen)�����e������̂ł͂Ȃ��B
���̖@���́A�R���̓�\footnote{
�V�񐹏��u���n�l�َ��^�v�ȂǂɌ����郆�_���E�L���X�g���̏I���_�v�z�ł́A
���E�ɂ́u�I���̓��v����߂��Ă���A���`�̐_�ɂ��u�Ō�̐R���v���s����B
�\���͂���Ă��邪�ǂ̂悤�Ȍ`��󋵂ł��“������邩�͒m�炳��Ă��Ȃ��B
�������A���‚ł����Ă�����ɂ��Ȃ��Đ����������ϗ��ɐ����Ȃ���΂Ȃ�Ȃ��B
 �����ł́A�ے��I�ɁA�M�͊w�̑��@���̎w�������u�M�I���v���u���E�̏I���v�̋N������̈�‚ƌ����ĂĂ���B}
�������Đl�ނ����������̂ł͂��邪�A���̓��̗��鎞�́A�K���ɂ��܂��x�[���ɕ�܂�Ă���B
�e�l�Ɠ����悤�ɁA�l�ނ��܂����̎��ł̍l���Ɋ����Ȃ���΂Ȃ�Ȃ��B
�������l�ނ́A���Ɏ��ł������̐����`�Ԃ����D���āA�����ϗ��I�ۑ�(h{\"o}here sittliche Aufgaben)�����‚��̂ł���A
���̉ۑ��w�����A������������邱�Ƃɂ���Ă��̎g�����ʂ������̂ł���B
�i�w�����z���c�O�f�j---
\end{quote}

\section{�u�����I�v�łȂ����l��10�{�y���߂Ȃ�}

�����͂����Ō����̐��E�ɖ߂�B
����������$x,\, y,\, \ldots$�͐��w�I�Ȃ��̂łȂ��A�����ɕԂ����ʂł���B

�����̓I�J�l�⃂�m�̉��l�����̗�($x$)�ɔ�Ⴕ�Ȃ����Ƃ�m���Ă���B
�m���Ă���ǂ��납�A����ȑO�ɁA�����łȂ����ӎ��Ɋ����܂������łȂ����X�s�Ȃ��Ă���B
�����̓I�J�l�⃂�m�ɔ�Ⴕ�Ċy���߂�킯���Ȃ��B
�e����ς�f��������������Ƃ�����10�H�ׂ��10�{���������킯�ł͂Ȃ����A
�s�[�i�b�c������Ƃ��āA�ی��Ȃ��H�ׂ�ΐ�̕��ł͂��悻�����̂������Ȃ��Ȃ�A���������Ăǂ����ŃX�g�b�v���邾�낤�B
�e�����m��f���Ƃ͗ϗ��A������̖��ȑO�ŁA10�{�̃I�J�l�⃂�m�Łu10�{�y���ށv�Ƃ��u10�{�K���v�ɂȂ�킯�ł͂Ȃ��͓̂��R�ł���B
�����́A�����I�A�����I�ȗ�$x$�ɂ��̂܂܊�Â��Đ����Ă���킯�ł͂Ȃ��A
���Ƃ����Ĕ�Ⴗ���$ax$�A���邢��$ax+b$�Ő����Ă���킯�ł��Ȃ��B

�����A����͏d�v�Ȃ��Ƃ����A������x�͈̔͂ɂ����ĂȂ�A��葽���y���݂�葽���K���ɂȂ邱�Ƃ��炢�͌��������ł���B
�p���ċꂵ���Ȃ�����݂��߂ŕs�K�ɂȂ邱�Ƃ͂Ȃ��B
���ہA$2>1$�ł���ȏ�2�‚̃����S��1�‚�艿�l�������������A
$10>5$�ł���ȏ�10���~��5���~��肢�낢��Ȃ��Ƃ��ł��邩��A���l���������Ƃ��m���ł���B
�����I��$x$�͌����Ė��֌W�ł͂Ȃ������Ă���Ƃ͌�����B
�ł́A$x$��\underline{�ǂ�}�����̂��B

\section{�����ΐ��_}

�x���k�[�C�Ƃ̐l�X�́A�N�w�̃g���[�j���O��ς�ł����B
�����̒m������l�X�̖񑩎��̂悤�Ȃ��̂ł���������A�����I�ɍl���邱�Ƃɂ͊���Ă����B
�܂��Ă�_�j�G���͐����w�����ɂ��Ă����B
�����ɏq�ׂ����������ǂ�������t���Ă������킩��Ȃ�
---�킩��؂��Ă��邪�䂦�ɂ������ē��---���ɂ�����قNj�J�͂��Ȃ������B

�e�l�Ԃɂ͐S��������A���l�͐S���ɂ��ƂÂ��f�Ƃ����΂��Ȃ�߂��̂����A�S���w�Ȃ���̂͂܂��Ȃ������B
�����A���Ȃ��Ƃ��A���炩�ɂ���́u���_�v�̌��ۂł���B
�����I�A�����I���݂����łȂ��A�u���_�v�̑��݂�����A
�I�J�l�⃂�m��\kenten{����}�����ŁA����ɉ��l�����o���̂́u���_�v�ł���B
���Ȃ킿�I�J�l�����m�Ɍ��o����鉿�l�̓_�j�G���E�x���k�[�C�́u���_�I���l�v�Ȃ̂ł���B

�u���_�I���l�vmoral value��moral�ɂ͂������u�����E�_���́v�Ƃ��������I�Ӗ������邪�A�����Ёw��p�a���T�x�ł́A
��3�ԖڂɁu�����I�Ȃ��̂ɑ΂��Đ��_�I�v�Ƃ�����ꂪ����B
�����͂��́e���_�I�f�Ƃ����`�e���͍L���p�����A�e�l�ԓI�f����Ɂe�Љ�I�f�܂ŋy��ł���B
���R�u�E�x���k�[�C��Ars Conjectandi�̋��ɖړI�́A�m���̉��p���Љ��l�Ԃ̒��Ɍ��o���\�z�ł������B
���̑�IV���́e�Љ�̒��ɂ���m���f
---�p�X�J����t�F���}�[�A�z�C�w���X�͂������낵�����ɂȂ�����---��ڂ������̂ŁA
�����Ƃ��Ă͌^�j��ɑ�_�ȍ\�z�������낤�B
���̍\�z�͂����̃_�j�G���ɂ����Ă��������m�Ȍ`���Ƃ����B
�I�J�l�̐��_�I���l�ł���B
�����炩�ɂ���͌o�ϊw�����ē��v�w�̎n�܂����������̂ł��������B

�e���n�f�̐l�ɂ́A�����ϕ��i���R�u�A���n���E�x���k�[�C�j�◬�̗͊w�A�M�͊w�i�_�j�G���E�x���k�[�C�j�Ɨ��āA
�������A�C�}�C�Ń~�X�e���A�X�Ȃ��̂ɘA�ꍞ�܂ꂽ�Ɗ�����l�͑����B
�u���_�v�͏��Ȃ��Ƃ��u�����v�ł͂Ȃ��A���m�ł����m�ł��Ȃ��B
����͐l�ԂɊւ��邱�Ƃł�����蓖�R�ł���̂����A���́A
���P�C�ɂ��킩��A���_�I���l���l���Ȃ���΂Ȃ�Ȃ����ۂ����݂���B
���ꂪ�_�j�G���E�x���k�[�C�����ɂ����u�Z���g�E�y�e���X�u���O�̋t��(St. Petersburg's paradox)�v�ł���B

\begin{quote}
---���܁A���z�I�ȍd�݁i$\text{�\�̊m��}=p=1/2$�j���A�\���o��܂œ����‚Â���B
1��ڂŏo���2�~�A2��ڂȂ��4�~�A3��ڂȂ�8�~�A$\ldots $�A$n$��ڂȂ�$2^n$�~���󂯎��B
�q�̎Q������$f$�~�Ƃ���B
���̓q���ɎQ�����ׂ����B---
\end{quote}

���̓q���瓾����z�̊��Ғl���v�Z���悤�B
$n$��ڂɂ͂��߂ĕ\���o��Ƃ����m����$(1/2)^n$�Ȃ̂ŁA$\text{�m��}\times \text{���z}$�̘a���Ƃ�΁A
\[
(1/2)   \times 2   +
(1/2)^2 \times 2^2 +\cdots +
(1/2)^n \times 2^n +\cdots =\infty
\]
\noindent
\Fig[����139��]{\textwidth}{5\baselineskip}
�ŁA������ƂȂ�B
���������āA$f$�������獂���Ă��q�ɎQ�������������ƂȂ邪�A�Ƃ��Ă������ł��ʃo�J�o�J�������_�ł���B
���ہA$f$�������Ȃ�΁A���̓q�ɎQ������l�̐��͌�������ł��낤�B
�Ⴆ��$f=1000$�~�Ƃ����$n \geqq 10$�łȂ���΂��̓q�͑��ƂȂ�A�e�܂��߂Ɂf�q����l�Ȃ�΂��̓q�ɂ͂܂��Q�����Ȃ�
�i�h�X�g�G�t�X�L�[�w�q���ҁx�ɂ������a�I�q���҂ɂ͒ʂ��Ȃ��j�B
�ǂ����Ă��̂悤�Ȍ�������w�I���ʂɂȂ�̂ł��낤���B
����𐳂����C��������@�͂Ȃ��̂��낤���B

�j�R���X�E�x���k�[�C��1713�N9��9����P. R. de�������[���ɂ��ĂĂ�������N�����B
���ꂪ��́u�Z���g�E�y�e���u���O�̋t���v�ł���B
1728�NG. Cramer��$\sqrt{x}$�֐���p���Ă���������i��q�j�A5��21���Ƀj�R���X�ɓ`�����B
�����āA�j�R���X��1732�N4��5���A�_�j�G���ɂ����`���Ă���B
�_�j�G���͌��ʂ�s�\���ƍl���A1738�N���̗L���Ș_���ƂȂ����B
�_���̓��e����i�ꕔ�t�����X��j��
\begin{quote}
Specimen theoriae novae de mensura sortis \\
{\it Commentarii Academiae Scientiarum Imperialis Petropolitanae}
\end{quote}
�����Ė󂹂΁w�q���̑���Ɋւ���V���_�̌��{�x�ƂȂ낤�B
specimen�i���{�j�́A�����������̂��쐬���܂����̂ň�‚������������A���炢�̈Ӗ��ŁA
�����ɂ͒P�ɋt���������Ƃ��������A�����ƑO�i���ĐV���_���Ă���C�T������\footnote{
���\�́u�Z���g�E�؃e���u���N�鍑�Ȋw�@�_�����v�i���邢�͋I�v�j��ŁA
Petropolitan�̓��V�A�鍑�̃y�e���u���N�̃M���V���������e����󂵂����́B}�B

���āA���������u�q���v�isors,\,$\text{�p}=\text{lottery}$�j�͂��̋��z�Œ��ڕ]���ł�����̂ł͂Ȃ��A
����K�؂ȉ��l�֐��i�ΐ��֐��j�Ɋ��Z���������ŕ��ρi���Ғl�j���Ƃ��āA
���̎������l�imoral value�S���I�F����̉��l�j���߂�ׂ��ł���B
���̖���͉��l�̖{���ɔ������I�Ȃ��̂ŁA���̎��Ɛ}�Ő��������B

$C,\, D,\, E,\, F,\, \ldots$�͓�����”\���̂�����z�A$m : n : p : q : \cdots$�͂���炪������”\���̔䗦�Ƃ���B
�K�؂Ȋ֐���p�ӂ��A���̊֐��ŋ��z�̎������l�]����$CG,\, DH,\, EL,\, FM,\, \ldots$�ƂȂ����Ƃ���΁A
���̓q���̎������l�͂��̊��Ғl
\[
\text{�i��������H�j}
\]
�ŕ\�����B
�����āA����͗L���Ȓl�ɂ��邱�Ƃ��ؖ��ł��邵�A���یv�Z���”\�ł���B

\noindent
\Fig[�}]{\textwidth}{10\baselineskip}

���ݎ��Y�z��$B$�_�i�傫��$AB$�j�Ȃ�΋��z�͂���Ŋ����Ă����Ă��悢�B
�������̉��l�֐���ΐ��֐�$\log x$�i��ʂ�$b \log x$�B������$b$�͌��ʓI�Ɍ����Ȃ��j����Ȃ�΁A
\[
\text{�i��������H�j}
\]
�ł���B
��������z�ɖ߂��Čv�Z�������Ȃ�
\[
\text{�i��������H�j}
\]
�ƂȂ�A���̏ꍇ�͂���ʼn��������B
�Ȃ����l�֐��͐��w�I�ɂ͑ΐ��֐��Ɍ��闝�R�͂Ȃ��A$\sqrt{}$�֐��ł������x���Ȃ��B
�������A���ݎ��Y���ʂ���肭����Ȃ��Ȃǂ̌��_������B

���̂悤�ȉ��l�֐�����ꂽ���Ȃ��Ȃ�A���̂悤�ɍl���Ă��悢�B
�t���̐ݒ�ł́A�q���̒񋟎ҁi�n�E�X�j�͖����̎��Y��ۗL����Ɖ��肳��Ă��邪�A����͌����I�łȂ��B
���ɑ傫�����L���̎��Y��ۗL���A�Q���҂͂�������x�Ƃ��ēq���̏܋��𓾂�ƍl����΁A
�q���̊��Ғl�͗L���ƂȂ�A�t���͉�������B
���ہA
\[
\text{�i��������H�j}
\]
�Ƃ����v�Z���ʂ�����B

�o�ϊw��P. �T�~���G���\���́A�s�ꃁ�J�j�Y���̂��Ƃł͂���قǂ̋���ȑ���������Z���i�͂���������������Ȃ�����A
���Ƃ��Ƌt���͐����Ȃ��Ƃ����B
�@
\section{�u���p�v�i�R�E���E�j�Ƃ́H}

���āA���_���ɓ����Ē��ׂ��̂����A��������߂��Ă݂悤�B
�ݕ��̉��l�Ƃ����̂�����܂��͌o�ϊw�̖��ł���B
�����Ȃ肾���A�u���p�Ƃ͉����낤���B�v
�_�j�G���E�x���k�[�C�́e���_�I���l�f�ƌ����Ă��鉿�l���A�����ł͌o�ϊw�Łu���p�vutility�ƍL���������炳��Ă���T�O�ł���B
���Ƃ��A�e���l�f�Ƃ́A�����P�����ƁA�ǂ����ƁA�C���������ƁA���ɗ��‚��ƁA�����������ƁA�K���Ȃ��Ɓc�c�Ȃǂ�
���̂ŁA��������炩�̌����߂����邱�ƂƂ��āu���p�v�Ɖ��Ɍ������Ƃɂ��悤
�i����Ȃ炻������Ȃ��Ă������̂����c�c�j�B
����ƁA�����܂ŏq�ׂ����Ƃ͂܂����̖���ƂȂ�B
$x$��$10x$�ɂȂ��Ă��V�A���Z�̌����ڂ��P�����I��10�{�ɂȂ�킯�ł͂Ȃ����A�����I�ɃV�A���Z���L�т�킯�ł͂Ȃ��B

\begin{quote}
---���m��I�J�l�̗ʂ�$x$�Ƃ���ƁA$x$�̌��p$u(x)$��$x$���邢��$ax$���邢��$ax+b$�Ƃ͕\�킳�ꂸ�A����֐�$u(x)$�ł���---
\end{quote}

�������Ƃ���ƁA$u(x)$�́e���_�I�Ȃ��́f��\�킷�䂦�A�����A�C�}�C�ŃG�^�C�̒m��Ȃ��������ƂȂ낤�B
�m���ɂ��������ʂ͂���A���ꂪ�u���p�֐��v$u(x)$���ƈ�ʂ莦�����Ƃ͂ł��Ȃ��B
���������l�ɂ��ꍇ�ɂ��܂�$x$�����ł��邩�ɂ�邾�낤�B
���Ȃ��Ƃ�
\[
x>y \text{�Ȃ�} \quad u(x)>u(y)
\]
�łȂ��Ă͂Ȃ�Ȃ����낤�B
����ȊO�Ƀn�b�L�����邱�Ƃ͂Ȃ��̂��낤���B

\section{�Z���g�E�y�e���u���O�̋t���̉���---�o�ϊw�̃X�^�[�g}

���������āA���_�I���l�̋Ȑ��́u���K�̌��p�Ȑ��v�ƂȂ�B
�}��������ł���B

�d�v�Ȃ��Ƃ́A���������z�i���Ƃ���5�~�j����1�~�����������ꍇ�ƁA
�傫�����z�i�Ⴆ��1000000�~�j����1�~�����������ꍇ�Ƃł́A����1�~�ɑ΂�����p�̑�����
---���E���p�Ƃ���---����҂̕��������Ə������B
�������قǁu�����v�̂��肪���݂��킩��Ȃ��B
����͓���̌o�����ł��邪�A�d���o���I�����Ōo�ϊw�Łu���E���p����\footnote{
�u�����v�Ƃ͂��񂾂񌸂�A�̈ӁB}
�̖@���v�Ƃ����Ă���B
�������A�x���k�[�C�̌��p�Ȑ�
\begin{align*}
& u(x)=k \log x \quad (k>0) \\
& \text{�i���R�ΐ��Ƃ��Ē��}e=2.71828\cdots \text{�j}
\end{align*}
�ł����̖@���͎�������Ă���B

�ȒP�̂��߂�$\mathrm{AB}=1, \, k=1$�Ƃ���ƁA$x\text{�~�̌��p}=\log x$�ƂȂ邪�A
\[
\log 2.72 x=\log 2.72+\log x=1+\log x
\]
���K����2.72�{��\kenten{�Ȃ邲�Ƃ�}�A���p��1���‚ӂ���B
\[
\log 10 x = \log 10 +\log x=2.3+\log x
\]
����āA10�{�ɂȂ��Ă���2.3���x�ӂ��邾���ł���B
�����āA10�{�V�A���Z�ɂȂ邱�Ƃ͂Ȃ��I

�Z���g�E�y�e���u���O�̖����A�q�ɂ���ċ��K���̂��̂𓾂�̂łȂ����K�̌��p---���_�I���l---�𓾂Ă���ƍl���A
���̉��l�̊��Ғl�Ƃ��āA���x��$\infty$�ł͂Ȃ�
\begin{align*}
&(1/2)\log 2+(1/2)^2 \log 2^2+(1/2)^3 \log 2^3+\cdots \\
&=(\log 2)\{ 1\times (1/2)+2\times (1/2)^2+3\times (1/2)^3+\cdots \} \\
&=\log 4
\end{align*}
�𓾂�Ƃ������ƂŁA�߂ł��������i�����j����i$\{ \quad \}\text{��}=2$�ɒ��Ӂj�B

�������Ƃ���������ς��A���̓q���̉��l�����K���Z����ƁA�e�m���ȁf4�~
---���̊��Z�l���u�m�����l�z�vcertainty (monetary) eqivalent�Ƃ���---
���^������p�ɓ������B
���p�ő��������̓q�̊��Z���Ғl�͂킸����4�~�ŁA�Q������4�~�ȉ��̂Ƃ������q�ɎQ������΂悢���A
�܂Ƃ��Ȍ��_�ł��낤�B
�Ɠ����ɁA�u���p�v�T�O�̗L�����A�Ó���������ŗ��j��͂��߂Ċm�������̂ł���B

\begin{itembox}[l]{�Ȃ��A�u�Z���g�E�y�e���u���N�v���B}
�_�j�G���E�x���k�[�C�̓Z���g�E�y�e���u���N��w�̋����ŁA�����ł��̉ۑ��m��A
�_���������̊w�m�@�I�v�ɏo�����B
�u�I�v�v�Ƃ͊w�Ғ��Ԃ̘_���W�ŁA�ӂ‚��̐l�X�͎��l���Ԃ̓��l�����v�������ׂ邩������Ȃ��B
�������A�w��̂��̕��ɂ��N�I���e�B�̐R���Ōf�ڋ��‚�����̂ŁA�Ȃ��Ȃ����т����B

�Ȃ��A���̒n���͎���I�ϑJ������A
\begin{quote}
1. �Z���g�E�y�e���u���N $\Longrightarrow$
2. �y�g���O���[�h�i�g�j \\
$\Longrightarrow$ 
3. ���j���O���[�h�i�g�j   $\Longrightarrow$
4. �T���N�g�E�y�e���u���N
\end{quote}
�ƕς����B
1.�Z���g�́u���v�ŃC�G�X�E�L���X�g�̒�q�y�e���𐹐l�Ƃ������A
���邢�͗��j��̌����҃s���[�g�����Ɉ��ނƂ����Ă��悢�B
2.��1.�̃��V�A���A�u�u���N�v�u�O���[�h�v�́c�s�A
3.�̓��V�A�v����A4.�̓\�A�����ŁA�e�T���N�g�f�e�u���N�f��1.�̉������ʂ������̂ŁA���ݖ��ł���B
�Ȃ��A�p�ꖼPetersburg�ł�s�𔭉����邪�A���V�A���ł͌�������̂ŁA
�����s���R�����As����ꂸ�������Ȃ���u�Z���g�v����ꂽ�n���ď̂Ƃ����B
\end{itembox}

\section{�u���p�v�̐؂ꖡ}

���ꂾ���L�p�Ȃ炻�̐؂ꖡ��m�肽�����̂����A�T�^�I�Ȍ���I�p�������Љ�悤�B���҂̒���������p����B

���X�N���ɂ�����l�ԍs�����݂Ă���ƁA���p�̊T�O�Ȃ��ɁA�ʏ�̋��K�̊��Ҋz
---���ҋ��z(expected monetary value)---�����ł́A
���ꖞ���ł���������ł��Ȃ����Ƃ́A�Z���g�E�y�e���u���O�̋t���ł݂��ʂ�ł���B
���ꂾ���ł͂Ȃ��B
�l�͂Ȃ��A���i�R�⃉�X�x�K�X�Ŗׂ���̂��H�@
�������A�J�W�m�̌o�c�҂͂߂����ɕ����Ĕj�Y���邱�Ƃ͂Ȃ��B
�΍Еی��Ƃ����e�q�f�ł̓v���~�A���i�ی����j�̓��v�́A�����̏ꍇ�A�x������ł��낤�ی�����荂�����A
�����łȂ���Εی���Ђ͓|�Y���Ă��܂��B
����ƁA�l�͂Ȃ��ی��ɓ���̂��H�@
�����̖��́A���p�̋Ȑ��̗l�q---����\ruby{����}{�����Ƃ�}---�ɂ���āA��������B

���p�Ȑ���\kenten{���^}�̂Ƃ��ɁA�\���ł���킳���悤�ȓq�i�����j���l���悤�B
�����A���s��50--50�Ƃ������X�N�̂��铊���ł����āA���������200���~�������A���s�����200���~��������B
�܂��A���������Ȃ��ꍇ�̌��݊z��300���~�ł���B

\noindent
\Fig[�\����140��]{\textwidth}{5\baselineskip}

�܂��A�����ɂ����ҋ��z�ōl����΁A
\[
\frac{1}{2}\times 500+\frac{1}{2}\times 100=300 \, \text{�i���~�j}
\]
�ƂȂ�A���������Ȃ�����z�ɓ������B
�䂦�Ɋ��ҋ��z�̊�́A��ʂ������A�����̉”ۂɂ‚��ĉ��������Ȃ��B
����A���K�z�łȂ����p�𓱓����āA���p�̊��Ғl�ōl���悤�B
�q�ɓ������Ƃ���ƁA
\[
u(500\text{���~})=\mathrm{QG}, \quad 
u(100\text{���~})=\mathrm{PF}
\]
�ł��邩��A��`$\mathrm{PFSQ}$�̒��_�A���藝�ɂ��A���p�̊��Ғl��
\[
\frac{1}{2} u(500\text{���~})+\frac{1}{2} u(100\text{���~})=\mathrm{S'M}
\]
�ƂȂ�B
����A�q�ɓ���Ȃ���΁A���p��
\[
u(300\text{���~})=\mathrm{SM}
\]
�����ŋȐ������^�ł��邩��A�K��
\[ \mathrm{S'M}< \mathrm{SM} \]
���Ȃ킿�A�q�i�����j��\kenten{���炸}�A���X�N��������邱�ƂɂȂ�B
���邢�́A�������Ƃ����K�z��---���p�ł͂Ȃ��A���̋��K��---���̂悤�ɂ����Ă��悢�B

���p�Ȑ����$\mathrm{R}$��$\mathrm{RS'}\parallel x$���Ȃ�悤�ɂƂ�A
$x$���ɉ����������̑���$\mathrm{N}$�A���̓_��$x$�̒l��$x_0$�Ƃ���B
�����������$\mathrm{RS'}=\mathrm{RN}$�Ȃ邱�Ƃ��l����ƁA�\���̓q�ɓ��邱�Ƃ́A
�z$x_0$�̋��K�i���i�}�j���m���Ɂ\�q�ɋ������Ɂ\�ێ����Ă��邱�ƂƁA����̌��p��^����z����̋��z�ł��邪�A
���p���l���ɓ��ꂽ�ꍇ�ɁA���X�N�̂��Ƃɂ��铊�����Y�̎����I
---���X�N�̂Ȃ����S�ȏꍇ�Ɋ��Z����---���l�ł���B
����$x_0$���A�u�m�����l�z�v�Ƃ������Ƃ͐�ɏq�ׂ��B
�������́A$x_0< 300$�A�‚܂�
\[
\text{�m�����l�z}<\text{����z}
\]
�Ɠ������Ƃ������Ă���B

���Ȃ݂ɁA�_�j�G���E�x���k�[�C�̌��p�֐��̂Ƃ��i�������A$k=1$�j�ɂ́A
\[
\mathrm{QG}=\log(5 \times 10^6), \qquad 
\mathrm{PF}=\log(1 \times 10^6)
\]
�ł��邩��A
\begin{align*}
\mathrm{RN}
&=\mathrm{S'M}
 =\dfrac{1}{2}(\mathrm{QG}+\mathrm{PF})
 =\log (\sqrt{5 \times 1 \times 10^{12}}) \\
x_0&=\sqrt{5}\times 10^6 
\fallingdotseq 223\text{��}6\text{��~} %\label{141}
\end{align*}
�‚܂�A���̓����̎����I���l�́A���������Ȃ�����z300���~���66���~����邱�ƂɂȂ�A���R�A
���X�N��������ē������T���邱�ƂɂȂ�̂ł���B

\section{�S�[���h�o�b�n�̖��}

\subsection{�u���͕�������vresurgo}

�o�[�[���́u�~�����X�^�[\footnote{
�����̏C���@�ɗR�������‘吹���B�ق��Ƀ{���A�R���X�^���X�A�G�b�Z���A�t���C�u���O�A�E�����Ȃǂɂ���B}�v�́A
������藬��郉�C��������Ɍ��鏬�����u�ɓV���Ղ��ė����Ă���B
�o�[�[����w�u�x���k�[�C���Ɂv�̃t���b�c���̊��߂ł��̑吹����K�ꂽ�B

\noindent
\Fig[�ʐ^]{\textwidth}{10\baselineskip}

���R�u�E�x���k�[�C�̕��͂��̑吹���̈�p�̒��ɍ��܂�A����̉��ɁA�ΐ��点��ƂƂ��ɁA�W��
\begin{quote}
{\it Eadem mutata resurgo}\footnote{
�Ăяオ��(rise again)�B
�����p��ŁA�����痧���オ��i��������j�̈ӁB} \\
(Though changed I shall rise the same.) \\
�ς��Ƃ����ǂ������`�Ŏ��͕�������
\end{quote}
���̐l�ւ���ƂƂ��ɏ������܂�Ă������B
�u��������v�i�p�Fresuurect�j�́A
�u������̕����v�‚܂�u\ruby{�S}{��݂���}��v�Ɖi���̐������Ӗ����L���X�g���M�‚̒��S���`�ł���B
���͂���ɂ͐��w�I�\�����^������B
�������ɕ��c�ȗ��̔M�S�ȐV���k�i���v�h����j�̖ʖږ��@������̂�����B

�����ŁA����������߂��B
\begin{quote}
IACOBUS BERNOULLI                         \\
MATHEMATICUS INCOMPARABILIS               \\
ACAD. BASIL.                              \\
VLTRA XVIII ANNOS PROF.                   \\
ACADEM. ITEM REGIAE PARIS. ET BEROLIN.    \\
SOCIUS                                    \\
EDITIS LUCUBRAT. INLUSTRIS.               \\
MORBO CHRONICO                            \\
MENTE AD EXTREMUM INTEGRA                 \\
ANNO SALUT. MDCCV. D. XVI. AUGUSTI        \\
AETATIS L. M. VII                         \\
EXTINCTUS                                 \\
RESURRECT.  PIOR. HIC  PRAESTOLATUR\footnote{
RESURRECT.\, PIOR. HIC PRAESTOLATUR 
���͐��߂��A�����Ă����Łi����ɂ���āj������҂��]�݁m���@�I�ɂ͎󓮑ԁn�c�c�̈ӁB}\\
IUDITHA STUPANA                           \\
XX ANNOR. UXOR                            \\
CUM DUOBUS LIBERIS                        \\
MARITO ET PARENTI                         \\
EHEU DESIDERATISS.                        \\
H.M.P.\footnote{
hoc monumentum posuit \,�u��������v�i���V�j�B���̌������L���ŏI���B} \\
���M�󒆁�
\end{quote}

\noindent
\Fig[�i�ʐ^�j]{\textwidth}{10\baselineskip}

\begin{itembox}[l]{�ΐ��点��}
�m����悤�ɁA���ʋȐ��̂点��ɂ͂��낢�날�邪�A
\noindent
\Fig[�\]{\textwidth}{5\baselineskip}

���R�u�E�x���k�[�C�͂��̑ΐ��点��
\[
\text{�i�ɕ\���Ŏ�������H�j}
\]
���C�ɓ����Ă����B
���̒�����������
\[
\text{�i��������H�j}
\]
�ł����āA�x���k�[�C�ɂ́u�����`�ōĐ�����v�Ƃ����@���I�e�[�}���Î����Ă����B
�܂��A���̂点��́u���p�点��vequiangular spiral�Ƃ������A
�e�_�ł̐ڐ������a�ƈ��p$\phi$���Ȃ��Ƃ������������邪�A�����$\theta,\, \theta +d\theta$�̓��a��
���O�p�`����e�Ղɏؖ������B
\end{itembox}

\noindent
\Fig[�}]{\textwidth}{10\baselineskip}

\noindent
\Fig[�x���k�[�C�ƔN���F�쐬��]{\textwidth}{10\baselineskip}


\include{Chap02}
\include{Chap03}
\include{Chap04}
\include{Chap05}
\end{document}

